\documentclass[12pt]{article}
\usepackage{amsmath,amssymb,amsthm,amsfonts,tabularx}
\usepackage[pdftex]{graphicx}
\usepackage[dvipsnames]{xcolor}
\usepackage{fancyhdr}
\usepackage{parskip}
\usepackage[shortlabels]{enumitem}
\pagestyle{fancy}
\usepackage{setspace}
\newcommand{\realname}[1]{\newcommand{\printrealname}{#1}}
\newcommand{\pset}[1]{\newcommand{\printpset}{#1}}
\newcommand{\mathclass}[1]{\newcommand{\printmathclass}{#1}}

%% Pagestyle setup
\setlength{\headheight}{0.75in}
\setlength{\oddsidemargin}{0in}
\setlength{\evensidemargin}{0in}
\setlength{\voffset}{-.5in}
\setlength{\headsep}{10pt}
\setlength{\textwidth}{6.5in}
\setlength{\headwidth}{6.5in}
\setlength{\textheight}{8in}
\lhead{Math \printmathclass}
\chead{\Large \textbf{Homework \printpset}}
\rhead{\printrealname}
\rfoot{Page \thepage}
\renewcommand{\headrulewidth}{0.5pt}
\renewcommand{\footrulewidth}{0.3pt}
\setlength{\textwidth}{6.5in}
\renewcommand{\baselinestretch}{1}
\setenumerate[0]{label=(\alph*)}
\newcommand{\todo}{\textcolor{red}{\textbf{TODO }}}

\newtheorem*{prop}{Proposition}
\newtheorem*{corollary}{Corollary}
\newtheorem*{lemma}{Lemma}
\theoremstyle{remark}
\newtheorem*{defn}{Definition}

\newtheoremstyle{named}{}{}{}{}{\bfseries}{.}{.5em}{\thmnote{Problem #3}}
\theoremstyle{named}
\newtheorem*{theorem}{Theorem}
\allowdisplaybreaks

%% DO NOT ALTER THE ABOVE LINES
%%%%%%%%%%%%%%%%%%%%%%%%%%%%%%%%%%%%%%%%%%%%


%% If you would like to use Asymptote within this document (which is optional), 
%% you can find out how at the following URL:
%%
%%   http://www.artofproblemsolving.com/Wiki/index.php/Asymptote:_Advanced_Configuration
%%
%% As explained there, you will want to uncomment the line below.  But be
%% sure to check the website because there are several other steps that must 
%% be followed.
%% \usepackage{asymptote}

%% Enter your real name here
%% Example: \realname{David Patrick}
\realname{Hanting Zhang}
\pset{13}
\mathclass{425A}

\renewcommand{\a}{\alpha}
\renewcommand{\b}{\beta}
\renewcommand{\d}{\delta}
\newcommand{\e}{\varepsilon}
\newcommand{\Z}{\mathbb Z}
\newcommand{\N}{\mathbb N}
\newcommand{\Q}{\mathbb Q}
\newcommand{\R}{\mathbb R}
\newcommand{\C}{\mathbb C}
\renewcommand{\bf}{\mathbf}
\newcommand{\id}[1]{\text{id}_{#1}}
\renewcommand{\implies}{\Rightarrow}
\newcommand{\coimplies}{\Leftarrow}
\renewcommand{\em}{\varnothing}
\renewcommand{\Im}{\text{Im}}
\newcommand{\abs}[1]{|#1|}
\newcommand{\bigabs}[1]{\left|#1\right|}

\begin{document}

Chapter 7; \# 1, 2, 3 (pg. 175)

\begin{theorem}[1]
    Prove that every uniformly convergent sequence of bounded functions is uniformly bounded. 
\end{theorem}

\begin{proof}
    Suppose \(f_n \to f\) uniformly on some set \(E\) such that each \(\abs{f_n(x)} \le M_n\) is bounded. Then there is some \(N\) such that \(\abs{f_n(x) - f(x)} < 1\) for all \(n \ge N\) and \(x \in E\). Thus we have \(\abs{f_n(x)} \le \abs{f_n(x) - f(x)} + \abs{f(x) - f_N(x)} + \abs{f_N(x)} \le 2 + M_N\) for all \(x \in E\). 

    Set \(M = \max(M_1, M_2, \dots, M_{N - 1}, 2 + M_N)\). Then \(f_n\) is uniformly bounded by \(M\), as desired. 
\end{proof}

\begin{theorem}[2]
    If \(\{f_n\}\) and \(\{g_n\}\) converge uniformly on a set \(E\), prove that \(\{f_n + g_n\}\) converges uniformly on \(E\). If, in addition, \(\{f_n\}\) and \(\{g_n\}\) are sequences of bounded functions, prove that \(\{f_ng_n\}\) converges uniformly on \(E\). 
\end{theorem}

\begin{proof}
    Let \(\e > 0\). Since \(f_n \to f\) and \(g_n \to g\) uniformly, we have \(N_f\) and \(N_g\) such that \(n \ge N_f\) implies \(\abs{f_n - f} < \e / 2\) and \(n \ge N_g\) implies \(\abs{g_n - g} < \e / 2\). Then set \(N = \max(N_f, N_g)\), so that for \(n \ge N\) we have \[\abs{(f_n + g_n} - (f + g) < \abs{f_n - f} + \abs{g_n - g} = \e.\]
    Thus \(f_n + g_n \to f + g\) uniformly.

    Now let \(\e > 0\) again, but assume that \(f_n\) and \(g_n\) are bounded. Suppose \(\abs{f} \le M\) and \(\abs{g} \le M'\). There are integers \(N_f\) and \(N_g\) such that \(n \ge N_f, N_g\) implies \(\abs{f_n - f} < 1\) and \(\abs{g_n - g} < 1\). Thus for all \(n \ge N = \max(N_f, N_g\), we have \(\abs{f_n} < 1 + M\) and \(\abs{g_n} < 1 + M'\).

    Now choose \(L_f\) and \(L_g\) such that \(n \ge L_f, L_g\) implies \(\abs{f_n - f} < \e / 2(1 + M')\) and \(\abs{g_n - g} < \e / 2(1 + M)\). For any \(n \ge \max(L_f, L_g, N)\), we have 
    \begin{align*}
        \abs{f_ng_n - fg} &\le \abs{f_n g_n - f_n g} + \abs{f_n g - fg} \\
        &\le \abs{f_n}\abs{g_n - g} + \abs{f_n - f} \abs{g} \\
        &\le \frac{(1 + M)\e}{2(1 + M)} + \frac{\e M'}{2(1 + M')} \\
        &\le \e.
    \end{align*}
    Hence \(f_n g_n \to fg\) uniformly, and we're done.
\end{proof}

\begin{theorem}[3]
    Construct sequences \(\{f_n\}\) and \(\{g_n\}\)  which converge uniformly on some set \(E\), but such that \(\{f_n g_n\}\) does not converge uniformly on \(E\). 
\end{theorem}

\begin{proof}
    Let \(f_n(x) = 1 - 1/n\) and \(g_n(x) = 1/x\) on \(E = (0, 1)\). The intuition is that although \(1/x\) is not uniformly continuous on \((0, 1)\), the \textit{constant sequence} \(g_n(x) = 1/x\) is trivially a uniformly convergent sequence. Thus we have \(f_n \to 1\) and \(g_n \to 1/x\).

    However, \(f_n g_n = (1 - \frac{1}{n})\frac{1}{x}\) does not converge uniformly to \(1/x\) since the pointwise sequences blow up as \(x \to 0\).
\end{proof}

\end{document}