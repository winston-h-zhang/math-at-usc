\documentclass[12pt]{article}
\usepackage{amsmath,amssymb,amsthm,amsfonts,tabularx}
\usepackage[pdftex]{graphicx}
\usepackage[dvipsnames]{xcolor}
\usepackage{fancyhdr}
\usepackage{parskip}
\usepackage[shortlabels]{enumitem}
\pagestyle{fancy}
\usepackage{setspace}
\usepackage{pgfplots}
\pgfplotsset{compat=1.18}
\newcommand{\realname}[1]{\newcommand{\printrealname}{#1}}
\newcommand{\pset}[1]{\newcommand{\printpset}{#1}}
\newcommand{\mathclass}[1]{\newcommand{\printmathclass}{#1}}

%% Pagestyle setup
\setlength{\headheight}{0.75in}
\setlength{\oddsidemargin}{0in}
\setlength{\evensidemargin}{0in}
\setlength{\voffset}{-.5in}
\setlength{\headsep}{10pt}
\setlength{\textwidth}{6.5in}
\setlength{\headwidth}{6.5in}
\setlength{\textheight}{8in}
\lhead{Math \printmathclass}
\chead{\Large \textbf{\printpset}}
\rhead{\printrealname}
\rfoot{Page \thepage}
\renewcommand{\headrulewidth}{0.5pt}
\renewcommand{\footrulewidth}{0.3pt}
\setlength{\textwidth}{6.5in}
\renewcommand{\baselinestretch}{1}
\setenumerate[0]{label=(\alph*)}
\newcommand{\todo}{\textcolor{red}{\textbf{TODO }}}

\newtheorem*{prop}{Proposition}
\newtheorem*{corollary}{Corollary}
\newtheorem{lemma}{Lemma}
\theoremstyle{remark}
\newtheorem*{defn}{Definition}
\newtheorem*{remark}{Remark}

\newtheoremstyle{named}{}{}{}{}{\bfseries}{.}{.5em}{\thmnote{Problem #3}}
\theoremstyle{named}
\newtheorem*{theorem}{Theorem}
\allowdisplaybreaks

%% DO NOT ALTER THE ABOVE LINES
%%%%%%%%%%%%%%%%%%%%%%%%%%%%%%%%%%%%%%%%%%%%


%% If you would like to use Asymptote within this document (which is optional), 
%% you can find out how at the following URL:
%%
%%   http://www.artofproblemsolving.com/Wiki/index.php/Asymptote:_Advanced_Configuration
%%
%% As explained there, you will want to uncomment the line below.  But be
%% sure to check the website because there are several other steps that must 
%% be followed.
%% \usepackage{asymptote}

%% Enter your real name here
%% Example: \realname{David Patrick}
\realname{Hanting Zhang}
\pset{W14P1}
\mathclass{425B}

\renewcommand{\a}{\alpha}
\renewcommand{\b}{\beta}
\renewcommand{\d}{\delta}
\newcommand{\e}{\varepsilon}
\newcommand{\Z}{\mathbb Z}
\newcommand{\N}{\mathbb N}
\newcommand{\Q}{\mathbb Q}
\newcommand{\R}{\mathbb R}
\newcommand{\C}{\mathbb C}
\renewcommand{\bf}{\mathbf}
\newcommand{\id}[1]{\text{id}_{#1}}
\renewcommand{\implies}{\Rightarrow}
\newcommand{\coimplies}{\Leftarrow}
\renewcommand{\em}{\varnothing}
\renewcommand{\Im}{\text{Im}}
\newcommand{\abs}[1]{|#1|}
\newcommand{\bigabs}[1]{\left|#1\right|}
\newcommand{\Rloc}{\mathcal R_{\text{loc}}}

%% should be reset

\begin{document}

\begin{theorem}[1.1]
    (Modified Single-Variable Inverse Function Theorem) Let \(f : (a, b) \to \R\) be a differentiable function with \(f'(x) > 0\) for all \(x \in (a, b)\).
    \begin{enumerate}
        \item Prove that \(f\) is injective, and argue that its image must be an open interval \((c, d)\) (with \(c\) and/or \(d\) possibly infinite).
        \item By part (a), there exists a function \(g : (c, d) \to (a, b)\) such that \(g(f(x)) = x\) for all \(x \in (a, b)\). Prove that \(g\) is continuous.
        \item Prove that \(g\) is differentiable, and the \(g'(f(x)) = \frac{1}{f'(x)}\), for all \(x \in (a, b)\). (Hint: Pick \(y \in (c, d)\), and let \((y_n)_{n = 1}^\infty\) be a sequence in \((c, d)\) that converges to \(y\). Write the difference quotient \(\frac{g(y_n) - g(y)}{y_n - y}\) in terms of \(f\) and a sequence \((x_n)_{n = 1}^\infty\) in \((a, b)\).)
    \end{enumerate}
\end{theorem}

\begin{theorem}[1.2]
    Define the function \(f : \R^2 \to \R\) by 
    \begin{align*}
        f(x, y) = 2x^3 - 3x^2 + 2y^3 + 3y^2 = (x + y)(2x^2 - 2xy + 2y^2 - 3x + 3y).
    \end{align*}
    \begin{enumerate}
        \item Find the four points in \(\R^2\) where the gradient of \(f\) is zero. Use the Second Derivative Test to show that \(f\) has exactly one local maximum and one local minimum in \(\R^2\). 
        \item Let \(S\) denote the level set \(\{(x, y) \in \R^2 : f(x, y) = 0\}\) of \(f\) at the value \(0\). Let \(S_1\) denote the subset of \(S\) consisting of those points \((x, y)\) of \(S\) at which \(\partial_1 f(x, y) = 0\). Determine \(S_1\) completely (it consists of four points).
        \item Failure of the hypotheses of the Implicit Function Theorem at a given point \((a, b)\) doesn't guarantee that one cannot `solve for one variable in terms of the other' near \((a, b)\). However, it turns out that in this particular example, one cannot solve for \(x\) as a function of \(y\) near any of the four points of \(S_1\). Give a heuristic argument for this statement, making reference to Figure 1(A).
        \item Pick one of the points \((a, b)\) in \(S_1\) and demonstrate rigorously that one cannot solve for \(x\) as a function of \(y\) near \((a, b)\). That is for any neighborhood \(U\) of \((a, b)\), show that there exist \(y \in \R\) such that \((x_1, y), (x_2, y) \in U\), \(f(x_1, y) = f(x_2, y) = 0\), and \(x_1 \neq x_2\). 
    \end{enumerate}
\end{theorem}

\begin{theorem}[1.3]
    Let \(X\) be a real normed vector space and let \(U\) be a open subset of \(X\). Assume that \(f : U \to \R\) is continuous and let \(a\) be a point of \(U\). 
    \begin{enumerate}
        \item Show that if \(f\) achieves a local minimum at \(a\), then \(f\) is not injective on any neighborhood of \(a\). Hint: Reduce this to a problem about a function of a single real variable, and use the Intermediate Value Theorem. Not that \(f\) is not assumed to be differentiable anywhere. (For future reference, note also that the same conclusion holds if \(f\) has a local maximum at \(a\) instead of a local minimum.)
    \end{enumerate}
\end{theorem}

\end{document}