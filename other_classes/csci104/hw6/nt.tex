\documentclass[12pt]{article}
\usepackage{amsmath,amssymb,amsthm,amsfonts,tabularx}
\usepackage[pdftex]{graphicx}
\usepackage[dvipsnames]{xcolor}
\usepackage{fancyhdr}
\usepackage{parskip}
\usepackage[shortlabels]{enumitem}
\pagestyle{fancy}
\usepackage{setspace}
\newcommand{\realname}[1]{\newcommand{\printrealname}{#1}}
\newcommand{\pset}[1]{\newcommand{\printpset}{#1}}
\newcommand{\mathclass}[1]{\newcommand{\printmathclass}{#1}}

%% Pagestyle setup
\setlength{\headheight}{0.75in}
\setlength{\oddsidemargin}{0in}
\setlength{\evensidemargin}{0in}
\setlength{\voffset}{-.5in}
\setlength{\headsep}{10pt}
\setlength{\textwidth}{6.5in}
\setlength{\headwidth}{6.5in}
\setlength{\textheight}{8in}
\lhead{\printmathclass}
\chead{\Large \textbf{Homework \printpset}}
\rhead{\printrealname}
\rfoot{Page \thepage}
\renewcommand{\headrulewidth}{0.5pt}
\renewcommand{\footrulewidth}{0.3pt}
\setlength{\textwidth}{6.5in}
\renewcommand{\baselinestretch}{1}
\setenumerate[0]{label=(\alph*)}
\newcommand{\todo}{\textcolor{red}{\textbf{TODO }}}

\newtheorem*{prop}{Proposition}
\newtheorem*{corollary}{Corollary}
\newtheorem*{lemma}{Lemma}
\theoremstyle{remark}
\newtheorem*{defn}{Definition}

\newtheoremstyle{named}{}{}{}{}{\bfseries}{.}{.5em}{\thmnote{Problem #3}}
\theoremstyle{named}
\newtheorem*{theorem}{Theorem}
\allowdisplaybreaks

%% DO NOT ALTER THE ABOVE LINES
%%%%%%%%%%%%%%%%%%%%%%%%%%%%%%%%%%%%%%%%%%%%


%% If you would like to use Asymptote within this document (which is optional), 
%% you can find out how at the following URL:
%%
%%   http://www.artofproblemsolving.com/Wiki/index.php/Asymptote:_Advanced_Configuration
%%
%% As explained there, you will want to uncomment the line below.  But be
%% sure to check the website because there are several other steps that must 
%% be followed.
%% \usepackage{asymptote}

%% Enter your real name here
%% Example: \realname{David Patrick}
\realname{Hanting Zhang}
\pset{6}
\mathclass{CSCI 104}

\renewcommand{\a}{\alpha}
\renewcommand{\b}{\beta}
\renewcommand{\d}{\delta}
\newcommand{\e}{\varepsilon}
\newcommand{\Z}{\mathbb Z}
\newcommand{\N}{\mathbb N}
\newcommand{\Q}{\mathbb Q}
\newcommand{\R}{\mathbb R}
\newcommand{\C}{\mathbb C}
\renewcommand{\bf}{\mathbf}
\newcommand{\id}[1]{\text{id}_{#1}}
\renewcommand{\implies}{\Rightarrow}
\newcommand{\coimplies}{\Leftarrow}
\renewcommand{\em}{\varnothing}
\renewcommand{\Im}{\text{Im}}
\renewcommand{\mod}{\text{ mod }}
\newcommand{\abs}[1]{|#1|}
\newcommand{\bigabs}[1]{\left|#1\right|}
\newcommand{\floor}[1]{\left\lfloor#1\right\rfloor}

\begin{document}

\begin{theorem}[1]
    Use the recursive pattern \(x_{n + 1} = (a x_n + c) \mod m\) to generate the first 5 pseudorandom numbers \(x_1, x_2, \dots, x_5\) is the sequence given \(a = 13, c = 7, x_0 = -5, m = 12\).
\end{theorem}

Note that \(x_{n + 1} = 13 x_n + 7 \mod 12 = x_n + 7 \mod 12\). Then just compute: 
\begin{align*}
    x_1 &= x_0 + 7 \mod 12 = -5 + 7 \mod 12 = 2 \\
    x_2 &= x_1 + 7 \mod 12 = 9 \\
    x_3 &= x_2 + 7 \mod 12 = 4 \\
    x_4 &= x_3 + 7 \mod 12 = 11 \\
    x_5 &= x_4 + 7 \mod 12 = 6 \\
\end{align*}

\begin{theorem}[2]
    How many zeros are at the end of \(100!\)?
\end{theorem}

The number of zeros is equal to the number of factors of 10 in \(100!\). Every factor of 10 is made from exactly one factor of 2 and 5. Since there are more factors of 2 than factors of 5, the number of zeros is equal to the number of factors of 5. There are 
\[\floor{\frac{100}{5}} + \floor{\frac{100}{5^2}} = 20 + 4 = 24\]
factors of 5, hence there are 24 zeros at the end of \(100!\).
\newline

\begin{theorem}[3]
    Prove that for any integer \(n\), \(n^5 - 5n^3 + 4n\) is divisible by 5.
\end{theorem}

\begin{proof}
    The term \(-5n^3\) is always divisble by 5, so it suffices to prove that \(n^5 - 4n\) is divisible by 5. Note that Fermat's Little Theorem, \(n^4 \equiv 1 \mod 5\) for any \(n\). Thus 
    \[n^5 + 4n \equiv n(n^4 + 4) \equiv n(1 + 4) \equiv 5n \equiv 0 \mod 5,\]
    which is what we want. Hence \(n^5 - 5n^3 + 4n\) is divisible by 5 for any \(n\), as desired.
\end{proof}

\begin{theorem}[4]
    Compute \(1333^{42} \mod 11\).
\end{theorem}

Note \(11^3 = 1331\) so \(1333 \equiv 2 \mod 11\). Then \(\varphi(11) = 10\), so 
\[1333^{42} \equiv 2^{42} \equiv (2^{10})^4 (2^2) \equiv 1^4 \cdot 2^2 \equiv 4 \mod 11.\]

\newpage
\begin{theorem}[5]
    Two integers \(x, y \in \Z\) are said to be relatively prime if their greatest common divisor is 1. Use (and show the steps to) the Euclidean algorithm to determine if 309 and 112 are relatively prime.
\end{theorem}

Just compute:
\begin{align*}
    309 &= 2(112) + 85 \\
    112 &= 1(85) + 27 \\
    85 &= 3(27) + 4 \\
    27 &= 6(4) + 3 \\
    4 &= 1(3) + 1 \\
    3 &= 3(1).
\end{align*}
Hence \(\gcd(309, 112) = 1\) and 309 and 112 are relatively prime.
\newline

\begin{theorem}[6]
    Solve \(54x + 16y = \gcd(54, 16)\). Show your work in such a way that allows the grader to recognize that you understand the relevant lecture material.
\end{theorem}

Just compute:
\begin{align*}
    54 &= 3(16) + 6 \\
    16 &= 2(6) + 4 \\
    6 &= 1(4) + 2 \\
    4 &= 2(2).
\end{align*}
Hence \(\gcd(54, 16) = 2\). Now to find \(x\) and \(y\):
\begin{align*}
    2 &= 6 - 1(4) \\
    2 &= 6 - (16 - 2(6)) = 3(6) - 16 \\
    2 &= 3(54 - 3(16)) - 16 = 3(54) - 9(16) - 16 = 3(54) - 10(16) \\
    &\Rightarrow x = 3, y = 10.
\end{align*}

\begin{theorem}[7]
    Find the multiplicative inverse of \(x \equiv 33 \mod 112\).
\end{theorem}

We want to solve \(33x + 112y = 1\). Just compute:
\begin{align*}
    112 &= 3(33) + 13 \\
    33 &= 2(13) + 7 \\
    13 &= 1(7) + 6 \\
    7 &= 1(6) + 1 \\
    6 &= 6(1).
\end{align*}
Hence \(\gcd(112, 33) = 1\). Now find \(x\) and \(y\):
\begin{align*}
    1 &= 7 - 1(6) \\
    &= 7 - 1(13 - 1(7)) = 2(7) - 13 \\
    &= 2(33 - 2(13)) - 13 = 2(33) - 5(13) \\
    &= 2(33) - 5(112 - 3(33)) = 17(33) - 5(112) \\
    &\Rightarrow x = 17, y = -5.
\end{align*}
So \(17\) is the inverse of \(33\). 


\end{document}