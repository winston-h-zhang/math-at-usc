%%%%%%%%%%%%%%%%%%%%%%%%%%%%%%%%%%%%%%%%%%%%%%
%%                                          %%
%% USE THIS FILE TO SUBMIT YOUR SOLUTIONS   %%
%%                                          %%
%% You must have the usamts.tex file in     %%
%% the same directory as this file.         %%
%% You do NOT need to submit this file or   %%
%% usamts.tex with your solutions.  You     %%
%% only need to submit the output PDF file. %%
%%                                          %%
%% DO NOT ALTER THE FILE usamts.tex         %%
%%                                          %%
%% If you have any questions or problems    %%
%% using this file, or with LaTeX in        %%
%% general, please go to the LaTeX          %%
%% forum on the Art Of Problem Solving      %%
%% web site, and post your problem.         %%
%%                                          %%
%%%%%%%%%%%%%%%%%%%%%%%%%%%%%%%%%%%%%%%%%%%%%%

%%%%%%%%%%%%%%%%%%%%%%%%%%%%%%%%%%%%%%%%%%%%
%% DO NOT ALTER THE FOLLOWING LINES
\documentclass[12pt]{article}
\usepackage{amsmath,amssymb,amsthm,amsfonts,tabularx}
\usepackage[pdftex]{graphicx}
\graphicspath{ {./images/} }
\usepackage{fancyhdr}
\pagestyle{fancy}
\usepackage{setspace}
\usepackage{csquotes}
%%%%%%%%%%%%%%%%%%%%%%%%%%%%%%%%%%%%%%%%%%%
%%                                       %%
%% Students: DO NOT MODIFY THIS FILE!!!  %%
%%                                       %%
%%%%%%%%%%%%%%%%%%%%%%%%%%%%%%%%%%%%%%%%%%%

%% USAMTS style sheet
%% last modified: 23-Jul-2004

%% Student and Year/Round data
\newcommand{\realname}[1]{\newcommand{\printrealname}{#1}}
\newcommand{\pset}[1]{\newcommand{\printpset}{#1}}

%% Pagestyle setup
\setlength{\headheight}{0.75in}
\setlength{\oddsidemargin}{0in}
\setlength{\evensidemargin}{0in}
\setlength{\voffset}{-.5in}
\setlength{\headsep}{10pt}
\setlength{\textwidth}{6.5in}
\setlength{\headwidth}{6.5in}
\setlength{\textheight}{8in}
\lhead{Math 410}
\chead{\Large \textbf{Homework \printpset}}
\rhead{\printrealname}
\rfoot{Page \thepage}
\renewcommand{\headrulewidth}{0.5pt}
\renewcommand{\footrulewidth}{0.3pt}
\setlength{\textwidth}{6.5in}


\renewcommand{\baselinestretch}{1}
%% DO NOT ALTER THE ABOVE LINES
%%%%%%%%%%%%%%%%%%%%%%%%%%%%%%%%%%%%%%%%%%%%


%% If you would like to use Asymptote within this document (which is optional), 
%% you can find out how at the following URL:
%%
%%   http://www.artofproblemsolving.com/Wiki/index.php/Asymptote:_Advanced_Configuration
%%
%% As explained there, you will want to uncomment the line below.  But be
%% sure to check the website because there are several other steps that must 
%% be followed.
%% \usepackage{asymptote}

\newtheorem*{prop}{Proposition}
\newtheorem*{corollary}{Corollary}
\newtheorem*{lemma}{Lemma}
\theoremstyle{remark}
\newtheorem*{defn}{Definition}

\newtheoremstyle{named}{}{}{}{}{\bfseries}{.}{.5em}{\thmnote{Problem #3}}
\theoremstyle{named}
\newtheorem*{theorem}{Theorem}

%% Enter your real name here
%% Example: \realname{David Patrick}
\realname{Hanting Zhang}
\pset{6}
\mathclass{410}

\renewcommand{\bf}{\mathbf}
\renewcommand{\implies}{\Rightarrow}
\newcommand{\coimplies}{\Leftarrow}
\newcommand{\Aut}{\text{Aut}}

\begin{document}
Exercises 16, 17, pp. 138.

\begin{theorem}[16]
    Prove that \((\mathbb Z / 24 \mathbb Z)^\times\) is an elementary abelian group of order 8.
\end{theorem}

\begin{theorem}[17]
    Let \(\rangle G \langle\) be a cyclic group of order \(n\). For \(n = 2, 3, 4, 5, 6,\) write out the elements of \(\Aut(G)\) explicitly.
\end{theorem}

Exercises 3, 5, 6, 7, 8, 14 pp. 184-187.

\begin{theorem}[3]
    Continue for Example 1. Prove that every element of \(G - H\) has order 2. Prove that \(G\) is abelian if and only if \(h^2 = 1\) for all \(h \in H\). 
\end{theorem}

\begin{theorem}[5]
    Let \(G = \text{Hol}(\mathbb Z_2 \times \mathbb Z_2)\).
    \begin{enumerate}
        \item[(a)] Prove that \(G = H \rtimes K\) where \(H = \mathbb Z_2 \times \mathbb Z_2\) and \(K \cong S_3\). Deduce that \(|G| = 24\).
        \item[(b)] Prove that \(G\) is isomorphic to \(S_4\).  
    \end{enumerate}
\end{theorem}

\begin{theorem}[6]
    
\end{theorem}

\begin{theorem}[7]
    
\end{theorem}

\begin{theorem}[8]
    Construct an non-abelian group of order 75. Classify all groups of order 75. 
\end{theorem}

\begin{theorem}[14]
    
\end{theorem}

Exercises 2, 5 pp. 165-167.



\begin{theorem}[5]
    Let \(G\) be a finite abelian group of type \((n_1, n_2, \dots, n_t)\). Prove that \(G\) contains an element of order \(m\) if and only if \(m \div n_1\). Deduce that \(G\) is of exponent \(n_1\).
\end{theorem}

Exercise 15 p. 174.

\begin{theorem}[15]
    If \(A\) and \(B\) are normal subgroups of \(G\) such that \(G / A\) and \(G / B\) is both abelian, prove that \(G / A \cap B\) is abelian.
\end{theorem}

\end{document}