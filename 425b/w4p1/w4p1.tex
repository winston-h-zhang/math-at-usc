\documentclass[12pt]{article}
\usepackage{amsmath,amssymb,amsthm,amsfonts,tabularx}
\usepackage[pdftex]{graphicx}
\usepackage[dvipsnames]{xcolor}
\usepackage{fancyhdr}
\usepackage{parskip}
\usepackage[shortlabels]{enumitem}
\pagestyle{fancy}
\usepackage{setspace}
\newcommand{\realname}[1]{\newcommand{\printrealname}{#1}}
\newcommand{\pset}[1]{\newcommand{\printpset}{#1}}
\newcommand{\mathclass}[1]{\newcommand{\printmathclass}{#1}}

%% Pagestyle setup
\setlength{\headheight}{0.75in}
\setlength{\oddsidemargin}{0in}
\setlength{\evensidemargin}{0in}
\setlength{\voffset}{-.5in}
\setlength{\headsep}{10pt}
\setlength{\textwidth}{6.5in}
\setlength{\headwidth}{6.5in}
\setlength{\textheight}{8in}
\lhead{Math \printmathclass}
\chead{\Large \textbf{\printpset}}
\rhead{\printrealname}
\rfoot{Page \thepage}
\renewcommand{\headrulewidth}{0.5pt}
\renewcommand{\footrulewidth}{0.3pt}
\setlength{\textwidth}{6.5in}
\renewcommand{\baselinestretch}{1}
\setenumerate[0]{label=(\alph*)}
\newcommand{\todo}{\textcolor{red}{\textbf{TODO }}}

\newtheorem*{prop}{Proposition}
\newtheorem*{corollary}{Corollary}
\newtheorem{lemma}{Lemma}
\theoremstyle{remark}
\newtheorem*{defn}{Definition}
\newtheorem*{remark}{Remark}

\newtheoremstyle{named}{}{}{}{}{\bfseries}{.}{.5em}{\thmnote{Problem #3}}
\theoremstyle{named}
\newtheorem*{theorem}{Theorem}
\allowdisplaybreaks

%% DO NOT ALTER THE ABOVE LINES
%%%%%%%%%%%%%%%%%%%%%%%%%%%%%%%%%%%%%%%%%%%%


%% If you would like to use Asymptote within this document (which is optional), 
%% you can find out how at the following URL:
%%
%%   http://www.artofproblemsolving.com/Wiki/index.php/Asymptote:_Advanced_Configuration
%%
%% As explained there, you will want to uncomment the line below.  But be
%% sure to check the website because there are several other steps that must 
%% be followed.
%% \usepackage{asymptote}

%% Enter your real name here
%% Example: \realname{David Patrick}
\realname{Hanting Zhang}
\pset{W4P1}
\mathclass{425B}

\renewcommand{\a}{\alpha}
\renewcommand{\b}{\beta}
\renewcommand{\d}{\delta}
\newcommand{\e}{\varepsilon}
\newcommand{\Z}{\mathbb Z}
\newcommand{\N}{\mathbb N}
\newcommand{\Q}{\mathbb Q}
\newcommand{\R}{\mathbb R}
\newcommand{\C}{\mathbb C}
\renewcommand{\bf}{\mathbf}
\newcommand{\id}[1]{\text{id}_{#1}}
\renewcommand{\implies}{\Rightarrow}
\newcommand{\coimplies}{\Leftarrow}
\renewcommand{\em}{\varnothing}
\renewcommand{\Im}{\text{Im}}
\newcommand{\abs}[1]{|#1|}
\newcommand{\bigabs}[1]{\left|#1\right|}
\newcommand{\Rloc}{\mathcal R_{\text{loc}}}
\newcommand{\infint}{\int_{-\infty}^{\infty}}
\begin{document}

\begin{theorem}[2.1]
    Let \(V\) be a real vector space, and let \(E\) be a convex subset of \(V\). Then \(f : E \to \R\) is a convex function if and only if for every \(a, b \in E\) and \(\lambda \in [0, 1]\), we have 
    \[f(\lambda a + (1 - \lambda)b) \leq \lambda f(a) + (1 - \lambda) f(b).\]
\end{theorem}

\begin{proof}
    We want to show that for every pair of points \((a, f(a)), (b, f(b)) \in \text{epi}(f)\) and every \(\lambda \in [0, 1]\), we have \(\lambda (a, f(a)) + (1 - \lambda) (b, f(b)) \in \text{epi}(f)\).

    Indeed, we have 
    \begin{align*}
        \lambda (a, f(a)) + (1 - \lambda) (b, f(b)) &= (\lambda a + (1 - \lambda) b, \lambda f(a) + (1 - \lambda) f(b)).
    \end{align*}

    Since \(E\) is convex, \(\lambda a + (1 - \lambda) b = c\) for some point \(c \in E\). Then by assumption, 
    \begin{align*}
        y = \lambda f(a) + (1 - \lambda) f(b) \geq f(\lambda a + (1 - \lambda b)) = f(c).
    \end{align*}
    Thus we satisfy the conditions that show
    \[(\lambda a + (1 - \lambda) b, \lambda f(a) + (1 - \lambda) f(b)) \in \text{epi}(f).\]
\end{proof}

\begin{theorem}[2.2]
    Given an example of a convex function \(f : \R \to \R\) which is convex, but whose square in not convex.
\end{theorem}

\begin{proof}
    The function \(x^2 - 1\) works. This function is a parabola and obivously convex. Its square is \((x + 1)^2 (x - 1)^2\). This is not convex since letting \(a = (-1, 0), b = (1, 0)\) and \(\lambda = 1/2\), the point \((0, 0)\) is not in the epigraph of the square. Essentially the ``squaring'' makes a W shape that destroys convexity.
\end{proof}

\begin{theorem}[2.3]
    (H\"older's inequality.) Assume \(f, g \in \Rloc(\R)\), and let \(p\) and \(q\) be positive real numbers satisfying \(\frac{1}{p} + \frac{1}{q} = 1\), and assume that the integrals \(\infint\abs{f(x)}^p dx\) and \(\infint\abs{g(x)}^q dx\) converge. Then \(\infint f(x) g(x) dx\) converges absolutely, and 
    \[\infint\abs{f(x)}\abs{g(x)} dx \leq \left(\infint \abs{f(x)}^p dx\right)^{\frac{1}{p}} \left(\infint \abs{g(x)}^q dx\right)^{\frac{1}{q}}.\] 
\end{theorem}

\newpage 

\begin{lemma}
    Suppose \(f\) and \(g\) are nonnegative functions satisfying \(\infint f(x)^p dx = \infint g(x)^q dx = 1\). Then in this special case, H\"older's inequality holds.
\end{lemma}

\begin{proof}
    By Young's inequality, we have for all \(x\), \(f(x) g(x) \leq \frac{f(x)^p}{p} + \frac{g(x)^q}{q}\).
    Thus, noting that \(f = |f|\) and \(g = |g|\),
    \begin{align*}
        \infint f(x) g(x) dx &\leq \frac{1}{p}\infint f(x)^p dx + \frac{1}{q}\infint g(x)^q dx \\
        &= \frac{1}{p} + \frac{1}{q} = 1 = (1)^\frac{1}{p} (1)^\frac{1}{q}\\
        &= \left(\infint f(x)^p dx\right)^{\frac{1}{p}} \left(\infint g(x)^q dx\right)^{\frac{1}{q}},
    \end{align*}
    we've shown H\"older's inequality as desired.
\end{proof}

\begin{proof}
    Now for H\"older's inequality in the general case of arbitrary functions \(f, g \in \Rloc(\R)\). Denote \(\|f(x) \| = \left(\infint \abs{f(x)}^p dx\right)^\frac{1}{p}\). If \(\|f\| = 0\) or \(\|g\| = 0\), then by Exercise 1.6, the LHS of the inequality is 0, and thus trivial. Otherwise, define \(F = |f| / \| f \|\) and \(G = |g| / \|g\|\). We have \(f'\) and \(g'\) are nonnegative functions which satisfy
    \begin{align*}
        \infint F(x)^p dx = \frac{1}{\|f\|^p}\infint \abs{f(x)}^p dx = \frac{\|f\|^p}{\|f\|^p} = 1.
    \end{align*} 
    and
    \begin{align*}
        \infint G(x)^q dx = \frac{1}{\|g\|^q}\infint \abs{g(x)}^q dx = \frac{\|g\|^q}{\|g\|^q} = 1.
    \end{align*}
    Thus we may apply Lemma 1 and see that, 
    \begin{align*}
        \infint F(x) G(x) dx &\leq \left(\infint F(x)^p dx\right)^{\frac{1}{p}} \left(\infint G(x)^q dx\right)^{\frac{1}{q}},
    \end{align*}
    which implies
    \begin{align*}
        \frac{1}{\|f\|\|g\|}\infint \abs{f(x)}^p \abs{g(x)}^q dx \leq \frac{1}{\|f\|^{p (1/p)}\|g\|^{q (1/q)}} \left(\infint \abs{f(x)}^p dx\right)^{\frac{1}{p}} \left(\infint \abs{g(x)}^q dx\right)^{\frac{1}{q}}.
    \end{align*}
    Clearing \(\|f\|\|g\|\) on boths sides gives H\"older's inequality, as desired.
\end{proof}

\begin{theorem}[2.4]
    Tweak H\"older's inequality slightly to prove that for \(\lambda \in [0, 1]\), we have 
    \begin{align*}
        \infint |f(x)|^\lambda |g(x)|^{1 - \lambda} dx \leq \left(\infint |f(x)| dx\right)^\lambda \left(\infint |g(x)|\right)^{1 - \lambda},
    \end{align*}
    whenever \(\lambda \in [0, 1]\) and \(f, g \in \Rloc(\R)\).
\end{theorem}

\begin{proof}
    This trivially follows from Exercise 2.3 with setting \(p = 1/\lambda\), \(q = 1 / (1 - \lambda)\), \(f \leftarrow f^{1/p}\), and \(g \leftarrow g^{1/q}\). There is a slight hiccup with the case \(\lambda = 0, 1\), but in these special cases, the inequality reduces to a trivial one.
\end{proof}

\begin{theorem}[2.5]
    This Exercise establishes some properties of log-convex functions.
    \begin{enumerate}
        \item Let \(f : (\a, \b) \to (0, \infty)\) (with \(-\infty \leq \a \leq \b \leq \infty\)) be a function. Then \(f\) is log-convex if and only if for any \(a, b \in (\a, \b)\), and \(\lambda \in [0, 1]\), we have
        \[f(\lambda a + (1 - \lambda) b) \leq f(a)^\lambda f(b)^{1 - \lambda}.\]
        \item Tweak Young's inequality to prove that for \(A, B \geq 0\) and \(\lambda \in [0, 1]\), one has 
        \[A^\lambda B^{1 - \lambda} \leq \lambda A + (1 - \lambda) B.\]
        Using this inequality and part (a), conclude that every log-convex function \(f : (\a, \b) \to (0, \infty)\) (with \(-\infty \leq \a \leq \b \leq \infty\)) is convex.
        \item Prove that the product of log-convex functions is log-convex.
    \end{enumerate}
\end{theorem}

\begin{proof}
    We proceed with each part.
    \begin{enumerate}
        \item We have \(f\) is log-convex iff \(\log \circ f\) is convex. This occurs iff 
        \[\log(f(\lambda a + (1 - \lambda) b)) \leq \lambda \log(f(a)) + (1 - \lambda) \log(f(b)),\]
        iff (use \(\log\) rules)
        \[\log(f(\lambda a + (1 - \lambda) b)) \leq \log(f(a)^\lambda f(b)^{1 - \lambda}),\]
        iff (monoticity of \(\log\))
        \[f(\lambda a + (1 - \lambda) b) \leq f(a)^\lambda f(b)^{1 - \lambda}.\]
        \item Just modify the proof of Young's inequality:
        \[A^\lambda B^{1 - \lambda} = \exp(\log A^\lambda B^{1 - \lambda}) = \exp(\lambda \log A + (1 - \lambda) \log B) \leq \lambda A + (1 - \lambda) B.\]
        Thus with part (a), log-convex functions satisfy
        \[f(\lambda a + (1 - \lambda) b) \leq f(a)^\lambda f(b)^{1 - \lambda} \leq \lambda f(a) + (1 - \lambda) f(b).\]
        i.e. they are convex, as desired.
        \item Let \(f\) and \(g\) be log-convex functions. We want to show that \(\log \circ fg\) is convex. Indeed, for all \(a, b \in (\a, \b)\), \(\lambda \in [0, 1]\), and \(c = \lambda a + (1 - \lambda) b)\),
        \begin{align*}
            \log ((f \cdot g) (c)) &= \log(f(c)) + \log(g(c)) \\
            &\leq \lambda \log(f(a)) + (1 - \lambda)\log(f(b)) + \log(g(a)) + (1 - \lambda)\log(g(b)) \\
            &= \lambda \log ((f \cdot g)(a)) + (1 - \lambda) \log ((f \cdot g) (b)).
        \end{align*}
        Thus the condition for convexity holds, as desired.
    \end{enumerate}
\end{proof}

\end{document}