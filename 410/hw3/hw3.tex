
%%%%%%%%%%%%%%%%%%%%%%%%%%%%%%%%%%%%%%%%%%%%%%
%%                                          %%
%% USE THIS FILE TO SUBMIT YOUR SOLUTIONS   %%
%%                                          %%
%% You must have the usamts.tex file in     %%
%% the same directory as this file.         %%
%% You do NOT need to submit this file or   %%
%% usamts.tex with your solutions.  You     %%
%% only need to submit the output PDF file. %%
%%                                          %%
%% DO NOT ALTER THE FILE usamts.tex         %%
%%                                          %%
%% If you have any questions or problems    %%
%% using this file, or with LaTeX in        %%
%% general, please go to the LaTeX          %%
%% forum on the Art Of Problem Solving      %%
%% web site, and post your problem.         %%
%%                                          %%
%%%%%%%%%%%%%%%%%%%%%%%%%%%%%%%%%%%%%%%%%%%%%%

%%%%%%%%%%%%%%%%%%%%%%%%%%%%%%%%%%%%%%%%%%%%
%% DO NOT ALTER THE FOLLOWING LINES
\documentclass[12pt]{article}
\usepackage{amsmath,amssymb,amsthm,amsfonts,tabularx}
\usepackage[pdftex]{graphicx}
\graphicspath{ {./images/} }
\usepackage{fancyhdr}
\pagestyle{fancy}
\usepackage{setspace}
\usepackage{csquotes}
%%%%%%%%%%%%%%%%%%%%%%%%%%%%%%%%%%%%%%%%%%%
%%                                       %%
%% Students: DO NOT MODIFY THIS FILE!!!  %%
%%                                       %%
%%%%%%%%%%%%%%%%%%%%%%%%%%%%%%%%%%%%%%%%%%%

%% USAMTS style sheet
%% last modified: 23-Jul-2004

%% Student and Year/Round data
\newcommand{\realname}[1]{\newcommand{\printrealname}{#1}}
\newcommand{\pset}[1]{\newcommand{\printpset}{#1}}
\newcommand{\mathclass}[1]{\newcommand{\printmathclass}{#1}}

%% Pagestyle setup
\setlength{\headheight}{0.75in}
\setlength{\oddsidemargin}{0in}
\setlength{\evensidemargin}{0in}
\setlength{\voffset}{-.5in}
\setlength{\headsep}{10pt}
\setlength{\textwidth}{6.5in}
\setlength{\headwidth}{6.5in}
\setlength{\textheight}{8in}
\lhead{Math \printmathclass}
\chead{\Large \textbf{Homework \printpset}}
\rhead{\printrealname}
\rfoot{Page \thepage}
\renewcommand{\headrulewidth}{0.5pt}
\renewcommand{\footrulewidth}{0.3pt}
\setlength{\textwidth}{6.5in}


\renewcommand{\baselinestretch}{1}
%% DO NOT ALTER THE ABOVE LINES
%%%%%%%%%%%%%%%%%%%%%%%%%%%%%%%%%%%%%%%%%%%%


%% If you would like to use Asymptote within this document (which is optional), 
%% you can find out how at the following URL:
%%
%%   http://www.artofproblemsolving.com/Wiki/index.php/Asymptote:_Advanced_Configuration
%%
%% As explained there, you will want to uncomment the line below.  But be
%% sure to check the website because there are several other steps that must 
%% be followed.
%% \usepackage{asymptote}

\newtheorem*{prop}{Proposition}
\newtheorem*{corollary}{Corollary}
\newtheorem*{lemma}{Lemma}
\theoremstyle{remark}
\newtheorem*{defn}{Definition}

\newtheoremstyle{named}{}{}{}{}{\bfseries}{.}{.5em}{\thmnote{Problem #3}}
\theoremstyle{named}
\newtheorem*{theorem}{Theorem}

%% Enter your real name here
%% Example: \realname{David Patrick}
\realname{Hanting Zhang}
\pset{3}
\mathclass{410}

\renewcommand{\bf}{\mathbf}
\renewcommand{\implies}{\Rightarrow}
\newcommand{\coimplies}{\Leftarrow}

\begin{document}

Exercises 3, 8, 20, 23, 25, pp. 40-41;

\begin{enumerate}
    \item [3.] ($\Rightarrow$): If $H$ is abelian, then for any $a, b \in G$, $\varphi(ab) = \varphi(a)\varphi(b) = \varphi(b)\varphi(a) = \varphi(ba)$, where we use all our hypotheses. But $\varphi$ is injective, so $ab = ba$. Hence $G$ is abelian.
    
    ($\Leftarrow$): If $G$ is abelian, then do the same argument with $\varphi^{-1}$. For any $a, b \in H$, $\varphi^{-1}(ab) = \varphi^{-1}(a) \varphi^{-1}(b) = \varphi^{-1}(b)\varphi^{-1}(a) = \varphi^{-1}(ba)$. Now $\varphi^{-1}$ is injective, so $G$ is abelian.
    \item [8.] The orders of $S_n$ and $S_m$ are $n!$ and $m!$, respectively. Since sizes are non-equal so there cannot exist a bijection between $S_n$ and $S_m$.
    
    \item [20.] We prove the group axioms for $\text{Aut}(G)$.
    
    \begin{enumerate}
        \item \textit{Identity}: Let $\text{id}_G : G \to G$ be the identity. Clearly for any $\varphi \in \text{Aut}(G)$, $\varphi \circ \text{id}_G = \text{id}_G \circ \varphi = \varphi$. Hence $\text{id}_G$ is the identity. 
        \item \textit{Associativity}: Note that function composition is associative, so multiplication in $\text{Aut}(G)$ is by definition associative.
        \item \textit{Closure}: We need to prove that for any $\varphi, \phi \in \text{Aut}(G)$, $\varphi \circ \phi \in \text{Aut}(G)$. Indeed, if $\varphi$ and $\phi$ are isomorphisms, then for any $g, h \in G$, $\varphi(\phi(gh)) = \varphi(\phi(g)\phi(b)) = \varphi(\phi(g)) \varphi(\phi(h))$. Hence $\varphi \circ \phi$ is a homomorphism. Furthermore, the composition of two bijections is a bijection, so $\varphi \circ \phi$ is a isomorphism as well.
        \item \textit{Inverses}: If $\varphi \in \text{Aut}(G)$, then the function inverse $\varphi^{-1} G \to G$ is also the inverse of $\varphi$ in $\text{Aut}(G)$.
    \end{enumerate}

    \item [23.] 
    \item [25.] 
\end{enumerate}

Exercises 4, 5, 6, 20, 21, pp. 44-45.

\begin{enumerate}
    \item [4.]
    \item [5.]
    \item [6.]
    \item [20.]
    \item [21.]
\end{enumerate}

Exercises 8, 10, 15 ,17, pp. 48-49.

\begin{enumerate}
    \item [8.]
    \item [10.]
    \item [15.]
    \item [17.]
\end{enumerate}

Exercises 5, 10, 12, pp. 52-53.

\begin{enumerate}
    \item [5.]
    \item [10.]
    \item [12.]
\end{enumerate}

\end{document}