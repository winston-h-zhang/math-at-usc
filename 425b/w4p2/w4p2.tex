\documentclass[12pt]{article}
\usepackage{amsmath,amssymb,amsthm,amsfonts,tabularx}
\usepackage[pdftex]{graphicx}
\usepackage[dvipsnames]{xcolor}
\usepackage{fancyhdr}
\usepackage{parskip}
\usepackage[shortlabels]{enumitem}
\pagestyle{fancy}
\usepackage{setspace}
\newcommand{\realname}[1]{\newcommand{\printrealname}{#1}}
\newcommand{\pset}[1]{\newcommand{\printpset}{#1}}
\newcommand{\mathclass}[1]{\newcommand{\printmathclass}{#1}}

%% Pagestyle setup
\setlength{\headheight}{0.75in}
\setlength{\oddsidemargin}{0in}
\setlength{\evensidemargin}{0in}
\setlength{\voffset}{-.5in}
\setlength{\headsep}{10pt}
\setlength{\textwidth}{6.5in}
\setlength{\headwidth}{6.5in}
\setlength{\textheight}{8in}
\lhead{Math \printmathclass}
\chead{\Large \textbf{\printpset}}
\rhead{\printrealname}
\rfoot{Page \thepage}
\renewcommand{\headrulewidth}{0.5pt}
\renewcommand{\footrulewidth}{0.3pt}
\setlength{\textwidth}{6.5in}
\renewcommand{\baselinestretch}{1}
\setenumerate[0]{label=(\alph*)}
\newcommand{\todo}{\textcolor{red}{\textbf{TODO }}}

\newtheorem*{prop}{Proposition}
\newtheorem*{corollary}{Corollary}
\newtheorem{lemma}{Lemma}
\theoremstyle{remark}
\newtheorem*{defn}{Definition}
\newtheorem*{remark}{Remark}

\newtheoremstyle{named}{}{}{}{}{\bfseries}{.}{.5em}{\thmnote{Problem #3}}
\theoremstyle{named}
\newtheorem*{theorem}{Theorem}
\allowdisplaybreaks

%% DO NOT ALTER THE ABOVE LINES
%%%%%%%%%%%%%%%%%%%%%%%%%%%%%%%%%%%%%%%%%%%%


%% If you would like to use Asymptote within this document (which is optional), 
%% you can find out how at the following URL:
%%
%%   http://www.artofproblemsolving.com/Wiki/index.php/Asymptote:_Advanced_Configuration
%%
%% As explained there, you will want to uncomment the line below.  But be
%% sure to check the website because there are several other steps that must 
%% be followed.
%% \usepackage{asymptote}

%% Enter your real name here
%% Example: \realname{David Patrick}
\realname{Hanting Zhang}
\pset{W4P2}
\mathclass{425B}

\renewcommand{\a}{\alpha}
\renewcommand{\b}{\beta}
\renewcommand{\d}{\delta}
\newcommand{\e}{\varepsilon}
\newcommand{\Z}{\mathbb Z}
\newcommand{\N}{\mathbb N}
\newcommand{\Q}{\mathbb Q}
\newcommand{\R}{\mathbb R}
\newcommand{\C}{\mathbb C}
\renewcommand{\bf}{\mathbf}
\newcommand{\id}[1]{\text{id}_{#1}}
\renewcommand{\implies}{\Rightarrow}
\newcommand{\coimplies}{\Leftarrow}
\renewcommand{\em}{\varnothing}
\renewcommand{\Im}{\text{Im}}
\newcommand{\abs}[1]{|#1|}
\newcommand{\bigabs}[1]{\left|#1\right|}
\newcommand{\Rloc}{\mathcal R_{\text{loc}}}

%% should be reset
\newcommand{\infint}{\int_{-\infty}^{\infty}}
\newcommand{\g}{\Gamma}
\renewcommand{\l}{\lambda}

\begin{document}

\begin{theorem}[3.1]
    Prove that the integral defining \(\g(x)\) converges for all \(x > 0\).
\end{theorem}

\begin{proof}
    Split the integral up between \((0, 1)\) and \((1, \infty)\) and deal with the convergence of each with different strategies. 

    On the unit interval, we have \(\int_{0}^{1} t^{x - 1} dt = \left . \frac{t^x}{x}\right |_{0}^1 = \frac{1}{x}\). But \(t^{x - 1}e^{-t} < t^{x - 1}\) on \(t \in (0, 1)\), thus \(\int_{0}^{1} t^{x - 1} e^t dt < \frac{1}{x}\) converges.

    On the \((1, \infty)\) ray, consider \(t^{x - 1} e^{-t} \leq e^{-t/2}\). This inequality is not outright true, but
    \[t^{x - 1} e^{-t} \leq e^{-t/2} \iff \frac{t^{x - 1}}{e^{t/2}} \leq 1 \hspace*{4mm} \text{and} \hspace*{4mm} \lim_{t \to \infty} \frac{t^{x - 1}}{e^{t/2}} = 0.\] 
    So for each \(t\), there is some \(N\) such that the inequality holds for all \(x \geq N\). Thus we may split the integral again,
    \begin{align*}
        \int_{1}^{\infty} t^{x - 1} e^{-t} dt &= \int_{1}^{N} t^{x - 1} e^{-t} dt + \int_{N}^{\infty} t^{x - 1} e^{-t} dt \\
        &\leq X + \int_{N}^{\infty} e^{-t/2} dt \\
        &= X - \left . 2e^{-t/2}\right |_N^\infty \\ 
        &= X + 2e^{-N/2} \\
        &< \infty,
    \end{align*} 
    where \(X\) is just some finite value. Thus \(\g(x)\) converges on both \((0, 1)\) and \((1, \infty)\), so it is a convergent integral, as desired.
\end{proof}

\begin{theorem}[3.2]
    Using H\"older's inequality, show that the Gamma function \(\g : (0, \infty) \to \R\) is log-convex.
\end{theorem}

\begin{proof}
    We want to show that \(\log(\g(\lambda a + (1 - \lambda) b)) \leq \l \log(\g(a)) + (1 - \l) \log(\g(b))\). Indeed, we use H\"older's inequality in the form of Exercise 2.3:
    \begin{align*}
        \log(\g(\lambda a + (1 - \lambda) b)) &= \log \left(\int_{0}^{\infty} t^{\l a + (1 - \l)b - 1} e^{-t} dt\right) \\
        &= \log \left(\int_{0}^{\infty} t^{(a - 1)\l} e^{-t\l}t^{(b - 1)(1 - \l)} e^{-t(1 - \l)} dt\right) \\
        &= \log \left(\int_{0}^{\infty} (t^{a - 1} e^{-t})^\l(t^{b - 1} e^{-t})^{1 - \l} dt\right) \\
        &\leq \log \left[\left(\int_{0}^{\infty}t^{a - 1}e^{-t} dt\right)^\l \left(\int_{0}^{\infty}t^{b - 1}e^{-t} dt\right)^{1 - \l}\right] \\
        &= \l \log \left(\int_{0}^{\infty}t^{a - 1}e^{-t} dt\right) + (1 - \l) \log \left(\int_{0}^{\infty}t^{b - 1}e^{-t} dt\right) \\
        &= \l \log(\g(a)) + (1 - \l) \log(\g(b)),
    \end{align*}
\end{proof}

\begin{theorem}[3.3]
    Let \(B : (0, \infty) \times (0, \infty) \to \R\) be defined by
    \[B(x, y) = \int_{0}^{1} t^{x - 1} (1 - t)^{y - 1} dt.\]
    This is called the \textit{beta function}.
    \begin{enumerate}
        \item Prove that \(B(1, y) = \frac{1}{y}\), for \(y \in (0, \infty)\).
        \item Prove that for each fixed \(y \in (0, \infty)\), the function \(x \mapsto B(x, y)\) is log-convex on \((0, \infty)\).
        \item Using an integration-by-parts on the identity
        \[B(x + 1, y) = \int_{0}^{1} \left(\frac{t}{1 - t}\right)^x ( 1- t)^{x + y - 1} dt,\]
        prove that \(B(x + 1, y) = \frac{x}{x + y} B(x, y)\).
        \item Argue that for each \(y > 0\), the function \(f_y : (0, \infty) \to \R\) defined by 
        \[f_y(x) = \frac{\g(x + y)}{\g(y)} B(x, y)\]
        satisfies the hypotheses of the Bohr-Mollerup Theorem. Conclude that 
        \[B(x, y) = \frac{\g(x)\g(y)}{\g(x + y)}, \hspace*{4mm} x, y > 0.\]
        \item Using the conclusion of part (d), and the substitution \(t = \sin^2 \theta\) in the definition of the beta function, show that \(B(\frac{1}{2}, \frac{1}{2}) = (\g(\frac{1}{2}))^2 = \pi\).
        \item Using the conclusion of part (e), and the substitution \(t = s^2\) in the definition of the Gamma function, conclude that 
        \[\infint e^{-s^2} dx = \sqrt{\pi}.\]
    \end{enumerate}
\end{theorem}

\begin{proof}
    We proceed with each part.
    \begin{enumerate}
        \item We have \[B(1, y) = \int_{0}^{1} (1 - t)^{y - 1} dt = \left . -\frac{(1 - t)^y}{y}\right |_0^1 = - \frac{0}{y} + \frac{1}{y} = \frac{1}{y}.\]
        
        \item We want to show that \(\log(B(\l a + (1 - \l)b, y)) \leq \l \log(B(a, y)) + (1 - \l)\log(B(b, y))\). Indeed, we have again by H\"older's inequality,
        \begin{align*}
            B(\l a + (1 - \l)b, y) &= \int_{0}^{1} t^{\l a + (1 - \l) b - 1}(1 - t)^{y - 1} dt \\
            &= \int_{0}^{1} t^{(a - 1)\l} (1 - t)^{(y - 1)\l}t^{(b - 1)(1 - \l)} (1 - t)^{(y - 1)(1 - \l)} dt \\
            &= \int_{0}^{1} (t^{a - 1} (1 - t)^{y - 1})^\l(t^{b - 1} (1 - t)^{y - 1})^{1 - \l} dt \\
            &\leq \left(\int_{0}^{1} t^{a - 1} (1 - t)^{y - 1} dt\right)^\l \left(\int_{0}^{1} t^{b - 1} (1 - t)^{y - 1}\right)^{1 - \l} \\
            &= B(a, y)^\l B(b, y)^{1 - \l}.
        \end{align*}
        Thus, 
        \begin{align*}
            \log (B(\l a + (1 - \l)b, y)) &\leq \log(B(a, y)^\l B(b, y)^{1 - \l}) \\
            &= \l \log(B(a, y)) + (1 - \l)\log(B(b, y)),
        \end{align*}
        as desired.

        \item Integrating by parts with \(u = \left(\frac{t}{1 - t}\right)^x\) and \(dv = (1 - t)^{x + y - 1}\), we compute first that \(du = \left(\frac{t}{1 - t}\right)^{x - 1}\frac{x dt}{(1 - t)^2}\) and \(v = -\frac{(1 - t)^{x + y}}{x + y}\). Then,
        \begin{align*}
            B(x + 1, y) &= \int_{0}^{1} \left(\frac{t}{1 - t}\right)^x ( 1- t)^{x + y - 1} dt \\
            &= \left . uv \right |_0^1 - \int_{0}^{1}v du \\
            &= \left . -\left(\frac{t}{1 - t}\right)^x \frac{(1 - t)^{x + y}}{x + y}  \right |_0^1 + \int_{0}^{1} \frac{(1 - t)^{x + y}}{x + y} \left(\frac{t}{1 - t}\right)^{x - 1}\frac{x dt}{(1 - t)^2} \\
            &= \left . - \frac{t^x(1 - t)^{y}}{x + y}  \right |_0^1 + \frac{x}{x + y}\int_{0}^{1} t^{x - 1}(1 - t)^{y - 1} dt \\
            &= 0 - 0 + \frac{x}{x + y} B(x, y) \\
            &= \frac{x}{x + y} B(x, y),
        \end{align*}
        as desired.
        \item We prove properties \(\g(\text{1-3})\).
        \begin{enumerate}
            \item[\((\g1)\):] We have 
            \begin{align*}
                f_y(x + 1) &= \frac{\g(x + 1 + y)}{\g(y)} B(x + 1, y) \\
                &= \frac{(x + y)\g(x + y)}{\g(y)} \frac{x}{x + y} B(x, y) \\
                &= x \frac{\g(x + y)}{\g(y)} B(x, y) = x f_y(x).
            \end{align*}
            \item[\((\g2)\):] We have 
            \begin{align*}
                f_y(1) &= \frac{\g(1 + y)}{\g(y)} B(1, y) = \frac{y \g(y)}{\g(y)}\frac{1}{y} = 1. \\
            \end{align*}
            \item[\((\g3)\):] Since the product of log-convex functions are convex, it suffices to show that for fixed \(y\), \(\frac{\g(x)\g(y)}{\g(x + y)}\) and \(B(x, y)\) are log-convex. The latter was proven in part (b). The former is a translation and scaling of a log-convex function, which is clearly log-convex. Thus \(f_y(x)\) must also be log-convex.
        \end{enumerate}
        So \(f_y(x)\) satisfies the hypotheses of the Bohr-Mollerup Theorem. Therefore, it must be true that
        \begin{align*}
            f_y(x) = \frac{\g(x + y)}{\g(y)} B(x, y) = \g(x).
        \end{align*}
        We conclude that 
        \[B(x, y) = \frac{\g(x)\g(y)}{\g(x + y)}, \hspace*{4mm} x, y > 0.\]
        \item With the substitution \(t = \sin^2 \theta\), we have \(dt = 2 \cos \theta \sin \theta d \theta\). Thus
        \begin{align*}
            \int_{0}^{1} t^{x - 1} (1 - t)^{y - 1} dt &= 2\int_{0}^{\pi/2} (\sin \theta)^{2x - 1} (\cos \theta)^{2y - 1} d \theta.
        \end{align*}
        Setting \(x = y = 1/2\), we have
        \begin{align*}
            B\left(\frac{1}{2}, \frac{1}{2}\right) &= 2\int_{0}^{\pi/2} (\sin \theta)^{2(1/2) - 1} (\cos \theta)^{2(1/2) - 1} d \theta \\
            &= \int_{0}^{\pi/2}d \theta = 2\pi / 2 = \pi.
        \end{align*}
        From part (d), \(B(\frac{1}{2}, \frac{1}{2}) = \frac{\g(1/2)\g(1/2)}{\g(1)} = (\g(\frac{1}{2}))^2\). Thus \(B(\frac{1}{2}, \frac{1}{2}) = (\g(\frac{1}{2}))^2 = \pi\), as desired.

        \item With the substitution \(t = s^2\), we have \(dt = 2 s ds\). Fixing \(x = 1/2\), we have 
        \begin{align*}
            \g(1/2) = \int_{0}^{\infty} t^{1/2 - 1} e^{-t} dt &= \int_{0}^{\infty} s^{2(1/2 - 1)} e^{-s^2} 2s ds = 2\int_{0}^{\infty} e^{- s^2} ds. \\
        \end{align*}
        Thus, remarkably, we conclude with part (d) that
        \[\infint e^{-s^2} dx = 2\int_{0}^{\infty} e^{- x^2} dx = \g(1/2) = \sqrt{\pi}.\]
    \end{enumerate}
\end{proof}

\end{document}