%%%%%%%%%%%%%%%%%%%%%%%%%%%%%%%%%%%%%%%%%%%%%%
%%                                          %%
%% USE THIS FILE TO SUBMIT YOUR SOLUTIONS   %%
%%                                          %%
%% You must have the usamts.tex file in     %%
%% the same directory as this file.         %%
%% You do NOT need to submit this file or   %%
%% usamts.tex with your solutions.  You     %%
%% only need to submit the output PDF file. %%
%%                                          %%
%% DO NOT ALTER THE FILE usamts.tex         %%
%%                                          %%
%% If you have any questions or problems    %%
%% using this file, or with LaTeX in        %%
%% general, please go to the LaTeX          %%
%% forum on the Art Of Problem Solving      %%
%% web site, and post your problem.         %%
%%                                          %%
%%%%%%%%%%%%%%%%%%%%%%%%%%%%%%%%%%%%%%%%%%%%%%

%%%%%%%%%%%%%%%%%%%%%%%%%%%%%%%%%%%%%%%%%%%%
%% DO NOT ALTER THE FOLLOWING LINES
\documentclass[12pt]{article}
\usepackage{amsmath,amssymb,amsthm,amsfonts,tabularx}
\usepackage[pdftex]{graphicx}
\usepackage[dvipsnames]{xcolor}
\usepackage{fancyhdr}
\usepackage{parskip}
\usepackage[shortlabels]{enumitem}
\pagestyle{fancy}
\usepackage{setspace}
\newcommand{\realname}[1]{\newcommand{\printrealname}{#1}}
\newcommand{\pset}[1]{\newcommand{\printpset}{#1}}
\newcommand{\mathclass}[1]{\newcommand{\printmathclass}{#1}}

%% Pagestyle setup
\setlength{\headheight}{0.75in}
\setlength{\oddsidemargin}{0in}
\setlength{\evensidemargin}{0in}
\setlength{\voffset}{-.5in}
\setlength{\headsep}{10pt}
\setlength{\textwidth}{6.5in}
\setlength{\headwidth}{6.5in}
\setlength{\textheight}{8in}
\lhead{Math \printmathclass}
\chead{\Large \textbf{Homework \printpset}}
\rhead{\printrealname}
\rfoot{Page \thepage}
\renewcommand{\headrulewidth}{0.5pt}
\renewcommand{\footrulewidth}{0.3pt}
\setlength{\textwidth}{6.5in}
\renewcommand{\baselinestretch}{1}
\setenumerate[0]{label=(\alph*)}

\newtheorem*{prop}{Proposition}
\newtheorem*{corollary}{Corollary}
\newtheorem*{lemma}{Lemma}
\theoremstyle{remark}
\newtheorem*{defn}{Definition}

\newtheoremstyle{named}{}{}{}{}{\bfseries}{.}{.5em}{\thmnote{Problem #3}}
\theoremstyle{named}
\newtheorem*{theorem}{Theorem}

%% DO NOT ALTER THE ABOVE LINES
%%%%%%%%%%%%%%%%%%%%%%%%%%%%%%%%%%%%%%%%%%%%


%% If you would like to use Asymptote within this document (which is optional), 
%% you can find out how at the following URL:
%%
%%   http://www.artofproblemsolving.com/Wiki/index.php/Asymptote:_Advanced_Configuration
%%
%% As explained there, you will want to uncomment the line below.  But be
%% sure to check the website because there are several other steps that must 
%% be followed.
%% \usepackage{asymptote}

%% Enter your real name here
%% Example: \realname{David Patrick}
\realname{Hanting Zhang}
\pset{8}
\mathclass{410}

\newcommand{\todo}{\textcolor{red}{\textbf{TODO} }}
\renewcommand{\a}{\alpha}
\renewcommand{\b}{\beta}
\newcommand{\Z}{\mathbb Z}
\renewcommand{\bf}{\mathbf}
\renewcommand{\implies}{\Rightarrow}
\newcommand{\coimplies}{\Leftarrow}
\newcommand{\rad}{\text{rad }}
\renewcommand{\em}{\varnothing}
\renewcommand{\mod}{\text{ mod }}

\begin{document}

Exercises 7, 11, 13, 14, 16, 30, 31 (expect (e)), pp. 256-260.

\begin{theorem}[7]
    Let \(R\) be a commutative ring with 1. Prove that the principal ideal generated by \(x\) in the polynomial ring \(R[x]\) is a prime ideal if and only if \(R\) is an integral domain. Prove that \((x)\) is a maximal ideal if and only if \(R\) is a field.
\end{theorem}

\begin{proof}
    \todo
\end{proof}

\begin{theorem}[11]
    Assume \(R\) is commutative. Prove that if \(P\) is a prime ideal of \(R\) and \(P\) contains no zero divisors then \(R\) is an integral domain.
\end{theorem}

\begin{proof}
    \todo
\end{proof}

\begin{theorem}[13]
    Let \(\varphi : R \to S\) be a homomorphism of commutative rings. 
    \begin{enumerate}
        \item Prove that if \(P\) is a prime ideal of \(S\) then either \(\varphi^{-1}(P) = R\) or \(\varphi^{-1}(P)\) is a prime ideal of \(R\). Apply this to the special case when \(R\) is a subring of \(S\) then \(P \cap R\) is either \(R\) or a prime ideal of \(R\).
        \item Prove that if \(M\) is a maximal ideal of \(S\) and \(\varphi\) is surjective then \(\varphi^{-1}(M)\) is a maximal ideal of \(R\). Give an example to show that this need not be the case if \(\varphi\) is not surjective.
    \end{enumerate}
\end{theorem}

\begin{proof}
    \todo
\end{proof}

\begin{theorem}[14]
    \todo
\end{theorem}

\begin{proof}
    \todo
\end{proof}

\begin{theorem}[16]
    Let \(x^2 - 16\) be an element of the polynomial ring \(E = \Z[x]\) and use the bar notation to denote passage to the quotient ring \(\Z[x]/(x^3 - 2x + 1)\). Let \(p(x) = 2x^7 - 7x^5 + 4x^3 - 9x + 1\) and let \(q(x) = (x - 1)^4\).
    \begin{enumerate}
        \item Express each of the following elements of \(\overline{E}\) in the form \(\overline{f(x)}\) for some polynomial \(f(x)\) of degree \(\le 2\): \(\overline{p(x)}, \overline{q(x)}, \overline{p(x) + q(x)}\), and \(\overline{p(x)q(x)}\).
        \item Prove that \(\overline{E}\) is not an integral domain.
        \item Prove that \(\overline{x}\) is a unit in \(\overline{E}\).
    \end{enumerate}
\end{theorem}

\begin{proof}
    \todo
\end{proof}

\begin{theorem}[30]
    Let \(I\) be an ideal of the commutative ring \(R\) and define 
    \[\rad I = \{r \in R \mid r^n \in I \text{ for some } n \in \Z^+\}\]
    called the \textit{nilradical} of \(I\). Prove that \(\rad I\) is an ideal containing \(I\) and that \((\rad I)/I\) is the nilradical of the quotient ring \(R / I\), i.e. \((\rad I / I) = \mathfrak{R}(R / I)\).
\end{theorem}

\begin{proof}
    \todo
\end{proof}

\begin{theorem}[31]
    An ideal \(I\) of the commutative ring \(R\) is called a \textit{radical ideal} if \(\rad I = I\). 
    \begin{enumerate}
        \item Prove that every prime ideal of \(R\) is a radical ideal.
        \item Let \(n > 1\) be an integer. Prove that 0 is a radical ideal in \(\Z / n \Z\) if and only if \(n\) is a product of distinct primes to the first power (i.e. \(n\) is square free). Deduce that \((n)\) is a radical ideal of \(\Z\) if and only if \(n\) is a product of distinct primes in \(\Z\). 
    \end{enumerate}
\end{theorem}

\begin{proof}
    \todo
\end{proof}

Exercises 1, 2, 5 pp. 267-269.

\begin{theorem}[1]
    An element \(e \in R\) is called an \textit{idempotent} if \(e^2 = e\). Assume that \(e\) is an idempotent in \(R\) and \(er = re\) for all \(r \in R\). Prove that \(Re\) and \(R(1 - e)\) and two-sided ideals of \(R\) and that \(R \cong Re \times R(1 - e)\). Show that \(e\) and \(1 - e\) are identities for the subrings \(Re\) and \(R(1 - e)\) respectively.
\end{theorem}

\begin{proof}
    \todo
\end{proof}

\begin{theorem}[2]
    Let \(R\) be a finite Boolean ring with identity \(1 \neq 0\). Prove that \(R \cong \Z / 2 \Z \times \Z / 2 \Z \times \dots \times \Z / 2 \Z\).
\end{theorem}

\begin{proof}
    \todo
\end{proof}

\begin{theorem}[5]
    Let \(n_1, n_2, \cdots, n_k\) be integers which are relatively prime in pairs: \((n_i, n_j = 1\) for all \(i \neq j\). 
    \begin{enumerate}
        \item Show that the Chinese Remainder Theorem implies that for any \(a_1, \cdots, a_k \in \Z\) there is a solution \(x \in \Z\) to the simultaneous congruences
        \[x \equiv a_1 \mod n_1, \hspace*{5mm} x \equiv a_2 \mod n_2, \hspace*{5mm} \cdots, \hspace*{5mm} x \equiv a_k \mod n_k\]
        and that the solution \(x\) is unique mod \(n = n_1n_2\cdots n_k\).
        \item Let \(n_i' = n / n_i\) and \(t_i\) be the inverse of \(n_i' \mod n_i\). Prove that the solution \(x\) in (a) is given by \[x = a_1 t_1 n_1' + a_2 t_2 n_2' + \cdots + a_k t_k n_k' \mod n.\]
        \item Solve the simultaneous system of congruences
        \[x \equiv 1 \mod 8, \hspace*{5mm} x \equiv 2 \mod 25, \hspace*{5mm} x \equiv 3 \mod 81\]
        and \[y \equiv 5 \mod 8, \hspace*{5mm} y \equiv 12 \mod 25, \hspace*{5mm} y \equiv 47 \mod 81.\]
    \end{enumerate}
\end{theorem}

\begin{proof}
    \todo
\end{proof}

More to be added...? \todo

\end{document}