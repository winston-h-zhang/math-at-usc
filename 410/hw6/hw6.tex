%%%%%%%%%%%%%%%%%%%%%%%%%%%%%%%%%%%%%%%%%%%%%%
%%                                          %%
%% USE THIS FILE TO SUBMIT YOUR SOLUTIONS   %%
%%                                          %%
%% You must have the usamts.tex file in     %%
%% the same directory as this file.         %%
%% You do NOT need to submit this file or   %%
%% usamts.tex with your solutions.  You     %%
%% only need to submit the output PDF file. %%
%%                                          %%
%% DO NOT ALTER THE FILE usamts.tex         %%
%%                                          %%
%% If you have any questions or problems    %%
%% using this file, or with LaTeX in        %%
%% general, please go to the LaTeX          %%
%% forum on the Art Of Problem Solving      %%
%% web site, and post your problem.         %%
%%                                          %%
%%%%%%%%%%%%%%%%%%%%%%%%%%%%%%%%%%%%%%%%%%%%%%

%%%%%%%%%%%%%%%%%%%%%%%%%%%%%%%%%%%%%%%%%%%%
%% DO NOT ALTER THE FOLLOWING LINES
\documentclass[12pt]{article}
\usepackage{amsmath,amssymb,amsthm,amsfonts,tabularx}
\usepackage[pdftex]{graphicx}
\graphicspath{ {./images/} }
\usepackage{fancyhdr}
\pagestyle{fancy}
\usepackage{setspace}
\usepackage{csquotes}
%%%%%%%%%%%%%%%%%%%%%%%%%%%%%%%%%%%%%%%%%%%
%%                                       %%
%% Students: DO NOT MODIFY THIS FILE!!!  %%
%%                                       %%
%%%%%%%%%%%%%%%%%%%%%%%%%%%%%%%%%%%%%%%%%%%

%% USAMTS style sheet
%% last modified: 23-Jul-2004

%% Student and Year/Round data
\newcommand{\realname}[1]{\newcommand{\printrealname}{#1}}
\newcommand{\pset}[1]{\newcommand{\printpset}{#1}}

%% Pagestyle setup
\setlength{\headheight}{0.75in}
\setlength{\oddsidemargin}{0in}
\setlength{\evensidemargin}{0in}
\setlength{\voffset}{-.5in}
\setlength{\headsep}{10pt}
\setlength{\textwidth}{6.5in}
\setlength{\headwidth}{6.5in}
\setlength{\textheight}{8in}
\lhead{Math 410}
\chead{\Large \textbf{Homework \printpset}}
\rhead{\printrealname}
\rfoot{Page \thepage}
\renewcommand{\headrulewidth}{0.5pt}
\renewcommand{\footrulewidth}{0.3pt}
\setlength{\textwidth}{6.5in}


\renewcommand{\baselinestretch}{1}
%% DO NOT ALTER THE ABOVE LINES
%%%%%%%%%%%%%%%%%%%%%%%%%%%%%%%%%%%%%%%%%%%%


%% If you would like to use Asymptote within this document (which is optional), 
%% you can find out how at the following URL:
%%
%%   http://www.artofproblemsolving.com/Wiki/index.php/Asymptote:_Advanced_Configuration
%%
%% As explained there, you will want to uncomment the line below.  But be
%% sure to check the website because there are several other steps that must 
%% be followed.
%% \usepackage{asymptote}

\newtheorem*{prop}{Proposition}
\newtheorem*{corollary}{Corollary}
\newtheorem*{lemma}{Lemma}
\theoremstyle{remark}
\newtheorem*{defn}{Definition}

\newtheoremstyle{named}{}{}{}{}{\bfseries}{.}{.5em}{\thmnote{Problem #3}}
\theoremstyle{named}
\newtheorem*{theorem}{Theorem}

%% Enter your real name here
%% Example: \realname{David Patrick}
\realname{Hanting Zhang}
\pset{6}
\mathclass{410}

\renewcommand{\bf}{\mathbf}
\renewcommand{\implies}{\Rightarrow}
\newcommand{\coimplies}{\Leftarrow}
\newcommand{\Aut}{\text{Aut}}
\newcommand{\normal}{\trianglelefteq}

\begin{document}
Exercises 16, 17, pp. 138.

\begin{theorem}[16]
    Prove that \((\mathbb Z / 24 \mathbb Z)^\times\) is an elementary abelian group of order 8.
\end{theorem}

\begin{proof}
    By the Chinese Remainder Theorem, we have 
    \[(\mathbb Z / 24 \mathbb Z)^\times \cong (\mathbb Z / 8 \mathbb Z)^\times \times (\mathbb Z / 3 \mathbb Z)^\times.\]
    We know that \((\mathbb Z / 3 \mathbb Z)^\times \cong \mathbb Z_2\). Furthermore, \((\mathbb Z / 8 \mathbb Z)^\times\) consists of the elements \(a\) such that \(\gcd(a, 8) = 1\). This give \(a = 1, 3, 5, 7\). We can compute their orders directly:
    \begin{align*}
        1^1 \equiv 3^2 \equiv 5^2 \equiv 7^2 \equiv 1 \mod 8 \\
        \implies |1| = 1 \hspace{2mm} \text{and} \hspace{2mm} |3| = |5| = |7| = 2.
    \end{align*}
    Thus \((\mathbb Z / 8 \mathbb Z)^\times\) is a group of order 4 with 3 elements of order 2. The only possible such group is \(\mathbb Z_2^2\), so
    \[(\mathbb Z / 24 \mathbb Z)^\times \cong \mathbb Z_2^2 \times \mathbb Z_2 \cong \mathbb Z_2^3.\] 

    Clearly \(\mathbb Z_2^3\) is an elementary group with \(p = 2\) that has order 8 and is abelian.
\end{proof}

\begin{theorem}[17]
    Let \(\langle G \rangle\) be a cyclic group of order \(n\). For \(n = 2, 3, 4, 5, 6,\) write out the elements of \(\Aut(G)\) explicitly.
\end{theorem}
\begin{proof}
    For each case, let \(x\) generate the group with \(|x| = n\). Notice that we only need to focus on the image of \(x\), as it determines the entire map. 
    \begin{enumerate}
        \item[]\(n = 2\): \(x\) can only map to itself. Hence any automorphism must be the identity: 
        \[1 \mapsto 1, \hspace{2mm} x \mapsto x.\] 
        
        \item[]\(n = 3\): We can map \(x\) to itself or \(x^2\). This gives two maps, which one can easily verify are also homomorphisms:
        \begin{align*}
            1 \mapsto 1, \hspace{2mm} x \mapsto x, \hspace{2mm} x^2 \mapsto x^2 \\
            1 \mapsto 1, \hspace{2mm} x \mapsto x^2, \hspace{2mm} x^2 \mapsto x.
        \end{align*} 
        
        \item[]\(n = 4\): We can map \(x\) to itself and \(x^3\), but not \(x^2\) (the map would not be bijective). This again gives two maps, which one can easily verify are also homomorphisms:
        \begin{align*}
            1 \mapsto 1, \hspace{2mm} x \mapsto x, \hspace{2mm} x^2 \mapsto x^2 \hspace{2mm} x^3 \mapsto x^3 \\
            1 \mapsto 1, \hspace{2mm} x \mapsto x^3, \hspace{2mm} x^2 \mapsto x^2 \hspace{2mm} x^3 \mapsto x.
        \end{align*} 
        
        \item[]\(n = 5\): We know that \((\mathbb Z_5)^\times \cong \mathbb Z_4\). Hence there are 4 maps, each corresponding \(x\) being mapped to an non-identity element:
        \begin{align*}
            1 \mapsto 1, \hspace{2mm} x \mapsto x, \hspace{2mm} x^2 \mapsto x^2 \hspace{2mm} x^3 \mapsto x^3 \hspace{2mm} x^4 \mapsto x^4 \\
            1 \mapsto 1, \hspace{2mm} x \mapsto x^2, \hspace{2mm} x^2 \mapsto x^4 \hspace{2mm} x^3 \mapsto x \hspace{2mm} x^4 \mapsto x^3 \\
            1 \mapsto 1, \hspace{2mm} x \mapsto x^3, \hspace{2mm} x^2 \mapsto x \hspace{2mm} x^3 \mapsto x^4 \hspace{2mm} x^4 \mapsto x^2 \\
            1 \mapsto 1, \hspace{2mm} x \mapsto x^4, \hspace{2mm} x^2 \mapsto x^3 \hspace{2mm} x^3 \mapsto x^2 \hspace{2mm} x^4 \mapsto x.
        \end{align*} 
        
        \item[]\(n = 6\): If \(x\) is mapped to any of \(x^2, x^3\), or \(x^4\), the generated map is not bijective. Hence there are only two possible maps, which we can see are isomorphisms:
        \begin{align*}
            1 \mapsto 1, \hspace{2mm} x \mapsto x, \hspace{2mm} x^2 \mapsto x^2 \hspace{2mm} x^3 \mapsto x^3 \hspace{2mm} x^4 \mapsto x^4 \hspace{2mm} x^5 \mapsto x^5 \\
            1 \mapsto 1, \hspace{2mm} x \mapsto x^5, \hspace{2mm} x^2 \mapsto x^4 \hspace{2mm} x^3 \mapsto x^3 \hspace{2mm} x^4 \mapsto x^2 \hspace{2mm} x^5 \mapsto x.
        \end{align*} 
        \item[] And thus we are done.
    \end{enumerate}
\end{proof}

Exercises 3, 5, 6, 8, 14 pp. 184-187.

\begin{theorem}[3]
    Continue from Example 1. Prove that every element of \(G - H\) has order 2. Prove that \(G\) is abelian if and only if \(h^2 = 1\) for all \(h \in H\). 
\end{theorem}

\begin{proof}
    We prove the statements separately:
    \begin{enumerate}
        \item[] \textit{Every element of \(G - H\) has order 2}. Let \(g \in G - H\). Then \(g\) must be of the form \(hk\) for some \(h \in H\) and \(k \in K\) and \textit{not} of the form \(g = h\). Thus we must have \(g = hx\). Then \(g^2 = hxhx\). The action implies that \(xhx^{-1} = xhx = h^{-1}\), therefore \(g^2 = hh^{-1} = 1\). Thus \(|g| = 2\).
    
        \item[] \textit{\(G\) is abelian \(\iff\) \(\forall h \in H, h^2 = 1\)}. 
    
        (\(\implies\)): If \(G\) is abelian, then for any \(h \in H\), \(h(hx) = (hx)h\). Then 
        \[hhx = hxh \implies hh = hxhx^{-1} = hxhx = 1 \implies h^2 = 1.\]
    
        (\(\coimplies\)): If \(h^2 = 1\) for all \(h \in H\), then every element of \(G\) has order 2. Then for any \(g_1, g_2 \in G\), 
        \[(g_1g_2)^2 = g_1^2 g_2^2 = 1 \implies g_1 g_2 g_1 g_2 = g_1 g_1 g_2 g_2 \implies g_2 g_1 = g_1 g_2.\]
        Thus \(G\) is abelian.
    \end{enumerate}
\end{proof}

\begin{theorem}[5]
    Let \(G = \text{Hol}(\mathbb Z_2 \times \mathbb Z_2)\).
    \begin{enumerate}
        \item[(a)] Prove that \(G = H \rtimes K\) where \(H = \mathbb Z_2 \times \mathbb Z_2\) and \(K \cong S_3\). Deduce that \(|G| = 24\).
        \item[(b)] Prove that \(G\) is isomorphic to \(S_4\).  
    \end{enumerate}
\end{theorem}

\begin{proof}
    We proceed by proving each part:
    \begin{enumerate}
        \item[(a)] Let \(K = \Aut(\mathbb Z_2 \times \mathbb Z_2)\). If \(\varphi : \mathbb Z_2 \to \mathbb Z_2\) is an isomorphism, then \(\varphi\) must fix \((0, 0)\) while permuting \(\{(0, 1), (1, 0), (1, 1)\}\). Thus the action of \(\varphi\) on the 3 non-identity elements can be associated with a element of \(S_3\). So we have a map
        \[\Phi : \Aut(\mathbb Z_2 \times \mathbb Z_2) \to S_3.\] 
        Now, the composition of two maps \(\varphi_2 \circ \varphi_1\) will permute the 3 non-identity elements by the composition of the permutations associated with \(\varphi_1\) and \(\varphi_2\), so we have \[\Phi(\varphi_2 \circ \varphi_1) = \Phi(\varphi_1) \Phi(\varphi_2),\]
        which shows that \(\Phi\) is a homomorphism. 
        
        Furthermore, clearly two automorphisms \(\varphi_1 \neq \varphi_2\) will permute the 3 non-identity elements differently, so \(\Phi\) is also injective. 
        To see that \(\Phi\) is surjective, we must show that any permutation of the 3 non-identity elements gives a automorphism of \(\mathbb Z_2 \times \mathbb Z_2\). This is not hard to check directly, but there it is tedious so we shall omit it. 
        Thus \(\Phi\) is bijective, and hence a isomorphism. 
        
        We conclude that \(K \cong S_3\). Since \(|S_3| = 6\) and \(|H \rtimes K| = |H||K|\), we may deduce \(|G| = 4 \times 6 = 24\).

        \item[(b)] Let \(G\) act on the 4 left cosets of \(K\), so that we may define the associated homomorphism \(G \to S_4\). Note that each left coset may be written as \(hk K\) for some \(h \in H\) and \(k \in K\). Since \(kK = K\), we may forget about the factors of \(k\) and realize that the 4 cosets are identified by the 4 elements of \(H\). 
        
        We want to show that \(G\) acts faithfully and conclude that \(G \to S_4\) is injective. For any \(g \in G\), if \(g \cdot hK = hK\) for all left cosets of \(K\), then 
        \[h^{-1}gh K = K \implies h^{-1}gh \in K \implies h^{-1}gh = 1,\]
        where the last implication follows from the fact that \(h^{-1}gh \in H\) and \(H \cap K = 1\). Thus \(g = 1\), proving that \(G\) acts faithfully and \(G \to S_4\) is injective. But \(|G| = |S_4| = 24\), so \(G \to S_4\) must also be bijective, and therefore an isomorphism.
    \end{enumerate}
\end{proof}

\begin{theorem}[6]
    Assume that \(K\) is a cyclic group, \(H\) is an arbitrary group and \(\varphi_1\) and \(\varphi_2\) are homomorphisms from \(K\) into \(\Aut(H)\) such that \(\varphi_1(K)\) and \(\varphi_2(K)\) are conjugate subgroups of \(\Aut(H)\). If \(K\) is infinite assume \(\varphi_1\) and \(\varphi_2\) are injective. Prove by constructing an explicit isomorphism that \(H \rtimes_{\varphi_1} K \cong H \rtimes_{\varphi_2} K\).
\end{theorem}

\begin{proof}
    Suppose that \(\sigma \varphi_1(K) \sigma^{-1} = \varphi_2(K)\). In particular, this can also be seen as the image of an automorphism on \(\varphi_2(K)\). Since \(K\) is cyclic, any automorphism has the form \(k \mapsto k^a\) for some \(a \in \mathbb Z\). Thus we have \(\sigma \varphi_1(k) \sigma^{-1} = \varphi_2(k)^a\) for all \(k \in K\). 

    We claim that the map from \(\psi : H \rtimes_{\varphi_1} K \to H \rtimes_{\varphi_2} K\) defined by \((h, k) \mapsto (\sigma(h), k^a)\) is a homomorphism. Indeed, let \((h_1, k_1), (h_2, k_2) \in H \rtimes_{\varphi_1} K\). Then we have,
    \begin{align*}
        \psi((h_1, k_1) \bullet_{\varphi_1} (h_2, k_2)) &= \psi((h_1\varphi_1(k_1)(h_2), k_1 k_2)) \\
        &= (\sigma h_1\varphi_1(k_1)(h_2) \sigma^{-1}, (k_1 k_2)^a) \\ 
        &= (\sigma h_1 \sigma^{-1} \sigma \varphi_1(k_1)(h_2) \sigma^{-1}, k_1^a k_2^a) \\
        &= (\sigma h_1 \sigma^{-1} \varphi_2(k_1)(h_2)^a, k_1^a k_2^a) \\ 
        &= (\sigma h_1 \sigma^{-1} \varphi_2(k_1)(h_2^a), k_1^a k_2^a) \\ 
        &= (\sigma h_1 \sigma^{-1}, k_1^a) \bullet_{\varphi_2} (h_2^a, k_2^a) \\ 
        &= (\sigma h_1 \sigma^{-1}, k_1^a) \bullet_{\varphi_2} (\sigma h_2 \sigma^{-1}, k_2^a) \\ 
        &= \psi(h_1, k_1) \bullet_{\varphi_2} \psi(h_2, k_2).
    \end{align*}
    Thus \(\psi\) is a homomorphism. 

    Furthermore, we can consider the map \(\phi : H \rtimes_{\varphi_2} K \to H \rtimes_{\varphi_1} K\) in the opposite direction given by \(\phi((h, k)) = (\sigma^{-1}h\sigma, k^{a^{-1}})\). Since \(\sigma^{-1}\varphi_2(K)\sigma = \varphi_1(K)\) and this forms the inverse automorphism which maps \(k \mapsto k^{a^{-1}}\), we similarly deduce that \(\phi\) is a homomorphism as well. Now note that 
    \begin{align*}
        \psi \circ \phi ((h, k)) = \psi((\sigma^{-1}h\sigma, k^{a^{-1}})) = (\sigma\sigma^{-1}h\sigma\sigma^{-1}, (k^{a^{-1}})^a) = (h, k); \\
        \phi \circ \psi ((h, k)) = \phi((\sigma h\sigma^{-1}, k^{a})) = (\sigma^{-1}\sigma h\sigma^{-1}\sigma, (k^{a})^{a^{-1}}) = (h, k).
    \end{align*}
    So \(\psi\) and \(\phi\) are two-sided inverses of each other. Thus \(\psi\) is an isomorphism and 
    \[H \rtimes_{\varphi_1} K \cong H \rtimes_{\varphi_2} K.\]
\end{proof}

\begin{theorem}[8]
    Construct an non-abelian group of order 75. Classify all groups of order 75. 
\end{theorem}

\begin{theorem}[14]
    Classify groups of order 60.
\end{theorem}

\begin{proof}
    Let \(G\) be a group of order 60, let \(P\) be a Sylow 5-subgroup of \(G\) and let \(Q\) be a Sylow 3-subgroup of \(G\).
    \begin{enumerate}
        \item[(a)] \textbf{TODO}.
        \item[(b)] \textbf{TODO}.
        \item[(c)] \textbf{TODO}.
    \end{enumerate}
\end{proof}

Exercises 2, 5 pp. 165-167.

\begin{theorem}[5]
    Let \(G\) be a finite abelian group of type \((n_1, n_2, \dots, n_t)\). Prove that \(G\) contains an element of order \(m\) if and only if \(m \mid n_1\). Deduce that \(G\) is of exponent \(n_1\).
\end{theorem}

Exercise 15 p. 174.

\begin{theorem}[15]
    If \(A\) and \(B\) are normal subgroups of \(G\) such that \(G / A\) and \(G / B\) are both abelian, prove that \(G / A \cap B\) is abelian.
\end{theorem}

\begin{proof}
    Let \(G' = [G, G]\) be the commutator subgroup of \(G\). By Proposition 7 (4) from the textbook, since both \(A, B \normal G\) and \(G / A\) and \(G / B\) are abelian, we have \(G' \le A, B\). Thus \(G' \le A \cap B\), and Proposition 7 (4) once again tells us that \(G / (A \cap B)\) must be abelian.
\end{proof}

\end{document}