\documentclass[12pt]{article}
\usepackage{amsmath,amssymb,amsthm,amsfonts,tabularx}
\usepackage[pdftex]{graphicx}
\usepackage[dvipsnames]{xcolor}
\usepackage{fancyhdr}
\usepackage{parskip}
\usepackage[shortlabels]{enumitem}
\pagestyle{fancy}
\usepackage{setspace}
\usepackage{pgfplots}
\pgfplotsset{compat=1.18}
\newcommand{\realname}[1]{\newcommand{\printrealname}{#1}}
\newcommand{\pset}[1]{\newcommand{\printpset}{#1}}
\newcommand{\mathclass}[1]{\newcommand{\printmathclass}{#1}}

%% Pagestyle setup
\setlength{\headheight}{0.75in}
\setlength{\oddsidemargin}{0in}
\setlength{\evensidemargin}{0in}
\setlength{\voffset}{-.5in}
\setlength{\headsep}{10pt}
\setlength{\textwidth}{6.5in}
\setlength{\headwidth}{6.5in}
\setlength{\textheight}{8in}
\lhead{Math \printmathclass}
\chead{\Large \textbf{\printpset}}
\rhead{\printrealname}
\rfoot{Page \thepage}
\renewcommand{\headrulewidth}{0.5pt}
\renewcommand{\footrulewidth}{0.3pt}
\setlength{\textwidth}{6.5in}
\renewcommand{\baselinestretch}{1}
\setenumerate[0]{label=(\alph*)}
\newcommand{\todo}{\textcolor{red}{\textbf{TODO }}}

\newtheorem*{prop}{Proposition}
\newtheorem*{corollary}{Corollary}
\newtheorem{lemma}{Lemma}
\theoremstyle{remark}
\newtheorem*{defn}{Definition}
\newtheorem*{remark}{Remark}

\newtheoremstyle{named}{}{}{}{}{\bfseries}{.}{.5em}{\thmnote{Problem #3}}
\theoremstyle{named}
\newtheorem*{theorem}{Theorem}
\allowdisplaybreaks

%% DO NOT ALTER THE ABOVE LINES
%%%%%%%%%%%%%%%%%%%%%%%%%%%%%%%%%%%%%%%%%%%%


%% If you would like to use Asymptote within this document (which is optional), 
%% you can find out how at the following URL:
%%
%%   http://www.artofproblemsolving.com/Wiki/index.php/Asymptote:_Advanced_Configuration
%%
%% As explained there, you will want to uncomment the line below.  But be
%% sure to check the website because there are several other steps that must 
%% be followed.
%% \usepackage{asymptote}

%% Enter your real name here
%% Example: \realname{David Patrick}
\realname{Hanting Zhang}
\pset{W13P1}
\mathclass{425B}

\renewcommand{\a}{\alpha}
\renewcommand{\b}{\beta}
\renewcommand{\d}{\delta}
\newcommand{\e}{\varepsilon}
\newcommand{\Z}{\mathbb Z}
\newcommand{\N}{\mathbb N}
\newcommand{\Q}{\mathbb Q}
\newcommand{\R}{\mathbb R}
\newcommand{\C}{\mathbb C}
\renewcommand{\bf}{\mathbf}
\newcommand{\id}[1]{\text{id}_{#1}}
\renewcommand{\implies}{\Rightarrow}
\newcommand{\coimplies}{\Leftarrow}
\renewcommand{\em}{\varnothing}
\renewcommand{\Im}{\text{Im}}
\newcommand{\abs}[1]{|#1|}
\newcommand{\bigabs}[1]{\left|#1\right|}
\newcommand{\Rloc}{\mathcal R_{\text{loc}}}

%% should be reset

\begin{document}

\begin{theorem}[1.6]
    Let \(X\) be a normed vector space. Let \(U\) be an open subset of \(X\), and let \(f : U \to \R\) be a function. Recall that we say that \(f\) has a \textit{local maximum} at \(a \in U\) if there exists a neighborhood \(V\) of \(a\) such that \(f(x) \leq f(a)\) for all \(x \in V\). Prove that if \(a\) is a local maximum of \(f\) at which \(f\) is differentiable, then \(f'(a) = 0\). 
\end{theorem}

\begin{proof}
    Choose a set of basis vectors \(\{e_i\}_{i \in I}\) appropriately such that \(x + e_i \in V\) for all \(i \in I\). Then, we know that for each \(i \in I\),
    \begin{align*}
        \lim_{h \to 0^+} \frac{\|f(x + h e_i) - f(x) - f'(x)h e_i\|}{\|h e_i\|} = \lim_{h \to 0^-} \frac{\|f(x + h e_i) - f(x) - f'(x)h e_i\|}{\|h e_i\|} = 0.
    \end{align*}
    Because \(f(x) \leq f(x + h e_i)\), the first limit imposes \(D_i f(x) \geq 0\), bit the second limit imposes \(D_i f(x) \leq 0\). Thus we must have \(D_i f(x) = 0\) for all \(i \in I\). Therefore, \(f'(x) = 0\), as desired.
\end{proof}

\begin{theorem}[2.1]
    Differentiation past the integral.
    \begin{enumerate}
        \item Let \(Y\) be a metric space; suppose \(f : [a, b] \times Y \to \R\) is continuous. Show that the function \(F : Y \to \R\) defined by \(F(y) = \int_{a}^{b} f(x, y) dx\) is continuous.
        \item Assume \(f : [a, b] \to \R\) is continuous; assume \(\partial_2 f(x, y)\) also exists and is continuous on \([a, b] \times (c, d)\). Define \(F : (c, d) \to \R\) by \(F(y) = \int_{a}^{b} f(x, y) dx\). Prove that \(F \in C^1((c, d);\R)\), with \(F'(y) = \int_{a}^{b} \partial_2 f(x, y)\). 
    \end{enumerate}
\end{theorem}

\begin{proof}
    We proceed with each part separately.
    \begin{enumerate}
        \item Let \(\e > 0\). Since \(f\) is continuous, there exists an \(\d\) such that for all \(\|z - y\| < \d\), we have \(|f(z) - f(y)| < \e / (b - a)\). Then we have 
        \begin{align*}
            |F(z) - F(y)| &= \bigabs{\int_{a}^{b} [f(x, z) - f(x, y)] dx} \\
            &\leq \int_{a}^{b} \abs{f(x, z) - f(x, y)} dx \\
            &< \int_{a}^{b} \frac{\e}{b - a} dx = \e,
        \end{align*}
        as desired.
        \item We show that \(F'(y)\) exists and is equal to \(\int_{a}^{b} \partial_2 f(x, y) dx\). Indeed,
        \begin{align*}
            \lim_{h \to 0} \frac{F(y + h) - F(y)}{h} &= \lim_{h \to 0} \int_{a}^{b} \frac{f(x, y + h) - f(x, y)}{h} dx \\
            &= \int_{a}^{b} \lim_{h \to 0} \frac{f(x, y + h) - f(x, y)}{h} dx \\
            &= \int_{a}^{b} \partial_2 f(x, y) dx,
        \end{align*}
        where swapping the order of the limit and integration relies on the fact that continuous functions on compact intervals are also uniformly continuous. By part (a), since \(\partial_2 f(x, y)\) is continuous, so is \(\int_{a}^{b} \partial_2 f(x, y) dx\), so \(F(y) \in C^1\), as desired.
    \end{enumerate}
\end{proof}

\begin{theorem}[2.2]
    Define a function \(f : \R^2 \to \R\) bia the following rule when \(y \geq 0\):
    \begin{align*}
        f(x, y) = 
        \begin{cases}
            x, & 0 \leq x \leq \sqrt{y} \\
            -x + 2\sqrt{y}, & \sqrt{y} \leq x \leq 2\sqrt{y} \\
            0, & x \geq 2\sqrt{y} \text{ or } x \leq 0. 
        \end{cases}
    \end{align*}
    Extend \(f(x, y)\) to all of \(\R^2\) by letting it be odd in the second argument: \(f(x, y) = -f(x, -y)\) for \(y < 0\).
    \begin{enumerate}
        \item Show that \(f\) is continuous everywhere, and \((\partial_2 f)(x, 0) = 0\) for all \(x\).
        \item Define \(F(y) = \int_{-1}^{1} f(x, y) dx\). Prove that \(F(y) = y\) for \(\abs{y} < \frac 1 4\). Conclude that \(F'(0) \neq \int_{-1}^{1}\partial_2 f(x, 0) dx\).
        \item Compare part (b) of the present exercise with part (b) of Exercise 2.1. Why are these two not in contradiction with one another?
    \end{enumerate}
\end{theorem}

\begin{proof}
    We proceed with each part separately.
    \begin{enumerate}
        \item Within each region, \(f\) is continuous, so we just need to show that it is continuous at the boundaries \(x = \sqrt y\) and \(x = 2 \sqrt y\). Indeed, we have if \(x = \sqrt y\), then \(x = -x + 2x = -x + 2 \sqrt y\); if \(x = 2 \sqrt y\), then \(-x + 2 \sqrt y = - 2 \sqrt y + 2 \sqrt y = 0\). Thus \(f\) has no discontinuities at the boundaries, so it is continuous everywhere.
        
        Furthermore, consider \(\partial_2 f(x, 0) = \lim_{h \to 0} (f(x, h) - f(x, 0)) / h = \lim_{h \to 0} f(x, h) / h\). For any \(x\), we may choose \(h\) small enough that \(2 \sqrt h \leq x\), so the limit \(\partial_2 f(x, 0)\) will always be 0.

        \item This is fairly painful, but if we draw out the graph its easier to visualize. We have for \(y > 0\):
        \begin{align*}
            F(y) &= \int_{-1}^{1} f(x, y) dx \\
            &= \int_{0}^{\sqrt{y}} x dx + \int_{\sqrt y}^{2 \sqrt y} (-x + 2\sqrt{y}) dx \\
            &= \left(\frac{x^2}{2}\right)_0^{\sqrt y} + \left(\frac{-x^2}{2}\right)_{\sqrt y}^{2 \sqrt y} + 2 y \\
            &= \frac{y}{2} + \frac{-3y}{2} + 2y = y.
        \end{align*}
        Since \(f\) is odd, for \(y < 0\), we have \(F(y) = F(-|y|) = -F(|y|) = -|y| = y\) as well, as desired.
        \item The reason why this doesn't contradict part (b) of Exercise 2.1 is because \(\partial_2 f(x, y)\) is not continuous.
    \end{enumerate}
\end{proof}

\end{document}