\documentclass[12pt]{article}
\usepackage{amsmath,amssymb,amsthm,amsfonts,tabularx}
\usepackage[pdftex]{graphicx}
\usepackage[dvipsnames]{xcolor}
\usepackage{fancyhdr}
\usepackage{parskip}
\usepackage[shortlabels]{enumitem}
\pagestyle{fancy}
\usepackage{setspace}
\newcommand{\realname}[1]{\newcommand{\printrealname}{#1}}
\newcommand{\pset}[1]{\newcommand{\printpset}{#1}}
\newcommand{\mathclass}[1]{\newcommand{\printmathclass}{#1}}

%% Pagestyle setup
\setlength{\headheight}{0.75in}
\setlength{\oddsidemargin}{0in}
\setlength{\evensidemargin}{0in}
\setlength{\voffset}{-.5in}
\setlength{\headsep}{10pt}
\setlength{\textwidth}{6.5in}
\setlength{\headwidth}{6.5in}
\setlength{\textheight}{8in}
\lhead{Math \printmathclass}
\chead{\Large \textbf{\printpset}}
\rhead{\printrealname}
\rfoot{Page \thepage}
\renewcommand{\headrulewidth}{0.5pt}
\renewcommand{\footrulewidth}{0.3pt}
\setlength{\textwidth}{6.5in}
\renewcommand{\baselinestretch}{1}
\setenumerate[0]{label=(\alph*)}
\newcommand{\todo}{\textcolor{red}{\textbf{TODO }}}

\newtheorem*{prop}{Proposition}
\newtheorem*{corollary}{Corollary}
\newtheorem{lemma}{Lemma}
\theoremstyle{remark}
\newtheorem*{defn}{Definition}
\newtheorem*{remark}{Remark}

\newtheoremstyle{named}{}{}{}{}{\bfseries}{.}{.5em}{\thmnote{Problem #3}}
\theoremstyle{named}
\newtheorem*{theorem}{Theorem}
\allowdisplaybreaks

%% DO NOT ALTER THE ABOVE LINES
%%%%%%%%%%%%%%%%%%%%%%%%%%%%%%%%%%%%%%%%%%%%


%% If you would like to use Asymptote within this document (which is optional), 
%% you can find out how at the following URL:
%%
%%   http://www.artofproblemsolving.com/Wiki/index.php/Asymptote:_Advanced_Configuration
%%
%% As explained there, you will want to uncomment the line below.  But be
%% sure to check the website because there are several other steps that must 
%% be followed.
%% \usepackage{asymptote}

%% Enter your real name here
%% Example: \realname{David Patrick}
\realname{Hanting Zhang}
\pset{W3P2}
\mathclass{425B}

\renewcommand{\a}{\alpha}
\renewcommand{\b}{\beta}
\renewcommand{\d}{\delta}
\newcommand{\e}{\varepsilon}
\newcommand{\Z}{\mathbb Z}
\newcommand{\N}{\mathbb N}
\newcommand{\Q}{\mathbb Q}
\newcommand{\R}{\mathbb R}
\newcommand{\C}{\mathbb C}
\renewcommand{\bf}{\mathbf}
\newcommand{\id}[1]{\text{id}_{#1}}
\renewcommand{\implies}{\Rightarrow}
\newcommand{\coimplies}{\Leftarrow}
\renewcommand{\em}{\varnothing}
\renewcommand{\Im}{\text{Im}}
\newcommand{\abs}[1]{|#1|}
\newcommand{\bigabs}[1]{\left|#1\right|}
\newcommand{\Rloc}{\mathcal R_{\text{loc}}}

\begin{document}

\begin{theorem}[1.5]
    Assume \(f \in \Rloc(\R)\) and \(\int_{-\infty}^{\infty} |f(x)| dx < + \infty\). Then given \(\e > 0\) there exists \(R > 0\) such that 
    \[\int_{-\infty}^{\infty} |f(x) - f(x) 1_{[-R, R]}(x)| dx < \e.\]
\end{theorem}

\begin{proof}
    Note that it suffices to show that there is some \(R > 0\) such that
    \[\int_{R}^{\infty} |f(x)| dx < \e,\]
    since then we can replicate the argument on both sides with \(\e / 2\) and combine them to achieve the full claim.

    Assume for the sake of contradiction that there is some \(\e > 0\) such that for all \(R > 0\), 
    \[\int_{R}^{\infty} |f(x)| dx \geq \e.\]
    Now let \(R_0 = 0\). Since \(\int_{R_0}^{\infty} |f(x)| dx \geq \e\), there is some \(R_1\) such that \(\int_{R_0}^{R_1} |f(x)| dx = \e\). But then again \(\int_{R_1}^{\infty} |f(x)| dx \geq \e\), so there exists some \(R_2\) such that \(\int_{R_1}^{R_2} |f(x)| dx = \e\). Continuing with this, we may construct a sequence \(R_0 < R_1 < \cdots < R_n < \cdots\) such that 
    \[\int_{R_n}^{R_{n + 1}} |f(x)| dx = \e.\]
    Thus we have
    \begin{align*}
        \int_{0}^{\infty} |f(x)| dx = \sum_{n = 0}^\infty \int_{R_n}^{R_{n + 1}} |f(x)| dx  = \sum_{n = 0}^\infty \e = \infty.
    \end{align*}
    Contradiction! Thus there must be some \(R > 0\) such that \(\int_{R}^{\infty} |f(x)| dx < \e\), as desired.
\end{proof}

\begin{theorem}[1.6]
    This Exercise outlines a proof of Theorem 1.6. For all parts of the problem, assume that \(f \in \Rloc(\R)\).
    \begin{enumerate}
        \item Assume that \(\int_{-\infty}^{\infty}|f(x)|^2 dx = 0\). Prove that \(\int_{-\infty}^{\infty} f(x) \overline{g(x)} dx = 0\) whenever \(g \in \Rloc(\R)\) and \(\int_{-\infty}^{\infty}|g(x)|^2 dx < + \infty\).
        \item Assume that \(\int_{-\infty}^{\infty}|f(x)|^p dx = 0\) for some \(p > 0\). Prove that \(\int_{-\infty}^{\infty}|f(x)|^q dx = 0\) for all \(q > p\). 
        \item Assume that \(\int_{-\infty}^{\infty}|f(x)|^p dx = 0\) for some \(p > 0\). Prove that \(\int_{-\infty}^{\infty}|f(x)|^q dx = 0\) whenever \(q = 2^{-n}p\) for some \(n \in \N\).
        \item Combine parts (b) and (c) of this Exercise to prove Theorem 1.6(a).
        \item Prove Theorem 1.6(b) by combining Theorem 1.6(a) with part (a) of this Exercise.
    \end{enumerate}
\end{theorem}

\begin{proof}
    We proceed with each step
    \begin{enumerate}
        \item \todo
        \item We have \(q = p + (q - p)\), so we can break up the integral into \(\int_{-\infty}^{\infty} |f(x)|^p |f(x)|^{q - p} dx\). In particular, on every bounded interval, we know that \(|f(x)|^{q - p}\) must be bounded by some \(L\), since \(f(x) \in \Rloc(\R)\). Thus for all intervals \([a, b]\),
        \[\int_{a}^{b} |f(x)|^q dx \leq \int_{a}^{b} |f(x)|^p L dx = L \int_{a}^{b} |f(x)|^p = 0.\]
        This implies \(\int_{-\infty}^{\infty} |f(x)|^q dx = 0\), as desired.
        \item Consider \(|f(x)|^{p/2}\) applied in part (a). Then we have \(\int_{-\infty}^{\infty}|f(x)|^{p/2}\overline{g(x)} dx = 0\). The problem is how to choose \(g(x)\). We would like to choose \(g = 1\), but then we would have \(\int_{-\infty}^{\infty} g(x) dx = \infty\). So instead we choose \(g(x) = 1_{[-R, R]}\) for some \(R > 0\). Thus we have 
        \[\int_{-\infty}^{\infty}|f(x)|^{p/2} dx = \lim_{R \to \infty} \int_{-\infty}^{\infty}|f(x)|^{p/2} \overline{1_{[-R, R]}} dx = \lim_{R \to \infty} 0 = 0.\]
        Inducting on this, we have 
        \[\int_{-\infty}^{\infty}|f(x)|^{p/2^n} dx = 0\]
        for all \(n \in \N\).

        \item Putting parts (b) and (c) together, let \(q > 0\). Choose \(n\) such that \(2^n > p/q\), so that \(q > 2^{-n} p\). Thus from part (c) \(\int_{-\infty}^{\infty}|f(x)|^{p} dx = 0\) implies \(\int_{-\infty}^{\infty}|f(x)|^{p/2^n} dx = 0\), which from part (b) implies \(\int_{-\infty}^{\infty}|f(x)|^{q} dx = 0\), as desired.
        \item \todo
    \end{enumerate}
\end{proof}

\begin{theorem}[1.7]
    Let \(f\) be a nonnegative, continuous function defined on \(\R\). If \(\int_{-\infty}^{\infty} f(x) dx = 0\), then \(f(x) = 0\) for all \(x \in \R\).
\end{theorem}

\begin{proof}
    Assume for the sake of contradiction that \(f \neq 0\), i.e. there is some \(x_0\) such that \(f(x_0) \neq 0\). Set \(\e = f(x_0)/2 > 0\). Since \(f\) is continuous, there exists some \(\d > 0\) such that \(f([x_0 - \d, x_0 + \d]) \subseteq [f(x_0) - \e, f(x_0) + \e]\). In particular, we conclude that \(f(x) \geq f(x_0) - \e = x_0 / 2\) on the interval \([x_0 - \d, x_0 + \d]\). Thus
    \[\int_{-\infty}^{\infty} f(x) dx \geq \int_{x_0 - \d}^{x_0 + \d} (f(x_0) - \e) dx = 2\d (f(x_0) - \e) > 0,\]
    contradiction! Thus we conclude that \(f = 0\).
\end{proof}

\end{document}