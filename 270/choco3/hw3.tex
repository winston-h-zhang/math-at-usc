\documentclass[10pt]{article}
\usepackage{fullpage}
\usepackage[noend]{algorithmic}
\usepackage{algorithm}

\usepackage{amsfonts}
\usepackage{amsmath,amsthm,amssymb}
\usepackage{nicefrac}
\usepackage{latexsym}
\usepackage{epic}
\usepackage{epsfig}
\usepackage{amscd}
\usepackage{url}
\usepackage{verbatim}
\usepackage{setspace}
\usepackage{hyperref}
\usepackage{enumerate}
\usepackage{times}

% \usepackage{CS270}

\newtheorem{lemma}{Lemma}
\newtheorem{fact}[lemma]{Fact}
\newtheorem{theorem}[lemma]{Theorem}
\newtheorem{corollary}[lemma]{Corollary}
\newtheorem{proposition}[lemma]{Proposition}
\newtheorem{claim}[lemma]{Claim}

\newtheorem{definition}{Definition}
\newtheorem{example}[definition]{Example}
\newtheorem{remark}[definition]{Remark}

\providecommand{\half}{\ensuremath{\nicefrac{1}{2}}\xspace}
\providecommand{\third}{\ensuremath{\nicefrac{1}{3}}\xspace}
\providecommand{\quarter}{\ensuremath{\nicefrac{1}{4}}\xspace}

\providecommand{\SetCard}[1]{\ensuremath{| #1 |}\xspace}
\providecommand{\SET}[1]{\ensuremath{\{ #1 \}}\xspace}
\providecommand{\Abs}[1]{\ensuremath{| #1 |}\xspace}
\providecommand{\Set}[2]{\ensuremath{\SET{#1 \mid #2}}\xspace}

\providecommand{\Kth}[1]{\ensuremath{{#1}^{\rm th}}}
\providecommand{\Ceiling}[1]{\ensuremath{\lceil {#1} \rceil}\xspace}
\providecommand{\Floor}[1]{\ensuremath{\lfloor {#1} \rfloor}\xspace}

\providecommand{\PROB}{\ensuremath{{\rm Prob}}\xspace}
\providecommand{\Prob}[2][]{\ensuremath{%
\ifthenelse{\equal{#1}{}}{\PROB[#2]}{\PROB_{#1}[#2]}}\xspace}
\providecommand{\ProbC}[3][]{\Prob[#1]{#2\;|\;#3}}
\providecommand{\Expect}[2][]{\ensuremath{%
\ifthenelse{\equal{#1}{}}{\mathbb{E}}{\mathbb{E}_{#1}}%
\left[#2\right]}\xspace}
\providecommand{\ExpectC}[3][]{\Expect[#1]{#2\;|\;#3}}

\title{\bf Chocolate 3}
\author{Steve Vott \& Winston (Hanting) Zhang}

\begin{document}
\maketitle

\subsection*{Problem 3. [0]}
\textbf{Chocolate Problem: 2 chocolate bars}

Reminder: If you solve a chocolate problem (which you can do in groups of size up to 3), please e-mail David with the solution --- do not submit it on Gradescope. Also, feel free to list preferences or dietary restrictions for/against particular types of chocolate.

Exercise 4.31 in the textbook. Notice that Part (b) is really the interesting thing here --- Part (a) is basically a slightly harder regular problem.

\begin{proposition}
  Prove that for every pair of nodes \(u, v \in V\), the length of the shortest \(u-v\) path in \(H\) is at most 3 times the length of the shortest \(u-v\) path in \(G\). 
\end{proposition}

\begin{proof}
  Denote the weight of a edge \(e = (u, v)\) by \(w(e) = w(u, v)\). For some subgraph \(K \subseteq G\), denote the length of the shortest \(u - v\) path in \(K\) by \(d_K(u, v)\). Let \(H\) be the output of our algorithm. Suppose for the sake of contradiction that there exists some \(u, v \in V\) such that \(d_H(u, v) > 3 d_G(u, v)\). Let the \(u - v\) path in \(G\) be made up of the sequence of vertices \(u = v_1, v_2, \dots, v_k = v\). Since \(H\) is connected, for each edge of the path \((v_i, v_{i+1})\), for \(1 \leq i < n\), there is a path in \(H\) connecting \(v_i\) and \(v_{i + 1}\). 

  We claim that \(d_H(v_i, v_{i + 1}) \leq 3 d_G(v_i, v_{i + 1}) = 3 w(v_i, v_{i + 1})\). The equality holds because the edge \((v_i, v_{i + 1})\) itself is the shortest path between \(v_i\) and \(v_{i+1}\). If it weren't, then we could improve \(d_G(u, v)\) by taking the shorter path between \(v_i\) and \(v_{i + 1}\). As a corollary, this implies that \(d_G(u, v) = \sum_{i = 1}^{n - 1} w(v_i, v_{i + 1}) = \sum_{i = 1}^{n - 1} d_G(v_i, v_{i + 1})\).
  
  We split into two cases. 
  \begin{enumerate}
    \item If \((v_i, v_{i + 1}) \in H\), then clearly \(d_H(v_i, v_{i + 1}) \leq 3 d_G(v_i, v_{i + 1})\).
    \item If \((v_i, v_{i + 1}) \notin H\), then consider the step of our algorithm when we are have the (incomplete) graph \(H'\) and are considering adding the edge \((v_i, v_{i + 1})\). We deduce that \(v_i\) and \(v_{i + 1}\) must be connected at this point, else we would add \((v_i, v_{i+1})\) into \(H'\). Because \((v_i, v_{i + 1})\) was not added, we deduce that \(d_{H'}(v_i, v_{i + 1}) \le  3 w(v_i, v_{i + 1})\). Since \(H' \subseteq H\), we have \(d_{H}(v_i, v_{i + 1}) \le d_{H'}(v_i, v_{i + 1})\) (intuitively, we can always do better when we have more edges). Thus \(d_{H}(v_i, v_{i + 1}) \le d_{H'}(v_i, v_{i + 1}) \le d_{G}(v_i, v_{i + 1})\), as desired.
  \end{enumerate}

  Thus we have,
  \[d_H(u, v) \le \sum_{i = 1}^{n - 1}d_H(v_i, v_{i + 1}) \le 3 \sum_{i = 1}^{n - 1} d_G(v_i, v_{i + 1}) = 3 d_G(u, v),\]
  and the proof is complete.
\end{proof}

\begin{proposition}
  Despite its ability to approximately preserve shortest-path distances, the subgraph \(H\) produced by the algorithm cannot be too dense. Let \(f(n)\) denote the maximum number of edges that can possibly be produced as the output of this algorithm, over all \(n\)-node input graphs with edge lengths. Prove that
  \[\lim_{n \to \infty} \frac{f(n)}{n^2} = 0.\]
\end{proposition}

\begin{proof}
  We have no idea. 
\end{proof}

\end{document}