%%%%%%%%%%%%%%%%%%%%%%%%%%%%%%%%%%%%%%%%%%%%%%
%%                                          %%
%% USE THIS FILE TO SUBMIT YOUR SOLUTIONS   %%
%%                                          %%
%% You must have the usamts.tex file in     %%
%% the same directory as this file.         %%
%% You do NOT need to submit this file or   %%
%% usamts.tex with your solutions.  You     %%
%% only need to submit the output PDF file. %%
%%                                          %%
%% DO NOT ALTER THE FILE usamts.tex         %%
%%                                          %%
%% If you have any questions or problems    %%
%% using this file, or with LaTeX in        %%
%% general, please go to the LaTeX          %%
%% forum on the Art Of Problem Solving      %%
%% web site, and post your problem.         %%
%%                                          %%
%%%%%%%%%%%%%%%%%%%%%%%%%%%%%%%%%%%%%%%%%%%%%%

%%%%%%%%%%%%%%%%%%%%%%%%%%%%%%%%%%%%%%%%%%%%
%% DO NOT ALTER THE FOLLOWING LINES
\documentclass[12pt]{article}
\usepackage{amsmath,amssymb,amsthm,amsfonts,tabularx}
\usepackage[pdftex]{graphicx}
\graphicspath{ {./images/} }
\usepackage{fancyhdr}
\pagestyle{fancy}
\usepackage{setspace}
\usepackage{csquotes}
%%%%%%%%%%%%%%%%%%%%%%%%%%%%%%%%%%%%%%%%%%%
%%                                       %%
%% Students: DO NOT MODIFY THIS FILE!!!  %%
%%                                       %%
%%%%%%%%%%%%%%%%%%%%%%%%%%%%%%%%%%%%%%%%%%%

%% USAMTS style sheet
%% last modified: 23-Jul-2004

%% Student and Year/Round data
\newcommand{\realname}[1]{\newcommand{\printrealname}{#1}}
\newcommand{\pset}[1]{\newcommand{\printpset}{#1}}

%% Pagestyle setup
\setlength{\headheight}{0.75in}
\setlength{\oddsidemargin}{0in}
\setlength{\evensidemargin}{0in}
\setlength{\voffset}{-.5in}
\setlength{\headsep}{10pt}
\setlength{\textwidth}{6.5in}
\setlength{\headwidth}{6.5in}
\setlength{\textheight}{8in}
\lhead{Math 410}
\chead{\Large \textbf{Homework \printpset}}
\rhead{\printrealname}
\rfoot{Page \thepage}
\renewcommand{\headrulewidth}{0.5pt}
\renewcommand{\footrulewidth}{0.3pt}
\setlength{\textwidth}{6.5in}


\renewcommand{\baselinestretch}{1}
%% DO NOT ALTER THE ABOVE LINES
%%%%%%%%%%%%%%%%%%%%%%%%%%%%%%%%%%%%%%%%%%%%


%% If you would like to use Asymptote within this document (which is optional), 
%% you can find out how at the following URL:
%%
%%   http://www.artofproblemsolving.com/Wiki/index.php/Asymptote:_Advanced_Configuration
%%
%% As explained there, you will want to uncomment the line below.  But be
%% sure to check the website because there are several other steps that must 
%% be followed.
%% \usepackage{asymptote}

\newtheorem*{prop}{Proposition}
\newtheorem*{corollary}{Corollary}
\newtheorem*{lemma}{Lemma}
\theoremstyle{remark}
\newtheorem*{defn}{Definition}

\newtheoremstyle{named}{}{}{}{}{\bfseries}{.}{.5em}{\thmnote{Problem #3}}
\theoremstyle{named}
\newtheorem*{theorem}{Theorem}

%% Enter your real name here
%% Example: \realname{David Patrick}
\realname{Hanting Zhang}
\pset{2}
\mathclass{425A}

\renewcommand{\bf}{\mathbf}
\newcommand{\pvec}[2]{(#1_1, #1_2, \dots, #1_{#2})}

\begin{document}
\begin{theorem}[13]
If $x, y$ are complex, prove that \[ ||x| - |y||  \le |x - y|.\]
\end{theorem}

\begin{proof}
    Notice that the sides of the triangle formed by $0$, $x$, and $y$ are $|x|$, $|y|$, and $|x - y|$. Then we can apply the triangle inequality and deduce that $|x| \le |x - y| + |y|$ and $|y| \le |y - x| + |x|$. (In the second equation we replace $|x - y| = |y - x|$.) Then $|x| - |y|$ and $|y| - |x|$ are both less than $|y - x|$. Hence $||x| - |y|| = |x - y|$, as desired.
\end{proof}

\begin{theorem}[16]
Suppose $k \ge 3$, $|\bf x - \bf y| = d > 0,$ and $r > 0$. Prove:
    \begin{enumerate}
        \item If $2r > d$, there are infinitely many $\bf z \in \mathbb R^k$ such that \[|\bf z - \bf x| = |\bf z - \bf y| = r.\]
        \item If $2r = d$, there is exactly one such $\bf z $.
        \item If $2r < d$, there is no such $\bf z$. 
    \end{enumerate}
How must these statements be modified if $k$ is 2 or 1.
\end{theorem}

\begin{proof}
    This will be easier if we use Exercise 15, i.e. equality holds in the Schwarz inequality if and only if $\bf x = \alpha \bf y$ for some real $\alpha$. The proof is fairly involved so I will leave it out. 

    \begin{enumerate}
        \item[(a)] Let $\bf x = \pvec{x}{k}$ and $\bf y = \pvec{y}{k}$. Let $\bf m = \frac{\bf x + \bf y}{2}$ be their midpoint. Consider the set of vectors $A$ such that $\bf a \in A$ if $\bf a \cdot (\bf x - \bf m) = 0$ and $|\bf a| = \sqrt {r^2 - d^2 / 4}$. The first equation defines a $k-1$ dimensional hyperplane in $\mathbb R^k$. The second equation defines a sphere, which will intesect the plane to form a $k-2$ dimensional space. Since $k \ge 3$, this space will have an infinite number of points. 
        
        Now we clain that every point of the form $\bf a + (\bf m - \bf x)$, for $\bf a \in A$, satisfies our condition. Indeed, since $a \cdot (\bf x - \bf y) = 0$, $\bf a$ is orthogonal to both $\bf m - \bf x$ and $\bf m - \bf y$. Hence we may apply the Pythagorean theorem: $|\bf a + (\bf m - \bf x)|^2 = |\bf a|^2 + |\bf m - \bf x|^2 = r^2 - d^2 / 4 + d^2 / 4 = r^2$; similarly $|\bf a + (\bf m - \bf y)|^2 = r^2$. Thus $|\bf a + (\bf m - \bf x)| = |\bf a + (\bf m - \bf y)| = r$, as desired. 
        
        \item[(b)] First, I claim that equality in the triangle inequality holds if and only if $\bf x = \alpha \bf y$ for some real $\alpha$. Indeed, 
        \begin{align*}
            |\bf u + \bf v|^2 &= (\bf u + \bf v) \cdot (\bf u + \bf v) \\
            &= \bf u \cdot \bf u + \bf v \cdot \bf v + \bf u \cdot \bf v + \bf v \cdot \bf u \\
            &= |\bf u|^2 + |\bf v|^2 + 2 |\bf u \cdot \bf v| \\
            &= |\bf u|^ 2 + |\bf v|^2 + 2 |\bf u| |\bf v| \\
            &= (|\bf u| + |\bf v|)^2,
        \end{align*}
        where the 4th and 5th equalities follow from the equality of the Schwarz inequality. Taking the root of both sides give equality of the triangle inequality.

        Now back to the problem. Since $|\bf z - \bf x| + |\bf z - \bf y| = |\bf x - \bf y|$, we must have $\bf z - \bf y = \alpha (\bf z - \bf x)$. The two have the same magnitude, so $\alpha = -1$. Then solving for $\bf z$, we have the unique solution $\bf z = \frac{\bf x + \bf y}{2}$.
        
        \item[(c)] The triangle inequality is not satisfied, so no such $\bf z$ may exist. 
    \end{enumerate}

    For lower dimensions we can just try to picture it. If $k = 1$, then we have two points in $\mathbb R$. Part 1 would be impossible. Parts 2 and 3 would remain the same. If $k = 2$, then part 1 would be equivalent to finding the intersection between two circles, which gives 2 points. Parts 2 and 3 would remain the same.

\end{proof}

\end{document}