\documentclass[12pt]{article}
\usepackage{amsmath,amssymb,amsthm,amsfonts,tabularx}
\usepackage[pdftex]{graphicx}
\usepackage[dvipsnames]{xcolor}
\usepackage{fancyhdr}
\usepackage{parskip}
\usepackage[shortlabels]{enumitem}
\pagestyle{fancy}
\usepackage{setspace}
\newcommand{\realname}[1]{\newcommand{\printrealname}{#1}}
\newcommand{\pset}[1]{\newcommand{\printpset}{#1}}
\newcommand{\mathclass}[1]{\newcommand{\printmathclass}{#1}}

%% Pagestyle setup
\setlength{\headheight}{0.75in}
\setlength{\oddsidemargin}{0in}
\setlength{\evensidemargin}{0in}
\setlength{\voffset}{-.5in}
\setlength{\headsep}{10pt}
\setlength{\textwidth}{6.5in}
\setlength{\headwidth}{6.5in}
\setlength{\textheight}{8in}
\lhead{Math \printmathclass}
\chead{\Large \textbf{Homework \printpset}}
\rhead{\printrealname}
\rfoot{Page \thepage}
\renewcommand{\headrulewidth}{0.5pt}
\renewcommand{\footrulewidth}{0.3pt}
\setlength{\textwidth}{6.5in}
\renewcommand{\baselinestretch}{1}
\setenumerate[0]{label=(\alph*)}
\newcommand{\todo}{\textcolor{red}{\textbf{TODO }}}

\newtheorem*{prop}{Proposition}
\newtheorem*{corollary}{Corollary}
\newtheorem*{lemma}{Lemma}
\theoremstyle{remark}
\newtheorem*{defn}{Definition}

\newtheoremstyle{named}{}{}{}{}{\bfseries}{.}{.5em}{\thmnote{Problem #3}}
\theoremstyle{named}
\newtheorem*{theorem}{Theorem}

%% DO NOT ALTER THE ABOVE LINES
%%%%%%%%%%%%%%%%%%%%%%%%%%%%%%%%%%%%%%%%%%%%


%% If you would like to use Asymptote within this document (which is optional), 
%% you can find out how at the following URL:
%%
%%   http://www.artofproblemsolving.com/Wiki/index.php/Asymptote:_Advanced_Configuration
%%
%% As explained there, you will want to uncomment the line below.  But be
%% sure to check the website because there are several other steps that must 
%% be followed.
%% \usepackage{asymptote}

%% Enter your real name here
%% Example: \realname{David Patrick}
\realname{Hanting Zhang}
\pset{11}
\mathclass{425A}

\renewcommand{\a}{\alpha}
\renewcommand{\b}{\beta}
\renewcommand{\d}{\delta}
\newcommand{\e}{\varepsilon}
\newcommand{\Z}{\mathbb Z}
\newcommand{\N}{\mathbb N}
\newcommand{\Q}{\mathbb Q}
\newcommand{\R}{\mathbb R}
\newcommand{\C}{\mathbb C}
\renewcommand{\bf}{\mathbf}
\newcommand{\id}[1]{\text{id}_{#1}}
\renewcommand{\implies}{\Rightarrow}
\newcommand{\coimplies}{\Leftarrow}
\renewcommand{\em}{\varnothing}
\renewcommand{\Im}{\text{Im}}
\newcommand{\abs}[1]{|#1|}
\newcommand{\bigabs}[1]{\left|#1\right|}

\begin{document}

Chapter 6, \# 9, 10 \& 13

\begin{theorem}[10]
    Let \(p\) and \(q\) be positive real numbers such that 
    \[\frac{1}{p} + \frac{1}{q} = 1.\]
    Prove the following statements.
    \begin{enumerate}
        \item If \(u \ge 0\) and \(v \ge 0\), then 
        \[uv \le \frac{u^p}{p} + \frac{v^q}{q}.\]
        Equality holds if and only if \(u^p = v^q\). 
        \item If \(f \in \mathcal R(\a), g \in \mathcal (\a), f \ge 0, g \ge 0\), and
        \[\int_a^b f^p d\a = 1 = \int_a^b g^q d\a,\]
        then
        \[\int_a^b fg d \a \le 1.\]
        \item If \(f\) and \(g\) are complex functions in \(\mathcal R (\a)\), then 
        \[\bigabs{\int_a^b fg d \a} \le \left\{\int_a^b \abs{f}^p d\a\right\}^{1/p} \left\{\int_a^b \abs{g}^q d\a\right\}^{1/q}.\]
        This is \textit{H\"{o}lder's inequality}. When \(p = q = 2\) it is usually called the Schwarz inequality. (Note that Theorem 1.35 is a very special case of this.)
        \item Show that H\"{o}lder's inequality is also true for the ``improper'' integrals described in Exercises 7 and 8.
    \end{enumerate}
\end{theorem}

\begin{proof}
    We proceed with each part:
    \begin{enumerate}
        \item We use Jensen's inequality. (I know this is slightly illegal since we haven't defined what \(\log(x)\) is, but I couldn't figure out any other way to do it.)
        If \(a = 0\) or \(b = 0\) then there's nothing to prove, so assume \(a, b > 0\). By Jensen's inequality,
        \begin{align*}
            \log\left(\frac{u^p}{p} + \frac{v^q}{q}\right) \ge \frac 1 p \log (u^p) + \frac 1 q \log(v^q) = \log u + \log v = \log(uv).
        \end{align*}
        Since \(\log x\) is monotonically increasing, we conclude that \[uv \le \frac{u^p}{p} + \frac{v^q}{q}.\]
        \item From part (a) we know that \[f(x)g(x) \le \frac{f(x)^p}{p} + \frac{g(x)^q}{q}.\]
        Integrating both sides gives
        \begin{align*}
            \int_a^b f(x)g(x) dx &\le \int_a^b \frac{f(x)^p}{p} + \frac{g(x)^q}{q} dx \\
            &= \frac 1 p \int_a^b f(x)^p dx + \frac 1 q \int_a^b g(x)^q dx \\
            &= \frac{1}{p} + \frac 1 q = 1.
        \end{align*}\
        \item Normalize \(f(x)\) and \(g(x)\) by taking 
        \[\frac{\abs{f(x)}}{\left(\int_a^b \abs{f(x)}^p\right)^{1/p}} \hspace*{5mm} \text{and} \hspace*{5mm} \frac{\abs{g(x)}}{\left(\int_a^b \abs{g(x)}^q\right)^{1/q}},\]
        so we can apply part (b). We have,
        \[\int_a^b\frac{\abs{f(x)}}{\left(\int_a^b \abs{f(x)}^p\right)^{1/p}}\frac{\abs{g(x)}}{\left(\int_a^b \abs{g(x)}^q\right)^{1/q}} \le 1,\]
        which implies 
        \[\bigabs{\int_a^b fg d \a} \le \int_a^b \abs{f}\abs{g} d \a \le \left\{\int_a^b \abs{f}^p d\a\right\}^{1/p} \left\{\int_a^b \abs{g}^q d\a\right\}^{1/q},\]
        as desired.
        \item There is no need for any other assumptions. We can just push the limits around since everything is continuous:
        \begin{align*}
            \bigabs{\int_a^\infty fg d \a} &= \bigabs{\lim_{b \to \infty}\int_a^b fg d \a} \\
            &\le \lim_{b \to \infty}\left\{\int_a^b \abs{f}^p d\a\right\}^{1/p} \left\{\int_a^b \abs{g}^q d\a\right\}^{1/q} \\
            &= \lim_{b \to \infty}\left\{\int_a^b \abs{f}^p d\a\right\}^{1/p} \lim_{b \to \infty}\left\{\int_a^b \abs{g}^q d\a\right\}^{1/q} \\
            &= \left\{\lim_{b \to \infty}\int_a^b \abs{f}^p d\a\right\}^{1/p} \left\{\lim_{b \to \infty}\int_a^b \abs{g}^q d\a\right\}^{1/q} \\
            &= \left\{\int_a^\infty \abs{f}^p d\a\right\}^{1/p} \left\{\int_a^\infty \abs{g}^q d\a\right\}^{1/q},
        \end{align*}
        as desired.
    \end{enumerate}
\end{proof}

\begin{theorem}[13]
    Define 
    \[f(x) = \int_x^{x + 1} \sin(t^2)dt.\]
    \begin{enumerate}
        \item Prove that \(\abs{f(x)} < 1 / x\) if \(x > 0\).
        \item Prove that 
        \[2xf(x) = \cos(x^2) - \cos[(x + 1)^2] + r(x)\]
        where \(\abs{r(x)} < c / x\) and \(c\) is a constant.
        \item Find the upper and lower limits of \(x f(x)\), as \(x \to \infty\).
        \item Does \(\int_0^\infty \sin(t^2)dt\) converge?
    \end{enumerate}
\end{theorem}

\begin{proof}
    We proceed with each part:
    \begin{enumerate}
        \item Assume that \(x > 0\) throughout. We begin with the hint from the book. Make the substitution \(u = t^2\) with \(dt = du / 2\sqrt u\) to obtain
        \[f(x) = \int_{x^2}^{(x + 1)^2} \frac{\sin u}{2\sqrt{u}}du.\]
        We integrate by parts with \(f(x) = \sin x\) and \(G(x) = 1 / 2\sqrt x\). So \(F(x) = -\cos x\) and \(g(x) = -1 / 4 x^{3/2}\). Hence 
        \[f(x) = \frac{\cos(x^2)}{2x} - \frac{\cos((x + 1)^2)}{2(x + 1)} - \int_{x^2}^{(x + 1)^2}\frac{\cos u}{4 u^{3/2}}du.\]
        Now we try to bound \(f(x)\) above and below by \(1/x\) and \(-1/x\), respectively. To bound \(f(x) < 1/x\), note that we can simplify the last integral term with the inequality by replacing \(\cos u\) with \(1\):
        \begin{align*}
            \int_{x^2}^{(x + 1)^2}\frac{\cos u}{4 u^{3/2}}du < \int_{x^2}^{(x + 1)^2}\frac{1}{4 u^{3/2}}du = \frac{1}{2(x + 1)} - \frac{1}{2x}.
        \end{align*}
        Thus 
        \begin{align*}
            f(x) &< \frac{\cos(x^2)}{2x} - \frac{\cos((x + 1)^2)}{2(x + 1)} + \frac{1}{2x} - \frac{1}{2(x + 1)} \\
            &= \frac{1 - \cos(x^2)}{2x} - \frac{1 - \cos((x + 1)^2)}{2(x + 1)} \\
            &\le \frac{1 - \cos(x^2)}{2x} \\
            &\le \frac{1}{x}.
        \end{align*}
        On the other hand, we can also replace \(\cos u\) with \(-1\), which gives the opposite effect:
        \begin{align*}
            f(x) &> \frac{\cos(x^2)}{2x} - \frac{\cos((x + 1)^2)}{2(x + 1)} + \frac{1}{2(x + 1)} - \frac{1}{2x} \\
            &= \frac{1 - \cos((x + 1)^2)}{2(x + 1)} - \frac{1 - \cos(x^2)}{2x} \\
            &\ge \frac{1 - \cos((x + 1)^2)}{2(x + 1)} \\
            &\ge -\frac{1}{x}.
        \end{align*}
        Thus \(\abs{f(x)} < 1 / x\). 
        \item We have 
        \begin{align*}
            2xf(x) &= \cos(x^2) - \frac{2x\cos((x + 1)^2)}{2(x + 1)} - 2x\int_{x^2}^{(x + 1)^2}\frac{\cos u}{4 u^{3/2}}du \\
            &= \cos(x^2) - \frac{2(x + 1)\cos((x + 1)^2) - 2\cos((x + 1)^2)}{2(x + 1)} - 2x\int_{x^2}^{(x + 1)^2}\frac{\cos u}{4 u^{3/2}}du \\
            &= \cos(x^2) - \cos((x + 1)^2) + \frac{\cos((x + 1)^2)}{x + 1} - 2x\int_{x^2}^{(x + 1)^2}\frac{\cos u}{4 u^{3/2}}du.
        \end{align*}
        Thus we may identify 
        \[r(x) = \frac{\cos((x + 1)^2)}{x + 1} - \frac{x}{2}\int_{x^2}^{(x + 1)^2}\frac{\cos u}{u^{3/2}}du.\]
        Now we cannot bound the last intgral again by \(\mathcal O (1/x)\) again because of the factor of \(x\) in the front. Hence the messy (but somewhat natural) thing to do is simply integrate by parts again. Let \(du = \sin x dx\) and \(v = \frac{1}{x^{3/2}}\), so that \(u = \cos x\) and \(dv = -\frac{3}{2x^{5/2}}dv\). Thus 
        \begin{align*}
            \int_{x^2}^{(x + 1)^2}\frac{\cos u}{u^{3/2}}du &= \frac{\sin((x + 1)^2)}{(x + 1)^3} -  \frac{\sin(x^2)}{x^3} + \int_{x^2}^{(x + 1)^2}\frac{3\sin u}{2 u^{5/2}}du.
        \end{align*}
        Now we may use the same technique as in part (a) to bound 
        \[-\frac{3}{2x^3} < \int_{x^2}^{(x + 1)^2}\frac{\cos u}{u^{3/2}}du < \frac{3}{2x^3}.\]
        Thus we can bound \(r(x)\) loosely with 
        \begin{align*}
            \abs{r(x)} &= \bigabs{\frac{\cos((x + 1)^2)}{x + 1}} + \bigabs{\frac{x}{2}\int_{x^2}^{(x + 1)^2}\frac{\cos u}{u^{3/2}}du} \\
            &< \frac{1}{x} + \frac{3}{2x^2} \\
            &< \frac{2}{x}.
        \end{align*}
        \item We claim that the lower and upper limits of \(xf(x)\) are \(\pm 1\). Indeed, since \(r(x) \to 0\) as \(x \to \infty\) we can not worry about it. So consider the behavior of 
        \begin{align*}
            \frac{\cos(x^2) - \cos((x + 1)^2)}{2} &= -\sin\left(\frac{x^2 + (x + 1)^2}{2}\right)\sin\left(\frac{x^2 - (x + 1)^2}{2}\right) \\
            &= \sin\left(x^2 + x + \frac 1 2\right)\sin\left(x + \frac 1 2\right).
        \end{align*}
        Intuitively, we need both arguments inside the sines to be close to \(2\pi (n + 1/2)\). This will \textit{probably (?)} never happen exactly due to the transendental nature of \(\pi\). But because we are looking at only the upper and lower limits, we have ``an epsilon of room'' to work with. In particular, intuitively, if \(x + 1/2 = 2 \pi (n + 1/2)\) for some \(n\) and \(n\) is large enough, the neighbourhood around 
        \[2 \pi \left(n + \frac 1 2\right) - \frac 1 2 \pm \e\]
        will map to some interval with length proportional to \(\e n\) under \(x \mapsto x^2\). To be precise, if we have 
        \[x^- = 2 \pi \left(n + \frac 1 2\right) - \frac 1 2 - \e \hspace*{5mm} \text{and} \hspace*{5mm} x^+ = 2 \pi \left(n + \frac 1 2\right) - \frac 1 2 + \e,\]
        then (I will just skip all the calculation...)
        \begin{align*}
            (x^+)^2 + x^+ + 1 - (x^-)^2 - x^- - 1 &= (x^+)^2 - (x^-)^2 + 2\e \\
            &= 2\e(\pi (4n + 2) + 1) + 2\e \\
            &\ge 2\pi \e (4n + 2)
        \end{align*}
        So for any \(\e > 0\), we may choose \(n > \frac{2 - \e}{8\e}\) so that \(2\pi \e (4n + 2) > 2\pi\). Thus there exist \(a, b\) in the interval such that \(\sin(a^2 + a + 1) = 1\) and \(\sin(b^2 + b + 1) = -1\), where \(\abs{x - a} < \e\) and \(\abs{x - b} < \e\). So we have \(af(a) > 1 - \e\) and \(bf(b) < -1 + \e\). (I may have lost some factors in there somewhere.) This holds for any \(\e > 0\), so the upper and lower limits of \(xf(x)\) are \(\pm 1\).
        \item The integral does converge. For any integer \(N\) we have 
        \begin{align*}
            \int_0^{N + 1} \sin(t^2)dt &= \sum_{n = 0}^N f(n) \\
            &= f(0) + \sum_{n = 1}^N \frac{1}{2n}\left(\cos(n^2) - \cos((n + 1)^2) + r(n)\right) \\
            &= f(0) + \sum_{n = 1}^N \frac{r(n)}{2n} + \frac 1 2 \sum_{n = 1}^N \frac{\cos(n^2)}{n} - \frac 1 2 \sum_{n = 2}^N \frac{\cos(n^2)}{n - 1} \\
            &= f(0) + \sum_{n = 1}^N \frac{r(n)}{2n} + \frac{\cos 1}{2} - \frac{\cos((N + 1)^2)}{2} + \sum_{n = 2}^N \frac{\cos(n^2)}{n(n - 1)}
        \end{align*}
        Since \(\abs{r(n)} < 2/n\) and \(\abs{\cos(n^2)} \le 1\), both sums are comparable to \(\sum_{n = 0}^N 1 / n^2\), we conclude that they converge in the limit \(n \to \infty\). Hence \[\int_0^\infty \sin(t^2)dt\] converges in the limit as well.
    \end{enumerate}
\end{proof}

\end{document}