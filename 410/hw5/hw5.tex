%%%%%%%%%%%%%%%%%%%%%%%%%%%%%%%%%%%%%%%%%%%%%%
%%                                          %%
%% USE THIS FILE TO SUBMIT YOUR SOLUTIONS   %%
%%                                          %%
%% You must have the usamts.tex file in     %%
%% the same directory as this file.         %%
%% You do NOT need to submit this file or   %%
%% usamts.tex with your solutions.  You     %%
%% only need to submit the output PDF file. %%
%%                                          %%
%% DO NOT ALTER THE FILE usamts.tex         %%
%%                                          %%
%% If you have any questions or problems    %%
%% using this file, or with LaTeX in        %%
%% general, please go to the LaTeX          %%
%% forum on the Art Of Problem Solving      %%
%% web site, and post your problem.         %%
%%                                          %%
%%%%%%%%%%%%%%%%%%%%%%%%%%%%%%%%%%%%%%%%%%%%%%

%%%%%%%%%%%%%%%%%%%%%%%%%%%%%%%%%%%%%%%%%%%%
%% DO NOT ALTER THE FOLLOWING LINES
\documentclass[12pt]{article}
\usepackage{amsmath,amssymb,amsthm,amsfonts,tabularx}
\usepackage[pdftex]{graphicx}
\graphicspath{ {./images/} }
\usepackage{fancyhdr}
\pagestyle{fancy}
\usepackage{setspace}
\usepackage{csquotes}
%%%%%%%%%%%%%%%%%%%%%%%%%%%%%%%%%%%%%%%%%%%
%%                                       %%
%% Students: DO NOT MODIFY THIS FILE!!!  %%
%%                                       %%
%%%%%%%%%%%%%%%%%%%%%%%%%%%%%%%%%%%%%%%%%%%

%% USAMTS style sheet
%% last modified: 23-Jul-2004

%% Student and Year/Round data
\newcommand{\realname}[1]{\newcommand{\printrealname}{#1}}
\newcommand{\pset}[1]{\newcommand{\printpset}{#1}}

%% Pagestyle setup
\setlength{\headheight}{0.75in}
\setlength{\oddsidemargin}{0in}
\setlength{\evensidemargin}{0in}
\setlength{\voffset}{-.5in}
\setlength{\headsep}{10pt}
\setlength{\textwidth}{6.5in}
\setlength{\headwidth}{6.5in}
\setlength{\textheight}{8in}
\lhead{Math 410}
\chead{\Large \textbf{Homework \printpset}}
\rhead{\printrealname}
\rfoot{Page \thepage}
\renewcommand{\headrulewidth}{0.5pt}
\renewcommand{\footrulewidth}{0.3pt}
\setlength{\textwidth}{6.5in}


\renewcommand{\baselinestretch}{1}
%% DO NOT ALTER THE ABOVE LINES
%%%%%%%%%%%%%%%%%%%%%%%%%%%%%%%%%%%%%%%%%%%%


%% If you would like to use Asymptote within this document (which is optional), 
%% you can find out how at the following URL:
%%
%%   http://www.artofproblemsolving.com/Wiki/index.php/Asymptote:_Advanced_Configuration
%%
%% As explained there, you will want to uncomment the line below.  But be
%% sure to check the website because there are several other steps that must 
%% be followed.
%% \usepackage{asymptote}

\newtheorem*{prop}{Proposition}
\newtheorem*{corollary}{Corollary}
\newtheorem*{lemma}{Lemma}
\theoremstyle{remark}
\newtheorem*{defn}{Definition}

\newtheoremstyle{named}{}{}{}{}{\bfseries}{.}{.5em}{\thmnote{Problem #3}}
\theoremstyle{named}
\newtheorem*{theorem}{Theorem}

%% Enter your real name here
%% Example: \realname{David Patrick}
\realname{Hanting Zhang}
\pset{5}
\mathclass{410}

\renewcommand{\bf}{\mathbf}
\renewcommand{\bar}{\overline}
\renewcommand{\implies}{\Rightarrow}
\newcommand{\coimplies}{\Leftarrow}
\newcommand{\normal}{\trianglelefteq}

\begin{document}
Exercises 9, 10, pp. 116-117.

\begin{theorem}[9]
    Assume \(G\) acts transitively on the finite set \(A\) and let \(H\) be a normal subgroup of \(G\). Let \(\mathcal O_1, \mathcal O_2, \dots, \mathcal O_r\) be the distinct orbits of \(H\) on \(A\).

    \begin{enumerate}
        \item[(a)] Prove that \(G\) permutes the sets \(\mathcal O_1, \mathcal O_2, \dots, \mathcal O_r\) in the sense that for each \(g \in G\) and each \(i \in \{1, \dots, r\}\) there is a \(j\) such that \(g \mathcal O_i = \mathcal O_j\), where \(g \mathcal O = \{g \cdot a \mid a \in \mathcal O\}\). Prove that \(G\) is transitive on \(\{\mathcal O_1, \dots \mathcal O_r\}\). Deduce that all orbits of \(H\) on \(A\) have the same cardinality.
        
        \item[(b)] Prove that if \(a \in \mathcal O_1\) then \(|\mathcal O_1| = |H : H \cap G_a|\) and prove that \(r = |G : H G_a|\). 
    \end{enumerate}
\end{theorem}

\begin{proof}
    Part (a): 
\end{proof}

\begin{theorem}[10]
    Let \(H\) and \(K\) be subgroups of the group \(G\). For each \(x \in G\) define the \(HK\) \textit{double coset} of \(x\) in \(G\) to be the set
    \[HxK = \{hxk \mid h \in H, k \in K\}.\]
\end{theorem}

Exercises 8, 14, pp. 122-123.

\begin{theorem}[8]
    Prove that if \(H\) has finite index \(n\) then there is a normal subgroup \(K \normal G\) with \(K \le H\) and \(|G : K| \le n!\). 
\end{theorem}

\begin{proof}
    Let \(G\) act on the left cosets of \(H\) with permutation representation \(\pi_{G / H} : G \to S_{|G/H|}\). Then let \(K = \ker (\pi_{G/H})\), which is clearly a subset of \(K\) and also a normal subgroup of \(G\). By the first isomorphism theorem, \(G / \ker(\pi_{G/H}) = G / K \cong \text{im}(\pi_{G/H})\). Comparing the cardinality of both sides, we have \
    \[|G : K| = |G / K| = |\text{im}(\pi_{G / H})| \le |S_{|G / H|}| = n!.\]
\end{proof}

\begin{theorem}[14]
    Let \(G\) be a finite group of composite order \(n\) with the property that \(G\) has a subgroup of order \(k\) dividing \(n\) for each positive integer \(k\) dividing \(n\). Prove that \(G\) is not simple.
\end{theorem}

\begin{proof}
    Choose \(k\) such that \(n / k = p\), where \(p\) is the minimum prime dividing \(n\). Then by Corollary 5, 
\end{proof}

\end{document}