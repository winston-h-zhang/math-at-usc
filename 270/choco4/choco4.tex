\documentclass[10pt]{article}
\usepackage{fullpage}
\usepackage[noend]{algorithmic}
\usepackage{algorithm}

\usepackage{amsfonts}
\usepackage{amsmath,amsthm,amssymb}
\usepackage{nicefrac}
\usepackage{latexsym}
\usepackage{epic}
\usepackage{epsfig}
\usepackage{amscd}
\usepackage{url}
\usepackage{verbatim}
\usepackage{setspace}
\usepackage{hyperref}
\usepackage{enumerate}
\usepackage{times}

% \usepackage{CS270}

\newtheorem{lemma}{Lemma}
\newtheorem{fact}[lemma]{Fact}
\newtheorem{theorem}[lemma]{Theorem}
\newtheorem{corollary}[lemma]{Corollary}
\newtheorem{proposition}[lemma]{Proposition}
\newtheorem{claim}[lemma]{Claim}

\newtheorem{definition}{Definition}
\newtheorem{example}[definition]{Example}
\newtheorem{remark}[definition]{Remark}

\providecommand{\half}{\ensuremath{\nicefrac{1}{2}}\xspace}
\providecommand{\third}{\ensuremath{\nicefrac{1}{3}}\xspace}
\providecommand{\quarter}{\ensuremath{\nicefrac{1}{4}}\xspace}

\providecommand{\SetCard}[1]{\ensuremath{| #1 |}\xspace}
\providecommand{\SET}[1]{\ensuremath{\{ #1 \}}\xspace}
\providecommand{\Abs}[1]{\ensuremath{| #1 |}\xspace}
\providecommand{\Set}[2]{\ensuremath{\SET{#1 \mid #2}}\xspace}

\providecommand{\Kth}[1]{\ensuremath{{#1}^{\rm th}}}
\providecommand{\Ceiling}[1]{\ensuremath{\lceil {#1} \rceil}\xspace}
\providecommand{\Floor}[1]{\ensuremath{\lfloor {#1} \rfloor}\xspace}

\providecommand{\PROB}{\ensuremath{{\rm Prob}}\xspace}
\providecommand{\Prob}[2][]{\ensuremath{%
\ifthenelse{\equal{#1}{}}{\PROB[#2]}{\PROB_{#1}[#2]}}\xspace}
\providecommand{\ProbC}[3][]{\Prob[#1]{#2\;|\;#3}}
\providecommand{\Expect}[2][]{\ensuremath{%
\ifthenelse{\equal{#1}{}}{\mathbb{E}}{\mathbb{E}_{#1}}%
\left[#2\right]}\xspace}
\providecommand{\ExpectC}[3][]{\Expect[#1]{#2\;|\;#3}}

\title{\bf Chocolate 4}
\author{Winston (Hanting) Zhang}

\begin{document}
\maketitle

\subsection*{Problem 0. [0]}
\textbf{Chocolate Problem: 1 chocolate bar}

Reminder: If you solve a chocolate problem (which you can do in groups of size up to 3), please e-mail David with the solution --- do not submit it on Gradescope. Also, feel free to list preferences or dietary restrictions for/against particular types of chocolate.

In discussion section, you saw how to generalize the idea of Binary Search to trees: each tree has a node $v$ such that when you query that node and the answer points to one of the subtrees from that node, there are at most $n/2$ nodes in that subtree. Thus, repeating this querying at most $\log_2 (n)$ times, you can find any node in a tree from the answers to such queries.

Here, we want to generalize the idea further, from trees to arbitrary undirected graphs. In non-tree graphs, it does not make sense to talk about ``the subgraph containing the node''. For instance, if your graph is a cycle, then no matter which node you query, you can go around the cycle in either direction to get to any other node. So we will clarify the answer as follows: when a node $v$ is queried, and the correct answer is $t$, the answer will reveal an edge out of $v$ that lies on a \emph{shortest} path from $v$ to $t$. If there are multiple shortest paths from $v$ to $t$ (with different edges out of $v$), then any of them could be returned.

You now play the following game: both you and your opponent know the undirected graph $G$. Your opponent picks a node $t \in G$, without telling you what it is. In each round, you get to point to a vertex $v$. If $v=t$, then the game is over. Otherwise, your opponent reveals an edge $e$ incident on $v$ that lies on a shortest path from $v$ to $t$. The game repeats until you've found $t$. Your goal is to get there with few queries.
Prove the following:

\begin{lemma} \label{lem:median}
  For every graph $G=(V,E)$, there exists a vertex $v$ you can query such that never mind which edge $e$ incident on $v$ your opponent reveals, the set of remaining vertices that are consistent with this answer shrinks by at least a factor 2. That is, if $S$ is the set of all nodes $t$ such that $e$ is on a shortest path from $v$ to $t$, then $|S| \leq |V|/2$.
\end{lemma}

To round out the answer, show how to use the lemma to guarantee that you can find the node $t$ in at most $\log_2 n$ queries --- this part is basically trivial once you have proved the lemma.

\end{document}