%%%%%%%%%%%%%%%%%%%%%%%%%%%%%%%%%%%%%%%%%%%%%%
%%                                          %%
%% USE THIS FILE TO SUBMIT YOUR SOLUTIONS   %%
%%                                          %%
%% You must have the usamts.tex file in     %%
%% the same directory as this file.         %%
%% You do NOT need to submit this file or   %%
%% usamts.tex with your solutions.  You     %%
%% only need to submit the output PDF file. %%
%%                                          %%
%% DO NOT ALTER THE FILE usamts.tex         %%
%%                                          %%
%% If you have any questions or problems    %%
%% using this file, or with LaTeX in        %%
%% general, please go to the LaTeX          %%
%% forum on the Art Of Problem Solving      %%
%% web site, and post your problem.         %%
%%                                          %%
%%%%%%%%%%%%%%%%%%%%%%%%%%%%%%%%%%%%%%%%%%%%%%

%%%%%%%%%%%%%%%%%%%%%%%%%%%%%%%%%%%%%%%%%%%%
%% DO NOT ALTER THE FOLLOWING LINES
\documentclass[12pt]{article}
\usepackage{amsmath,amssymb,amsthm,amsfonts,tabularx}
\usepackage[pdftex]{graphicx}
\usepackage[dvipsnames]{xcolor}
\usepackage{fancyhdr}
\usepackage{parskip}
\usepackage[shortlabels]{enumitem}
\pagestyle{fancy}
\usepackage{setspace}
\newcommand{\realname}[1]{\newcommand{\printrealname}{#1}}
\newcommand{\pset}[1]{\newcommand{\printpset}{#1}}
\newcommand{\mathclass}[1]{\newcommand{\printmathclass}{#1}}

%% Pagestyle setup
\setlength{\headheight}{0.75in}
\setlength{\oddsidemargin}{0in}
\setlength{\evensidemargin}{0in}
\setlength{\voffset}{-.5in}
\setlength{\headsep}{10pt}
\setlength{\textwidth}{6.5in}
\setlength{\headwidth}{6.5in}
\setlength{\textheight}{8in}
\lhead{Math \printmathclass}
\chead{\Large \textbf{Homework \printpset}}
\rhead{\printrealname}
\rfoot{Page \thepage}
\renewcommand{\headrulewidth}{0.5pt}
\renewcommand{\footrulewidth}{0.3pt}
\setlength{\textwidth}{6.5in}
\renewcommand{\baselinestretch}{1}
\setenumerate[0]{label=(\alph*)}

\newtheorem*{prop}{Proposition}
\newtheorem*{corollary}{Corollary}
\newtheorem*{lemma}{Lemma}
\theoremstyle{remark}
\newtheorem*{defn}{Definition}

\newtheoremstyle{named}{}{}{}{}{\bfseries}{.}{.5em}{\thmnote{Problem #3}}
\theoremstyle{named}
\newtheorem*{theorem}{Theorem}

%% DO NOT ALTER THE ABOVE LINES
%%%%%%%%%%%%%%%%%%%%%%%%%%%%%%%%%%%%%%%%%%%%


%% If you would like to use Asymptote within this document (which is optional), 
%% you can find out how at the following URL:
%%
%%   http://www.artofproblemsolving.com/Wiki/index.php/Asymptote:_Advanced_Configuration
%%
%% As explained there, you will want to uncomment the line below.  But be
%% sure to check the website because there are several other steps that must 
%% be followed.
%% \usepackage{asymptote}

%% Enter your real name here
%% Example: \realname{David Patrick}
\realname{Hanting Zhang}
\pset{8}
\mathclass{410}

\newcommand{\todo}{\textcolor{red}{\textbf{TODO} }}
\renewcommand{\a}{\alpha}
\renewcommand{\b}{\beta}
\newcommand{\Z}{\mathbb Z}
\newcommand{\F}{\mathbb F}
\renewcommand{\bf}{\mathbf}
\renewcommand{\implies}{\Rightarrow}
\newcommand{\coimplies}{\Leftarrow}
\newcommand{\rad}{\text{rad}\hspace*{0.7mm}}
\renewcommand{\em}{\varnothing}
\renewcommand{\mod}{\text{ mod }}

\begin{document}

Exercises 7, 11, 13, 14, 16, 30, 31 (expect (e)), pp. 256-260.

Let \(R\) be a ring with identity \(1 \neq 0\).

\begin{theorem}[7]
    Let \(R\) be a commutative ring with 1. Prove that the principal ideal generated by \(x\) in the polynomial ring \(R[x]\) is a prime ideal if and only if \(R\) is an integral domain. Prove that \((x)\) is a maximal ideal if and only if \(R\) is a field.
\end{theorem}

\begin{proof}
    The ideal \((x)\) is prime if \(ab \in (x) \Rightarrow a \in (x) \lor b \in (x)\) by definition. We apply the equivalence that \(r \in (x) \iff \overline{r} = \overline{0} \in R[x] / (x)\). Thus the definition \((x)\) being prime is equivalent to \(\overline{ab} = \overline{a} \overline{b} = \overline{0} \Rightarrow \overline{a} = \overline{0} \lor \overline{b} = \overline{0}\), i.e. \(R[x]/(x)\) is an integral domain.
\end{proof}

\begin{theorem}[11]
    Assume \(R\) is commutative. Prove that if \(P\) is a prime ideal of \(R\) and \(P\) contains no zero divisors then \(R\) is an integral domain.
\end{theorem}

\begin{proof}
    Let \(a, b \in R\) be any elements such that \(ab = 0\). Note that \(ab \in P\), and since \(P\) is prime, we have that either \(a \in P\) or \(b \in P\). Suppose \(a \in P\). Then since \(P\) has no zero-divisors, \(ab = 0\) forces \(a = 0\). The same argument applies when \(b \in P\) to show that \(b = 0\). In any case, either \(a = 0\) or \(b = 0\). Hence \(R\) is an integral domain.
\end{proof} 

\begin{theorem}[13]
    Let \(\varphi : R \to S\) be a homomorphism of commutative rings. 
    \begin{enumerate}
        \item Prove that if \(P\) is a prime ideal of \(S\) then either \(\varphi^{-1}(P) = R\) or \(\varphi^{-1}(P)\) is a prime ideal of \(R\). Apply this to the special case when \(R\) is a subring of \(S\) then \(P \cap R\) is either \(R\) or a prime ideal of \(R\).
        \item Prove that if \(M\) is a maximal ideal of \(S\) and \(\varphi\) is surjective then \(\varphi^{-1}(M)\) is a maximal ideal of \(R\). Give an example to show that this need not be the case if \(\varphi\) is not surjective.
    \end{enumerate}
\end{theorem}

\begin{proof}
    We proceed with each separately:
    \begin{enumerate}
        \item Let \(P \le S\) be a prime ideal. We can split into two cases: \(\varphi^{-1}(P) = R\) or \(\varphi^{-1}(P) < R\).
        
        In the first case, we're just done. 

        In the second case, let \(a, b \in R\) and \(ab \in \varphi^{-1}(P)\). Then we can do some map manipulations to see that 
        \[\varphi(ab) = \varphi(a)\varphi(b) \in P \Rightarrow \varphi(a) \in P \lor \varphi(b) \in P \Rightarrow a \in \varphi^{-1}(P) \lor b \in \varphi^{-1}(P),\]
        where the first implication is due to the fact that \(S\) is integral. Hence we have both conditions, so \(\varphi^{-1}(P)\) is integral.

        In the special case where we consider the inclusion \(\iota : R \hookrightarrow S\), we have \(\varphi^{-1}(P) = P \cap R\); so \(P \cap R\) is either \(R\) or a prime ideal of \(R\).
        \item Let \(I\) be any ideal such that \(\varphi^{-1}(M) \le I \le R\). Then we have \(M \le \varphi(I) \le \varphi(R)\). Since \(\varphi\) is surjective, we may identify \(\varphi(R) = S\). Since \(M\) is maximal, we deduce that \(\varphi(I)\) must be either \(M\) or \(S\). Thus \(I\) must be either \(\varphi^{-1}(M)\) or \(\varphi^{-1}(S) = R\), which means exactly that \(\varphi^{-1}(M)\) is maximal.
    \end{enumerate}
\end{proof}

\begin{theorem}[14]
    Assume \(R\) is commutative. Let \(x\) be an indeterminate, let \(f(x)\) be a monic polynomial in \(R[x]\) of degree \(n \ge 1\) and use the bar notation to denote passage to the passage to the quotient ring \(R[x]/(f(x))\).
    \begin{enumerate}
        \item Show that every element of \(R[x]/(f(x))\) is of the form \(\overline{p(x)}\) for some polynomial \(p(x) \in R[x]\) of degree less than \(n\).
        \item Prove that if \(p(x)\) and \(q(x)\) are distinct polynomials in \(R[x]\) which are both of degree less than \(n\), then \(\overline{p(x)} \neq \overline{q(x)}\).
        \item If \(f(x) = a(x)b(x)\) where both \(a(x)\) and \(b(x)\) have degree less than \(n\), prove that \(\overline{a(x)}\) is a zero divisor in \(R[x]/(f(x))\).
        \item If \(f(x) = x^n - a\) for some nilpotent element \(a \in R\), prove that \(\overline{x}\) is nilpotent in \(R[x]/(f(x))\).
        \item Let \(p\) be prime, assume \(R = \F_p\) and \(f(x) = x^p - a\) for some \(a \in \F_p\). Prove that \(\overline{x - a}\) is nilpotent in \(R[x]/(f(x))\).
    \end{enumerate}
\end{theorem}

\begin{proof}
    We proceed with each part separately:
    \begin{enumerate}
        \item We proceed by induction on the degree to show that for any \(q(x)\in R[x]\) we have \(\overline{q(x)} = \overline{p(x)}\) for some \(p(x)\) of degree less than \(n\).
        
        Consider the base case \(m < n\), then there is nothing to prove.

        Now assume for the sake of induction that for some \(k \ge n\) all polynomials \(r(x) \in R[x]\) with \(\deg r = k\) satisfy \(\overline{r(x)} = \overline{p(x)}\) for some \(p(x)\) of degree less than \(n\).

        Let \(q(x) = a_{k + 1} x^{k + 1} + a_k x^k + \cdots + a_1 x + a_0\) be any polynomial of degree \(k + 1\). If \(f(x) = x^n + b_{n - 1} x^{n - 1} + \cdots + b_0\). Notice that we have the relation 
        \[\overline{x^n} = \overline{-(b_{n - 1} x^{n - 1} + \cdots + b_0)}.\]
        Hence we may erase the leading coefficient of \(q(x)\):
        \begin{align*}
            \overline{q(x)} &= \overline{a_{k + 1} x^{k + 1} + a_k x^k + \cdots + a_1 x + a_0} \\
            &= \overline{a_{k + 1} x^{k + 1}} + \overline{a_k x^k + \cdots + a_1 x + a_0} \\
            &= \overline{x^n}\left(\overline{a_{k + 1} x^{k + 1 - n}}\right) + \overline{a_k x^k + \cdots + a_1 x + a_0} \\
            &= \left(\overline{-(b_{n - 1} x^{n - 1} + \cdots + b_0)}\right)\left(\overline{a_{k + 1} x^{k + 1 - n}}\right) + \overline{a_k x^k + \cdots + a_1 x + a_0} \\
            &= -\left(\overline{a_{k + 1}b_{n - 1} x^{n - 1}x^{k + 1 - n} + \cdots + a_{k + 1}b_0x^{k + 1 - n}}\right) + \overline{a_k x^k + \cdots + a_1 x + a_0} \\
            &= -\overline{a_{k + 1}b_{n - 1} x^{k} + \cdots + a_{k + 1}b_0x^{k + 1 - n}} + \overline{a_k x^k + \cdots + a_1 x + a_0}.
        \end{align*}
        Hence we see that \(\overline{q(x)} = \overline{r(x)}\) for some polynomial \(r(x)\) of degree \(k\)! The induction hypothesis states that \(\overline{q(x)} = \overline{r(x)} = \overline{p(x)}\) for some \(p(x)\) of degree less than \(n\). This completes the induction and we are done.
        \item We have \(\deg(p - q) < n\). Thus \(p - q \notin (f(x)) \Rightarrow \overline{p - q} \neq \overline{0}\). Hence \(\overline{p(x)} \neq \overline{q(x)}\).
        \item Since both \(\deg a(x), \deg b(x) < n\), we have \(\overline{a(x)}, \overline{b(x)} \neq 0\). But clearly we also have \(\overline{a(x)}\overline{b(x)} = \overline{a(x)b(x)} = \overline{f(x)} = \overline{0}\). Thus \(\overline{a(x)}\) is a zero divisor of \(R[x]/(f(x))\).
        \item We have: \[f(x) = x^n - a \Rightarrow \overline{0} = \overline{x^n - a} \Rightarrow \overline{x^n} = \overline{a}.\]
        But \(a\) is nilpotent, so there is some \(m \in \Z^+\) such that \(a^m = 0\). Thus, \[\overline{0} = \overline{a^m} = \overline{a}^m = \overline{x^n}^m = \overline{x}^{mn}.\]
        So indeed \(\overline{x}\) is nilpotent as well.
        \item From Exercise 26 from Section 3 we know that \((x - a)^p = x^p + (-a)^p\). Note that \(\F_p^\times\) is a group of order \(p - 1\), so we have \((-a)^{p - 1} = 1\). Thus \((x - a)^p = x^p - a\). But this exactly shows that \(\overline{(x - a)^p} = \overline{x^p - a} = \overline{0}\), as desired!
    \end{enumerate}
\end{proof}

\begin{theorem}[16]
    Let \(x^2 - 16\) be an element of the polynomial ring \(E = \Z[x]\) and use the bar notation to denote passage to the quotient ring \(\Z[x]/(x^3 - 2x + 1)\). Let \(p(x) = 2x^7 - 7x^5 + 4x^3 - 9x + 1\) and let \(q(x) = (x - 1)^4\).
    \begin{enumerate}
        \item Express each of the following elements of \(\overline{E}\) in the form \(\overline{f(x)}\) for some polynomial \(f(x)\) of degree \(\le 2\): \(\overline{p(x)}, \overline{q(x)}, \overline{p(x) + q(x)}\), and \(\overline{p(x)q(x)}\).
        \item Prove that \(\overline{E}\) is not an integral domain.
        \item Prove that \(\overline{x}\) is a unit in \(\overline{E}\).
    \end{enumerate}
\end{theorem}

\begin{proof}
    We proceed with each separately:
    \begin{enumerate}
        \item Do polynomial long division to figure out \(\overline{p(x)}\) and \(\overline{q(x)}\):
        \begin{align*}
            p(x) &= (2x^4 - 3x^2 - 2x - 2) (x^3 - 2x + 1) + (-x^2 - 11x + 3) \\
            &\Rightarrow \overline{p(x)} = \overline{-x^2 - 11x + 3}; \\
            q(x) &= (x - 4)(x^3 - 2x + 1) + (8x^2 - 13x + 5) \\
            &\Rightarrow \overline{q(x)} = \overline{8x^2 - 13x + 5}.
        \end{align*}
        Then we have \(\overline{p(x) + q(x)} = \overline{7x^2 - 24x + 8}\) and 
        \begin{align*}
            \overline{p(x)q(x)} &= \overline{(-x^2 - 11x + 3)(8x^2 - 13x + 5)} \\
            &= \overline{-8x^4-75x^3+162x^2-94x+15} \\
            \overline{p(x)q(x)} &= \overline{(-8x - 75)(x^3 - 2x + 1) + (146x^2 - 236x + 90)} \\
            &\Rightarrow   \overline{p(x)q(x)} = \overline{146x^2 - 236x + 90}
        \end{align*}
        \item Note that \(x^3 - 2x + 1\) has a root at 1 so we may factor \(x^3 - 2x + 1 = (x - 1)(x^2 + x - 1)\). However in the quotient, both \(\overline{x - 1}\) and \(\overline{x^2 + x - 1}\) are nonzero while \(\overline{x^3 - 2x + 1} = \overline{0}\). Thus \(\overline{E}\) is not an integral domain.
        \item We need \(xf(x) = qd + 1\) where \(d = x^3 - 2x + 1\) and \(q\) is some resulting quotient. Note that the LHS has no constant factor; hence a good guess for \(q\) would be \(-1\), since that eliminates the \(+1\) on the RHS. Indeed, \(xf(x) = -d + 1 = -x^3 + 2x = x(-x^2 + 2)\). So clearly \(f(x) = -x^2 + 2\) works. Then \(\overline{f(x)} = \overline{-x^2 + 2}\) is the inverse of \(\overline{x}\), proving that it is a unit.
    \end{enumerate}
\end{proof}

\begin{theorem}[30]
    Let \(I\) be an ideal of the commutative ring \(R\) and define 
    \[\rad I = \{r \in R \mid r^n \in I \text{ for some } n \in \Z^+\}\]
    called the \textit{radical} of \(I\). Prove that \(\rad I\) is an ideal containing \(I\) and that \((\rad I)/I\) is the nilradical of the quotient ring \(R / I\), i.e. \((\rad I / I) = \mathfrak{R}(R / I)\).
\end{theorem}

\begin{proof}
    \(\rad I\) contains \(I\): Clearly for any \(r \in I\) we have \(r^1 \in I\), so \(r \in \rad I\). Thus \(I \le \rad I\).

    Recall that the nilradical of \(R / I\) is defined as
    \[\{\overline{r} \in R / I \mid \overline{r}^n = 0 \text{ for some } n \in \Z^+\}.\]
    Thus \(\overline{r} \in \mathfrak{R}(R / I)\) if and only if there is \(n \in \Z^+\) such that \(\overline{r}^n = 0\). This occurs if and only if there is \(n \in \Z^+\) such that \(r^n \in I\), i.e. \(r \in \rad I\). Thus we may chain the if and only if statements to conclude that \((\rad I / I) = \mathfrak{R}(R / I)\).
\end{proof}

\begin{theorem}[31]
    An ideal \(I\) of the commutative ring \(R\) is called a \textit{radical ideal} if \(\rad I = I\). 
    \begin{enumerate}
        \item Prove that every prime ideal of \(R\) is a radical ideal.
        \item Let \(n > 1\) be an integer. Prove that 0 is a radical ideal in \(\Z / n \Z\) if and only if \(n\) is a product of distinct primes to the first power (i.e. \(n\) is square free). Deduce that \((n)\) is a radical ideal of \(\Z\) if and only if \(n\) is a product of distinct primes in \(\Z\). 
    \end{enumerate}
\end{theorem}

\begin{proof}
    We proceed with each part separately:
    \begin{enumerate}
        \item Let \(P\) be a prime ideal of \(R\). We already know that \(P \le \rad P\), so it suffices to only show that \(\rad P \le P\). Let \(r \in \rad P\) and \(n \in \Z^+\) such that \(r^n \in P\). 
        
        We proceed by induction to prove that \(r^n \in P \Rightarrow r \in P\) for all \(n \in \Z^+\). The base case \(n = 1\) is trivial: \(r \in P \Rightarrow r \in P\). Now assume for that sake of induction that \(r^k \in P \Rightarrow r \in P\) is true for some \(k \in \Z^+\). Then consider \(r^{k + 1} = rr^k\in P\). Since \(P\) is prime, we have either \(r \in P\), in which case we are done, or \(r^k \in P\), in which case we may apply our IH to conclude that \(r \in P\). This completes the induction. 

        Therefore we see that \(r^n \in P \Rightarrow r \in P\), so \(\rad P \le P\). Hence \(\rad P = P\) and \(P\) is a radical ideal.
        
        \item Recall the following theorem from homework 7, problem 13 (b):
        
        \textit{If \(a \in \Z\) is an integer, the element \(\overline{a} \in \Z / n \Z\) is nilpotent if and only if every prime divisor of \(n\) is also a prime divisor of \(a\).}

        Note that trying to find the radical of \(0\) is equivalent to finding all elements \(r \in R\) such that \(r^n = 0\) for some \(n \in \Z^+\), i.e. the nilpotent elements of \(R\). Thus here we have \(a \in \rad 0\) if and only if every prime divisor of \(n\) is also a prime divisor of \(a\).

        (\(\Rightarrow\)): If \(n = p_1 \cdots p_k\) is the product of distinct primes, and each of those primes must divide \(a\), then \(\forall i, p_i \mid a \Rightarrow p_1 \cdots p_k \mid a \Rightarrow n \mid a\). Thus \(\overline{a} = \overline{0}\); we conclude that \(\rad 0 = 0\) is a radical ideal. 

        (\(\Leftarrow\)): We show that contrapositive. Suppose \(n\) is not the product of distinct primes, i.e. there is some prime \(p\) such that \(p^2 \mid n\). Then \(a = p\cdot p_1' \cdots p_k'\), where \(p_1', \cdots, p_k'\) are all the other prime factors of \(n\) other than \(p\). But \(p^2\) does not divide \(a\) so \(n \nmid a\); hence \(a \neq 0 \in \rad 0\). We conclude that \(\rad 0\) is not a radical ideal, as desired.
    \end{enumerate}
\end{proof}

Exercises 1, 2, 5 pp. 267-269.

\begin{theorem}[1]
    An element \(e \in R\) is called an \textit{idempotent} if \(e^2 = e\). Assume that \(e\) is an idempotent in \(R\) and \(er = re\) for all \(r \in R\). Prove that \(Re\) and \(R(1 - e)\) and two-sided ideals of \(R\) and that \(R \cong Re \times R(1 - e)\). Show that \(e\) and \(1 - e\) are identities for the subrings \(Re\) and \(R(1 - e)\) respectively.
\end{theorem}

\begin{proof}
    \textit{\(Re\) and \(R(1 - e)\) are two-sided ideals of \(R\)}: Clearly \(Re\) is a two-sided ideal since \(re = er \Rightarrow Re = eR\). Note that \((1 - e)^2 = 1 - 2e + e^2 = 1 - 2e = e = 1 - e\), so \(1 - e\) is an idempotent of \(R\) as well. Furthermore, for any \(r \in R\) we have \(r(1 - e) = r - re = r - er - (1 - e)r\), so \(1 - e\) commutes with everything. Clearly this shows that \(R(1 - e) = (1 - e)R\); hence \(R(1 - e)\) is a two-sided ideal.

    \(R \cong Re \times R(1 - e)\): Define the map \(\varphi : R \to Re \times R(1 - e)\) by \(r \mapsto (re, r(1 - e))\). Clearly \(\varphi\) is a surjective ring homomorphism, since both \(r \mapsto re\) and \(r \mapsto r(1 - e)\) are surjective ring homomorphisms.

    Thus it remains only to show that \(\varphi\) is injective. Indeed, suppose \(\varphi(r) = (re, r(1 - e)) = (0, 0)\). We have \(re = 0\) and \(r(1 - e) = 0\); hence \(r(1 - e) = r - re = r = 0\), which shows that \(\ker \varphi = 0\), as desired. We conclude that \(\varphi\) is an isomorphism and that \[R \cong Re \times R(1 - e).\]

    View \(Re\) as a ring. Any element in \(Re\) has the form \(re\) for some \(r \in R\). We can check that \(e\) is the identity directly: \((re)e = ree = re\) and \(e(re) = e(er) = eer = er\). Similarily, view \(R(1 - e)\) as a ring. Since we've already shown that \(1 - e\) is an idempotent of \(R\) and \(r(1 - e) = (1 - e)r\) for all \(r \in R\), we have the same logic to show that \(1 - e\) is the identity: \(r(1 - e)(1 - e) = r(1 - e)\) and \((1 - e)r(1 - e) = r(1 - e)(1 - e) = r(1 - e)\).
\end{proof}

\begin{theorem}[2]
    Let \(R\) be a finite Boolean ring with identity \(1 \neq 0\). Prove that \(R \cong \Z / 2 \Z \times \Z / 2 \Z \times \dots \times \Z / 2 \Z\).
\end{theorem}

\begin{proof}
    We proceed by induction on the cardinality of \(R\). Consider the base case \(|R| = 2\). Then there is only one choice for \(R\), namely \(\Z_2\) (we shorten \(\Z / 2 \Z\)), so our base case is correct. 

    Now assume for the sake of strong induction that our hypothesis holds for all \(n < k\) for some \(k > 2\). We want to show that any Boolean ring with size \(|R| = n\) is isomorphic to some \(\Z_2^r\).

    Indeed, let \(e \in R\) be any nonzero, non-identity element. Then \(e^2 = e\) by definition, so \(e\) is an idempotent of \(R\). We apply the previous exercise to give \(R = Re \times R(1 - e)\). In particular, both \(Re\) and \(R(1 - e)\) have at least two elements (zero and identity), so \(|Re|, |R(1 - e)| < |R|\). Thus we may apply the induction hypothesis to see that \[R \cong \Z_2^a \times \Z_2^b = \Z_2^{a + b}.\]
    So we have \(r = a + b\), and the induction is complete.
\end{proof}

\begin{theorem}[5]
    Let \(n_1, n_2, \cdots, n_k\) be integers which are relatively prime in pairs: \((n_i, n_j = 1\) for all \(i \neq j\). 
    \begin{enumerate}
        \item Show that the Chinese Remainder Theorem implies that for any \(a_1, \cdots, a_k \in \Z\) there is a solution \(x \in \Z\) to the simultaneous congruences
        \[x \equiv a_1 \mod n_1, \hspace*{5mm} x \equiv a_2 \mod n_2, \hspace*{5mm} \cdots, \hspace*{5mm} x \equiv a_k \mod n_k\]
        and that the solution \(x\) is unique mod \(n = n_1n_2\cdots n_k\).
        \item Let \(n_i' = n / n_i\) and \(t_i\) be the inverse of \(n_i' \mod n_i\). Prove that the solution \(x\) in (a) is given by \[x = a_1 t_1 n_1' + a_2 t_2 n_2' + \cdots + a_k t_k n_k' \mod n.\]
        \item Solve the simultaneous system of congruences
        \[x \equiv 1 \mod 8, \hspace*{5mm} x \equiv 2 \mod 25, \hspace*{5mm} x \equiv 3 \mod 81\]
        and \[y \equiv 5 \mod 8, \hspace*{5mm} y \equiv 12 \mod 25, \hspace*{5mm} y \equiv 47 \mod 81.\]
    \end{enumerate}
\end{theorem}

\begin{proof}
    We proceed with each:
    \begin{enumerate}
        \item If \((n_i, n_j) = 1\) then there are integers \(x, y\) such that \(xn_i + yn_j = 1\); thus we know that \((n_i, n_j) = \Z\) and that \((n_i)\) and \((n_j)\) are comaximal for any \(i\) and \(j\). Hence the Chinese Remainder Theorem gives that 
        \[\Z / (n_1) \times \cdots \times \Z / (n_k) \cong \Z / (n_1 \cdots n_k) = \Z / (n).\]
        This exactly means that there is a unique \(x \in \Z / (n)\) which solves the system. 
        \item Since \(x\) is unique, it suffices to show that \(x\) satisfies the above equivalences. For any \(i\) and \(j \neq i\), \(n_i \mid n_j'\), so 
        \[x \equiv \sum_{j = 1}^k a_j t_j n_j' \equiv a_i t_i n_i' \equiv a_i \mod{n_i},\]
        which means we're done.
        \item I do the computation elsewhere and won't include it since it's just very mechanical and boring. We have \(x \equiv 3601 \mod 16200\) and \(y \equiv 8269 \mod 16200\).
    \end{enumerate}
\end{proof}

More to be added...?

\end{document}