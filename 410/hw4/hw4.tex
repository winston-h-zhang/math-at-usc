%%%%%%%%%%%%%%%%%%%%%%%%%%%%%%%%%%%%%%%%%%%%%%
%%                                          %%
%% USE THIS FILE TO SUBMIT YOUR SOLUTIONS   %%
%%                                          %%
%% You must have the usamts.tex file in     %%
%% the same directory as this file.         %%
%% You do NOT need to submit this file or   %%
%% usamts.tex with your solutions.  You     %%
%% only need to submit the output PDF file. %%
%%                                          %%
%% DO NOT ALTER THE FILE usamts.tex         %%
%%                                          %%
%% If you have any questions or problems    %%
%% using this file, or with LaTeX in        %%
%% general, please go to the LaTeX          %%
%% forum on the Art Of Problem Solving      %%
%% web site, and post your problem.         %%
%%                                          %%
%%%%%%%%%%%%%%%%%%%%%%%%%%%%%%%%%%%%%%%%%%%%%%

%%%%%%%%%%%%%%%%%%%%%%%%%%%%%%%%%%%%%%%%%%%%
%% DO NOT ALTER THE FOLLOWING LINES
\documentclass[12pt]{article}
\usepackage{amsmath,amssymb,amsthm,amsfonts,tabularx}
\usepackage[pdftex]{graphicx}
\graphicspath{ {./images/} }
\usepackage{fancyhdr}
\pagestyle{fancy}
\usepackage{setspace}
\usepackage{csquotes}
%%%%%%%%%%%%%%%%%%%%%%%%%%%%%%%%%%%%%%%%%%%
%%                                       %%
%% Students: DO NOT MODIFY THIS FILE!!!  %%
%%                                       %%
%%%%%%%%%%%%%%%%%%%%%%%%%%%%%%%%%%%%%%%%%%%

%% USAMTS style sheet
%% last modified: 23-Jul-2004

%% Student and Year/Round data
\newcommand{\realname}[1]{\newcommand{\printrealname}{#1}}
\newcommand{\pset}[1]{\newcommand{\printpset}{#1}}

%% Pagestyle setup
\setlength{\headheight}{0.75in}
\setlength{\oddsidemargin}{0in}
\setlength{\evensidemargin}{0in}
\setlength{\voffset}{-.5in}
\setlength{\headsep}{10pt}
\setlength{\textwidth}{6.5in}
\setlength{\headwidth}{6.5in}
\setlength{\textheight}{8in}
\lhead{Math 410}
\chead{\Large \textbf{Homework \printpset}}
\rhead{\printrealname}
\rfoot{Page \thepage}
\renewcommand{\headrulewidth}{0.5pt}
\renewcommand{\footrulewidth}{0.3pt}
\setlength{\textwidth}{6.5in}


\renewcommand{\baselinestretch}{1}
%% DO NOT ALTER THE ABOVE LINES
%%%%%%%%%%%%%%%%%%%%%%%%%%%%%%%%%%%%%%%%%%%%


%% If you would like to use Asymptote within this document (which is optional), 
%% you can find out how at the following URL:
%%
%%   http://www.artofproblemsolving.com/Wiki/index.php/Asymptote:_Advanced_Configuration
%%
%% As explained there, you will want to uncomment the line below.  But be
%% sure to check the website because there are several other steps that must 
%% be followed.
%% \usepackage{asymptote}

\newtheorem*{prop}{Proposition}
\newtheorem*{corollary}{Corollary}
\newtheorem*{lemma}{Lemma}
\theoremstyle{remark}
\newtheorem*{defn}{Definition}

\newtheoremstyle{named}{}{}{}{}{\bfseries}{.}{.5em}{\thmnote{Problem #3}}
\theoremstyle{named}
\newtheorem*{theorem}{Theorem}

%% Enter your real name here
%% Example: \realname{David Patrick}
\realname{Hanting Zhang}
\pset{4}
\mathclass{410}

\renewcommand{\bf}{\mathbf}
\renewcommand{\bar}{\overline}
\renewcommand{\implies}{\Rightarrow}
\newcommand{\coimplies}{\Leftarrow}
\newcommand{\normal}{\trianglelefteq}

\begin{document}
Exercises 6, 12, pp. 52-53.

\begin{enumerate}
    \item [6.]
    \begin{enumerate}
        \item [(a)] If \(H\) is a subgroup of \(G\), then for any \(h, h' \in H\), we have \(h^{-1}h'h \in H\). Hence \(h^{-1}Hh = H\), and \(h \in N_G(H)\). Therefore \(H \le N_G(H)\). 
        
        If \(H\) is not a subgroup of \(G\), then multiplication fails so we have no reason to expect \(h^{-1}h'h \in H\). For example, let 
        \[H = \left\{\begin{pmatrix}
            1 & 1 \\ 0 & 1
        \end{pmatrix}, \begin{pmatrix}
            2 & 3 \\ 1 & 2
        \end{pmatrix}\right\}.\]
    Then 
    \[\begin{pmatrix}
        1 & 1 \\ 0 & 1
    \end{pmatrix}^{-1} \begin{pmatrix}
        2 & 3 \\ 1 & 2
    \end{pmatrix} \begin{pmatrix}
        1 & 1 \\ 0 & 1
    \end{pmatrix} = \begin{pmatrix}
        2 & -5 \\ 0 & 3
    \end{pmatrix} \notin H.\]
    Hence \(H \nleq N_G(H)\).

    \item [(b)] (\(\implies\)): If \(H\) is abelian, then clearly \(h'h = hh'\) for all \(h, h' \in H\). Then \(H \le C_G(H)\).

    (\(\Leftarrow\)): If \(H \le C_G(H)\), then for any \(h, h' \in H\), we have \(h^{-1}h'h = h' \implies h'h = h h'\). Hence \(H\) is abelian, as desired.

    \end{enumerate}
    \item [12.] We do each part:
    \begin{enumerate}
        \item[(a)] Define 
        \[p(x_1, \dots, x_4) = 12x_1^5 x_2^7 x_3 - 18 x_2^3 x_3 + 11 x_1^6 x_2 x_3^3 x_4^{23}.\]
        
        Direct calculation shows that:
        \begin{align*}
            \sigma \cdot p &= (1 2 3 4) \cdot p \\
            &= 12x_2^5 x_3^7 x_4 - 18 x_3^3 x_4 + 11 x_1^{23} x_2^6 x_3 x_4^3 \\ 
            \\
            \tau \cdot (\sigma \cdot p) &= (123) \cdot 12x_2^5 x_3^7 x_4 - 18 x_3^3 x_4 + 11 x_1^{23} x_2^6 x_3 x_4^3 \\
            &= 12 x_1^7 x_3^5 x_4 - 18 x_1^3 x_4 + 11 x_1 x_2^{23} x_3^6 x_4^3 \\ 
            \\
            (\tau \circ \sigma) \cdot p &= (1342) \cdot p \\ 
            &= 12 x_1^7 x_3^5 x_4 - 18 x_1^3 x_4 + 11 x_1 x_2^{23} x_3^6 x_4^3 \\ 
            \\
            (\sigma \circ \tau) \cdot p &= (1324) \cdot p \\
            &= 12 x_2 x_3^5 x_4^7 - 18 x_4^3 + 11 x_1^{23} x_2^3 x_3^6 x_4.
        \end{align*}

        \item[(b)] This definition gives a left group action of \(S_4\) on \(R\). If \(\sigma, \tau \in S_4\) and \(p \in R\), then 
        \begin{align*}
            \tau \cdot (\sigma \cdot p) &= \tau \cdot (p(x_{\sigma(1)}, \dots, x_{\sigma(4)})) \\
            &= p(x_{\tau(\sigma(1))}, \dots, x_{\tau(\sigma(4))}) \\
            &= p(x_{(\tau \circ \sigma) (1)}, \dots, x_{(\tau \circ \sigma) (1)}) \\
            &= (\tau \circ \sigma) \cdot p.
        \end{align*}
        Hence composition in \(S_4\) is compatible with its action on \(R\). Clearly \(e \cdot p = p\). Thus we have satisfied the axioms for a group action, as desired.

        \item[(c)] The permutations that stabilize \(x_4\) are the ones that fix \(4\). The subset of \(S_4\) that does this is: \(\{e, (12), (23), (13), (123), (231)\}\). Looking at these permutations in cycle notation, clearly they are isomorphic to \(S_3\). 
        (They are the image of the embedding \(\iota : S_3 \hookrightarrow S_4\).)
        
        \item[(d)] An element \(\sigma\) stabilizes \(x_1 + x_2\) satisfy \(x_1 + x_2 = x_{\sigma(1)} + x_{\sigma(2)}\). Hence we must have either \((\sigma(1), \sigma(2)) = (1, 2)\) or \((\sigma(1), \sigma(2)) = (2, 1)\). In the first case, \(\sigma\) fixes 1 and 2, so the possible values are \(\{e, (34)\}\). 
        In the second case, \(\sigma\) must permute \((12)\), so the possible values are \(\{3, (12), (12)(34)\}\). Letting \(x = (12), y = (34)\), we see that the stabilizer of \(x_1+ x_2\) is \(\{e, x, y, xy\}\) with \(x^2 = y^2 = e\). This is clearly the (abelian) group \(\mathbb Z_2 \times \mathbb Z_2\). 

        \item[(e)] If \(\sigma\) stabilizes \(x_1 x_2 + x_3 x_4\), then 
        \[x_{\sigma(1)}x_{\sigma(2)} + x_{\sigma(3)} x_{\sigma(4)} = x_1 x_2 + x_3 x_4.\]
        Then there are two cases:
        \begin{align*}
            (\sigma(1), \sigma(2)) &\in \{(1, 2), (2, 1)\} \land (\sigma(3), \sigma(4)) \in \{(3, 4), (4, 3)\}, \\
            (\sigma(1), \sigma(2)) &\in \{(3, 4), (4, 3)\} \land (\sigma(3), \sigma(4)) \in \{(1, 2), (2, 1)\}
        \end{align*}
        The first case has solutions \(\sigma \in \{e, (12), (34), (12)(34)\}\). And the second case has solutions \(\sigma \in \{(13)(24), (1324), (1423), (14)(23)\}\). 

        To see that these two sets combine to form \(D_8\), map \((12) \mapsto s\) and \(r \mapsto (1324)\). Then we have:
        \begin{align*}
            e \mapsto e, \hspace{2mm} (1324) \mapsto r, \hspace{2mm} (12)(34) \mapsto r^2, \hspace{2mm} (1423) \mapsto r^3, \\ 
            (12) \mapsto r, \hspace{2mm} (13)(24) \mapsto sr, \hspace{2mm} (34) \mapsto sr^2, \hspace{2mm} (14)(23) \mapsto sr^3.
        \end{align*}
        It can be checked that this is an isomorphism. Hence the stabilizer of \(x_1 x_2 + x_3 x_4\) is indeed isomorphic to \(D_8\).

        \item[(f)] Again, for the map to stablize, we must have either \(x_1 + x_2 = x_{\sigma(1)} + x_{\sigma(2)}\) or \(x_1 + x_2 = x_{\sigma(3)} + x_{\sigma(4)}\). The same equations follow for \(x_3 + x_4\). So there are two cases:
        \begin{align*}
            (\sigma(1), \sigma(2)) &\in \{(1, 2), (2, 1)\} \land (\sigma(3), \sigma(4)) \in \{(3, 4), (4, 3)\}, \\
            (\sigma(1), \sigma(2)) &\in \{(3, 4), (4, 3)\} \land (\sigma(3), \sigma(4)) \in \{(1, 2), (2, 1)\}
        \end{align*}

        As we've seen in part (e), this subset is isomorphic to \(D_8\).
    \end{enumerate}
\end{enumerate}

Exercises 16, 17, pp. 65-66.

\begin{enumerate}
    \item [16.] 
    \begin{enumerate}
        \item[(a)] Since \(G\) is finite there can only be a finite amount of subgroups. In particular, there are only a finite amount of subgroups \(\{H_i\}_{i = 1}^n\) containing \(H\). Then any chain \(H \le H_{i_1} \le H_{i_2} \le \dots \le H_{i_k} \le G\) is finite, and we may prescribe \(H_{i_k}\) as the maximal subgroup containing \(H\).
        
        \item[(b)] Suppose \(\langle r \rangle \le K\). Then \(|\langle r \rangle| \le |K|\) while \(|K| \mid |G|\). But \(\langle r \rangle\) has order \(n\) and \(G\) has order \(2n\). Hence \(|K|\) can only be \(n\), in which case \(H = K\), or \(2n\), in which case \(K = G\). This is exactly the definition of \(H\) being maximal, as desired.
         
        \item[(c)] The order of \(x^p\) is \(n / p\), so \(|\langle x^p \rangle| = n / p\). If \(K\) contains \(\langle x^p \rangle\), then \(n/p \le |K| \implies n / |K| \le p\) while \(|K| \mid n \implies n / |K| = a\) for some \(a \in \mathbb Z\). But the only possible factors of \(p\) are \(1\) and \(p\), and \(|K| \neq n\), so we must have \(a = p\). Because their orders are equal and one is a subset of the other, \(K = \langle x^p \rangle\). Hence \(\langle x^p \rangle\) is maximal.
    \end{enumerate}
    \item [17.] 
    \begin{enumerate}
        \item[(a)] The chain \(\mathcal C\) is a set of subgroups \(\{H_i\}_{i \in \mathcal I}\) on a total order \(\mathcal I\) such that \(H_i \le H_j\) for all \(i \le j\). 
        
        We first show that if 
        \(x, y \in \bigcup_{i \in \mathcal I} H_i = H\), then \[xy \in \bigcup_{i \in \mathcal I} H_i = H.\]

        Since \(x \in H\), we have \(x \in H_i\) for some \(i \in I\). Similiarly \(y \in H_j\) for some \(j \in \mathcal I\). Furthermore, \(I\) is a total order so either \(i \le j\) or \(i \ge j\). Without loss of generality assume that \(i \le j\), since we could just swap the labels if instead \(j \le i\). Then \(H_i \le H_j\), so \(x \in H_i \le H_j\) and \(y \in H_j\) imply \(xy \in H_j \le H\).

        The other subgroup axioms are straightforward: \(e \in H\) since every \(H_i\) is a subgroup. For any \(x \in H\), \(\exists i, x \in H_i \implies x^{-1} \in H_i \le H\). 
        
        Hence \(H\) is a subgroup of \(G\).

        \item[(b)] Assume for the sake of contradiction that \(H\) is \textit{not} a proper subgroup, i.e. \(H = G\). Then each \(g_i\) must lie in some \(H_{\alpha_i}\). There are only finite \(g_i\), therefore we can compute the finite maximum \(\max (\alpha_i) = \alpha_j\) for some fixed \(j\). Then \(H_{\alpha_j}\) is both in \(\mathcal C\) and contains each \(g_i\). 
        Then \(\langle g_1, \dots, g_n\rangle \subset H_{\alpha_j}\). But \(\langle g_1, \dots, g_n\rangle = G\)! So \(H_{\alpha_j}\) is not proper, contradicting our assuptions about \(\mathcal C\).
          
        \item[(c)] Part (b) shows that for any chain \(\mathcal C\), the union of all subgroups in the chain \(H\) is an upper bound on \(\mathcal C\) that is proper. In other words, \(H \in \mathcal S\), and hence we may apply Zorn's lemma to deduce that \(\mathcal S\) contains at least one maximal element. This concludes the proof.
    \end{enumerate}
\end{enumerate}

Exercises 1, 18, 24, 40, 41 pp. 85-89.

\begin{theorem}[1]
    Let \(\varphi : G \to H\) be a homomorphism and let \(E\) be a subgroup of \(H\). Prove that \(\varphi^{-1}(E) \le G\). If \(E \trianglelefteq H\), then \(\varphi^{-1}(E) \normal G\). Deduce that \(\ker \varphi \normal G\).
\end{theorem}

\begin{proof}
    Part 1: We show that \(\varphi^{-1}(E)\) is a subgroup with the subgroup property. Suppose \(g, h \in \varphi^{-1}(E)\). Then by definition \(\varphi(g), \varphi(h) \in E\), and since \(E\) is a subgroup, we have in particular \(\varphi(h)^{-1} = \varphi(h^{-1}) \in E\). Thus \(\varphi(g)\varphi(h^{-1}) = \varphi(gh^{-1}) \in E\). By definiton this means \(gh^{-1} \in \varphi^{-1}(E)\), which proves that \(\varphi^{-1}(E)\) is indeed a subgroup.

    Part 2: Now suppose that \(E \normal H\). To show that \(\varphi^{-1}(E) \normal G\), we have to prove \(gng^{-1} \in \varphi^{-1}(E)\) for all \(n \in \varphi^{-1}(E)\) and \(g \in G\). By definition, \(\varphi(n) \in E\), and also the normality of \(E\) implies \(\varphi(g)\varphi(n)\varphi(g)^{-1} \in H\). Applying the properties of homomorphisms, we have can deduce \(\varphi(gng^{-1}) \in H\), and so \(gng^{-1} \in \varphi^{-1}(E)\). Hence \(\varphi^{-1}(E)\) is normal in \(G\).

    Part 3: Immediately from part 2, since \(\{e\} \normal H\) and by definition \(\ker \varphi = \varphi^{-1}(e)\), we have \(\ker \varphi \normal G\).
\end{proof}

\begin{theorem}[18]
    Let \(G\) be the quasidihedral group of order 16:
    \[G = \langle \sigma \tau \mid \sigma^8 = \tau^2 = 1, \sigma \tau = \tau \sigma^3\rangle\]
    and let \(\overline{G} = G / \langle \sigma^4 \rangle\) be the quotient of \(G\) by the subgroup generated by \(\sigma^4\) (this subgroup is the center of \(G\), hence is normal).
\end{theorem}

\begin{proof}
    We do each part in this proof:
    \begin{enumerate}
        \item[(a)] The subgroup \(\langle \sigma^4 \rangle\) has order 2, so Lagrange's theorem implies that \(|\overline{G}| = |G| / |\langle \sigma^2 \rangle| = 16 / 2 = 8\).
 
        \item[(b)] We have \(\overline{G} = \{\overline{\tau}^a\overline{\sigma}^b \mid a = 0, 1, \hspace{2mm} b = 0, 1, 2, 3\}\). This gives 8 elements which cannot be further reduced with the rules \(\overline{\tau}^2 = \overline{\sigma}^4 = 1\) and \(\sigma \tau = \tau \sigma^3\). Hence these must \textit{exactly} be the elements of \(\overline{G}\).
 
        \item[(c)] Let \(x = \overline{\tau}\) and \(y = \overline{\sigma}\). The orders can be computed pretty easily:
        \begin{align*}
            x^0 y^0 &= 1 \implies |x^0y^0| = 1 \\
            x^1 y^0 &= 1 \implies |x^1y^0| = |x| = 2 \\
            x^0 y^1 &= 1 \implies |x^0y^1| = |y| = 4 \\
            x^1 y^1 &= 1 \implies xyxy = xxy^3y = 1 \implies |x^1y^1| = 2 \\
            x^0 y^2 &= 1 \implies |x^0y^2| = 2 \\
            x^1 y^2 &= 1 \implies |x^1y^2| = 2 \\
            x^0 y^3 &= 1 \implies |x^0y^3| = |y^{-1}| = 4 \\
            x^1 y^3 &= 1 \implies |x^1y^3| = |xy^{-1}| = 2
        \end{align*}

        \item[(d)] Again let \(x = \overline{\tau}\) and \(y = \overline{\sigma}\). Then 
        \begin{align*}
            yx &= xy^3 \\
            xy^{-2}x &= xy^2x = xyyx = xyxy^3 = xxy^3y^3 = y^2 \\
            x^{-1}y^{-1}xy &= xy^3xy = xy^2 yxy = xy^2 xy^3 y = xy^2x = y^2
        \end{align*}
        
        \item[(e)] Consider the map \(\varphi\) such that \(x \mapsto s\) and \(y \mapsto r\). Then, looking at (c), clearly \(x\) and \(y\) interact in the same way \(s\) and \(r\) do hence \(\overline{G} \cong D_8\).
    \end{enumerate}
\end{proof}

\begin{theorem}[24]
    Prove that if \(N \normal G\) and \(H\) is any subgroup of \(G\) then \(N \cap H \normal H\).
\end{theorem}

\begin{proof}
    Suppose \(N \normal G\) and \(H \le G\). Let \(n \in H \cap N\) and \(h \in H\). So \(h \in G\) and \(n \in N\), for which we duduce that \(hnh^{-1} \in N\). Also \(n \in H\), and so \(hnh^{-1} \in H\). Hence \(hnh^{-1} \in N \cap H\). Hence \(N \cap H \normal H\). 
\end{proof}

\begin{theorem}[40]
    Let \(G\) be a group, let \(N\) be a normal subgroup of \(G\) and let \(\bar{G} = G / N\). Prove that \(\bar x\) and \(\bar y\) commute in \(\bar G\) if and only if \(x^{-1}y^{-1}xy \in N\).
\end{theorem}

\begin{proof}
    \((\implies)\): If \(x^{-1}y^{-1}xy \in N\), then \((x^{-1}y^{-1}xy)N = N \implies xyN = Nyx\). But \(N\) is normal, so we can swap the left and right cosets. Thus \(xN yN = xyN = N yx = yx N = yN xN\), as desired.

    \((\coimplies)\): If \(xN yN = yN xN\), then we can just run the argument backwards:
    \[xyN = yxN \implies x^{-1}y^{-1}xyN = N \implies x^{-1}y^{-1}xy \in N.\]
\end{proof}

\begin{theorem}[41]
    Let \(G\) be a group. Prove that \(N = \langle x^{-1}y^{-1}xy \mid x, y \in G \rangle\) is a normal subgroup of \(G\) and \(G/N\) is abelian. 
\end{theorem}

\begin{proof}
    \textit{\(N\) is a normal subgroup of \(G\)}: Let \(\varphi_g(n) = g^{-1}ng\). Note that conjugation by \(g\) is a homomorphism. Let \(n \in N\), which will have the form \(a_1^{\epsilon_1} a_2^{\epsilon_2} \dots a_n^{\epsilon_n}\), where each \(a_i = x^{-1}y^{-1}xy\) for some \(x, y \in G\) and \(\epsilon_i = \pm 1\). Now we want to show that \(g^{-1}ng = \varphi_g(n) \in N\) for any \(g \in G\). Since \(\varphi_g\) is a homomorphism, we have 
    \[\varphi_g(n) = \varphi_g(a_1)^{\epsilon_1} \varphi_g(a_2)^{\epsilon_2} \dots \varphi_g(a_n)^{\epsilon_n}.\]
    Because \(N\) is a subgroup, it suffices now to prove that each \(\varphi_g(a_i) \in N\). We have \[\varphi_g(a_i) = \varphi_g(x^{-1}y^{-1}xy) = \varphi_g(x^{-1}y^{-1}xy) = \varphi_g(x)^{-1}\varphi_g(y)^{-1}\varphi_g(x)\varphi_g(y).\]
    The LHS is of the form \(x'^{-1}y'^{-1}x'y'\) for \(x' = \varphi_g(x)\) and \(y' = \varphi_g(y)\), so it must be in \(N\). Hence \(\varphi_g(a_i) \in N\). By extension, \(\varphi_g(n) \in N\). Therefore \(N\) is normal.
\end{proof}

\begin{proof}
    \textit{\(N\) is abelian}: By Exercise 40 we have that \(\bar x\) and \(\bar y\) commute in \(G/N\) is and only if \(x^{-1}y^{-1}xy \in N\). But this implies that \(\bar{x^{-1}y^{-1}xy} = 1\) in \(G / N\). Rearranging gives \(\bar x \bar y = \bar y \bar x\), as desired. 
\end{proof}

Exercise 4, pp. 111.

\begin{theorem}[4]
    Prove that \(S_n = \langle (12), (123\dots n)\rangle\) for all \(n \ge 2\).
\end{theorem}

\begin{proof}
    It suffices to show that every transposition can be generated from \(x = (12)\) and \(y = (123 \dots n)\). Indeed, direct calculation shows that we can obtain transpositions of the form \((i, i+1)\) by conjugating \(y^{i - 1} x y^{1-i}\):
    \begin{align*}
        &i \mapsto 1 \mapsto 2 \mapsto i + 1 \\
        &i + 1 \mapsto 2 \mapsto 1 \mapsto i \\
        &j \mapsto j - i + 1 \notin \{1, 2\} \mapsto j \hspace{4mm} \forall j \ne i, i + 1 \\
    \end{align*}
    
    Next, transpositions of the form \(1i\) can be generated recursively using \((1, i + 1) = (1i)(i, i + 1)(1i)\), starting with the base case \((12)\). Finally, general transpositions of the form \((ij)\) can be computed with \((1i)(1j)(1i)\), which clearly maps \(i \mapsto j\) and \(j \mapsto i\). And with all the transpositions, we're done.
\end{proof}

\end{document}