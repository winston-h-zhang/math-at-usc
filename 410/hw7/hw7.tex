%%%%%%%%%%%%%%%%%%%%%%%%%%%%%%%%%%%%%%%%%%%%%%
%%                                          %%
%% USE THIS FILE TO SUBMIT YOUR SOLUTIONS   %%
%%                                          %%
%% You must have the usamts.tex file in     %%
%% the same directory as this file.         %%
%% You do NOT need to submit this file or   %%
%% usamts.tex with your solutions.  You     %%
%% only need to submit the output PDF file. %%
%%                                          %%
%% DO NOT ALTER THE FILE usamts.tex         %%
%%                                          %%
%% If you have any questions or problems    %%
%% using this file, or with LaTeX in        %%
%% general, please go to the LaTeX          %%
%% forum on the Art Of Problem Solving      %%
%% web site, and post your problem.         %%
%%                                          %%
%%%%%%%%%%%%%%%%%%%%%%%%%%%%%%%%%%%%%%%%%%%%%%

%%%%%%%%%%%%%%%%%%%%%%%%%%%%%%%%%%%%%%%%%%%%
%% DO NOT ALTER THE FOLLOWING LINES
\documentclass[12pt]{article}
\usepackage{amsmath,amssymb,amsthm,amsfonts,tabularx}
\usepackage[pdftex]{graphicx}
\usepackage[dvipsnames]{xcolor}
\usepackage{fancyhdr}
\usepackage{parskip}
\usepackage[shortlabels]{enumitem}
\pagestyle{fancy}
\usepackage{setspace}
\newcommand{\realname}[1]{\newcommand{\printrealname}{#1}}
\newcommand{\pset}[1]{\newcommand{\printpset}{#1}}
\newcommand{\mathclass}[1]{\newcommand{\printmathclass}{#1}}

%% Pagestyle setup
\setlength{\headheight}{0.75in}
\setlength{\oddsidemargin}{0in}
\setlength{\evensidemargin}{0in}
\setlength{\voffset}{-.5in}
\setlength{\headsep}{10pt}
\setlength{\textwidth}{6.5in}
\setlength{\headwidth}{6.5in}
\setlength{\textheight}{8in}
\lhead{Math \printmathclass}
\chead{\Large \textbf{Homework \printpset}}
\rhead{\printrealname}
\rfoot{Page \thepage}
\renewcommand{\headrulewidth}{0.5pt}
\renewcommand{\footrulewidth}{0.3pt}
\setlength{\textwidth}{6.5in}
\renewcommand{\baselinestretch}{1}
\setenumerate[0]{label=(\alph*)}

\newtheorem*{prop}{Proposition}
\newtheorem*{corollary}{Corollary}
\newtheorem*{lemma}{Lemma}
\theoremstyle{remark}
\newtheorem*{defn}{Definition}

\newtheoremstyle{named}{}{}{}{}{\bfseries}{.}{.5em}{\thmnote{Problem #3}}
\theoremstyle{named}
\newtheorem*{theorem}{Theorem}

%% DO NOT ALTER THE ABOVE LINES
%%%%%%%%%%%%%%%%%%%%%%%%%%%%%%%%%%%%%%%%%%%%


%% If you would like to use Asymptote within this document (which is optional), 
%% you can find out how at the following URL:
%%
%%   http://www.artofproblemsolving.com/Wiki/index.php/Asymptote:_Advanced_Configuration
%%
%% As explained there, you will want to uncomment the line below.  But be
%% sure to check the website because there are several other steps that must 
%% be followed.
%% \usepackage{asymptote}

%% Enter your real name here
%% Example: \realname{David Patrick}
\realname{Hanting Zhang}
\pset{7}
\mathclass{410}

\newcommand{\todo}{\textcolor{red}{\textbf{TODO} }}
\renewcommand{\a}{\alpha}
\renewcommand{\b}{\beta}
\newcommand{\Z}{\mathbb Z}
\renewcommand{\bf}{\mathbf}
\renewcommand{\implies}{\Rightarrow}
\newcommand{\coimplies}{\Leftarrow}
\renewcommand{\em}{\varnothing}

\begin{document}

Exercises 6, 13, 14, 21, 25, 26, pp. 230-233.

\textbf{Problem 6.} Are the following subrings of the ring of all functions from the closed interval \([0, 1]\) to \(\mathbb R\).
\begin{enumerate}
    \item the set of all functions \(f(x)\) such that \(f(q) = 0\) for all \(q \in \mathbb Q \cap [0, 1]\): 
    
    \textbf{Yes.} We check that these functions form an nonempty set closed under subtraction and multiplication. Nonemptiness is immediate. 

    If \(f, g\) are two functions in this set, then for any \(q \in \mathbb Q \cap [0, 1]\), \((f - g)(q) = f(q) - g(q) = 0 - 0 = 0\). Hence \(f - g\) is in this set. Futhermore, \((fg)(q) = f(q)g(q) = 0 * 0 = 0\), so \(fg\) is in this set. 

    \item the set of all polynomial functions: 
    
    \textbf{Yes.} Let \(p, q\) be polynomials. Clearly \(p - q\) and \(pq\) are still polynomials, so polynomial functions are closed under subtraction and multiplication.
    
    \item the set of all functions which have only finite number of zeros, together with the zero function: 
    
    \textbf{No.} Let \(f(x) = 1\) be the constant function and \(g(x) = 1\) if \(x \le 1/2\) and \(g(x) = -1\) if \(x > 1/2\). Both \(f\) and \(g\) have a finite amount of zeros. However, \((f + g)(x) = 2\) if \(x \le 1/2\) else \((f - g)(x) = 0\), which has an infinite amount of zeros. Hence this set is not closed under \(+\).
    
    \item the set of all functions which have an infinite number of zeros: 
    
    \textbf{No.} Let \(f(x) = 0\) if \(x \le 1/2\) else \(1\) and let \(g(x) = 1\) if \(x \le 1/2\) else \(0\). Then it is clear that \((f + g)(x) = 1\), which has no zeros. Hence this set is not closed under \(+\). 
    
    \item the set of all functions \(f\) such that \(\lim_{x \to 1^-} f(x) = 0\): 
    
    \textbf{Yes.} We have from analysis that 
    \[\lim_{x \to 1^-} f(x) - \lim_{x \to 1^-} g(x) = \lim_{x \to 1^-} (f(x) - g(x))\]
    and 
    \[\lim_{x \to 1^-} f(x) * \lim_{x \to 1^-} g(x) = \lim_{x \to 1^-} (f(x)g(x)).\]
    Thus if \(\lim_{x \to 1^-} f(x) = 0\) and \(\lim_{x \to 1^-} g(x) = 0\), then \[\lim_{x \to 1^-} f(x) - \lim_{x \to 1^-} g(x) = \lim_{x \to 1^-} f(x) * \lim_{x \to 1^-} g(x) = 0,\]
    as desired.
    
    \item the set of all rational linear combinations of the functions \(\sin nx\) and \(\cos nx\), where \(m, n \in \{0, 1, 2, \dots\}\): 
    
    \textbf{Yes.} Let $f(x)$ and $g(x)$ be rational linear combinations of \(\sin nx\) and \(\cos nx\), i.e.
    $$\sum_{i = 0}^n (a_i \sin ix + b_i \cos ix) \hspace{2mm} \text{and} \hspace{2mm} \sum_{j = 0}^m (c_j \sin jx + d_j \cos jx).$$
    Then we have $f(x) - g(x)$ equal to:
    $$\sum_{k = 0}^{\max(n, m)} (a_k + c_k) \sin kx + (b_k + d_k) \cos kx),$$
    which is another rational linear combinations of \(\sin nx\) and \(\cos nx\).
    
    I won't type out $f(x)g(x)$ because the calculation is so messy, but essentially it boils down to a rational linear combination of the functions $\sin nx \sin mx$, $\cos nx \cos mx$, $\cos nx \sin mx$, $\sin nx \cos mx$. But these can be further broken down into \(\sin nx\) and \(\cos nx\) by applying sum to product formulas
    \begin{align*}
        2 \sin \a \cos \b &= \sin(\a - \b) + \sin (\a + \b) \\
        2 \cos \a \cos \b &= \cos(\a - \b) + \cos (\a + \b) \\
        2 \cos \a \sin \b &= \sin(\a + \b) - \sin (\a - \b) \\
        2 \sin \a \sin \b &= \cos(\a - \b) - \cos (\a + \b).
    \end{align*}
    Thus $f(x) g(x)$ can be written rational linear combinations of \(\sin nx\) and \(\cos nx\), and thus this set is a subring.
\end{enumerate}
And we're done.
\newline

\begin{theorem}[13]
    An element \(x\) in \(R\) is called \textit{nilpotent} if \(x^m = 0\) for some \(m \in \Z^+\).
    \begin{enumerate}
        \item Show that if \(n = a^kb\) for some integers \(a\) and \(b\) then \(\overline{ab}\) is a nilpotent element of \(\Z / n \Z\). 
        \item If \(a \in \Z\) is an integer, show that the element \(\overline{a} \in \Z / n \Z\) is nilpotent if and only if every prime divisor of \(n\) is also a prime divisor of \(a\). 
        In particular, determine the nilpotent elements of \(\Z / 72 \Z\) explicitly.
        \item Let \(R\) be the ring of functions from a nonempty set \(X\) to a field \(F\). Prove that \(R\) contains no nonzero nilpotent elements.
    \end{enumerate}
\end{theorem}

\begin{proof}
    We proceed with each separately: 
    \begin{enumerate}
        \item Let \(n = a^kb\). If \(k = 0\), then \(n = b\) so \(\overline{ab} = \overline{an} = \overline{0}\) is trivially nilpotent. 
        Otherwise, if \(k \ge 1\), we have \(\overline{ab}^k = \overline{a^kb^k} = \overline{a^k b * b^{k - 1}} = \overline{nb^{k - 1}} = \overline{0}\), as desired. Thus \(\overline{ab}\) is nilpotent.

        \item (\(\Rightarrow\)): Suppose every prime divisor of \(n\) is a divisor of \(a\), i.e. for any prime \(p\), \(p \mid n \Rightarrow p \mid a\). By the fundamental theorem of arithmetic, write \(n = p_1^{e_1} p_2^{e_2} \dots p_k^{e_k}\), where \(p_i\) are primes and \(e_i > 0\). Then \(p_1 \dots p_k \mid a\) by our hypothesis. Write \(a = p_1 \dots p_k * b\) and \(e = \max(e_1, \dots, e_k)\). Then 
        \begin{align*}
            a^e = (p_1 \dots p_k * b)^e = p_1^e \dots p_k^e * b^e.
        \end{align*}
        Since \(e_i \le e\), we can safely pull out each factor of \(p_i^{e_i}\) to have \(a^e = n * p_1^{e - e_1} \dots p_k^{e - e_k} * b^e\). Thus \(\overline{a}\) is nilpotent:
        \[\overline{a}^e = \overline{n * p_1^{e - e_1} \dots p_k^{e - e_k} * b^e} = \overline{0}.\]

        (\(\Leftarrow\)): If \(a\) is a nilpotent element of \(\Z /n\Z\), then \(a^m = 0\) for some \(m \in \Z^+\). By the fundamental theorem of arithmetic, let \(a = p_1^{d_i} \dots p_k^{d_k}\) for primes \(p_i\) and \(d_i > 0\). Then \(a^m = 0\) in \(\Z / n\Z\) implies \(n \mid a^m\) in \(\Z\). Thus if \(p\) is any prime divisor of \(n\), we have \(p \mid n\) implies 
        \[p \mid n * b = a^m = p_1^{d_i m} \dots p_k^{d_k m} \Rightarrow \exists i, p \mid p_i^{d_i m} \Rightarrow \exists i, p = p_i.\]
        Thus \(p\) is also a prime divisor of \(a\), and our proof is complete.
        
        We may find the nilpotent elements of \(\Z / 72\Z\) using the above parts:
        $$\{2 \cdot 3, 2^2 \cdot 3, 2^3 \cdot 3, 2 \cdot 3^2, 2^2 \cdot 3^2\} = \{6, 12, 24, 18, 36\}.$$
        \item We claim that any integral domain \(D\) has no nonzero nilpotent elements. Indeed, suppose \(x \in D\) in nilpotent. Let \(m \in \Z^+\) be the minimum number such that \(x^m = 0\). If \(m = 1\), then \(x = x^1 = 0\), so we're done. Otherwise, if \(m \ge 2\), then write \(x^m = x x^{m - 1} = 0\). By the minimality of \(m\), we know that \(x^{m - 1} \neq 0\). Since \(D\) is an integral domain and therefore has no nonzero zero-divisors, we must have \(x = 0\).
        
        Now we return to the problem. Since \(F\) is a field it is also an integral domain. Then \(F\) has no nonzero nilpotent elements. Let \(f \in R\) be nilpotent. Then \(f^m = 0\) for some \(m \in Z^+\). Thus for any \(x \in X\), \(f(x)^m = 0 \Rightarrow f(x) = 0\), i.e. \(f = 0\), as desired.
    \end{enumerate}
\end{proof}

\begin{theorem}[14]
    Let \(x\) be a nilpotent element of the commutative ring \(R\).
    \begin{enumerate}
        \item Prove that \(x\) is either zero or a zero divisor.
        \item Prove that \(rx\) is nilpotent for all \(r \in R\).
        \item Prove that \(1 + x\) is a unit in \(R\). 
        \item Deduce that the sum of a nilpotent element and a unit is a unit.
    \end{enumerate}
\end{theorem}

\begin{proof}
    Let \(x\) be nilpotent and \(x^m = 0\) for some minimal number \(m \in \Z^+\).
    \begin{enumerate}
        \item Either \(x = 0\) or \(x^m = xx^{m - 1} = 0\). In the second case, the minimality of \(m\) guarantees that \(x^{m - 1} \neq 0\); thus \(x\) is a zero-divisor. 
         
        \item We may rewrite \((rx)^m = r^mx^m\) because they commute in \(R\). Thus \((rx)^m = r^mx^m = r^m*0 = 0\), as desired.
        
        \item We motivate ourselves with the well-known factorization of \(1 - x^m\):
        \[1 - x^m = (1 + x)\left(\sum_{k = 0}^{m - 1}(-1)^kx^k\right).\] 
        Since \(x^m = 0\), the LHS is just \(1\), showing that \(1 + x\) is a unit.

        \item Let \(u \in R\) be a unit. Then \(u + x = u(1 + u^{-1}x)\). Part (b) gives that \(u^{-1}x\) is nilpotent; then part (c) gives that \(1 + u^{-1}x\) is a unit. The product of units is a unit, thus \(u(1 + u^{-1}x) = u + x\) is a unit. 
    \end{enumerate}
\end{proof}

\begin{theorem}[21]
    Let \(X\) be any nonempty set.
    \begin{enumerate}
        \item Prove that \(\mathcal P (X)\) is a ring under the addition and multiplication given by the textbook.
        \item Prove that this ring is commutative, has an identity and is a Boolean ring.
    \end{enumerate}
\end{theorem}

\begin{proof}
    We proceed with each separately:
    \begin{enumerate}
        \item In the following let \(A, B \subseteq X\).
        \begin{enumerate}[1.]
            \item \textit{\((X, +)\) is a abelian group}:
            
            Closure: \((A - B \cap (B - A)\) is another subset of \(X\), so addition is closed.

            Abelian: We have \(A + B = (A - B) \cap (B - A) = (B - A) \cap (A - B) = B + A\).
            
            Identity: \(\em\), because \(\em + A = A + \em = (A - \em) \cup (\em - A) = A \cup \em = A\).

            Associativity: This is painful to check but it is true.

            \item \textit{\((X, \times)\) is a monoid}: 
            
            Closure: \(A \cap B\) is a subset of \(X\), so multiplication is closed. 

            Associativity: This is painful to check but it is true.

            \item \textit{Distribution laws}: Let \(C \subseteq X\). Then 
            \begin{align*}
                C \times (A + B) &= C \cap [(A - B) \cup (B - A)] \\
                &= C \cap (A - B) \cup C \cap (B - A) \\
                &= (C \cap A - C \cap B) \cup (C \cap B - C \cap A) \\
                &= (C \times A - C \times B) \cup (C \times B - C \times A) \\
                &= C \times A + C \times B.
            \end{align*}
        \end{enumerate} 
        Thus \(\mathcal P(X)\) is a ring.

        \item Clearly \(A^2 = A \cap A = A\) for any \(A \in \mathcal P(X)\). Hence \(\mathcal P(X)\) is a boolean ring. By Exercise 15 in this section, every boolean ring is commutative. Thus \(\mathcal P(X)\) is commutative. We have \(X\) is the identity, since \(XA = X \cap A = A \cap X = AX = A\) for all \(A \in \mathcal P(X)\). 
    \end{enumerate}
\end{proof}

\begin{theorem}[25]
    Let \(I\) be the ring of integral Hamilton Quaterions and define 
    \[N : I \to \Z \hspace{2mm} \text{by} \hspace{2mm} N(a + bi + cj + dk) = a^2 + b^2 + c^2 + d^2\]
    (the map \(N\) is called the \textit{norm}).
    \begin{enumerate}
        \item Prove that \(N(\alpha) = \alpha\overline{\alpha}\) for all \(\alpha \in I\), where if \(\alpha = a + bi + cj + dk\) then \(\overline{\alpha} = a - bi - cj - dk\).
        \item Prove that \(N(\alpha\beta) = N(\alpha)(\beta)\) for all \(\a, \b \in I\).
        \item Prove that an element of \(I\) is a unit if and only if it has norm \(+1\). Show that \(I^\times\) is isomprphic to the quaterion group of order 8.
    \end{enumerate}
\end{theorem}

\begin{proof}
    We proceed with each separately:
    \begin{enumerate}
        \item We have $\a \overline{\a}$ is equal to:
        $${\displaystyle {
        \begin{alignedat}{4}
        &aa&&+bb&&+cc&&+dd\\
        {}+{}(&ab&&-ba&&+cd&&-dc) {i} \\
        {}+{}(&ac&&-bd&&-ca&&+db) {j} \\
        {}+{}(&ad&&+bc&&-cb&&-da) {k} 
        \end{alignedat}}},$$
        from which it is easy to see that everything cancels, leaving $a^2 + b^2 + c^2 + d^2$. Thus $N(\a) = \a \overline{\a}$.
        \item From part (a) we know that $N(\a\b) = \a \b \overline{\a\b} = \a\b \overline{\b} \overline{\a} = \a N(\b) \overline{\a}$. But $N(\a)$ is an integer, so it commutes with $\overline{\a}$; hence we can pull it out to see that $N(\a\b) = \a N(\b) \overline{\a} = \a \overline{\a} N(\b) = N(\a) N(\b)$, as desired.
        \item ($\Rightarrow$): If $\a \in I$ has unit norm, then $a^2 + b^2 + c^2 + d^2 = 1$. But we know that $a^2, b^2, c^2, d^2 \ge 0$ and $a, b, c, d$ are integers, so the only solutions occur when exactly one of $a, b, c, d$ is equal to $\pm 1$ and everyone else is zero. Thus there are 8 possible values of $\a$: $\pm a, \pm bi, \pm cj, \pm dk$. All of these are clearly units.
        
        ($\Leftarrow$): If $\a \in I$ is a unit, then its inverse can be written as $ \overline{\a} / (a^2 + b^2 + c^2 + d^2)$. The inverse must have integer coefficients, so we need $a / (a^2 + b^2 + c^2 + d^2) \in \Z$. This can only occur when $a^2 \le (a^2 + b^2 + c^2 + d^2) \le a \Rightarrow a = 0, \pm 1$. Thus we see that the only possible values of $N(\a)$ are 0 and 1. It can't be 0 because we assumed that $\a$ is a unit; hence we conclude $N(\a) = 1$.
    \end{enumerate}
\end{proof}

\begin{theorem}[26]
    Let \(K\) be a field and \(\nu : K^\times \to \Z\) a discrete valuation on \(K\). Let \(R\) be the valuation ring of \(\nu\).
    \begin{enumerate}
        \item Prove that \(R\) is a subring of \(K\) which contains the identity.
        \item Prove that for each nonzero element \(x \in K\) either \(x\) or \(x^{-1}\) is in \(R\). 
        \item Prove that an element \(x\) is a unit of \(R\) if and only if \(\nu(x) = 0\).
    \end{enumerate}
\end{theorem}

\begin{proof}
    We proceed with each separately:
    \begin{enumerate}
        \item Note that $\nu(1) = \nu(1 \cdot 1) = \nu(1) + \nu(1) \Rightarrow \nu(1) = 0$. Thus the identity is in $R$.
        \item Since we have $\nu(x) + \nu(x^{-1})\nu(xx^{-1}) = \nu(1) = 0$, either one or the other is non-negative. Thus one must be in $R$.
        \item ($\Rightarrow$): If $\nu(x) = 0$, then $\nu(x^{-1}) = \nu(x^{-1}) + \nu(x) - \nu(x) = \nu(xx^{-1}) - \nu(x) = \nu(1) - \nu(x) = 0 - 0 = 0$. Thus $x^{-1} \in R$.
        
        ($\Leftarrow$): If $x$ is a unit in $R$, then $x^{-1}$ is in $R$, so $\nu(x), \nu(x^{-1}) \ge 0$. BUt we also know that $\nu(x) + \nu(x^{-1}) = 0$, so the only values they can be is $\nu(x) = \nu(x^{-1}) = 0$.
    \end{enumerate}
\end{proof}

Exercises 3, 4, 10, 11, pp. 238-239.

\begin{theorem}[3]
    Let \(R[[x]]\) be the \textit{formal power series} of \(R\) in \(x\). Define addition and multiplication as the textbook does.
    \begin{enumerate}
        \item Prove that \(R[[x]]\) is a commutative ring with 1.
        \item Show that \(1 - x\) is a unit in \(R[[x]]\) with inverse \(1 + x + x^2 + \cdots\).
        \item Prove that \(\sum_{n = 0}^\infty a_n x^n\) is a unit in \(R[[x]]\) if and only if \(a_0\) is a unit in \(R\).
    \end{enumerate}
\end{theorem}

\begin{proof}
    We proceed with each separately:
    \begin{enumerate}
        \item This proof is largely an extension of the proof that the power series \(R[x]\) is a commutative ring. There is not much change in the fact that we may now have infinite nonzero indices. 
        
        \item We have 
        \begin{align*}
            (1 - x) \left(\sum_{n = 0}^\infty x^n\right) &= \sum_{n = 0}^\infty x^n - \sum_{n = 1}^\infty x^n \\
            &= (1 + x + x^2 + \dots) - (x + x^2 + \dots) \\
            &= 1.
        \end{align*}
        One may convince themselves that the sums telescope to ``infinity,'' so the only term left is 1.
        \item Let \(f(x) = \sum_{i = 0}^\infty a_i x^i\). We want to find \(g(x) = \sum_{j = 0}^\infty b_j x^j\) such that \(f(x)g(x) = 1\). Expanding the product, we have
        \begin{align*}
            f(x)g(x) &= \left(\sum_{i = 0}^\infty a_i x^i\right)\left(\sum_{j = 0}^\infty b_j x^j\right) = \sum_{k = 0}^\infty \left(\sum_{i = 0}^k a_i b_{k - i}\right) x^k.
        \end{align*}
        Comparing the coefficients, we see that \(a_0 b_0 = 1\) and \(\sum_{i = 0}^k a_i b_{k - i} = 0\) for all \(k \ge 1\). Hence if \(f(x)\) is a unit, then \(a_0\) is a unit in \(R\). 
        
        Conversely, suppose \(a_0\) is a unit in \(R\). We proceed to constuct \(b_k\) for each \(k \ge 1\) by recursion. We may rewrite each of the remaining equations as \(a_0 b_k = - \sum_{i = 1}^k a_i b_{k - i}\); multiplying by \(b_0\) on both sides gives
        \[b_k = -b_0\sum_{i = 1}^k a_i b_{k - i}.\]
        Indeed, assume for the sake of strong induction that \(b_i\) is known for all \(i < k\). Then clearly we can construct \(b_k\). The base case \(k = 0\) holds with \(b_0 = a_0^{-1}\). Thus induction yields a solution for \(f(x)g(x) = 1\). Therefore \(f(x)\) is a unit. 
    \end{enumerate}
\end{proof}

\begin{theorem}[4]
    Prove that if \(R\) is an integral domain then the ring of formal power series \(R[[x]]\) is also an integral domain.
\end{theorem}

\begin{proof}
    Let \(f(x) = \sum_{i = 0}^\infty a_i x^i\) and \(g(x) = \sum_{j = 0}^\infty b_j x^j\). We want to show that if \(f(x)g(x) = 0\), then either \(f(x) = 0\) or \(g(x) = 0\). 

    Expanding the product, again we have 
    \begin{align*}
        f(x)g(x) &= \left(\sum_{i = 0}^\infty a_i x^i\right)\left(\sum_{j = 0}^\infty b_j x^j\right) = \sum_{k = 0}^\infty \left(\sum_{i = 0}^k a_i b_{k - i}\right) x^k.
    \end{align*}
    We see that each coefficient \(\sum_{i = 0}^k a_i b_{k - i}\) must be zero. In particular, \(a_0 b_0 = 0\); \(R\) is an integral domain, so \(a_0 = 0\) or \(b_0 = 0\). Without loss of generality assume that \(b_0 = 0\). Note here that we may always assume \(a_0 \neq 0\) by finding the smallest nonzero monomial \(a_i x^i\) in \(f(x)\) and factoring out \(x^i\) to write \(f(x) = x^i f'(x)\). As proving \(f(x)g(x) = 0\) is equivalent to \(f'(x)g(x) = 0\), we may restart the proof with \(f'(x)\) instead of \(f(x)\) to guarentee \(a_0' \neq 0\). 

    We proceed to show that \(b_k = 0\) for all \(k\) via induction. Assume for the sake of induction that for all \(i < k\), \(b_i = 0\). Rewrite the coefficient equations as \(a_0 b_k = - \sum_{i = 1}^k a_i b_{k - i}\). The terms of the RHS each have a term \(b_i\) for \(i < k\), so it collapses to zero. On the LHS, we know that \(a_0 \neq 0\), therefore \(b_k = 0\). Adding the base case \(b_0 = 0\) completes the induction. Thus we have shown that \(g(x) = 0\), and that \(R[[x]]\) is an integral domain.  
\end{proof}

\begin{theorem}[10]
    Consider the following elements of the integral group ring \(\Z S_3\):
    \[\a = 3(1, 2) - 5(2, 3) + 14(1, 2, 3) \hspace{2mm} \text{and} \hspace{2mm} \b = 6(1) + 2(2, 3) - 7(1, 3, 2)\]
    (where (1) is the identity of \(S_3\)). Compute the following elements:

    \textbf{(a)} \(\a + \b\), \textbf{(b)} \(2\a - 3\b\), \textbf{(c)} \(\a \b\), \textbf{(d)} \(\b\a\), \textbf{(e)} \(a^2\).
\end{theorem}

\begin{proof}
    Just apply the definitions given in the textbook. Note: I did the calculations on paper.
    \begin{enumerate}
        \item \(\a + \b = 6(1) + 3(1, 2) - 3(2, 3) + 14(1, 2, 3) - 7(1, 3, 2)\)
        \item $2\a - 3\b = -18(1) + 6(1, 2) - 22(2, 3) + 28 (1,2,3) + 21(1, 3, 2)$
        \item \(\a\b = -108 (1) + 81(1, 2) - 21(1, 3) - 30 (2, 3) + 90(1, 2, 3)\) 
        \item $\b\a = -108(1) + 18(1, 2) - 51(2, 3) + 63(1, 3) + 84(1, 2, 3) + 6(1, 3, 2)$
        \item $\a^2 = 34(1) + 70(1, 2) + 42(2, 3) + 112 (1, 3) - 15(1, 2, 3) + 181 (1, 3, 2)$
    \end{enumerate}
\end{proof}

\begin{theorem}[11]
    Repeat the preceding exercise under the assumption that the coefficients of \(\a\) and \(\b\) are in \(\Z / 3\Z\).
\end{theorem}

\begin{proof}
    We can just take the answers above and mod 3:
    \begin{enumerate}
        \item \(\a + \b = 2(1, 2, 3) + 2(1, 3, 2)\)
        \item $2\a - 3\b = (1, 2, 3) + (1, 3, 2)$
        \item \(\a\b = 0\) 
        \item $\b\a = 0$
        \item $\a^2 = 1(1) + 1(1, 2) + 1 (1, 3) + 1 (1, 3, 2)$
    \end{enumerate}
\end{proof}

Exercises 15, 17, 18, 19, 24, 26, pp. 247-251.

\begin{theorem}[15]
    Prove that the map \(\varphi : \mathcal P(X) \to R\) defined by \(A \mapsto \chi_A\) is a ring homomorphism, where \(\chi_A\) is the \textit{characteristic function} of \(A\). 
\end{theorem}

\begin{proof}
    Notice that for any sets $A, B \subseteq X$, we have $\{x \mid x \in A \text{ xor } x \in B\}$ equal to $(A - B) \cap (B - A)$. Rewrite $\{x \mid x \in A \text{ xor } x \in B\} = \{x \mid \chi_A(x) = 1 \text{ xor } \chi_B(x) = 1\}$, and notice that in $\Z / 2 \Z$, we can replace the xor with $+$, so $$(A - B) \cap (B - A) = \{x \mid \chi_A(x) + \chi_B(x) = 1\} \Rightarrow \varphi(A + B) = \varphi(A) + \varphi(B).$$
    
    For multiplication, we have $AB = A \cap B = \{x \mid \chi_A(x) = 1 \land \chi_B(x) = 1\}$. But $\chi_A(x) = 1 \land \chi_B(x) = 1 \iff \chi_A(x)\chi_B(x) = 1$, thus $\varphi(AB) = \varphi(A \varphi(B)$. 
    
    This proves that $\varphi$ is a ring homomorphism, as desired. 
\end{proof}

\begin{theorem}[17]
    Let \(R\) and \(S\) be nonzero rings with identity and denote their respective identities by \(1_R\) and \(1_S\). Let \(\varphi : R \to S\) be a nonzero homomorphism of rings. 
    \begin{enumerate}
        \item Prove that if \(\varphi(1_R) \neq 1_S\), then \(\varphi(1_R)\) is a zero divisor in \(S\). Deduce that if \(S\) is an integral domain then every ring homomorphism from \(R\) to \(S\) sends the identity of \(R\) to the identity of \(S\). 
        \item Prove that if \(\varphi(1_R) = 1_S\) then \(\varphi(u)\) is a unit in \(S\) and that \(\varphi(u^{-1}) = \varphi(u)^{-1}\) for each unit \(u \in R\). 
    \end{enumerate}
\end{theorem}

\begin{proof}
    We proceed with each separately:
    \begin{enumerate}
        \item We have \(\varphi(1_R) = \varphi(1_R 1_R) = \varphi(1_R)\varphi(1_R)\). Factoring gives \[0 = \varphi(1_R) - \varphi(1_R)\varphi(1_R) = \varphi(1_R)(1_S - \varphi(1_R)).\] If \(1_S \neq \varphi(1_R)\), then \(1_S - \varphi(1_R)\) is nonzero; thus \(\varphi(1_R)\) is a zero-divisor.
        
        From this we may deduce that if \(S\) is an integral domain, then we must instead have \(1_S - \varphi(1_R) = 0\). Thus \(1_S = \varphi(1_R)\).

        \item Suppose \(\varphi(1_R) = 1_S\) and let \(u \in R\) be a unit. Then \[\varphi(uu^{-1}) = \varphi(1_R) = 1_S = \varphi(u)\varphi(u^{-1}).\]
        Similarily, \(1_S = \varphi(u^{-1})\varphi(u)\). By definition then \(\varphi(u^{-1}) = \varphi(u)^{-1}\) and \(\varphi(u)\) is a unit.  
    \end{enumerate}
\end{proof}


\begin{theorem}[18]
    Let \(R\) be a ring.
    \begin{enumerate}
        \item If \(I\) and \(J\) are ideals of \(R\) prove that their intersection \(I \cap J\) is also an ideal of \(R\). 
        \item Prove that the intersection of an arbitrary nonempty collection of ideals is again an ideal of \(R\). 
    \end{enumerate}
\end{theorem}

\begin{proof}
    We proceed with each separately:
    \begin{enumerate}
        \item Let \(a \in I \cap J\) and \(r \in R\). Then because \(I\) and \(J\) are ideals, \(a \in I \Rightarrow ra \in I\) and \(a \in J \Rightarrow ra \in J\). Thus \(ra \in I \cap J\), as desired.
        \item Let \(\{I_\a\}_{\a \in A}\) be an arbitrary nonempty collection of ideals. Let \(a \in \bigcap_{\a \in A} I_\a\) and \(r \in R\). Then 
        \[\forall \a \in A, a \in I_\a \Rightarrow ra \in I_\a.\] 
        Thus \(ra \in \bigcap_{\a \in A}I_\a\) which proves that it is an ideal of \(R\). 
    \end{enumerate}
\end{proof}


\begin{theorem}[19]
    Prove that if \(I_1 \subseteq I_2 \subseteq \dots\) are ideals of \(R\) then \(\bigcup_{n = 1}^\infty I_n\) is an ideal of \(R\).
\end{theorem}

\begin{proof}
    Let \(a \in \bigcup_{n = 1}^\infty I_n\) and \(r \in R\). We have \(a \in I_m\) for some \(m \in \mathbb N\). Thus \(ra \in I_m \subseteq \bigcup_{n = 1}^\infty I_n\); hence \(\bigcup_{n = 1}^\infty I_n\) is an ideal of \(R\).
\end{proof}


\begin{theorem}[24]
    Let \(\varphi : R \to S\) be a ring homomorphism.
    \begin{enumerate}
        \item Prove that if \(J\) is an ideal of \(S\) then \(\varphi^{-1}(J)\) is an ideal of \(R\). Apply this to the special case when \(R\) is a subring of \(S\) and \(\varphi\) is the inclusion homomorphism to deduce that if \(J\) is an ideal of \(S\) then \(J \cap R\) is an ideal of \(R\). 
        \item Prove that if \(\varphi\) is surjective and \(I\) is an ideal of \(R\) then \(\varphi(I)\) is an ideal of \(S\). Give an example where this fails if \(\varphi\) is not surjective.
    \end{enumerate}
\end{theorem}

\begin{proof}
    We proceed with each separately:
    \begin{enumerate}
        \item Let \(a \in \varphi^{-1}(J)\) and \(r \in R\). We have \(\varphi(a) \in J\) and \(\varphi(r) \in S\), so since \(J\) is an ideal, \(\varphi(r)\varphi(a) = \varphi(ra) \in J\). Hence \(ra \in \varphi^{-1}(J)\), which proves that it is an ideal. 
        
        In the special case where \(R \subseteq S\) and \(\varphi = \iota\) is an inclusion, then \(\varphi^{-1}(J) = J \cap R\) is an ideal of \(R\).

        \item Let \(b \in \varphi(I)\) and \(s \in S\). Fix some \(a \in R\) such that \(\varphi(a) = b\). Since \(\varphi\) is surjective, there is some \(r \in R\) such that \(\varphi(r) = s\). Thus \(sb = \varphi(r)\varphi(a) = \varphi(ra) \in \varphi(I)\), where the last equality is because \(ra \in I\). Hence \(\varphi(I)\) is an ideal. 
        
        If \(\varphi\) was not surjective, then consider the example \(R = \Z\), \(S = \mathbb R\), and \(\varphi = \iota\) is the inclusion. Then \(2\Z\) is an ideal of \(\Z\), but not an ideal of \(\mathbb R\), as \(0.5 * 2 = 1 \notin 2\Z\).
    \end{enumerate}
\end{proof}


\begin{theorem}[26]
    Let \(R\) be a ring. For any \(n \in \Z\) and \(r \in R\), define \(nr = r + \dots + r\) (\(n\) times).
    \begin{enumerate}
        \item Prove that the map \(\Z \to R\) defined by \(k \mapsto k1_R\) is a ring homomorphism whose kernel is \(n \Z\), where \(n\) is the characteristic of \(R\). 
        \item Determine the characteristics of the rings \(\mathbb Q\), \(\Z[x]\), and \(\Z / n \Z [x]\).
        \item Prove that if \(p\) is a prime and if \(R\) is a commutative ring of characteristic \(p\), then \((a + b)^p = a^p + b^p\) for all \(a, b \in R\). 
    \end{enumerate}
\end{theorem}

\begin{proof}
    We proceed with each separately:
    \begin{enumerate}
        \item Denote the map by \(\varphi\). We have \(\ker \varphi \le \Z\), and the subgroup structure of \(Z\) gives us \(\ker \varphi = n \Z\) for some \(n \in \mathbb N\). Since \(n \Z\) is cyclic, we only need to look at where the generator, \(n\), maps to. We must have \(\varphi(n) = 0\) and furthermore \(n\) is the minimal number for which this occurs. Hence \(\varphi(n) = n1_R = 0\) implies \(\text{char}(R) = n\).
        \item There is no \(n \in \mathbb N\) such that \(n1_{\mathbb Q} = n = 0\). Thus \(\text{char}(\mathbb Q) = 0\). Similarly, the variable \(x\) doesn't affect \(n1_{\Z[x]}\), so \(\text{char}(\Z[x]) = 0\).
        
        For \(\Z / n \Z [x]\), we have \(n1 = 0\) in \(\Z / n\Z\), so the same holds in the polynomial ring. Thus \(\text{char}(\Z / n \Z[x]) = n\).

        \item As \(R\) is a commutative ring, we have enough structure to apply the binomial theorem:
        \[(a + b)^p = a^p + \binom{p}{1}a^{p - 1}b + \binom{p}{2}a^{p - 2}b^2 + \dots + \binom{p}{p - 1}ab^{p - 1} + b^p.\] 
        If \(k \neq 0, p\), consider \(\binom{p}{k} = \frac{p!}{k!(p - k)!}\). The factors in the denominator \(k!(p - k)!\) are strictly less than \(p\), and thus do not divide \(p\). Thus \(p! / k!(p - k)!\) must have a factor of \(p\). We have \(\text{char}(R) = p\), so all the terms but the first and last of \((a + b)^p\) are equal to zero. Thus \((a + b)^p = a^p + b^p\), as desired.
    \end{enumerate}
\end{proof}


\end{document}