\documentclass{article}
\usepackage[pdftex]{graphicx}
\usepackage{algorithm}
\usepackage[noend]{algorithmic}
\usepackage{tikz}
\usepackage{graphicx}
\usepackage{subcaption}
\usepackage{amsmath,amssymb}
\usepackage{amsfonts}
\usepackage{xcolor}
\usepackage{fancyvrb}
\usepackage{color}
\usepackage{blindtext}
\usepackage{titlesec}
\setlength{\topmargin}{0 in}
\setlength{\oddsidemargin}{0 in}
\setlength{\textheight}{8.5 in}
\setlength{\textwidth}{6.5 in}
\setlength{\parindent}{0 in}

\titleformat*{\section}{\LARGE\bfseries}
\renewcommand{\o}[1]{$\overline{#1}$}
\renewcommand{\O}[1]{$\mathcal{O}(#1)$} %% For the clear looking O notation.

\definecolor{darkgreen}{rgb}{0,0.42,0}
\newcommand{\answer}[1]{} 
\newcommand{\rubric}[1]{{\leavevmode\color{brown}#1}}
\newcommand{\ZZ}[1]{\mathbb Z / #1 \mathbb Z}
\newcommand{\notes}[1]{\textcolor{red}{#1}}

\DeclareUnicodeCharacter{2212}{-}

\title{\textbf{Math 410 Homework 2}}
\author{Due Date: \textbf{something}}
\date{} 
\begin{document}
\maketitle

Exercises 8, 9, 12, 26, 36, pp. 21-23.

\begin{enumerate}
    \item [8.] 
    \begin{enumerate}
        \item [(a)] 
        We show that $G$ satifies that group axioms under multiplcation. 

        \textit{Identity:} Clearly $1 \in G$. And $1$ is the identity since for any $g \in G \subset \mathbb C$, $g * 1 = 1 * g = g$.
        
        \textit{Associativity:} Note that $G \subset \mathbb C$. Since $\mathbb C$ is associative under normal $*$, and $G$ carries the same (restricted) $*$, $G$ must be associative as well. 

        \textit{Closure:} Let $g, h \in G$. Then by definition there exist some $n, m \in \mathbb Z^+$ such that $g ^ n = h ^ m = 1$. We want to find $N$ such that $(gh)^N = 1$. Let $N = nm$, directly calucating $(gh)^N$ yields $(gh)^{nm} = g^{nm} * h ^{nm} = (g^n)^m * (h^m)^n = 1^m * 1 ^ n = 1$. Therefore $gh \in G$. 

        \textit{Inverse:} Let $g \in G$, with $g^n = 1$ for some $n \in \mathbb Z^+$. Let $h = \overline{g}$, the complex conjugate of $g$. Then

        \item [(b)] 
    \end{enumerate}

    \item [9.]
    \item \begin{enumerate}
        \item [(a)]
        We show that $G$ satisfies the group axioms under addtion.

        \textit{Identity:} Clearly $0 = 0 + 0 \sqrt 2 \in G$. And $0$ is the identity since for any $g \in G \subset \mathbb R$, $g + 0 = 0 + g = g$.
        
        \textit{Associativity:} Note that $G \subset \mathbb R$. Since $\mathbb R$ is associative under normal $+$, and $G$ carries the same (restricted) $+$, $G$ must be associative as well. 

        \textit{Closure:} Let $g, h \in G$. Then by definition there exist some $p, q, r, s \in \mathbb Q$ such that $p + q\sqrt 2 = g$ and $r + s \sqrt 2 = h$. We want to find $x, y \in \mathbb Q$ such that $x + y \sqrt 2 = g + h$. Clearly we want $x = p + r$ and $y = q + s$, so that $g + h = (p + r) + (q + s)\sqrt 2 = x + y \sqrt 2$. 

        \textit{Inverse:} Let $g \in G$, with $a + b \sqrt 2 = g$ for some $a, b, \in \mathbb Q$. Then $-a + -b \sqrt 2 \in G$ is the inverse of $g$, since $a + b\sqrt 2 - a - b\sqrt 2 = 0$. Hence $G$ has inverses.

        \item [(b)] Let $g$ be a non-zero element of $G$ such that $a + b \sqrt 2 = g$ for some $a, b \in \mathbb Q$ (where $a$ and $B$ are not both $0$). Then note that $1 / g = 1 / (a + b \sqrt 2)$ is in $G$, since \[\frac 1 {a + b \sqrt 2} = \frac {a - b\sqrt 2} {(a + b \sqrt 2) (a - b \sqrt 2)} = \frac {a - b \sqrt 2} {a^2 - 2b^2}.\] Letting $x = \frac a {a^2 - 2b^2}$ and $y = \frac {-b} {a^2 - 2b^2}$, we have $1 / g = x + y \sqrt 2$. 
        Both $x$ and $y$ are rational, since they are made up of rational expressions. Hence $1 / g$ (in $\mathbb R$) is the inverse og $g$ in $G$. 

        \item[Note.] This makes $G$ a \textit{field}. In fact it is the field $\mathbb Q[\sqrt 2]$, the result of adjoining $\sqrt 2$ to $\mathbb Q$. 
    \end{enumerate}
    
    \item [12.] $1$
\end{enumerate}

Exercises 3, 9, pp. 27-28.
Exercises 2, 4, 13, 16, 20, pp. 32-34.
Exercises 17, 18, pp. 40.
Exercises 18, 19, pp. 45.

\end{document}