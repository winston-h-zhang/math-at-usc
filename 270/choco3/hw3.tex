\documentclass[10pt]{article}
\usepackage{fullpage}
\usepackage[noend]{algorithmic}
\usepackage{algorithm}

\usepackage{amsfonts}
\usepackage{amsmath,amsthm,amssymb}
\usepackage{nicefrac}
\usepackage{latexsym}
\usepackage{epic}
\usepackage{epsfig}
\usepackage{amscd}
\usepackage{url}
\usepackage{verbatim}
\usepackage{setspace}
\usepackage{hyperref}
\usepackage{enumerate}
\usepackage{times}

% \usepackage{CS270}

\newtheorem{lemma}{Lemma}
\newtheorem{fact}[lemma]{Fact}
\newtheorem{theorem}[lemma]{Theorem}
\newtheorem{corollary}[lemma]{Corollary}
\newtheorem{proposition}[lemma]{Proposition}
\newtheorem{claim}[lemma]{Claim}

\newtheorem{definition}{Definition}
\newtheorem{example}[definition]{Example}
\newtheorem{remark}[definition]{Remark}

\providecommand{\half}{\ensuremath{\nicefrac{1}{2}}\xspace}
\providecommand{\third}{\ensuremath{\nicefrac{1}{3}}\xspace}
\providecommand{\quarter}{\ensuremath{\nicefrac{1}{4}}\xspace}

\providecommand{\SetCard}[1]{\ensuremath{| #1 |}\xspace}
\providecommand{\SET}[1]{\ensuremath{\{ #1 \}}\xspace}
\providecommand{\Abs}[1]{\ensuremath{| #1 |}\xspace}
\providecommand{\Set}[2]{\ensuremath{\SET{#1 \mid #2}}\xspace}

\providecommand{\Kth}[1]{\ensuremath{{#1}^{\rm th}}}
\providecommand{\Ceiling}[1]{\ensuremath{\lceil {#1} \rceil}\xspace}
\providecommand{\Floor}[1]{\ensuremath{\lfloor {#1} \rfloor}\xspace}

\providecommand{\PROB}{\ensuremath{{\rm Prob}}\xspace}
\providecommand{\Prob}[2][]{\ensuremath{%
\ifthenelse{\equal{#1}{}}{\PROB[#2]}{\PROB_{#1}[#2]}}\xspace}
\providecommand{\ProbC}[3][]{\Prob[#1]{#2\;|\;#3}}
\providecommand{\Expect}[2][]{\ensuremath{%
\ifthenelse{\equal{#1}{}}{\mathbb{E}}{\mathbb{E}_{#1}}%
\left[#2\right]}\xspace}
\providecommand{\ExpectC}[3][]{\Expect[#1]{#2\;|\;#3}}

\title{\bf Chocolate 3}
\author{Winston (Hanting) Zhang}
\date{Friday, September 23, by 23:00}

\begin{document}
\maketitle

\subsection*{Problem 3. [0]}
\textbf{Chocolate Problem: 2 chocolate bars}

Reminder: If you solve a chocolate problem (which you can do in groups of size up to 3), please e-mail David with the solution --- do not submit it on Gradescope. Also, feel free to list preferences or dietary restrictions for/against particular types of chocolate.

Exercise 4.31 in the textbook. Notice that Part (b) is really the interesting thing here --- Part (a) is basically a slightly harder regular problem.

\begin{proposition}
  Prove that for every pair of nodes \(u, v \in V\), the length of the shortest \(u-v\) path in \(H\) is at most 3 times the length of the shortest \(u-v\) path in \(G\). 
\end{proposition}

\begin{proof}
  Denote the length of the shortest \(u-v\) path in \(H\) and \(G\) by \(d_H(u, v)\) and \(d_G(u, v)\), respectively. Suppose for the sake of contradiction that there exists some \(u, v \in V\) such that \(d_H(u, v) > 3 d_G(u, v)\). Furthermore set \(u, v\) to be the vertices such that \(d_H(u, v)\) is minimum with repsect to \(d_H(u, v) > 3 d_G(u, v)\). Note that such a minimum exists since edges have positive side lengths. (And the graph must be finite?) Consider the \(u-v\) path in \(G\) made up of the sequence of vertices \(u = v_1, v_2, \dots, v_k = v\). 
\end{proof}

\end{document}