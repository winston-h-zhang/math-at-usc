%%%%%%%%%%%%%%%%%%%%%%%%%%%%%%%%%%%%%%%%%%%%%%
%%                                          %%
%% USE THIS FILE TO SUBMIT YOUR SOLUTIONS   %%
%%                                          %%
%% You must have the usamts.tex file in     %%
%% the same directory as this file.         %%
%% You do NOT need to submit this file or   %%
%% usamts.tex with your solutions.  You     %%
%% only need to submit the output PDF file. %%
%%                                          %%
%% DO NOT ALTER THE FILE usamts.tex         %%
%%                                          %%
%% If you have any questions or problems    %%
%% using this file, or with LaTeX in        %%
%% general, please go to the LaTeX          %%
%% forum on the Art Of Problem Solving      %%
%% web site, and post your problem.         %%
%%                                          %%
%%%%%%%%%%%%%%%%%%%%%%%%%%%%%%%%%%%%%%%%%%%%%%

%%%%%%%%%%%%%%%%%%%%%%%%%%%%%%%%%%%%%%%%%%%%
%% DO NOT ALTER THE FOLLOWING LINES
\documentclass[12pt]{article}
\usepackage{amsmath,amssymb,amsthm,amsfonts,tabularx}
\usepackage[pdftex]{graphicx}
\usepackage[dvipsnames]{xcolor}
\usepackage{fancyhdr}
\usepackage{parskip}
\usepackage[shortlabels]{enumitem}
\pagestyle{fancy}
\usepackage{setspace}
\newcommand{\realname}[1]{\newcommand{\printrealname}{#1}}
\newcommand{\pset}[1]{\newcommand{\printpset}{#1}}
\newcommand{\mathclass}[1]{\newcommand{\printmathclass}{#1}}

%% Pagestyle setup
\setlength{\headheight}{0.75in}
\setlength{\oddsidemargin}{0in}
\setlength{\evensidemargin}{0in}
\setlength{\voffset}{-.5in}
\setlength{\headsep}{10pt}
\setlength{\textwidth}{6.5in}
\setlength{\headwidth}{6.5in}
\setlength{\textheight}{8in}
\lhead{Math \printmathclass}
\chead{\Large \textbf{Homework \printpset}}
\rhead{\printrealname}
\rfoot{Page \thepage}
\renewcommand{\headrulewidth}{0.5pt}
\renewcommand{\footrulewidth}{0.3pt}
\setlength{\textwidth}{6.5in}
\renewcommand{\baselinestretch}{1}
\setenumerate[0]{label=(\alph*)}

\newtheorem*{prop}{Proposition}
\newtheorem*{corollary}{Corollary}
\newtheorem*{lemma}{Lemma}
\theoremstyle{remark}
\newtheorem*{defn}{Definition}

\newtheoremstyle{named}{}{}{}{}{\bfseries}{.}{.5em}{\thmnote{Problem #3}}
\theoremstyle{named}
\newtheorem*{theorem}{Theorem}

%% DO NOT ALTER THE ABOVE LINES
%%%%%%%%%%%%%%%%%%%%%%%%%%%%%%%%%%%%%%%%%%%%


%% If you would like to use Asymptote within this document (which is optional), 
%% you can find out how at the following URL:
%%
%%   http://www.artofproblemsolving.com/Wiki/index.php/Asymptote:_Advanced_Configuration
%%
%% As explained there, you will want to uncomment the line below.  But be
%% sure to check the website because there are several other steps that must 
%% be followed.
%% \usepackage{asymptote}

%% Enter your real name here
%% Example: \realname{David Patrick}
\realname{Hanting Zhang}
\pset{8}
\mathclass{425A}

\newcommand{\todo}{\textcolor{red}{\textbf{TODO} }}
\renewcommand{\a}{\alpha}
\renewcommand{\b}{\beta}
\newcommand{\e}{\varepsilon}
\newcommand{\Z}{\mathbb Z}
\newcommand{\N}{\mathbb N}
\newcommand{\Q}{\mathbb Q}
\newcommand{\R}{\mathbb R}
\newcommand{\C}{\mathbb C}
\renewcommand{\bf}{\mathbf}
\renewcommand{\implies}{\Rightarrow}
\newcommand{\coimplies}{\Leftarrow}
\renewcommand{\em}{\varnothing}

\begin{document}

Chapter 5, \#1,2 3 and 26
\begin{theorem}[1]
    Let \(f\) be defined for all real \(x\), and suppose that 
    \[|f(x) - f(y)| \le (x - y)^2\]
    for all real \(x\) and \(y\). Prove that \(f\) is constant.
\end{theorem}

\begin{proof}
    Without loss of generality assume that \(x \ge y\) and \(x = y + \delta\). Then we may rewrite the given equation as
    \[\frac{|f(y + \delta) - f(y)|}{\delta} \le \delta.\]
    Then for any \(y\), as \(\delta \to 0\), we have \(\lim_{\delta \to 0} |(f(y + \delta) - f(y)) / \delta| \le 0\). Thus \(f'(y)\) is defined and equal to zero. Theorem 5.11 gives \(f'(x) = 0\) implies \(f\) is constant.
\end{proof}

\begin{theorem}[2]
    Suppose \(f'(x) > 0\) in \((a, b)\). Prove that \(f\) is strictly increasing on \((a, b)\), and let \(g\) be its inverse function. Prove that \(g\) is differentiable, and that 
    \[g'(f(x)) = \frac{1}{f'(x)} \hspace{5mm} (a < x < b).\]
\end{theorem}

\begin{proof}
    For any \(x, y \in (a, b)\), \(x > y\), apply MVT to see that \(f(y) - f(x) = (y - x)f'(c)\) for some \(x < c < y\). Both \(y - x > 0\) and \(f'(c) > 0\), so the RHS is positive; hence \(f(y) > f(x)\). This shows that \(f\) is strictly increasing.

    We deduce that \(f\) is injective, and therefore it is bijective on its image, \((f(a), f(b))\). Thus we may construct an inverse \(g\). For any \(y \in (f(a), f(b))\), consider the limit 
    \[\lim_{s \to y}\frac{g(s) - g(y)}{s - y} \hspace{5mm} (f(a) < y < f(b)).\]
    Since \(g\) is the inverse of \(f\), there is a unique mapping \(f(x) = y\) and \(f(t) = s\) such that 
    \[\lim_{s \to y} \frac{g(s) - g(y)}{s - y} = \lim_{t \to x}\frac{t - x}{f(t) - f(x)} = \lim_{t \to x}\left(\frac{f(t) - f(x)}{t - x}\right)^{-1} = \frac{1}{f'(x)}.\]
    Thus \(g\) is differentiable, and with \(y = f(x)\), we deduce that
    \[g'(f(x)) = \frac{1}{f'(x)} \hspace{5mm} (a < x < b).\]
\end{proof}

\begin{theorem}[3]
    Suppose that \(g\) is a real function on \(\R^1\), with bounded derivative (say \(|g'| \le M\)). Fix \(\e > 0\), and define \(f(x) = x + \e g(x)\). Prove that \(f\) is injective if \(\e\) is small enough. (A set of admissable values of \(\e\) can be determined which depends only on \(M\).)
\end{theorem}

\begin{proof}
    Let \(\e < 1 / M\). Then \(f'(x) = 1 + \e g'(x)\). Now \(|\e g'(x)| < (1/M) M = 1\), so we have \(f'(x) > 0\). Thus \(f\) is strictly increasing. The reals form a total order so this implies that \(f\) is injective.
\end{proof}

\begin{theorem}[26]
    Suppose \(f\) is differentiable on \([a, b]\), \(f(a) = 0\), and there is a real number \(A\) such that \(|f'(x)| \le A|f(x)|\) on \([a, b]\). Prove that \(f(x) = 0\) for all \(x \in [a, b]\). 
\end{theorem}

\begin{proof}
    Following the hint given by the textbook, let \(M_0 = \sup |f(x)|\) and \(M_1 = \sup |f'(x)|\) for \(x \in [a, b]\). For any \(x\), we have \(|f(x)| \le M_1(b - a) \le A(b - a)M_0\). We deduce that \(\sup |f(x)| = M_0 \le A(b - a)M_0\). Hence \(M_0 = 0\) if \(A(b - a) < 1\). So choose \(A < 1/ (b - a)\); then \(f = 0\).
\end{proof}

\end{document}