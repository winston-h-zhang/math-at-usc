%%%%%%%%%%%%%%%%%%%%%%%%%%%%%%%%%%%%%%%%%%%%%%
%%                                          %%
%% USE THIS FILE TO SUBMIT YOUR SOLUTIONS   %%
%%                                          %%
%% You must have the usamts.tex file in     %%
%% the same directory as this file.         %%
%% You do NOT need to submit this file or   %%
%% usamts.tex with your solutions.  You     %%
%% only need to submit the output PDF file. %%
%%                                          %%
%% DO NOT ALTER THE FILE usamts.tex         %%
%%                                          %%
%% If you have any questions or problems    %%
%% using this file, or with LaTeX in        %%
%% general, please go to the LaTeX          %%
%% forum on the Art Of Problem Solving      %%
%% web site, and post your problem.         %%
%%                                          %%
%%%%%%%%%%%%%%%%%%%%%%%%%%%%%%%%%%%%%%%%%%%%%%

%%%%%%%%%%%%%%%%%%%%%%%%%%%%%%%%%%%%%%%%%%%%
%% DO NOT ALTER THE FOLLOWING LINES
\documentclass[12pt]{article}
\usepackage{amsmath,amssymb,amsthm,amsfonts,tabularx}
\usepackage[pdftex]{graphicx}
\graphicspath{ {./images/} }
\usepackage{fancyhdr}
\pagestyle{fancy}
\usepackage{setspace}
\usepackage{csquotes}
%%%%%%%%%%%%%%%%%%%%%%%%%%%%%%%%%%%%%%%%%%%
%%                                       %%
%% Students: DO NOT MODIFY THIS FILE!!!  %%
%%                                       %%
%%%%%%%%%%%%%%%%%%%%%%%%%%%%%%%%%%%%%%%%%%%

%% USAMTS style sheet
%% last modified: 23-Jul-2004

%% Student and Year/Round data
\newcommand{\realname}[1]{\newcommand{\printrealname}{#1}}
\newcommand{\pset}[1]{\newcommand{\printpset}{#1}}

%% Pagestyle setup
\setlength{\headheight}{0.75in}
\setlength{\oddsidemargin}{0in}
\setlength{\evensidemargin}{0in}
\setlength{\voffset}{-.5in}
\setlength{\headsep}{10pt}
\setlength{\textwidth}{6.5in}
\setlength{\headwidth}{6.5in}
\setlength{\textheight}{8in}
\lhead{Math 410}
\chead{\Large \textbf{Homework \printpset}}
\rhead{\printrealname}
\rfoot{Page \thepage}
\renewcommand{\headrulewidth}{0.5pt}
\renewcommand{\footrulewidth}{0.3pt}
\setlength{\textwidth}{6.5in}


\renewcommand{\baselinestretch}{1}
%% DO NOT ALTER THE ABOVE LINES
%%%%%%%%%%%%%%%%%%%%%%%%%%%%%%%%%%%%%%%%%%%%


%% If you would like to use Asymptote within this document (which is optional), 
%% you can find out how at the following URL:
%%
%%   http://www.artofproblemsolving.com/Wiki/index.php/Asymptote:_Advanced_Configuration
%%
%% As explained there, you will want to uncomment the line below.  But be
%% sure to check the website because there are several other steps that must 
%% be followed.
%% \usepackage{asymptote}

\newtheorem*{prop}{Proposition}
\newtheorem*{corollary}{Corollary}
\newtheorem*{lemma}{Lemma}
\theoremstyle{remark}
\newtheorem*{defn}{Definition}

\newtheoremstyle{named}{}{}{}{}{\bfseries}{.}{.5em}{\thmnote{Problem #3}}
\theoremstyle{named}
\newtheorem*{theorem}{Theorem}

%% Enter your real name here
%% Example: \realname{David Patrick}
\realname{Hanting Zhang}
\pset{4}
\mathclass{410}

\renewcommand{\bf}{\mathbf}
\renewcommand{\implies}{\Rightarrow}
\newcommand{\coimplies}{\Leftarrow}

\begin{document}
Exercises 6, 12, pp. 52-53.

\begin{enumerate}
    \item [6.]
    \begin{enumerate}
        \item [(a)] If \(H\) is a subgroup of \(G\), then for any \(h, h' \in H\), we have \(h^{-1}h'h \in H\). Hence \(h^{-1}Hh = H\), and \(h \in N_G(H)\). Therefore \(H \le N_G(H)\). 
        
        If \(H\) is not a subgroup of \(G\), then multiplication fails so we have no reason to expect \(h^{-1}h'h \in H\). For example, let 
        \[H = \left\{\begin{pmatrix}
            1 & 1 \\ 0 & 1
        \end{pmatrix}, \begin{pmatrix}
            2 & 3 \\ 1 & 2
        \end{pmatrix}\right\}.\]
    Then 
    \[\begin{pmatrix}
        1 & 1 \\ 0 & 1
    \end{pmatrix}^{-1} \begin{pmatrix}
        2 & 3 \\ 1 & 2
    \end{pmatrix} \begin{pmatrix}
        1 & 1 \\ 0 & 1
    \end{pmatrix} = \begin{pmatrix}
        2 & -5 \\ 0 & 3
    \end{pmatrix} \notin H.\]
    Hence \(H \nleq N_G(H)\).

    \item [(b)] If \(H \le C_G(H)\), then for any \(h, h' \in H\), we have \(h^{-1}h'h = h' \implies h'h = h h'\). Hence \(H\) is abelian, as desired.
    \end{enumerate}
    \item [12.] Too much work for now.
\end{enumerate}

Exercises 16, 17, pp. 65-66.

\begin{enumerate}
    \item [16.] 
    \begin{enumerate}
        \item[(a)] Since \(G\) is finite there can only be a finite amount of subgroups. In particular, there are only a finite amount of subgroups \(\{H_i\}_{i = 1}^n\) containing \(H\). Then any chain \(H \le H_{i_1} \le H_{i_2} \le \dots \le H_{i_k} \le G\) is finite, and we may prescribe \(H_{i_k}\) as the maximal subgroup containing \(H\).
        
        \item[(b)] Suppose \(\langle r \rangle \le K\). Then \(|\langle r \rangle| \le |K|\) while \(|K| \mid |G|\). But \(\langle r \rangle\) has order \(n\) and \(G\) has order \(2n\). Hence \(|K|\) can only be \(n\), in which case \(H = K\), or \(2n\), in which case \(K = G\). This is exactly the definition of \(H\) being maximal, as desired.
         
        \item[(c)] The order of \(x^p\) is \(n / p\), so \(|\langle x^p \rangle| = n / p\). If \(K\) contains \(\langle x^p \rangle\), then \(n/p \le |K|\) while \(|K| \mid n \implies a|K| = n\) for some \(a\). And again, 
    \end{enumerate}
    \item [17.]
\end{enumerate}

Exercises 1, 18, 24, 40, 41 pp. 85-89.

\begin{enumerate}
    \item [1.]
    \item [18.]
    \item [24.]
    \item [40.]
    \item [41.]
\end{enumerate}

Exercise 4, pp. 111.

\begin{enumerate}
    \item [4.]
\end{enumerate}

\end{document}