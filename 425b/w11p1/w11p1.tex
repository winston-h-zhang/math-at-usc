\documentclass[12pt]{article}
\usepackage{amsmath,amssymb,amsthm,amsfonts,tabularx}
\usepackage[pdftex]{graphicx}
\usepackage[dvipsnames]{xcolor}
\usepackage{fancyhdr}
\usepackage{parskip}
\usepackage[shortlabels]{enumitem}
\pagestyle{fancy}
\usepackage{setspace}
\usepackage{pgfplots}
\pgfplotsset{compat=1.18}
\newcommand{\realname}[1]{\newcommand{\printrealname}{#1}}
\newcommand{\pset}[1]{\newcommand{\printpset}{#1}}
\newcommand{\mathclass}[1]{\newcommand{\printmathclass}{#1}}

%% Pagestyle setup
\setlength{\headheight}{0.75in}
\setlength{\oddsidemargin}{0in}
\setlength{\evensidemargin}{0in}
\setlength{\voffset}{-.5in}
\setlength{\headsep}{10pt}
\setlength{\textwidth}{6.5in}
\setlength{\headwidth}{6.5in}
\setlength{\textheight}{8in}
\lhead{Math \printmathclass}
\chead{\Large \textbf{\printpset}}
\rhead{\printrealname}
\rfoot{Page \thepage}
\renewcommand{\headrulewidth}{0.5pt}
\renewcommand{\footrulewidth}{0.3pt}
\setlength{\textwidth}{6.5in}
\renewcommand{\baselinestretch}{1}
\setenumerate[0]{label=(\alph*)}
\newcommand{\todo}{\textcolor{red}{\textbf{TODO }}}

\newtheorem*{prop}{Proposition}
\newtheorem*{corollary}{Corollary}
\newtheorem{lemma}{Lemma}
\theoremstyle{remark}
\newtheorem*{defn}{Definition}
\newtheorem*{remark}{Remark}

\newtheoremstyle{named}{}{}{}{}{\bfseries}{.}{.5em}{\thmnote{Problem #3}}
\theoremstyle{named}
\newtheorem*{theorem}{Theorem}
\allowdisplaybreaks

%% DO NOT ALTER THE ABOVE LINES
%%%%%%%%%%%%%%%%%%%%%%%%%%%%%%%%%%%%%%%%%%%%


%% If you would like to use Asymptote within this document (which is optional), 
%% you can find out how at the following URL:
%%
%%   http://www.artofproblemsolving.com/Wiki/index.php/Asymptote:_Advanced_Configuration
%%
%% As explained there, you will want to uncomment the line below.  But be
%% sure to check the website because there are several other steps that must 
%% be followed.
%% \usepackage{asymptote}

%% Enter your real name here
%% Example: \realname{David Patrick}
\realname{Hanting Zhang}
\pset{W11P1}
\mathclass{425B}

\renewcommand{\a}{\alpha}
\renewcommand{\b}{\beta}
\renewcommand{\d}{\delta}
\newcommand{\e}{\varepsilon}
\newcommand{\Z}{\mathbb Z}
\newcommand{\N}{\mathbb N}
\newcommand{\Q}{\mathbb Q}
\newcommand{\R}{\mathbb R}
\newcommand{\C}{\mathbb C}
\renewcommand{\bf}{\mathbf}
\newcommand{\id}[1]{\text{id}_{#1}}
\renewcommand{\implies}{\Rightarrow}
\newcommand{\coimplies}{\Leftarrow}
\renewcommand{\em}{\varnothing}
\renewcommand{\Im}{\text{Im}}
\newcommand{\abs}[1]{|#1|}
\newcommand{\bigabs}[1]{\left|#1\right|}
\newcommand{\Rloc}{\mathcal R_{\text{loc}}}

%% should be reset

\begin{document}

\begin{theorem}[1.1]
    Prove Proposition 1.5. In the right-most expression in (52), interpret \(\inf(\varnothing) = \infty\) when necessary, or equivalently, the the infimum over \(C \geq 0\) belonging to \(\overline{\R}\). Caution: You are not guaranteed the existence of an \(x \in X\) such that \(\|T_x\|_Y = \|T\|_{X \to Y}\|x\|_X\).
\end{theorem}

\begin{proof}
    Let's deal with each equality, starting with the middle \((2) = (3)\). We have
    \begin{align*}
        \{\|T x\|_Y : \|x\|_X = 1\} \subseteq \{\|T x\|_Y : \|x\|_X \leq 1\} \implies \sup_{\|x\| = 1} \|T x\|_Y \leq \sup_{\|x\| \leq 1} \|T x\|_Y.
    \end{align*}
    But also, by linearity, \(\|Tx_1\| \leq \|Tx_2\|\) for all \(\|x_1\| \leq \|x_2\|\), and for all \(x_1\) such that \(\|x_2\| \leq 1\), there is some \(x_2\) such that \(\|x_1\| \leq \|x_2\| = 1\). Thus also \(\sup_{\|x\| \leq 1} \|T x\|_Y \leq \sup_{\|x\| = 1} \|T x\|_Y,\) and we have
    \(\sup_{\|x\| = 1} \|T x\|_Y = \sup_{\|x\| \leq 1} \|T x\|_Y.\)

    Now \((3) = (4)\). We have by definition of \(\sup / \inf\):
    \begin{align*}
        \sup_{\|x\| = 1} \|T x\|_Y = \inf \{C : \|T x\|_Y \leq C, \|x\| = 1\}.
    \end{align*}
    Consider the map \(\psi\) such that \(x \mapsto x / \|x\|_X\). For all \(\|x\| = 1\), the union preimages \(\psi^{-1}(x)\) is the entire space \(X\). Furthermore, if \(\|T x\|_Y \leq C\) for \(\|x\| = 1\), then for every point \(w\) in the preimage, it holds that \(\|T w\| \leq C \|w\|\), by linearity. Thus we may write:
    \begin{align*}
        \sup_{\|x\| = 1} \|T x\|_Y &= \inf \{C : \|T x\|_Y \leq C, \|x\| = 1\} \\
        &= \inf \{C : \|T w\|_Y \leq C \|w\|, w \in X\},
    \end{align*}
    as desired.
    Finally, the right side. We have by definition of \(\sup / \inf\):
    \begin{align*}
        \sup_{x \neq 0} \frac{\|T x\|_Y}{\|x\|_X} = \inf \{C : \frac{\|T x\|_Y}{\|x\|_X} \leq C, x \neq 0\}.
    \end{align*}
    A little rearranging gives (note we can add in \(x = 0\) since it doesn't affect the answer):
    \begin{align*}
        \sup_{x \neq 0} \frac{\|T x\|_Y}{\|x\|_X} &= \inf \{C : \frac{\|T x\|_Y}{\|x\|_X} \leq C, x \neq 0\} \\&= \inf \{C : \|T x\|_Y \leq C \|x\|_X, x \in X\},
    \end{align*}
    as desired.
\end{proof}
\newpage
\begin{theorem}[1.2]
    Prove Proposition 1.9. For convenience, a ``checklist'' is provided below.
    \begin{enumerate}
        \item Start with a Cauchy sequence \((T_n)_{n = 1}^\infty\) in \((\mathcal B (X, Y), \|\cdot\|_{X \to Y})\).
        \item Find a candidate \(T : X \to Y\) for the limit. (Use the completeness of \((Y, \|\cdot\|_Y)\).)
        \item Prove that \(T\) is linear and continuous.
        \item Prove that \(\lim_{n \to \infty} \|T_n - T\|_{X \to Y} = 0\), and finish the argument.
    \end{enumerate}
\end{theorem}

\begin{proof}
    We proceed with the steps given:
    Let \((T_n)_{n = 1}^\infty\) be a Cauchy in \((\mathcal B (X, Y), \|\cdot\|_{X \to Y})\) and \(\e > 0\). Then there exists \(N\) such that for all \(n, m > N\), we have \(\|T_n - T_m\|_{X \to Y} < \e\). This implies for all \(x \in X\), we have \(\|T_n(x) - T_m(x)\|_{Y} < \e\). Thus \((T_n(x))_{n = 1}^\infty\) is shown to also be Cauchy; and knowing that \((Y, \|\cdot\|_Y)\) is complete, we must have \(T_n(x) \to T_x\) for all \(x \in X\). Then define \(T : X \to Y\) to be \(x \mapsto T_x\). 
    Indeed, \(T\) is linear and continuous. Let \(x_1, x_2 \in X\) and \(k \in F\). We have
    \begin{align*}
        T (x_1 + x_2) &= \lim_{n \to \infty} \left(T_n (x_1 + x_2)\right) \\ 
        &= \lim_{n \to \infty} (T_n(x_1) + T_n(x_2)) =  \lim_{n \to \infty} T_n(x_1) + \lim_{n \to \infty} T_n(x_2) \\
        &= T(x_1) + T(x_2)
    \end{align*}
    and
    \begin{align*}
        T (k x_1) &= \lim_{n \to \infty} \left(T_n (k x_1)\right) \\ 
        &= \lim_{n \to \infty} (k T_n(x_1)) = k \lim_{n \to \infty} T_n(x_1) \\
        &= k T(x_1).
    \end{align*}
    For continuity, it suffices to show that \(T\) is bounded. We know that all the \(T_n\)s are bounded uniformly by some \(K\). Then for all \(x \in X\), we abuse the limit to conlude:
    \begin{align*}
        \|T(x)\| = \left\|\lim_{n \to \infty} T_n (x)\right\| \leq K \|x\|.
    \end{align*}
    Thus \(T \in \mathcal B(X, Y)\). Finally, we must check that actually \(T_n \to T\) in the \(\|\cdot\|_{X \to Y}\) norm. Since \(T_n(x) \to T\), we have \(\|T_n(x) - T\| < \e\). Then \(\|T_n - T\| = \sup_{\|x\| = 1} \|T_n(x) - T(x)\| < \e\), as desired.
\end{proof}
\newpage
\begin{theorem}[1.3]
    Let \((X, \|\cdot\|_X)\) and \((Y, \|\cdot\|_Y)\) be finite-dimensional normed \(F\)-vector spaced, with \(F = \R\) or \(\C\). Give an explanation (as concise as possible) for why any linear bijection \(T : X \to Y\) is automatically a normed vector space isomorphism.
\end{theorem}

\begin{proof}
    Construct the space \((Y, \|\cdot\|_Y)\) by \(\|y\|_Y = \|\psi^{-1}y\|_X\). Then \(X\) and \(Y\) are isomorphic as normed vector spaces by construction. Recall that all norms on finite dimensional \(F\)-vector spaces are equivalent. Thus we have the isomorphism
    \begin{align*}
        (X, \|\cdot\|_X) \xrightarrow{\sim} (Y, \|\cdot\|_Y) \xrightarrow{\sim} (Y, \|\cdot\|),
    \end{align*}
    where the second map is id\(_Y\), but converts the topology.
\end{proof}

\end{document}