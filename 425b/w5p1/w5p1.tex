\documentclass[12pt]{article}
\usepackage{amsmath,amssymb,amsthm,amsfonts,tabularx}
\usepackage[pdftex]{graphicx}
\usepackage[dvipsnames]{xcolor}
\usepackage{fancyhdr}
\usepackage{parskip}
\usepackage[shortlabels]{enumitem}
\pagestyle{fancy}
\usepackage{setspace}
\newcommand{\realname}[1]{\newcommand{\printrealname}{#1}}
\newcommand{\pset}[1]{\newcommand{\printpset}{#1}}
\newcommand{\mathclass}[1]{\newcommand{\printmathclass}{#1}}

%% Pagestyle setup
\setlength{\headheight}{0.75in}
\setlength{\oddsidemargin}{0in}
\setlength{\evensidemargin}{0in}
\setlength{\voffset}{-.5in}
\setlength{\headsep}{10pt}
\setlength{\textwidth}{6.5in}
\setlength{\headwidth}{6.5in}
\setlength{\textheight}{8in}
\lhead{Math \printmathclass}
\chead{\Large \textbf{\printpset}}
\rhead{\printrealname}
\rfoot{Page \thepage}
\renewcommand{\headrulewidth}{0.5pt}
\renewcommand{\footrulewidth}{0.3pt}
\setlength{\textwidth}{6.5in}
\renewcommand{\baselinestretch}{1}
\setenumerate[0]{label=(\alph*)}
\newcommand{\todo}{\textcolor{red}{\textbf{TODO }}}

\newtheorem*{prop}{Proposition}
\newtheorem*{corollary}{Corollary}
\newtheorem{lemma}{Lemma}
\theoremstyle{remark}
\newtheorem*{defn}{Definition}
\newtheorem*{remark}{Remark}

\newtheoremstyle{named}{}{}{}{}{\bfseries}{.}{.5em}{\thmnote{Problem #3}}
\theoremstyle{named}
\newtheorem*{theorem}{Theorem}
\allowdisplaybreaks

%% DO NOT ALTER THE ABOVE LINES
%%%%%%%%%%%%%%%%%%%%%%%%%%%%%%%%%%%%%%%%%%%%


%% If you would like to use Asymptote within this document (which is optional), 
%% you can find out how at the following URL:
%%
%%   http://www.artofproblemsolving.com/Wiki/index.php/Asymptote:_Advanced_Configuration
%%
%% As explained there, you will want to uncomment the line below.  But be
%% sure to check the website because there are several other steps that must 
%% be followed.
%% \usepackage{asymptote}

%% Enter your real name here
%% Example: \realname{David Patrick}
\realname{Hanting Zhang}
\pset{W4P2}
\mathclass{425B}

\renewcommand{\a}{\alpha}
\renewcommand{\b}{\beta}
\renewcommand{\d}{\delta}
\newcommand{\e}{\varepsilon}
\newcommand{\Z}{\mathbb Z}
\newcommand{\N}{\mathbb N}
\newcommand{\Q}{\mathbb Q}
\newcommand{\R}{\mathbb R}
\newcommand{\C}{\mathbb C}
\renewcommand{\bf}{\mathbf}
\newcommand{\id}[1]{\text{id}_{#1}}
\renewcommand{\implies}{\Rightarrow}
\newcommand{\coimplies}{\Leftarrow}
\renewcommand{\em}{\varnothing}
\renewcommand{\Im}{\text{Im}}
\newcommand{\abs}[1]{|#1|}
\newcommand{\bigabs}[1]{\left|#1\right|}
\newcommand{\Rloc}{\mathcal R_{\text{loc}}}

%% should be reset
\newcommand{\supp}[1]{\text{supp}({#1})}

\begin{document}

\begin{theorem}[1.1]
    Suppose \(F = \R\) or \(\C\).
    \begin{enumerate}
        \item Show that \(F^X\) is finite-dimensional if and only if \(X\) is a finite set.
        \item The space of `sequences in \(F\) that are eventually zero' is an infinite-dimensional vector space. Give a more precise definition for this space; then give an example of a Hamel basis for it.
    \end{enumerate}
\end{theorem}

\begin{proof}
    We proceed with each part.
    \begin{enumerate}
        \item \(F^X\) is finite-dimensional if and only if there exists a basis of finite size. Note that all bases have the same cardinality (this may depend on AoC, I think). But \(X\) itself is trivially a basis of \(F^X\). Thus this occurs if and only if \(X\) is finite.
        \item Say a function \(f : A \to F\) has \textit{finite support} if the set \(\supp{f} = \{x \in A \mid f(x) \neq 0\}\) is finite. Sequences in \(F\) are just functions \(\N \to F\). They are eventually zero means that they must have finite support. Thus we must show that the set of finitely supported functions, \(\{f : \N \to F \mid \supp{f} \text{ is finite}\}\), is a vector space. 

        Indeed, let's just check the axioms. Let \(f g : \N \to F\), then it is easy to see that \(\supp{f + g} \subseteq \supp{f} \cup  \supp g\), which is finite. Thus addition is closed. Let \(k \in F\), it is also easy to see that \(\supp{kf} \subseteq \supp{f}\), which is finite. 
        Thus scalar multiplication is closed. We don't really need to check all the other commutative/associative/distributive axioms as they are tedious yet obvious. Thus \(\{f : \N \to F \mid \supp{f} \text{ is finite}\}\) is indeed a vector space, as desired.
    \end{enumerate}
\end{proof}

\begin{theorem}[2.1]
    Let \(F = \R\) or \(\C\). Prove that \((BC(X;F), \|\cdot\|_u)\) is a Banach space., using the Uniform Limit Theorem.
\end{theorem}

\begin{theorem}[2.2]
    Let \(F = \R\) or \(\C\). Use the results of this section, together with the completeness of \(F^n\) under the Euclidean norm \(\|\cdot\|\), to prove that any finite-dimensional normed \(F\)-vector space \((V, \|\cdot\|_V)\) is complete.
\end{theorem}

\begin{proof}
    Any finite-dimensional \(F\)-vector space \(V\) with dimension \(n\) is isomorphic (as a topological vector space) to \(F^n\). Thus any norm on \(V\) is equivalent to some norm on \(F^n\). Since \(F^n\) is complete, \(V\) is complete.
\end{proof}

\begin{theorem}[2.3]
    In this problem, we show that the \(L^2\) norm on \(C([0, 1])\) is strictly stronger than the \(L^1\) norm. (Actually, we show a little more.)
    \begin{enumerate}
        \item Assume \(1 \leq p < q < +\infty\). Prove that for any \(f \in C([a, b])\), we have
        \[\|f\|_{L^p([a, b])} \leq (b - a)^{\frac{1}{p} - \frac{1}{q}} \|f\|_{L^q([a, b])}.\]
        \item Prove that the continuous functions \(f_n(x) = n^2 (1/n - x) 1_{[-, 1/n]}(x)\) have constant \(L^1\) norm, but their \(L^2\) norm tends to \(+\infty\) as \(n \to \infty\). Conclude that the \(L^2\) norm is strictly stronger than the \(L^1\) norm on \(C([0, 1])\).
    \end{enumerate}
\end{theorem}

\begin{proof}
    We proceed with each part.
    \begin{enumerate}
        \item Apply H\"older's inequality on functions \(|f|^p\) and \(1\), with conjugates \(q / p\) and \(q / (q - p)\). Note that since \(p < q\), we have \(q / p, q / (q - p) > 1\) so these conjugates are valid. Now,
        \begin{align*}
            \int_{a}^{b} |f|^p dx \leq \left(\int_{a}^{b} (|f|^p)^{\frac{q}{p}} dx \right)^{\frac{p}{q}} \left(\int_{a}^{b} 1^{\frac{q}{q - p}} dx \right)^{\frac{q - p}{q}} = (b - a)^{\frac{q - p}{q}}\left(\int_{a}^{b} |f|^q dx \right)^{\frac{p}{q}}.
        \end{align*}
        Taking the \(p\)th root on both sides, we have
        \begin{align*}
            \|f\|_{L^p([a, b])} = \left(\int_{a}^{b} |f|^p dx\right)^\frac{1}{p} &\leq (b - a)^{\frac{q - p}{pq}}\left(\int_{a}^{b} |f|^q dx \right)^{\frac{1}{q}} \\ 
            &= (b - a)^{\frac{1}{p} - \frac{1}{q}}\left(\int_{a}^{b} |f|^q dx \right)^{\frac{1}{q}} \\
            &= (b - a)^{\frac{1}{p} - \frac{1}{q}} \|f\|_{L^q([a, b])},
        \end{align*}
        as desired.
        \item In the \(L^1\) norm, 
        \begin{align*}
            \|f_n\|_1 &= \int_{0}^{1/n} n^2 \left(\frac{1}{n} - x\right) dx = \int_{0}^{1/n} (n - n^2 x) dx \\
            &= 1 - n^2 \frac{(1/n)^2}{2} = \frac{1}{2}.
        \end{align*}
        So indeed the functions have constant \(L^1\) norm.

        In the \(L^2\) norm, 
        \begin{align*}
            \|f_n\|_2 &= \left(\int_{0}^{1/n} (n - n^2 x)^2 dx \right)^\frac{1}{2} \\
            &= \left(\left . \frac{n(nx - 1)^3}{3}\right |_0^{1/n}\right)^\frac{1}{2} \\
            &= \left(\frac{n}{3}\right)^\frac{1}{2}.
        \end{align*}
        Thus clearly \(\|f_n\|_2 \to \infty\) as \(n \to \infty\). This shows that \(L^1\) and \(L^2\) are not equivalent, and thus \(L^2\) must be stronger than \(L^1\).
    \end{enumerate}
\end{proof}

\begin{theorem}[2.4]
    What is the relationship between \(\|\cdot\|_{L^1([0, 1])}\) and \(\|\cdot\|_u\) on \(C([0, 1])\)? Justify your answer.
\end{theorem}

\begin{proof}
    \(\|\cdot\|_u\) is \textit{strictly stronger} than \(\|\cdot\|_1\). We simply have,
    \begin{align*}
        \|f\|_1 = \int_{0}^{1} |f| dx \leq \sup f \times 1 = \|f\|_u.
    \end{align*}
    And there are many functions where \(\sup f = \infty\) but \(\|f\|_1 < \infty\), so the two norms are definitely not equivalent. Thus \(\|\cdot\|_u\) is strictly stronger than \(\|\cdot\|_1\).
\end{proof}

\end{document}