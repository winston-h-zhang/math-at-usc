%%%%%%%%%%%%%%%%%%%%%%%%%%%%%%%%%%%%%%%%%%%%%%
%%                                          %%
%% USE THIS FILE TO SUBMIT YOUR SOLUTIONS   %%
%%                                          %%
%% You must have the usamts.tex file in     %%
%% the same directory as this file.         %%
%% You do NOT need to submit this file or   %%
%% usamts.tex with your solutions.  You     %%
%% only need to submit the output PDF file. %%
%%                                          %%
%% DO NOT ALTER THE FILE usamts.tex         %%
%%                                          %%
%% If you have any questions or problems    %%
%% using this file, or with LaTeX in        %%
%% general, please go to the LaTeX          %%
%% forum on the Art Of Problem Solving      %%
%% web site, and post your problem.         %%
%%                                          %%
%%%%%%%%%%%%%%%%%%%%%%%%%%%%%%%%%%%%%%%%%%%%%%

%%%%%%%%%%%%%%%%%%%%%%%%%%%%%%%%%%%%%%%%%%%%
%% DO NOT ALTER THE FOLLOWING LINES
\documentclass[12pt]{article}
\usepackage{amsmath,amssymb,amsthm,amsfonts,tabularx}
\usepackage[pdftex]{graphicx}
\graphicspath{ {./images/} }
\usepackage{fancyhdr}
\pagestyle{fancy}
\usepackage{setspace}
\usepackage{csquotes}
%%%%%%%%%%%%%%%%%%%%%%%%%%%%%%%%%%%%%%%%%%%
%%                                       %%
%% Students: DO NOT MODIFY THIS FILE!!!  %%
%%                                       %%
%%%%%%%%%%%%%%%%%%%%%%%%%%%%%%%%%%%%%%%%%%%

%% USAMTS style sheet
%% last modified: 23-Jul-2004

%% Student and Year/Round data
\newcommand{\realname}[1]{\newcommand{\printrealname}{#1}}
\newcommand{\pset}[1]{\newcommand{\printpset}{#1}}

%% Pagestyle setup
\setlength{\headheight}{0.75in}
\setlength{\oddsidemargin}{0in}
\setlength{\evensidemargin}{0in}
\setlength{\voffset}{-.5in}
\setlength{\headsep}{10pt}
\setlength{\textwidth}{6.5in}
\setlength{\headwidth}{6.5in}
\setlength{\textheight}{8in}
\lhead{Math 410}
\chead{\Large \textbf{Homework \printpset}}
\rhead{\printrealname}
\rfoot{Page \thepage}
\renewcommand{\headrulewidth}{0.5pt}
\renewcommand{\footrulewidth}{0.3pt}
\setlength{\textwidth}{6.5in}


\renewcommand{\baselinestretch}{1}
%% DO NOT ALTER THE ABOVE LINES
%%%%%%%%%%%%%%%%%%%%%%%%%%%%%%%%%%%%%%%%%%%%


%% If you would like to use Asymptote within this document (which is optional), 
%% you can find out how at the following URL:
%%
%%   http://www.artofproblemsolving.com/Wiki/index.php/Asymptote:_Advanced_Configuration
%%
%% As explained there, you will want to uncomment the line below.  But be
%% sure to check the website because there are several other steps that must 
%% be followed.
%% \usepackage{asymptote}

\newtheorem*{prop}{Proposition}
\newtheorem*{corollary}{Corollary}
\newtheorem*{lemma}{Lemma}
\theoremstyle{remark}
\newtheorem*{defn}{Definition}

\newtheoremstyle{named}{}{}{}{}{\bfseries}{.}{.5em}{\thmnote{Problem #3}}
\theoremstyle{named}
\newtheorem*{theorem}{Theorem}

%% Enter your real name here
%% Example: \realname{David Patrick}
\realname{Hanting Zhang}
\pset{MIDTERM}
\mathclass{425A}

\renewcommand{\bf}{\mathbf}
\renewcommand{\implies}{\Rightarrow}
\newcommand{\coimplies}{\Leftarrow}

\begin{document}

\begin{theorem}[1]
    If \(x \in \mathbb R^k\), \(k \ge 2\), show that there exists \(y \in \mathbb R^k\) such that \(x \cdot y = 0\).
\end{theorem}

\begin{proof}
    One possible solution is simply \(y = 0\); hence trivially \(x \cdot y = 0\). But that's not very interesting so I'll give a nontrivial example.

    If \(x\) is trivial then \(y\) can be anything, so assume that \(x\) is nontrivial. Then there is at least one index \(i\) where \(x_i \neq 0\). Since \(n \ge 2\), pick some \(j \neq i\). Set \(y_i = -x_j\) and \(y_j = x_i\), and \(y_k = 0\) for all \(k \neq i\) and \(k \neq j\). Hence this defines \(y\) as nontrivial. Then 
    \[x \cdot y = \sum_{k = 1}^n x_k y_k = x_i y_i + x_j + y_j = - x_i x_j + x_j x_i = 0.\]
\end{proof}

\begin{theorem}[2]
    True or False: If true prove it, if false counterexample it.
    \begin{enumerate}
        \item [(a)] Let \(\{F_n\}\) be a countable collection of closed subsets of \(\mathbb R\) such that for any finite sub-collection 
        \[F_{n_1} \cap F_{n_2} \cap \dots \cap F_{n_k} \neq \varnothing.\]
        Then 
        \[\bigcap_{n = 1}^\infty F_n \neq \varnothing.\]
        \item [(b)] Add the condition that each \(F_n\) is bounded and repeat (2a).
        \item [(c)] Repeat (2a) where closed and bounded \(F_n \subseteq X\), and arbitrary metric space.
    \end{enumerate}
\end{theorem}

\begin{proof}
    Part (a): This claim is \textbf{false}. Consider the subsets \(F_n = [n, \infty)\). Then for any finite sub-collection \(F_{n_1}, F_{n_2}, \dots, F_{n_k}\), let \(n = \max_{k} n_k\). We can compute the intersection to be: 
    \[F_{n_1} \cap F_{n_2} \cap \dots \cap F_{n_k} = F_{n} \neq \varnothing.\]
    However, since for any \(x \in \mathbb R\) we may find some \(n \ge x\), there is always some \(F_n\) such that \(x \notin F_n\). Hence 
    \[\bigcap_{n=1}^\infty F_n = \varnothing.\]
    This disproves the claim.
\end{proof}

\begin{proof}
    Part (b): This claim is \textbf{true}. If \(F_n\) are both closed and bounded subsets of \(\mathbb R\), then the Heine-Borel Theorem guarantees that \(F_n\) is compact. Now apply Theorem 2.36 fromthe textbook to conclude that 
    \[\bigcap_{n = 1}^\infty F_n \neq \varnothing.\]
\end{proof}

\begin{proof}
    Part (c): This claim is \textbf{false}. Let \(X = \mathbb Q\) with the relative topology inherited from \(\mathbb R\). Then consider the subsets \(F_n = \overline{B_{1/ n}(\sqrt 2)}\) as the closed balls centered at \(\sqrt 2\) with radius \(1 / n\), where \(p_n\) is tne \(n\)th prime. 
    In particular, since \(\sqrt 2 \pm 1 / n\) are irrational, the boundary points of \(F_n\) don't exist in \(\mathbb Q\), and hence we can drop them without changing anything: \(F_n = B_{1/n}(\sqrt 2)\).

    Now we check that finite intersections are nonempty. Indeed, if \(F_{n_1}, F_{n_2}, \dots, F_{n_k}\) are a finite sub-collection, then we their intersection is just the ball of minimum radius \(r = \min_{k} (1/n_k)\). This radius is clearly greater than \(0\), so we know that \(F_{n_1} \cap F_{n_2} \cap \dots \cap F_{n_k} \neq \varnothing\).

    However, if we consider \(\bigcap_{n = 1}^\infty F_n\), then for any \(x \neq \sqrt 2\), we can find a \(k\) such that \(k > 1 / |x - \sqrt 2|\). This implies \(1 / n < |x - \sqrt 2|\). Hence by definition \(x \notin F_k\), so \(x \notin \bigcap_{n = 1}^\infty F_n\). But \(\sqrt 2\) is not in \(\mathbb Q\)! Hence in \(\mathbb Q\), the intersection \(\bigcap_{n = 1}^\infty F_n\) is empty. This disproves the claim.
\end{proof}

\begin{theorem}[3]
    Consider the metric space \(\mathbb Q\) of all rationals on the real line with the Euclidean metric. Prove that if \(K \neq \varnothing\) is a compact subset of \(\mathbb Q\) then \(K\) cannot contain an open subset of \(\mathbb Q\). \textit{Hint: Consider the relative topology.}
\end{theorem}

\begin{proof}
    bruh
\end{proof}

\begin{theorem}[4]
    Let \(F\) and \(K\) be nonempty closed subsets of the metric space \(X\) with \(K \cap F = \varnothing\). Show that if \(K\) is compact there is a positive distance from \(F\) to \(K\), i.e.
    \[\inf \{d(x, y) \mid x \in F, y \in K\} = \delta > 0.\]
    Is is still true if \(K\) is only assumed to be closed? If not find a counterexample.
\end{theorem}

\begin{theorem}[5]
    A \textit{base} for a topological space \(X\) is a collection \(\{V_{\alpha} | \alpha \in A\}\) of open subsets of \(X\) such that for every open subset of \(G \subseteq X\), one has \(G = \bigcup_{\alpha \in \mathcal B} V_{\alpha}\) where \(\mathcal B \subseteq \mathcal A\).

    Prove that every compact metric space \(X\) has a countable base.
\end{theorem}

\begin{proof}
    For each \(q > 0 \in \mathbb Q\), consider the collection of subsets \(\mathcal C_q = \{B_q(x) \mid x \in X\}\). For each \(q\), this clearly defines an open cover of \(X\), so we may construct a finite subcover \(\mathcal D_q = \{B_{q_1}(x_1), B_{q_2}(x_2), \dots, B_{q_n}(x_n)\}\). In articular, we have a \textit{countable} amount \textit{finite} covers, so their union is countable. Given this, we claim that 
    \[\bigcup_{q \in \mathbb Q} \mathcal D_q\]
    forms a countable base of \(X\). 
\end{proof}

\end{document}