\documentclass[12pt]{article}
\usepackage{amsmath,amssymb,amsthm,amsfonts,tabularx}
\usepackage[pdftex]{graphicx}
\usepackage[dvipsnames]{xcolor}
\usepackage{fancyhdr}
\usepackage{parskip}
\usepackage[shortlabels]{enumitem}
\pagestyle{fancy}
\usepackage{setspace}
\usepackage{pgfplots}
\pgfplotsset{compat=1.18}
\newcommand{\realname}[1]{\newcommand{\printrealname}{#1}}
\newcommand{\pset}[1]{\newcommand{\printpset}{#1}}
\newcommand{\mathclass}[1]{\newcommand{\printmathclass}{#1}}

%% Pagestyle setup
\setlength{\headheight}{0.75in}
\setlength{\oddsidemargin}{0in}
\setlength{\evensidemargin}{0in}
\setlength{\voffset}{-.5in}
\setlength{\headsep}{10pt}
\setlength{\textwidth}{6.5in}
\setlength{\headwidth}{6.5in}
\setlength{\textheight}{8in}
\lhead{Math \printmathclass}
\chead{\Large \textbf{\printpset}}
\rhead{\printrealname}
\rfoot{Page \thepage}
\renewcommand{\headrulewidth}{0.5pt}
\renewcommand{\footrulewidth}{0.3pt}
\setlength{\textwidth}{6.5in}
\renewcommand{\baselinestretch}{1}
\setenumerate[0]{label=(\alph*)}
\newcommand{\todo}{\textcolor{red}{\textbf{TODO }}}

\newtheorem*{prop}{Proposition}
\newtheorem*{corollary}{Corollary}
\newtheorem{lemma}{Lemma}
\theoremstyle{remark}
\newtheorem*{defn}{Definition}
\newtheorem*{remark}{Remark}

\newtheoremstyle{named}{}{}{}{}{\bfseries}{.}{.5em}{\thmnote{Problem #3}}
\theoremstyle{named}
\newtheorem*{theorem}{Theorem}
\allowdisplaybreaks

%% DO NOT ALTER THE ABOVE LINES
%%%%%%%%%%%%%%%%%%%%%%%%%%%%%%%%%%%%%%%%%%%%


%% If you would like to use Asymptote within this document (which is optional), 
%% you can find out how at the following URL:
%%
%%   http://www.artofproblemsolving.com/Wiki/index.php/Asymptote:_Advanced_Configuration
%%
%% As explained there, you will want to uncomment the line below.  But be
%% sure to check the website because there are several other steps that must 
%% be followed.
%% \usepackage{asymptote}

%% Enter your real name here
%% Example: \realname{David Patrick}
\realname{Hanting Zhang}
\pset{W9P1}
\mathclass{425B}

\renewcommand{\a}{\alpha}
\renewcommand{\b}{\beta}
\renewcommand{\d}{\delta}
\newcommand{\e}{\varepsilon}
\newcommand{\Z}{\mathbb Z}
\newcommand{\N}{\mathbb N}
\newcommand{\Q}{\mathbb Q}
\newcommand{\R}{\mathbb R}
\newcommand{\C}{\mathbb C}
\renewcommand{\bf}{\mathbf}
\newcommand{\id}[1]{\text{id}_{#1}}
\renewcommand{\implies}{\Rightarrow}
\newcommand{\coimplies}{\Leftarrow}
\renewcommand{\em}{\varnothing}
\renewcommand{\Im}{\text{Im}}
\newcommand{\abs}[1]{|#1|}
\newcommand{\bigabs}[1]{\left|#1\right|}
\newcommand{\Rloc}{\mathcal R_{\text{loc}}}

%% should be reset

\begin{document}

\begin{theorem}[1.1]
    In the proof of Weierstrass's Polynomial Approximation Theorem above, we added the additional simplifying hypotheses that \([a, b] = [0, 1]\) and \(f(0) = f(1) = 0\). Finish the proof of the original statement, with this special case in hand.
\end{theorem}

\begin{proof}
    Let \(t : [0, 1] \to [a, b]\) be a bijection defined by \(\lambda \mapsto \lambda a + (1 - \lambda) b\). Let \(T : C([0, 1]) \to C([a, b])\) extend \(t\) via \(T(f)(x) = f(t^{-1}(x))\). Note that \(T\) is a bijection, thus for any \(f \in C([a, b])\), we have \(T^{-1}(f) \in C([0, 1])\), so there exists a sequence of polynomials \(P_n \to f\). Then 
    \begin{align*}
        T(P_n)(x) = P_n(t^{-1}(x)) \to f(t^{-1}(x)) = T(f)(x),
    \end{align*}
    as desired.
\end{proof}

\begin{theorem}[1.2]
    Assume \(f \in C([a, b])\).
    \begin{enumerate}
        \item Assume that \(\int_{a}^{b} f(x) x^n dx = 0\) for all \(n \in \N\). Prove that \(f(x) = 0\) for all \(x \in [a, b]\).

        \item Now assume that instead that \(\int_{a}^{b} f(x) x^n dx = 0\) for all \(n \geq N\), \(n \in \N\). Can you still conclude that \(f \equiv 0\) on \([a, b]\)? Prove your answer is correct.
    \end{enumerate}
\end{theorem}

\begin{proof}
    We proceed with each part separately.
    \begin{enumerate}
        \item Suppose \(P_n \to f\) by the Weierstrass Approximation Theorem. We have 
        \begin{align*}
            \int_{a}^{b} f(x) f(x) dx &= \int_{a}^{b} (\lim_{n \to \infty} P_n) f(x) dx \\
            \text{(because \(P_n \to f\) uniformly)} \hspace*{2mm} &= \lim_{n \to \infty} \int_{a}^{b} P_n f(x) dx \\
            &= \lim_{n \to \infty}\int_{a}^{b} \left(\sum_{k = 0}^{\infty} a^{(n)}_k x^k\right) f(x) dx \\
            &= \lim_{n \to \infty} \sum_{k = 0}^{\infty} a^{(n)}_k \int_{a}^{b} x^k f(x) dx \\
            &= \lim_{n \to \infty} \sum_{k = 0}^{\infty} a^{(n)}_k 0 = \lim_{n \to \infty} 0 = 0.
        \end{align*}
        But this means \(\int_{a}^{b} f^2(x) dx = 0\) and \(f^2(x) \geq 0\) for all \(x\), thus this is only possible if \(f^2(x) = 0\). This of course implies \(|f(x)| = 0\), and \(f(x) = 0\) for all \(x\).
        \item Yes. Apply part (a) to \(g(x) = x^N f(x)\) to conclude that \(g = 0\). Thus \(f = 0\).
    \end{enumerate}
\end{proof}

\begin{theorem}[1.3]
    Suppose that \(f \in C([1, \infty))\) and \(\lim_{x \to +\infty} f(x) = a \in \R\). Prove that for any \(\e > 0\), there is a polynomial \(p\) such that \(\sup_{x \in [1, \infty)} \abs{p(1/x) - f(x)} < \e\).
\end{theorem}

\begin{proof}
    Let \(g : [0, 1] \to \R\) and define \(g(x) = f(1/x)\) when \(x \in (0, 1]\) and \(g(0) = a\). Then \(f(x) = g(1/x)\). By the Weierstrass Approximation Theorem, there exists some polynomial \(p\) such that \(\|p - g\|_u < \e\). Now, then
    \begin{align*}
        \sup_{x \in [1, \infty)} \abs{p(1/x) - f(x)} = \sup_{x \in [1, \infty)} \abs{p(1/x) - g(1/x)} = \sup_{y \in [0, 1]} \abs{p(y) - g(y)} < \e,
    \end{align*} 
    as desired.
\end{proof}

\begin{theorem}[1.4]
    Suppose that \(f \in C^1([a, b])\), that is, \(f\) is continuously differentiable on the interval \([a, b]\). Recall the ``\(C^1\) norm,'' defined by 
    \begin{align*}
        \|f\|_{C^1([a, b])} = \sup_{x \in [a, b]}\abs{f(x)} + \sup_{x \in [a, b]} \abs{f'(x)}.
    \end{align*}
    Prove that for any \(\e > 0\), there exists a polynomial \(p\) such that \(\|f - p\|_{C^1([a, b])} < \e\).
\end{theorem}

\begin{proof}
    By the Weierstrass Approximation Theorem, there is some polynomial \(q\) such that \(\|q - f'\|_u < \frac{\e}{2(b - a)}\). Then let \(p(x) = \int_a^x q\) and
    \begin{align*}
        \|f - p\|_{C^1([a, b])} &= \sup_{x \in [a, b]}\abs{f(x) - p(x)} + \sup_{x \in [a, b]} \abs{f'(x) - q(x)} \\
        &< \sup_{x \in [a, b]}\bigabs{\int_{a}^{x} f'(y) - q(y) dy} + \frac{\e}{2(b - a)} \\
        &< \sup_{x \in [a, b]}\int_{a}^{x} \bigabs{\frac{\e}{2(b - a)}} dy + \frac{\e}{2(b - a)} \\
        &= \frac{\e}{2} + \frac{\e}{2(b - a)} < \e,
    \end{align*}
    as desired.
\end{proof}

\end{document}