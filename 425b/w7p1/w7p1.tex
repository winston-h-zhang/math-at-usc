\documentclass[12pt]{article}
\usepackage{amsmath,amssymb,amsthm,amsfonts,tabularx}
\usepackage[pdftex]{graphicx}
\usepackage[dvipsnames]{xcolor}
\usepackage{fancyhdr}
\usepackage{parskip}
\usepackage[shortlabels]{enumitem}
\pagestyle{fancy}
\usepackage{setspace}
\newcommand{\realname}[1]{\newcommand{\printrealname}{#1}}
\newcommand{\pset}[1]{\newcommand{\printpset}{#1}}
\newcommand{\mathclass}[1]{\newcommand{\printmathclass}{#1}}

%% Pagestyle setup
\setlength{\headheight}{0.75in}
\setlength{\oddsidemargin}{0in}
\setlength{\evensidemargin}{0in}
\setlength{\voffset}{-.5in}
\setlength{\headsep}{10pt}
\setlength{\textwidth}{6.5in}
\setlength{\headwidth}{6.5in}
\setlength{\textheight}{8in}
\lhead{Math \printmathclass}
\chead{\Large \textbf{\printpset}}
\rhead{\printrealname}
\rfoot{Page \thepage}
\renewcommand{\headrulewidth}{0.5pt}
\renewcommand{\footrulewidth}{0.3pt}
\setlength{\textwidth}{6.5in}
\renewcommand{\baselinestretch}{1}
\setenumerate[0]{label=(\alph*)}
\newcommand{\todo}{\textcolor{red}{\textbf{TODO }}}

\newtheorem*{prop}{Proposition}
\newtheorem*{corollary}{Corollary}
\newtheorem{lemma}{Lemma}
\theoremstyle{remark}
\newtheorem*{defn}{Definition}
\newtheorem*{remark}{Remark}

\newtheoremstyle{named}{}{}{}{}{\bfseries}{.}{.5em}{\thmnote{Problem #3}}
\theoremstyle{named}
\newtheorem*{theorem}{Theorem}
\allowdisplaybreaks

%% DO NOT ALTER THE ABOVE LINES
%%%%%%%%%%%%%%%%%%%%%%%%%%%%%%%%%%%%%%%%%%%%


%% If you would like to use Asymptote within this document (which is optional), 
%% you can find out how at the following URL:
%%
%%   http://www.artofproblemsolving.com/Wiki/index.php/Asymptote:_Advanced_Configuration
%%
%% As explained there, you will want to uncomment the line below.  But be
%% sure to check the website because there are several other steps that must 
%% be followed.
%% \usepackage{asymptote}

%% Enter your real name here
%% Example: \realname{David Patrick}
\realname{Hanting Zhang}
\pset{W7P1}
\mathclass{425B}

\renewcommand{\a}{\alpha}
\renewcommand{\b}{\beta}
\renewcommand{\d}{\delta}
\newcommand{\e}{\varepsilon}
\newcommand{\Z}{\mathbb Z}
\newcommand{\N}{\mathbb N}
\newcommand{\Q}{\mathbb Q}
\newcommand{\R}{\mathbb R}
\newcommand{\C}{\mathbb C}
\renewcommand{\bf}{\mathbf}
\newcommand{\id}[1]{\text{id}_{#1}}
\renewcommand{\implies}{\Rightarrow}
\newcommand{\coimplies}{\Leftarrow}
\renewcommand{\em}{\varnothing}
\renewcommand{\Im}{\text{Im}}
\newcommand{\abs}[1]{|#1|}
\newcommand{\bigabs}[1]{\left|#1\right|}
\newcommand{\Rloc}{\mathcal R_{\text{loc}}}

%% should be reset

\begin{document}

\begin{theorem}[4.2]
    Verify that the sequence (23) is Cauchy in \(L^p_{\mathcal R}(\R)\).
    \begin{align*}
        f_n = \frac{1_{(1/n, 1]}}{x^{1/2p}}.
    \end{align*}
\end{theorem}

\begin{proof}
    Without loss of generality for \(n > m > N\), \(N\) to be fixed later, we have
    \begin{align*}
        \|f_n - f_m\|_{L_p}^p &= \int_{-\infty}^{\infty} \bigabs{\frac{1_{(1/n, 1]} - 1_{(1/m, 1]}}{x^{1/2p}}}^p dx \\
        &= \int_{-\infty}^{\infty} \bigabs{\frac{1_{(1/n, 1/m]}}{x^{1/2}}} dx \\
        &= \int_{1/n}^{1/m} \frac{1}{|x|^{1/2}} dx \\
        &= 2 \sqrt{1/m} - 2 \sqrt{1/n}.
    \end{align*}
    Next, we may bound
    \begin{align*}
        2 \sqrt{1/m} - 2 \sqrt{1/n} = \frac{2(\sqrt n - \sqrt m)}{\sqrt{nm}} \leq \frac{2\sqrt n}{\sqrt{nm}} = \frac{2}{\sqrt m} < \frac{2}{\sqrt N}.
    \end{align*}
    Thus choose \(N > 4 / \e^2\), and we conclude \(\|f_n - f_m\|_{L_p}^p < \e\), which suffices.
\end{proof}

\begin{theorem}[5.1]
    Suppose \(f \in \mathcal{R}_{\text{loc}}(\R)\) is \(T\)-periodic, and assume that \(1 \leq p < \infty\). Show that given \(\e > 0\) there exists a \textit{continuous} \(T\)-periodic function \(g\) such that \(\|g\|_u \leq 4 \|f\|_u\) and 
    \begin{align*}
        \int_{0}^{T} \abs{f(x) - g(x)}^p dx < \e.
    \end{align*}
\end{theorem}

\begin{proof}
    We can follow the constructions of Corollary 5.2 and Lemma 5.3, while taking extra care to maintain the periodicity of each construction. Indeed, since \(f \in \Rloc(\R)\), we have \(f|_{[0, T]} \in \mathcal R([0, T])\). Thus there is some step function
    \begin{align*}
        \ell|_{[0, T]} = \sum_{j = 1}^n m_j 1_{[p_{j - 1}, p_j)}
    \end{align*}
    such that \(\|f|_{[0, T]} - \ell|_{[0, T]}\|_{L^1([0, T])} < \e\) and \(\|\ell|_{[0, T]}\|_u \leq 2 \|f|_{[0, T]}\|_u\). Since \(f\) is \(T\)-periodic, we can ``copy-paste'' this construction across every \(T\) interval and obtain an \(T\)-periodic extension \(\ell\) of \(\ell|_{[0, T]}\). We also have 
    \begin{align*}
        \|\ell\|_u = \max_n \|\ell|_{[nT, (n + 1)T]}\|_u \leq \max_n 2\|f|_{[nT, (n+1)T]}\| = 2\|f\|_u.
    \end{align*}

    Next, from Corollary 5.2 we may construct from \(\ell|_{[0, T]}\) a continuous function \(g|_{[0, T]} = \sum_{j = 1}^{n}c_j g_j\). Then it is easy to see that 
    \begin{align*}
        g = \sum_{n}^{} g|_{[nT, (n + 1)T]}
    \end{align*}
    is a \(T\)-periodic function. We also have \(\|\ell - g\|_{L^1([0, T])} < \e\) and 
    \begin{align*}
        \|g\|_u = \max_n \|g\|_{[nT, (n + 1)T]}\|_u \leq \max_n 2\|\ell|_{[nT, (n+1)T]}\| = 2\|\ell\|_u.
    \end{align*}
    (Actually, there is a bit of ``leakage'' across the boundary points at \(0, T, 2T, \ldots\), so the equations are not exactly true -- but these leakages don't effect the uniform norm.)

    Together, we conclude that there is some \(T\)-periodic function \(g\) such that \(\|g\|_u \leq 4 \|f\|_u\) and \(\|f - g\|_{L^1([0, T])} < \e\) (approx.).

    To extend this to all \(1 \leq p < \infty\), note that \(f \in \mathcal R([0, T])\) implies that \(f\) in bounded. Thus \(g\) is bounded by construction. Thus we can bound \(|f(x) - g(x)|^{p - 1} \leq L\) on \([0, T]\), and therefore
    \begin{align*}
        \int_{0}^{T} \abs{f(x) - g(x)}^p dx &= \int_{0}^{T} \abs{f(x) - g(x)}\abs{f(x) - g(x)}^{p - 1} dx \\
        &\leq TL \int_{0}^{T} \abs{f(x) - g(x)}dx < \e.
    \end{align*} 
\end{proof}

\begin{theorem}[1.1]
    Assume that \(\phi : \R \to \R\) and \(f : \R \to \R\) satisfy the following:
    \begin{enumerate}
        \item \(\phi\) is compactly supported and continuously differentiable.
        \item \(f\) is compactly supported and is Riemann integrable on an interval containing its support.
    \end{enumerate}
    Prove that the function \(g : \R \to \R\) defined \(g = \phi * f\) is continuously differentiable on \(\R\), with derivative \(g' = \phi' * f\). What can you say about the case where \(\phi\) and \(f\) are \(k\) and \(\ell\) times continuously differentiable, respectively (and still both be compactly supported)?
\end{theorem}

\begin{proof}
    We have 
    \begin{align*}
        g'(x) &= \lim_{h \to 0} \left(\frac{\phi * f (x + h) - \phi * f (x)}{h}\right) \\
        &= \lim_{h \to 0} \frac{1}{h} \left(\int_{-\infty}^{\infty}\phi(x + h - y) f(y) dy - \int_{-\infty}^{\infty}\phi(x - y) f(y) dy\right) \\
        &=  \lim_{h \to \infty} \int_{-\infty}^{\infty} \left( \frac{\phi(x - y + h) - \phi(x - y)}{h}\right)f(y) dy.
    \end{align*}
    Now we want to show that \(g'(x) - \phi' * f < \e\), so consider
    \begin{align*}
        g'(x) - \phi' * f &= \lim_{h \to \infty} \int_{-\infty}^{\infty} \left( \frac{\phi(x - y + h) - \phi(x - y)}{h} - \phi'(x - y) \right)f(y) dy \\
        &= \lim_{h \to \infty} \int_{-\infty}^{\infty} E_h(y)f(y) dy,
    \end{align*}
    where \(E_h(y) = \frac{\phi(x - y + h) - \phi(x - y)}{h} - \phi'(x - y)\).
    Now we can see that pointwise as \(h \to 0\),
    \begin{align*}
        \frac{\phi(x - y + h) - \phi(x - y)}{h} - \phi'(x - y) \to 0.
    \end{align*}
    However, this is insufficient to move change the order of the limit/integration. But note that 
    \begin{align*}
        E_h(y) &= \frac{\phi(x - y + h) - \phi(x - y)}{h} - \phi'(x - y) \\
        &= \frac{1}{h} \left( \int_{x - y}^{x - y + h} \phi'(t) dt - \int_{x - y}^{x - y + h} \phi'(x - y) dt \right) \\
        &= \frac{1}{h} \left( \int_{x - y}^{x - y + h} \phi'(t) - \phi'(x - y) dt \right).
    \end{align*}
    Since \(\phi\) is compactly supported and continuously differentiable, this implies that \(\phi'\) is uniformly continuous. Thus we see that in fact \(E_h(y) \to 0\) uniformly as \(h \to 0\). Furthermore, consider the fact that \(f\) has compact support, so we can limit the bounds of integration to some \([-L, L]\), and thus we may move the limit in:
    \begin{align*}
        g'(x) - \phi' * f &= \lim_{h \to \infty} \int_{-\infty}^{\infty} E_h(y)f(y) dy \\
        &= \lim_{h \to \infty} \int_{-L}^{L} E_h(y)f(y) dy \\
        &= \int_{-L}^{L} \lim_{h \to \infty} E_h(y) f(y) dy \\
        &= \int_{-L}^{L} 0 f(y) dy = 0.
    \end{align*}
    Thus \(g'(x) = \phi' * f\), as desired.
\end{proof}

\end{document}