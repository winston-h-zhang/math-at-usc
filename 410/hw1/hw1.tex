\documentclass{article}
\usepackage[pdftex]{graphicx}
\usepackage{algorithm}
\usepackage[noend]{algorithmic}
\usepackage{tikz}
\usepackage{graphicx}
\usepackage{subcaption}
\usepackage{amsmath,amssymb}
\usepackage{amsfonts}
\usepackage{xcolor}
\usepackage{fancyvrb}
\usepackage{color}
\usepackage{blindtext}
\usepackage{titlesec}
\setlength{\topmargin}{0 in}
\setlength{\oddsidemargin}{0 in}
\setlength{\textheight}{8.5 in}
\setlength{\textwidth}{6.5 in}
\setlength{\parindent}{0 in}

\titleformat*{\section}{\LARGE\bfseries}
\renewcommand{\o}[1]{$\overline{#1}$}
\renewcommand{\O}[1]{$\mathcal{O}(#1)$} %% For the clear looking O notation.

\definecolor{darkgreen}{rgb}{0,0.42,0}
\newcommand{\answer}[1]{} 
\newcommand{\rubric}[1]{{\leavevmode\color{brown}#1}}
\newcommand{\ZZ}[1]{\mathbb Z / #1 \mathbb Z}
\newcommand{\notes}[1]{\textcolor{red}{#1}}

\DeclareUnicodeCharacter{2212}{-}

\title{\textbf{Math 410 Homework 2}}
\author{Due Date: \textbf{something}}
\date{} 
\begin{document}
\maketitle

Exercises 8, 9, 12, 26, 36, pp. 21-23.

\begin{enumerate}
    \item [8.] 
    \begin{enumerate}
        \item [(a)] Since $G$ is a subset of $\mathbb C$, it suffices to prove that $G$ is a subgroup of $\mathbb C$. For any $g, h \in G$, we need to show that $gh^{-1} \in G$. By definiton there exist some $n, m \in \mathbb Z^+$ such that $g^n = h^m = 1$. 
        
        We want to find $N$ such that $(gh^{-1})^N = 1$. Let $N = nm$ and notice that $(gh^{-1})^{nm} = g^{nm} * h^{-nm} = (g^n)^m * (h^m)^{-n} = 1^m * 1 ^{-n} = 1$. Hence $gh^{-1} \in G$, and $G$ is a subgroup of $\mathbb C$, which makes it a group in general.
        \item [(b)] 
    \end{enumerate}

    \item [9.]
    \item \begin{enumerate}
        \item [(a)] Again we prove that $G$ is a group by proving it is a subgroup of $\mathbb R$. 
        
        For any $g, h \in G$, there are some $a, b, c, d \in \mathbb Q$ with $g = a + b \sqrt 2$ and $h = c \sqrt d$. then $g - h = (a - c) + (b - d)\sqrt 2$. Clearly $a - c$ and $b - d$ are rational, so $g - h \in G$, as desired. 

        \item [(b)] Let $g$ be a non-zero element of $G$ such that $a + b \sqrt 2 = g$ for some $a, b \in \mathbb Q$ (where $a$ and $B$ are not both $0$). Then note that $1 / g = 1 / (a + b \sqrt 2)$ is in $G$, since \[\frac 1 {a + b \sqrt 2} = \frac {a - b\sqrt 2} {(a + b \sqrt 2) (a - b \sqrt 2)} = \frac {a - b \sqrt 2} {a^2 - 2b^2}.\] Letting $x = \frac a {a^2 - 2b^2}$ and $y = \frac {-b} {a^2 - 2b^2}$, we have $1 / g = x + y \sqrt 2$. 
        Both $x$ and $y$ are rational, since they are made up of rational expressions. Hence $1 / g$ (in $\mathbb R$) is the inverse og $g$ in $G$. 

        \item[Note.] This makes $G$ a \textit{field}. In fact it is the field $\mathbb Q[\sqrt 2]$, the result of adjoining $\sqrt 2$ to $\mathbb Q$. 
    \end{enumerate}
    
    \item [12.] We can just calculate the orders:
    \begin{align*}    
        |\overline{1}| =& 0 \\
        \overline{-1}^2 = 1, |\overline {-1}| =& 2 \\
        
    \end{align*}
    
\end{enumerate}

Exercises 3, 9, pp. 27-28.
Exercises 2, 4, 13, 16, 20, pp. 32-34.
Exercises 17, 18, pp. 40.
Exercises 18, 19, pp. 45.

\end{document}