\documentclass[12pt]{article}
\usepackage{amsmath,amssymb,amsthm,amsfonts,tabularx}
\usepackage[pdftex]{graphicx}
\usepackage[dvipsnames]{xcolor}
\usepackage{fancyhdr}
\usepackage{parskip}
\usepackage[shortlabels]{enumitem}
\pagestyle{fancy}
\usepackage{setspace}
\newcommand{\realname}[1]{\newcommand{\printrealname}{#1}}
\newcommand{\pset}[1]{\newcommand{\printpset}{#1}}
\newcommand{\mathclass}[1]{\newcommand{\printmathclass}{#1}}

%% Pagestyle setup
\setlength{\headheight}{0.75in}
\setlength{\oddsidemargin}{0in}
\setlength{\evensidemargin}{0in}
\setlength{\voffset}{-.5in}
\setlength{\headsep}{10pt}
\setlength{\textwidth}{6.5in}
\setlength{\headwidth}{6.5in}
\setlength{\textheight}{8in}
\lhead{Math \printmathclass}
\chead{\Large \textbf{Final Exam}}
\rhead{\printrealname}
\rfoot{Page \thepage}
\renewcommand{\headrulewidth}{0.5pt}
\renewcommand{\footrulewidth}{0.3pt}
\setlength{\textwidth}{6.5in}
\renewcommand{\baselinestretch}{1}
\setenumerate[0]{label=(\alph*)}
\newcommand{\todo}{\textcolor{red}{\textbf{TODO }}}

\newtheorem*{prop}{Proposition}
\newtheorem*{corollary}{Corollary}
\newtheorem*{lemma}{Lemma}
\theoremstyle{remark}
\newtheorem*{defn}{Definition}

\newtheoremstyle{named}{}{}{}{}{\bfseries}{.}{.5em}{\thmnote{Problem #3}}
\theoremstyle{named}
\newtheorem*{theorem}{Theorem}
\allowdisplaybreaks

%% DO NOT ALTER THE ABOVE LINES
%%%%%%%%%%%%%%%%%%%%%%%%%%%%%%%%%%%%%%%%%%%%


%% If you would like to use Asymptote within this document (which is optional), 
%% you can find out how at the following URL:
%%
%%   http://www.artofproblemsolving.com/Wiki/index.php/Asymptote:_Advanced_Configuration
%%
%% As explained there, you will want to uncomment the line below.  But be
%% sure to check the website because there are several other steps that must 
%% be followed.
%% \usepackage{asymptote}

%% Enter your real name here
%% Example: \realname{David Patrick}
\realname{Hanting Zhang}
\pset{13}
\mathclass{425A}

\renewcommand{\a}{\alpha}
\renewcommand{\b}{\beta}
\renewcommand{\d}{\delta}
\newcommand{\e}{\varepsilon}
\newcommand{\Z}{\mathbb Z}
\newcommand{\N}{\mathbb N}
\newcommand{\Q}{\mathbb Q}
\newcommand{\R}{\mathbb R}
\newcommand{\C}{\mathbb C}
\renewcommand{\bf}{\mathbf}
\newcommand{\id}[1]{\text{id}_{#1}}
\renewcommand{\implies}{\Rightarrow}
\newcommand{\coimplies}{\Leftarrow}
\renewcommand{\em}{\varnothing}
\renewcommand{\Im}{\text{Im}}
\newcommand{\abs}[1]{|#1|}
\newcommand{\bigabs}[1]{\left|#1\right|}

\begin{document}

\begin{theorem}[1]
    True or False: If true prove it, if false counterexample it.
    \begin{enumerate}
        \item [(a)] Let \(\{F_n\}\) be a countable collection of closed subsets of \(\mathbb R\) such that for any finite sub-collection 
        \[F_{n_1} \cap F_{n_2} \cap \dots \cap F_{n_k} \neq \varnothing.\]
        Then 
        \[\bigcap_{n = 1}^\infty F_n \neq \varnothing.\]
        \item [(b)] Add the condition that each \(F_n\) is bounded and repeat (2a).
        \item [(c)] Repeat (1a) where closed and bounded \(F_n \subseteq X\), and arbitrary metric space.
    \end{enumerate}
\end{theorem}

\begin{proof}
    \hspace*{0in}
    \begin{enumerate}
        \item This claim is \textbf{false}. Consider the subsets \(F_n = [n, \infty)\). Then for any finite sub-collection \(F_{n_1}, F_{n_2}, \dots, F_{n_k}\), let \(n = \max_{k} (n_k)\). We can compute the intersection to be: 
        \[F_{n_1} \cap F_{n_2} \cap \dots \cap F_{n_k} = F_{n} \neq \varnothing.\]
        However, since for any \(x \in \mathbb R\) we may find some \(n \ge x\), there is always some \(F_n\) such that \(x \notin F_n\). Hence 
        \[\bigcap_{n=1}^\infty F_n = \varnothing.\]
        This disproves the claim.
        \item This claim is \textbf{true}. If \(F_n\) are both closed and bounded subsets of \(\mathbb R\), then the Heine-Borel Theorem guarantees that \(F_n\) is compact. Now apply Theorem 2.36 from the textbook to conclude that 
        \[\bigcap_{n = 1}^\infty F_n \neq \varnothing.\]
        \item This claim is \textbf{false}. Let \(X = \mathbb Q\) with the relative topology inherited from \(\mathbb R\). Then consider the subsets \(F_n = \overline{B_{1/ n}(\sqrt 2)}\) as the closed balls centered at \(\sqrt 2\) with radius \(1 / n\), where \(p_n\) is tne \(n\)th prime. 
        In particular, since \(\sqrt 2 \pm 1 / n\) are irrational, the boundary points of \(F_n\) don't exist in \(\mathbb Q\), and hence we can drop them without changing anything: \(F_n = B_{1/n}(\sqrt 2)\).
    
        Now we check that finite intersections are nonempty. Indeed, if \(F_{n_1}, F_{n_2}, \dots, F_{n_k}\) are a finite sub-collection, then their intersection is just the ball of minimum radius \(r = \min_{k} (1/n_k)\). This radius is clearly greater than \(0\), so we know that \(F_{n_1} \cap F_{n_2} \cap \dots \cap F_{n_k} \neq \varnothing\).
    
        However, if we consider \(\bigcap_{n = 1}^\infty F_n\), then for any \(x \neq \sqrt 2\), we can find a \(k\) such that \(k > 1 / |x - \sqrt 2|\). This implies \(1 / n < |x - \sqrt 2|\). Hence by definition \(x \notin F_k\), so \(x \notin \bigcap_{n = 1}^\infty F_n\). So all \(x \neq \sqrt 2\) are not in our intersection. But also \(\sqrt 2\) is not in \(\mathbb Q\)! Hence in \(\mathbb Q\), the intersection \(\bigcap_{n = 1}^\infty F_n\) is empty. This disproves the claim.    
    \end{enumerate}
\end{proof}

\begin{theorem}[2]
    Show that every compact metric space is complete. 
\end{theorem}

\begin{proof}
    Let \(X\) be a compact metric space. We must show that every Cauchy sequence \(\{x_n\}\) converges. Since \(X\) is compact, there is a convengent subsequence \(x_{n_k} \to x \in X\). We claim that in fact \(x_n \to x\).

    Indeed, since \(x_{n_k} \to x\), there is \(N_1\) such that \(n_k \ge N_1\) implies \(d(x_{n_k}, x) < \e / 2\). Furthermore, given that \(\{x_n\}\) is Cauchy, choose \(N_2\) such that \(n, m \ge N_2\) implies \(d(x_n, x_m) < \e / 2\).

    Set \(N = \max(N_1, N_2)\) and \(n_k \ge N\). Then 
    \[d(x_n, x) \le d(x_n, x_{n_k}) + d(x_{n_k}, x) < \e.\]
    Thus \(x_n \to x \in X\), and \(X\) is complete.
\end{proof}

\begin{theorem}[3]
    The following ``Theorem'' is not true. Find an error in the ``proof'' and construct a counterexample.

    \textbf{Theorem:} (Bogus) Let \(f : X \to Y\) be a continuous mapping from a metric space \(X\) to a metric space \(Y\). Let \(E \subseteq X\) be a closed subset and assume the diameter, \(\text{diam}(E) < 1\). Then \(f(E)\) is bounded. 

    \textbf{Proof:} (Junk) Since \(\text{diam}(E) < 1\), \(E\) can be contained in a ball \[B_2(x_0) = \{x \in X \mid d(x, x_0) < 2\}.\]
    Therefore \(E\) is bounded. Since \(E\) is assumed to be closed, \(E\) is therefore compact. Since \(F\) is continuous, \(f(E)\) is therefore compact and therefore bounded. 
\end{theorem}

\begin{proof}
    The error is in this step: ``Since \(E\) is assumed to be closed, \(E\) is therefore compact.'' Because \(X\) is any arbitrary metric space the equivalence between closed and bounded iff compact does not hold. Indeed, let \(f : (0, 1) \to \R\) with \(x \mapsto 1 / x\). The subspace topology gives that \((0, 1/2]\) is closed and bounded. But \(f((0, 1/2]) = (2, \infty)\) is clearly not bounded. 
\end{proof}

\begin{theorem}[4]
    Let \(I = [0, 1]\) and let \(f : I \to I\) be continuous. Prove that \(f\) has at least one fixed point.
\end{theorem}

\begin{proof}
    Extend the codomain of \(f\) to \(\R\) and consider the map \(g(x) = f(x) - x\). We have the bounds \(0 \le f(0) - 0 = f(0) \le 1\) and \(-1 \le f(1) - 1 \le 0\). Thus the interval \([g(0), g(1)]\) contains the point \(0\). The continuity of \(f(x)\) implies the continuity of \(g(x)\); the application of the intermediate value theorem guarentees the existence of \(x_0 \in [0, 1]\) such that \(g(x_0) = 0\). Thus \(f(x_0) = x_0\) and \(x_0\) is a fixed point. 
\end{proof}

\begin{theorem}[5]
    Let \(f : \R \to \R\) and suppose \[\abs{f(x) - f(y)} \le \abs{x - y}^{1 + \a}\]
    for all real \(x\) and some fixed real \(\a > 0\). Prove that \(f\) is a constant function.
\end{theorem}
    
\begin{proof}
    Without loss of generality assume that \(x \ge y\) and \(x = y + \delta\). Then we may rewrite the given equation as
    \[\frac{|f(y + \delta) - f(y)|}{\delta} \le \delta^\a.\]
    Note that \(\a > 0\) gives the important limit \(\lim_{\delta \to 0} \d^\a = 0\). Then for any \(y\), we have \(\lim_{\delta \to 0} |(f(y + \delta) - f(y)) / \delta| \le 0\). Thus \(f'(y)\) is defined and equal to zero. Theorem 5.11 gives that if \(f'(x) = 0\), then \(f\) must be constant.
\end{proof}

\begin{theorem}[6]
    Let \(g : \R \to \R\) and suppose that \(g'(x)\) exists for all \(x\). Also assume that there is a constant \(M > 0\) such that \(\abs{g'(x)} \le M\) for all \(x \in \R\). Define \(f(x) = x + \d g(x)\) where \(\d\) is a fixed real number. 
    \begin{enumerate}
        \item Show \(f\) is 1-to-1 if \(\abs{\d}\) is sufficiently small. Find an estimate \(\d\) must satisfy.
        \item Assuming \(\d\) satisfies the condition in (6a), find an expression for \(\frac{d}{dx}f^{-1}(x)\).
    \end{enumerate}
\end{theorem}

\begin{proof}
    \hspace*{0in}
    \begin{enumerate}
        \item Let \(\d < 1 / M\). Then \(f'(x) = 1 + \d g'(x)\). Now \(|\d g'(x)| < (1/M) M = 1\), so we have \(f'(x) > 0\). Thus \(f\) is strictly increasing. The reals form a total order so this implies that \(f\) is injective. Thus \(\d\) is about as small as \(1/M\).
        \item By definition \(f(f^{-1}(x)) = x\). Applying the chain rule, we see that 
        \[f'(f^{-1}(x))\cdot \frac{d}{dx}f^{-1}(x) = 1.\]
        Hence \[\frac{d}{dx}f^{-1}(x) = \frac{1}{f'(f^{-1}(x))}.\]
    \end{enumerate}
\end{proof}

\begin{theorem}[7]
    Define 
    \[\int_a^{\infty} f(x) d\a(x) = \lim_{N \to \infty}\int_a^{N} f(x) d\a(x)\]
    provided the limit exists. Let \(f(x) = 1 / x^2\) and \(\a(x) = \lfloor x/2\rfloor\).

    Show the above limit exists and compute \(\int_\frac{1}{2}^{\infty} f(x) d\a(x)\).
\end{theorem}

\begin{proof}
    Fix some \(N\). Now note that \[\a(x) = \sum_{a/2 \le n < N/2} I(x - 2n)\] on the interval \([a, N)\). Hence we may rewrite 
    \[\int_a^{N} f(x) d\a(x) = \sum_{a/2 \le n < N/2} f(2n) = \sum_{a/2 \le n < N/2} \frac{1}{4x^2}.\]
    This series is less than \(\sum 1/x^2\), so it converges as \(N \to \infty\). If \(a = 1/2\), then we have 
    \[\int_{\frac{1}{2}}^{\infty} f(x) d\a(x) = \sum_{n \ge 1} \frac{1}{4x^2} = \frac{6}{4\pi^2} = \frac{3}{2\pi^2}.\]
\end{proof}

\begin{theorem}[8]
    Let \(f \in C^1([0, 2\pi])\) and define 
    \[a_n = \int_0^{2\pi}f(x)\cos nx dx.\]
    Prove that \(a_n \to 0\) as \(n \to \infty\).
\end{theorem}

\begin{proof}
    Since \(f\) is differentiable we may use integration by parts to find that 
    \begin{align*}
        a_n &= \int_0^{2\pi}f(x)\cos nx dx \\
        &= \left(f(2\pi)\frac{\sin 2\pi n}{n} - f(0)\frac{\sin 0 n}{n}\right) - \int_0^{2\pi} f'(x) \frac{\sin n x}{n} dx \\
        &= - \int_0^{2\pi} f'(x) \frac{\sin n x}{n} dx \\
        &= -\frac{1}{n}\int_0^{2\pi} f'(x) \sin n x dx.
    \end{align*}
    The domain of \(f'\) is compact, so \(f'\) must be bounded. Since \(\sin nx\) is also bounded by some \(M > 0\), we conclude that \(f'(x) \sin nx\) is bounded. Therefore
    \[\abs{a_n} = \frac{1}{n} \bigabs{\int_0^{2\pi} f'(x) \sin n x dx} \le \frac{1}{n}\bigabs{\int_0^{2\pi}Mdx} \le \frac{2\pi M}{n}.\]
    Now it is clear that \(a_n \to 0\) as \(n \to \infty\).
\end{proof}

\begin{theorem}[9]
    Define \(\text{BUC} = \{f : \R \to \R \mid f \text{ is bounded and uniformly continuous on } \R\}\) and \(d(f, g) = \sup_\R \abs{f(t) - g(t)}\). For \(\d \in (0, 1)\) and \(f \in \text{BUC}\) define 
    \[f_\d(t) = \frac{1}{2\d}\int_{t - \d}^{t + \d} f(s)ds = \frac{1}{2\d}\int_{0}^{2\d} f(t - \d + \tau)d\tau.\]
    Show
    \begin{enumerate}
        \item \(f_\d \in \text{BUC}\),
        \item \(f_\d \in C^1\),
        \item the collection \(\{f_\d \mid 0 < \d < 1\}\) is dense in BUC, i.e. for each \(\e > 0\) there is a \(\d \in (0, 1)\) such that \(d(f, f_\d) < \e\).
    \end{enumerate}
\end{theorem}

\begin{proof}
    \hspace*{0in}
    \begin{enumerate}
        \item By assumption \(f\) is bounded by some \(M \ge 0\). Then 
        \[f_\d(t) = \frac{1}{2\d}\int_{t - \d}^{t + \d} f(s)ds \le \frac{1}{2\d}\int_{t - \d}^{t + \d} M ds = M.\]
        Hence \(f_\d\) is also bounded by \(M\). 

        Now let \(\e > 0\). Then choose \(\gamma < \frac{\e \d}{M}\). Then for all \(\abs{t_1 - t_2} < \gamma\), we have
        \begin{align*}
            \abs{f_\d(t_1) - f_\d(t_2)} &= \frac{1}{2\d}\bigabs{\int_{t_1 - \d}^{t_1 + \d} f(s)ds - \int_{t_2 - \d}^{t_2 + \d} f(s)ds} \\
            &= \frac{1}{2\d}\bigabs{\int_{t_1 - \d}^{t_2 - \d} f(s)ds - \int_{t_1 + \d}^{t_2 + \d} f(s)ds} \\
            &\le \frac{1}{2\d}\left(\bigabs{\int_{t_1 - \d}^{t_2 - \d} f(s)ds} + \bigabs{\int_{t_1 + \d}^{t_2 + \d} f(s)ds}\right)\\
            &\le \frac{1}{2\d}M\bigabs{t1 - t2} + \frac{1}{2\d}M\bigabs{t1 - t2} \\
            &= \frac{M\abs{t_1 - t_2}}{\d} \\
            &< \e.
        \end{align*}
        (Note that the second equality can be seen by drawing out the integrals geometrically.) This proves that \(f_\d\) is uniformly continous. Hence \(f_\d \in \text{BUC}\).
        \item Let \(F(x) = \int f(s) ds\). By the fundamental theorem of calculus we have \[f_\d(t) = \frac{1}{2\d}(F(t + \d) - F(t - \d)).\]
        Since \(F \in C^1\), we have \(f_\d \in C_1\) as well.
        \item Set \(\e > 0\). Since \(f\) is uniformly continuous we have some \(\gamma > 0\) such that \(\abs{t_1 - t_2} < \gamma\) implies \(\abs{f(t_1) - f(t_2)} < \e\). Now simjply choose \(\d = \gamma\). Then the integral of \(f(t)\) between \(t \pm \gamma\) is bounded above and below by \(f(t) \pm \e\), which implies  
        \[\frac{1}{2\gamma}(f(t) - \e)2\gamma \le \frac{1}{2\gamma}\int_{t - \gamma}^{t + \gamma} f(s)ds \le \frac{1}{2\gamma}(f(t) + \e)2\gamma.\]
        Thus \(f(t) - \e \le f_\gamma(t) \le f(t) + \e\) for all \(t\) implies \(d(f, f_\gamma) \le \e\), as desired.
    \end{enumerate}
\end{proof}

\end{document}