\documentclass[12pt]{article}
\usepackage{amsmath,amssymb,amsthm,amsfonts,tabularx}
\usepackage[pdftex]{graphicx}
\usepackage[dvipsnames]{xcolor}
\usepackage{fancyhdr}
\usepackage{parskip}
\usepackage[shortlabels]{enumitem}
\pagestyle{fancy}
\usepackage{setspace}
\newcommand{\realname}[1]{\newcommand{\printrealname}{#1}}
\newcommand{\pset}[1]{\newcommand{\printpset}{#1}}
\newcommand{\mathclass}[1]{\newcommand{\printmathclass}{#1}}

%% Pagestyle setup
\setlength{\headheight}{0.75in}
\setlength{\oddsidemargin}{0in}
\setlength{\evensidemargin}{0in}
\setlength{\voffset}{-.5in}
\setlength{\headsep}{10pt}
\setlength{\textwidth}{6.5in}
\setlength{\headwidth}{6.5in}
\setlength{\textheight}{8in}
\lhead{Math \printmathclass}
\chead{\Large \textbf{\printpset}}
\rhead{\printrealname}
\rfoot{Page \thepage}
\renewcommand{\headrulewidth}{0.5pt}
\renewcommand{\footrulewidth}{0.3pt}
\setlength{\textwidth}{6.5in}
\renewcommand{\baselinestretch}{1}
\setenumerate[0]{label=(\alph*)}
\newcommand{\todo}{\textcolor{red}{\textbf{TODO }}}

\newtheorem*{prop}{Proposition}
\newtheorem*{corollary}{Corollary}
\newtheorem*{lemma}{Lemma}
\theoremstyle{remark}
\newtheorem*{defn}{Definition}

\newtheoremstyle{named}{}{}{}{}{\bfseries}{.}{.5em}{\thmnote{Problem #3}}
\theoremstyle{named}
\newtheorem*{theorem}{Theorem}

%% DO NOT ALTER THE ABOVE LINES
%%%%%%%%%%%%%%%%%%%%%%%%%%%%%%%%%%%%%%%%%%%%


%% If you would like to use Asymptote within this document (which is optional), 
%% you can find out how at the following URL:
%%
%%   http://www.artofproblemsolving.com/Wiki/index.php/Asymptote:_Advanced_Configuration
%%
%% As explained there, you will want to uncomment the line below.  But be
%% sure to check the website because there are several other steps that must 
%% be followed.
%% \usepackage{asymptote}

%% Enter your real name here
%% Example: \realname{David Patrick}
\realname{Hanting Zhang}
\pset{Midterm 2}
\mathclass{425A}

\renewcommand{\a}{\alpha}
\renewcommand{\b}{\beta}
\newcommand{\e}{\varepsilon}
\newcommand{\Z}{\mathbb Z}
\newcommand{\N}{\mathbb N}
\newcommand{\Q}{\mathbb Q}
\newcommand{\R}{\mathbb R}
\newcommand{\C}{\mathbb C}
\renewcommand{\bf}{\mathbf}
\newcommand{\id}[1]{\text{id}_{#1}}
\renewcommand{\implies}{\Rightarrow}
\newcommand{\coimplies}{\Leftarrow}
\renewcommand{\em}{\varnothing}
\renewcommand{\Im}{\text{Im}}

\begin{document}

\begin{theorem}[1]
    Determine the convergence or divergence in (a)-(c).
    \begin{enumerate}
        \item[(a)] 
        \[\sum_{k = 1}^\infty (-1)^k \frac{\sqrt{k + 1} - \sqrt{k}}{\sqrt{k}}\]
        \item[(b)] \[\sum_{k = 1}^\infty \frac{k!}{(k + 2)!}\]
        \item[(c)] \[\sum_{k = 1}^\infty \frac{2^k}{k^k}\]
    \end{enumerate}
\end{theorem}

\begin{proof}
    We do each separately:
    \begin{enumerate}
        \item[(a)] \textbf{Converges.} We can simplify the terms with the difference of squares by multiplying the top and bottom by \(\sqrt{k + 1} + \sqrt{k}\):
        \[\sum_{k = 1}^\infty (-1)^k \frac{\sqrt{k + 1} - \sqrt{k}}{\sqrt{k}} \cdot \frac{\sqrt{k + 1} + \sqrt{k}}{\sqrt{k + 1} + \sqrt{k}} = \sum_{k = 1}^\infty \frac{(-1)^k}{\sqrt{k}(\sqrt{k + 1} + \sqrt{k})}.\] 
        The absolute magnitudes of the terms \(1 / \sqrt{k}(\sqrt{k + 1} + \sqrt{k})\) are monotonically decreasing, since \(\sqrt{k}(\sqrt{k + 1} + \sqrt{k})\) is monotonically increasing. Furthermore, \(1 / \sqrt{k}(\sqrt{k + 1} + \sqrt{k}) \to 0\) as \(k \to \infty\). Hence we can apply the alternating series test to conclude that 
        \[\sum_{k = 1}^\infty (-1)^k \frac{\sqrt{k + 1} - \sqrt{k}}{\sqrt{k}}\]
        converges.
        \item[(b)] \textbf{Converges.} We can bound each term:
        \[\frac{k!}{(k+2)!} = \frac{1}{(k + 1)(k + 2)} \le \frac{1}{k^2}.\] 
        Now we know \(\sum_{k = 1}^\infty \frac{1}{k^2} = \frac{\pi^2}{6}\) converges, so the comparsion test gives the convergence of \(\sum_{k = 1}^\infty \frac{2^k}{k^k}\).
        \item[(c)] \textbf{Converges.} For \(k \ge 3\), 
        \[\frac{2^k}{k^k} = \left(\frac{2}{k}\right)^k \le \left(\frac{2}{3}\right)^k.\] 
        Then 
        \[\sum_{k = 1}^\infty \frac{2^k}{k^k} = \frac{2}{1} + \frac{2^2}{2^2} + \sum_{k = 3}^\infty \frac{2^k}{k^k} \le 2 + 1 + \sum_{k = 1}^\infty \left(\frac{2}{3}\right)^k.\]
        The RHS is finite amount plus a geometric series with \(r < 1\), therefore it converges. Thus \(\sum_{k = 1}^\infty \frac{2^k}{k^k}\) converges.
    \end{enumerate}
\end{proof}

\begin{theorem}[2]
    Consider the power series \(\sum a_n z^n\) and assume that the coefficients \(a_n\) are integers, infinitely many of which are not zero. Prove that the radius of convergence \(R \le 1\).
\end{theorem}

\begin{proof}
    Recall the definition of \(R\) given in Theorem 3.39: 
    \[\a = \limsup_{n \to \infty}\sqrt[n]{|a_n|}, \hspace{5mm} R = \frac{1}{\a}.\] 
    Only finitely many of the \(a_n\) can be zero, so there is some \(N \in \N\) such that \(n \ge N\) implies \(|a_n| \ge 1\). Thus we have \(\sqrt[n]{|a_n|} \ge 1 \Rightarrow \limsup_{n \to \infty}\sqrt[n]{|a_n|} \ge 1\). This exactly means that \(R = 1 / \a \le 1\), as desired.
\end{proof}

\begin{theorem}[3]
    Consider a function \(f : M \to \mathbb R\). It's graph is the set,
    \[G(f) = \{(x, y) \in M \times \mathbb R \mid y = f(x)\}.\]
    \begin{enumerate}
        \item[(a)] Prove that if \(f\) is continuous then \(G(f)\) is closed as a subset of \(M \times \mathbb R\). 
        \item[(b)] Prove that if \(f\) is continuous and \(M\) is compact then \(G(f)\) is compact.
        \item[(c)] Prove that if \(G(f)\) is compact then \(f\) is continuous. 
    \end{enumerate}
\end{theorem}

\begin{proof}
    We do each part separately:
    \begin{enumerate}
        \item[(a)] If \(f\) is continuous, then any convergent sequence \(x_n \to x\) has a convergent image under \(f\), i.e. \(f(x_n) \to f(x)\). Thus \((x_n, f(x_n))\) converges to \((x, f(x))\) in the product space. Thus any subseqeuence of \(G(f)\) converges in \(G(f)\), which implies that \(G(f)\) is closed.
        \item[(b)] Define \(F : M \to M \times \R\) by mapping \(x \mapsto (x, f(x))\), i.e. \(F\) is the pair \((\id{M}, f)\). Both \(\id{M}\) and \(f\) are continuous, so \(F\) is continuous as well. The image of \(F\) is \(G(f)\); thus we may invoke Theorem 4.14 to give that \(F(M) = G(f)\) is compact.
        \item[(c)] We prove the contrapositive. Suppose \(f\) is not continuous. Then there exist points \(x \in M\) and \(\e > 0\) such that there is a sequence \(x_n \to x\) with \(|f(x_n) - f(x)| \ge \e\). Now consider the sequence \(\{(x_n, f(x_n))\} \subseteq G(f)\) obtained by mapping \(\{x_n\}\) under \(f\). Because \(G(f)\) is compact, we have a convergent subsequence \(\{(x_{n_i}, f(x_{n_i}))\}\) with limit \((x, y)\). We know that \(y \neq f(x)\) by construction and that \(G(f)\) must contain a unique pair of the form \((x, *)\) (namely \((x, f(x))\)). Thus \((x, y) \notin G(f)\), implying that \(G(f)\) is not compact, as desired.
    \end{enumerate}    
\end{proof}

\begin{theorem}[4]
    Let \(I = [0, 1]\) and let \(F : I \to I\) be continuous. Prove that \(F\) has at least one fixed point. Quoting a fixed points theorem is not acceptable.
\end{theorem}

\begin{proof}
    Extend the codomain of \(F\) to \(\R\) and consider the map \(f(x) = F(x) - x\). We have the bounds \(0 \le F(0) - 0 = F(0) \le 1\) and \(-1 \le F(1) - 1 \le 0\). Thus the interval \([f(0), f(1)]\) contains the point \(0\). The continuity of \(F(x)\) implies the continuity of \(f(x)\); the intermediate value theorem guarentees the existence of \(x_0 \in [0, 1]\) such that \(f(x_0) = 0\). Thus \(F(x_0) = x_0\) and \(x_0\) is a fixed point. 
\end{proof}

\begin{theorem}[5]
    Let \(X\) and \(Y\) be metric spaces and \(F : X \to Y\) be a continuous mapping onto \(Y\). If \(D\) is a dense subset of \(X\), prove that \(F(D)\) is a dense subset of \(Y\). 
\end{theorem}

\begin{proof}
    Let \(y \in F(X)\). If \(y \in F(D)\), then there is nothing to prove. Otherwise, if \(y \in F(X) \setminus F(D)\), then it suffices to show that \(y\) is a limit point of \(F(D)\). Let \(x \in F^{-1}(y)\). Since \(D\) is dense in \(X\), \(x\) is a limit point of \(X\). Thus there is some sequence \(x_n \to x\). The continuity of \(F\) implies that \(\lim_{x_n \to x} F(x_n) = y\), so \(F(x_n) = y_n \to y\). For each \(n\), \(F(x_n) \in F(D)\), so \(y\) is a limit point of \(F(D)\). 

    Now \(y\) was arbitrary; we have \(F(D)\) is dense in \(F(X)\). 
\end{proof}

\textbf{EXTRA CREDIT (10 POINTS):}

\begin{theorem}[6]
    Prove the convergence or divergence of
    \[\sum_{n = 1}^\infty\frac{\sin n}{n}.\]
\end{theorem}

\begin{proof}
    We begin by bounding the partial sums \(\sum_{n = 1}^N \sin n\).

    Consider \(S_N = \sum_{n = 1}^N \cos n + i \sin n\). Then \(\text{Im}(S_N) = \sum_{n = 1}^\infty \sin n\) is the quantity we need. Now \(|\Im(z)| \le |z|\) for any \(z \in \C\), so we have
    \begin{align*}
        |\Im(S_N)| \le |S_N| \le \left|\sum_{n = 1}^N e^{in}\right| = \left|e^i\frac{1 - e^{iN}}{1 - e^i}\right|,
    \end{align*}
    where the last equality follows from the formula for a geometric series. We can bound \(|1 - e^{iN}| \le 2\), so \(|\Im(S_N)| \le 2 / |1 - e^i| < \infty\).

    We can apply Theorem 3.42 with \(a_n = \sin n\) and \(b_n = 1/n\). We have the three needed conditions: (a) \(S_N\) is bounded, (b) \(1/n\) is decreasing, and (c) \(\lim_{n \to \infty} 1/ n = 0\). Thus \(\sum_{n = 1}^\infty \sin n / n\) converges.
\end{proof}
\end{document}