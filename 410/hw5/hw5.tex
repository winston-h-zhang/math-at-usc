%%%%%%%%%%%%%%%%%%%%%%%%%%%%%%%%%%%%%%%%%%%%%%
%%                                          %%
%% USE THIS FILE TO SUBMIT YOUR SOLUTIONS   %%
%%                                          %%
%% You must have the usamts.tex file in     %%
%% the same directory as this file.         %%
%% You do NOT need to submit this file or   %%
%% usamts.tex with your solutions.  You     %%
%% only need to submit the output PDF file. %%
%%                                          %%
%% DO NOT ALTER THE FILE usamts.tex         %%
%%                                          %%
%% If you have any questions or problems    %%
%% using this file, or with LaTeX in        %%
%% general, please go to the LaTeX          %%
%% forum on the Art Of Problem Solving      %%
%% web site, and post your problem.         %%
%%                                          %%
%%%%%%%%%%%%%%%%%%%%%%%%%%%%%%%%%%%%%%%%%%%%%%

%%%%%%%%%%%%%%%%%%%%%%%%%%%%%%%%%%%%%%%%%%%%
%% DO NOT ALTER THE FOLLOWING LINES
\documentclass[12pt]{article}
\usepackage{amsmath,amssymb,amsthm,amsfonts,tabularx}
\usepackage[pdftex]{graphicx}
\graphicspath{ {./images/} }
\usepackage{fancyhdr}
\pagestyle{fancy}
\usepackage{setspace}
\usepackage{csquotes}
%%%%%%%%%%%%%%%%%%%%%%%%%%%%%%%%%%%%%%%%%%%
%%                                       %%
%% Students: DO NOT MODIFY THIS FILE!!!  %%
%%                                       %%
%%%%%%%%%%%%%%%%%%%%%%%%%%%%%%%%%%%%%%%%%%%

%% USAMTS style sheet
%% last modified: 23-Jul-2004

%% Student and Year/Round data
\newcommand{\realname}[1]{\newcommand{\printrealname}{#1}}
\newcommand{\pset}[1]{\newcommand{\printpset}{#1}}

%% Pagestyle setup
\setlength{\headheight}{0.75in}
\setlength{\oddsidemargin}{0in}
\setlength{\evensidemargin}{0in}
\setlength{\voffset}{-.5in}
\setlength{\headsep}{10pt}
\setlength{\textwidth}{6.5in}
\setlength{\headwidth}{6.5in}
\setlength{\textheight}{8in}
\lhead{Math 410}
\chead{\Large \textbf{Homework \printpset}}
\rhead{\printrealname}
\rfoot{Page \thepage}
\renewcommand{\headrulewidth}{0.5pt}
\renewcommand{\footrulewidth}{0.3pt}
\setlength{\textwidth}{6.5in}


\renewcommand{\baselinestretch}{1}
%% DO NOT ALTER THE ABOVE LINES
%%%%%%%%%%%%%%%%%%%%%%%%%%%%%%%%%%%%%%%%%%%%


%% If you would like to use Asymptote within this document (which is optional), 
%% you can find out how at the following URL:
%%
%%   http://www.artofproblemsolving.com/Wiki/index.php/Asymptote:_Advanced_Configuration
%%
%% As explained there, you will want to uncomment the line below.  But be
%% sure to check the website because there are several other steps that must 
%% be followed.
%% \usepackage{asymptote}

\newtheorem*{prop}{Proposition}
\newtheorem*{corollary}{Corollary}
\newtheorem*{lemma}{Lemma}
\theoremstyle{remark}
\newtheorem*{defn}{Definition}

\newtheoremstyle{named}{}{}{}{}{\bfseries}{.}{.5em}{\thmnote{Problem #3}}
\theoremstyle{named}
\newtheorem*{theorem}{Theorem}

%% Enter your real name here
%% Example: \realname{David Patrick}
\realname{Hanting Zhang}
\pset{5}
\mathclass{410}

\renewcommand{\bf}{\mathbf}
\renewcommand{\bar}{\overline}
\renewcommand{\implies}{\Rightarrow}
\newcommand{\coimplies}{\Leftarrow}
\newcommand{\normal}{\trianglelefteq}

\begin{document}
Exercises 9, 10, pp. 116-117.

\begin{theorem}[9]
    Assume \(G\) acts transitively on the finite set \(A\) and let \(H\) be a normal subgroup of \(G\). Let \(\mathcal O_1, \mathcal O_2, \dots, \mathcal O_r\) be the distinct orbits of \(H\) on \(A\).

    \begin{enumerate}
        \item[(a)] Prove that \(G\) permutes the sets \(\mathcal O_1, \mathcal O_2, \dots, \mathcal O_r\) in the sense that for each \(g \in G\) and each \(i \in \{1, \dots, r\}\) there is a \(j\) such that \(g \mathcal O_i = \mathcal O_j\), where \(g \mathcal O = \{g \cdot a \mid a \in \mathcal O\}\). Prove that \(G\) is transitive on \(\{\mathcal O_1, \dots \mathcal O_r\}\). Deduce that all orbits of \(H\) on \(A\) have the same cardinality.
        
        \item[(b)] Prove that if \(a \in \mathcal O_1\) then \(|\mathcal O_1| = |H : H \cap G_a|\) and prove that \(r = |G : H G_a|\). 
    \end{enumerate}
\end{theorem}

\begin{proof}
    Part (a): Each orbit \(\mathcal O_i\) is of the form \(H \cdot x\). Then for any \(i, j \in \{1, \dots, r\}\). Since \(G\) acts transitively on \(A\), there is some \(g'\) such that \(x = gy\). we want \(H \cdot x = g \cdot (H \cdot y)\) for some \(g \in G\). Since \(G\) acts transitively on \(A\), there is some \(g'\) such that \(x = g \cdot y\). 
    Then \(g \cdot (H \cdot y) = (gH) \cdot y\). The normality of \(H\) implies \(gH = Hg\), thus \((gH) \cdot y = (Hg) \cdot y = H \cdot g \cdot y = H \cdot x\), which shows that \(G\) permutes \(\{\mathcal O_1, \dots, \mathcal O_r\}\). 

    Since we can find \(\mathcal O_i = g \mathcal O_j\) for any \(i, j\), clearly \(G\) acts transitively on \(\{\mathcal O_1, \dots, \mathcal O_r\}\). Furthermore this implies that both \(|\mathcal O_i| \subseteq |\mathcal O_j|\) and \(|\mathcal O_j| \subseteq |\mathcal O_i|\). Thus we conclude that all orbits have the same size.
\end{proof}

\begin{proof}
    Part (b): By the orbit-stabilizer theorem, we have \(|\mathcal O_1| = |H : H_a|\). But by definition \(H_a = H \cap G_a\), so \(|\mathcal O_1| = |H : H \cap G_a|\). 

    Next, since \(H \normal G\), we may apply the second isomorphism theorem to obtain \(H / (H \cap G_a) \cong HG_a / G_a\). Thus \(|\mathcal O_1| = |H G_a : G_a|\). Now we can use the orbit-stabilizer theorem again to compute \(r\):
    \[r = \frac{|A|}{|\mathcal O_1|} = \frac{|A|}{|H G_a : G_a|} = \frac{|G| /|G_a|}{|H G_a| / |G_a|} = \frac{|G|}{|HG_a|} = |G : H G_a|.\]
\end{proof}

\begin{theorem}[10]
    Let \(H\) and \(K\) be subgroups of the group \(G\). For each \(x \in G\) define the \(HK\) \textit{double coset} of \(x\) in \(G\) to be the set
    \[HxK = \{hxk \mid h \in H, k \in K\}.\]
    \begin{enumerate}
        \item[(a)] Prove that \(HxK\) is the union of left cosets in the orbit of \(xK\) generated by \(H\) acting on \(G / K\).
        \item[(b)] Prove that \(HxK\) is the union of right cosets of \(H\).
        \item[(c)] Show that the double cosets are disjoint and partition \(G\).
        \item[(d)] Prove that \(|HxK| = |K|\cdot |H : H \cap x K x^{-1}|\).
        \item[(e)] Prove that \(|HxK| = |H| \cdot |K : K \cap x H x^{-1}|\).
    \end{enumerate}
\end{theorem}

\begin{proof}
    Part (a): We want to show that
    \[HxK = \bigcup_{h \in H} hxK.\]
    If \(g \in HxK\), then there exist \(h \in H\) and \(k \in K\) such that \(g = hxk\). Then \(g \in hxK \subseteq \bigcup_{h \in H} hxK\). Hence we have one inclusion. This proves \(HxK \subseteq \bigcup_{h \in H} hxK\)

    Conversely, if \(g \in \bigcup_{h \in H} hxK\), then there exists \(h \in H\) such that \(g \in hxK\), which again implies there exists \(k \in K\) with \(g = hxk\). Hence \(g \in HxK\). This proves \(\bigcup_{h \in H} hxK \subseteq HxK\), and thus \(HxK = \bigcup_{h \in H} hxK\).
\end{proof}

\begin{proof}
    Part (b): This is very similar to part (a). Briefly, we have:
    \[g \in HxK \iff \exists h \in H, k \in K, g = hxk \iff g \in \bigcup_{k \in K} Hxk.\]
\end{proof}

\begin{proof}
    Part (c): Consider the relation \(x \sim y\) if and only if there exist \(h \in H\) and \(k \in K\) such that \(y = hxk\). We claim that this is an equivalence relation with classes \(HxK\). 

    Indeed, \(1 \in H\) and \(1 \in K\), so \(x = 1x1 \implies x \sim x\). 

    If \(x \sim y\), then \(y = hxk \implies h^{-1} y k^{-1}\). Hence \(y \sim x\). 

    If \(x \sim y\) and \(y \sim z\), then \(y = hxk\) and \(z = h'yk'\) implies \(z = h'hxkk'\). Clearly \(h'h \in H\) and \(kk' \in K\), \(x \sim z\).
    
    Thus the relation is an equivalence relation. Clearly the classes are \(\{hxk \mid h \in H, k \in K\} = HxK\), as desired.
\end{proof}

\begin{proof}
    Part (d): From part (a), take the set of coset \(\{hxK \mid h \in H\}\) whose union is \(HxK\). Note that for any \(h, h' \in H\), either \(hxK \cap h'xK = \varnothing\) or \(hxK = h'xK\). Define \(h \sim h'\) if and only if the latter \(hxK = h'xK\) holds. 
    Clearly this is an equivalence relation, because is it a subrelation on the cosets of \(xK\). Furthermore, if \(h \sim h'\), then we see that \[hxK = h'xK \iff x^{-1}(h'^{-1}h)x K \in K \iff x^{-1}(h'^{-1}h)x \in K \iff h'^{-1}h \in xKx^{-1}.\].

    Simultaneously, \(h'^{-1}h \in H\), so we may further see that \[h'^{-1}h \in H \cap xKx^{-1} \iff h \in h'(H \cap xKx^{-1}),\]
    where we view \(h'(H \cap xKx^{-1})\) as a coset of \(H / (H \cap xKx^{-1})\). Thus \(\{hxK \mid h \in H\}\) can be divided into \(|H / (H \cap xKx^{-1})|\) distinct classes. Since each class all contain the same cosets, we may rewrite our union with a transversal \(\mathcal C\) of the cosets \(H / (H \cap xKx^{-1})\).
    \[HxK = \bigcup_{h \in H} hxK = \bigsqcup_{h \in \mathcal C} hxK.\]
    Notice that now the second union is disjoint. Hence we may take the cardinality of both sides, with \(|\mathcal C| = |H / (H \cap xKx^{-1})|\), to obtain:
    \[|HxK| = \left|\bigsqcup_{h \in \mathcal C} hxK\right| = |K| \cdot |\mathcal C| = |K| \cdot |H : H \cap xKx^{-1}|.\]
\end{proof}

\begin{proof}
    Part (e): This proof is very similar to part (d). We may prove that \(Hxk = Hxk'\) if and only if \(k \in k'(K \cap (x^{-1} H x))\). Then again, with part (b),
    \[|HxK| = \left|\bigsqcup_{k \in \mathcal C} Hxk\right| = |H| \cdot |\mathcal C| = |H| \cdot |K : K \cap x^{-1}Hx|.\]
\end{proof}

Exercises 8, 14, pp. 122-123.

\begin{theorem}[8]
    Prove that if \(H\) has finite index \(n\) then there is a normal subgroup \(K \normal G\) with \(K \le H\) and \(|G : K| \le n!\). 
\end{theorem}

\begin{proof}
    Let \(G\) act on the left cosets of \(H\) with permutation representation \(\pi_{G / H} : G \to S_{|G/H|}\). Then let \(K = \ker (\pi_{G/H})\), which is clearly a subset of \(K\) and also a normal subgroup of \(G\). By the first isomorphism theorem, \(G / \ker(\pi_{G/H}) = G / K \cong \text{im}(\pi_{G/H})\). Comparing the cardinality of both sides, we have \
    \[|G : K| = |G / K| = |\text{im}(\pi_{G / H})| \le |S_{|G / H|}| = n!.\]
\end{proof}

\begin{theorem}[14]
    Let \(G\) be a finite group of composite order \(n\) with the property that \(G\) has a subgroup of order \(k\) dividing \(n\) for each positive integer \(k\) dividing \(n\). Prove that \(G\) is not simple.
\end{theorem}

\begin{proof}
    Choose \(k\) such that \(n / k = p\), where \(p\) is the minimum prime dividing \(n\). Then by Corollary 5, there is a normal subgroup of index \(p\) in \(G\). Thus \(G\) cannot be simple.
\end{proof}

\end{document}