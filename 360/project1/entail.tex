\documentclass[10pt]{article}
\usepackage{fullpage}
\usepackage[noend]{algorithmic}
\usepackage{algorithm}

\usepackage{amsfonts}
\usepackage{amsmath,amsthm,amssymb}
\usepackage{nicefrac}
\usepackage{latexsym}
\usepackage{epic}
\usepackage{epsfig}
\usepackage{amscd}
\usepackage{url}
\usepackage{verbatim}
\usepackage{setspace}
\usepackage{hyperref}
\usepackage{enumerate}
\usepackage{times}
\usepackage{graphicx}
\graphicspath{ {./images/} }

% \usepackage{CS270}

\newtheorem{lemma}{Lemma}
\newtheorem{fact}[lemma]{Fact}
\newtheorem{theorem}[lemma]{Theorem}
\newtheorem{corollary}[lemma]{Corollary}
\newtheorem{proposition}[lemma]{Proposition}
\newtheorem{claim}[lemma]{Claim}

\newtheorem{definition}{Definition}
\newtheorem{example}[definition]{Example}
\newtheorem{remark}[definition]{Remark}

\providecommand{\half}{\ensuremath{\nicefrac{1}{2}}\xspace}
\providecommand{\third}{\ensuremath{\nicefrac{1}{3}}\xspace}
\providecommand{\quarter}{\ensuremath{\nicefrac{1}{4}}\xspace}

\providecommand{\SetCard}[1]{\ensuremath{| #1 |}\xspace}
\providecommand{\SET}[1]{\ensuremath{\{ #1 \}}\xspace}
\providecommand{\Abs}[1]{\ensuremath{| #1 |}\xspace}
\providecommand{\Set}[2]{\ensuremath{\SET{#1 \mid #2}}\xspace}

\providecommand{\Kth}[1]{\ensuremath{{#1}^{\rm th}}}
\providecommand{\Ceiling}[1]{\ensuremath{\lceil {#1} \rceil}\xspace}
\providecommand{\Floor}[1]{\ensuremath{\lfloor {#1} \rfloor}\xspace}

\providecommand{\PROB}{\ensuremath{{\rm Prob}}\xspace}
\providecommand{\Prob}[2][]{\ensuremath{%
\ifthenelse{\equal{#1}{}}{\PROB[#2]}{\PROB_{#1}[#2]}}\xspace}
\providecommand{\ProbC}[3][]{\Prob[#1]{#2\;|\;#3}}
\providecommand{\Expect}[2][]{\ensuremath{%
\ifthenelse{\equal{#1}{}}{\mathbb{E}}{\mathbb{E}_{#1}}%
\left[#2\right]}\xspace}
\providecommand{\ExpectC}[3][]{\Expect[#1]{#2\;|\;#3}}

\newcommand\algorithmicprocedure{\textbf{procedure}}
\newcommand{\algorithmicendprocedure}{\algorithmicend\ \algorithmicprocedure}
\makeatletter
\newcommand\PROCEDURE[3][default]{%
  \ALC@it
  \algorithmicprocedure\ \textsc{#2}(#3)%
  \ALC@com{#1}%
  \begin{ALC@prc}%
}
\newcommand\ENDPROCEDURE{%
  \end{ALC@prc}%
  \ifthenelse{\boolean{ALC@noend}}{}{%
    \ALC@it\algorithmicendprocedure
  }%
}
\newenvironment{ALC@prc}{\begin{ALC@g}}{\end{ALC@g}}
\makeatother

\title{\bf Question 2}
\author{Winston (Hanting) Zhang}
\date{October 6, by 11:59}

\begin{document}

\subsection*{2. Resolution for First Order Logic - 20 points}
Consider the following knowledge base: (a) All programmers who do not have bugs in their code and comment their code write high-quality code. (b) Programmers who use best coding practices comment their code. (c) Programmers who write high-quality code solve programming problems quickly. (d) Amanda uses best coding practices and (e) does not have bugs in her code. (f) Bill does not comment his code.

\begin{enumerate}
    \item You want to find out whether the knowledge base entails that Amanda solves programming problems quickly. Use resolution for a proof by contradiction.
    \begin{enumerate}
        \item Write down the entailment that you want to prove, expressing the English statements in first-order logic. Use the predicates HasBugs(x), Comments(x), HighQuality(x), QuickSolve(x), and BestPractice(x), where x is a programmer.
        \item Do the transformations as explained in class.
        \item Use resolution to try to obtain false (that is, the empty clause).
        \item Interpret the result of the previous step in plain English.
    \end{enumerate}
    \item You want to find out whether the knowledge base entails that Bill solves programming problems quickly. Proceed using the steps given above.
\end{enumerate}

\begin{enumerate}
    \item We start with the knowledge base 
    \begin{align*} 
      (1) \hspace*{2mm}& \forall x, \neg \text{HasBugs}(x) \land \text{Comment}(x) \implies \text{HighQuality}(x) \\
      (2) \hspace*{2mm}& \forall x, \text{BestPractice}(x) \implies \text{Comment}(x) \\
      (3) \hspace*{2mm}& \forall x, \text{HighQuality}(x) \implies \text{QuickSolve}(x) \\
      (4) \hspace*{2mm}& \text{BestPractice}(A) \\
      (5) \hspace*{2mm}& \neg \text{HasBugs}(A) \\
      (6) \hspace*{2mm}& \neg \text{Comment}(B)
    \end{align*}
    By applying rewrite rules for implies, we can equivalently say 
    \begin{align*} 
      (1) \hspace*{2mm}& \forall x, \text{HasBugs}(x) \lor \neg \text{Comment}(x) \lor \text{HighQuality}(x) \\
      (2) \hspace*{2mm}& \forall x, \neg \text{BestPractice}(x) \lor \text{Comment}(x) \\
      (3) \hspace*{2mm}& \forall x, \neg \text{HighQuality}(x) \lor \text{QuickSolve}(x) \\
      (4) \hspace*{2mm}& \text{BestPractice}(A) \\
      (5) \hspace*{2mm}& \neg \text{HasBugs}(A) \\
      (6) \hspace*{2mm}& \neg \text{Comment}(B)
    \end{align*}
    To show QuickSolve\((A)\), we do the transformation of adding: (7) \(\neg \text{QuickSolve}(A)\), and do resolution to try to prove false. Indeed, we have: 
    \begin{align*}
      (2) + (5) \Rightarrow (8) \hspace*{2mm}& \text{Comment}(A) \\
      (1) + (8) \Rightarrow (9) \hspace*{2mm}& \text{HasBugs}(A) \lor \text{HighQuality}(A) \\
      (3) + (9) \Rightarrow (10) \hspace*{2mm}& \text{HasBugs}(A) \lor \text{QuickSolve}(A) \\
      (5) + (10) \Rightarrow (11) \hspace*{2mm}& \text{QuickSolve}(A) \\
      (7) + (11) \Rightarrow (12) \hspace*{2mm}& \bot 
    \end{align*}
    Since we have false, resolution succeeds. In plain english, this means that Alice can solve problems quickly.

    \item Start with the same 6 rules as we did for part 1 and add (7) \(\neg \text{QuickSolve}(B)\). We try to use resolution to prove false: 
    \begin{align*}
      (2) + (6) \Rightarrow (8) \hspace*{2mm}& \neg \text{BestPractice}(B) \\
      (3) + (7) \Rightarrow (9) \hspace*{2mm}& \neg \text{HighQuality}(B) \\
      (1) + (10) \Rightarrow (7) \hspace*{2mm}& \text{HasBugs}(B) \lor \neg \text{Comment}(B) \\
      & \text{No other resolutions can be made.}
    \end{align*}
    Since no other resolutions can be made, we can only conlude that the knowledge base does not entail that Bob solves problems quickly.
\end{enumerate}

\subsection*{3. Bayesian Networks}
\begin{enumerate}
  \item[Part 1]
  \begin{enumerate}
    \item[(1)] \(P(K = T) = 0.56215\)
    \item[(2)] \(P(K = T) = 0.61\)
  \end{enumerate}
  \item[Part 3]
\end{enumerate}

\end{document}