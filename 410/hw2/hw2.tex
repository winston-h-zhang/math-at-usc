\documentclass{article}
\usepackage[pdftex]{graphicx}
\usepackage{algorithm}
\usepackage[noend]{algorithmic}
\usepackage{tikz}
\usepackage{graphicx}
\usepackage{subcaption}
\usepackage{amsmath,amssymb}
\usepackage{amsfonts}
\usepackage{xcolor}
\usepackage{fancyvrb}
\usepackage{color}
\usepackage{blindtext}
\usepackage{titlesec}
\setlength{\topmargin}{0 in}
\setlength{\oddsidemargin}{0 in}
\setlength{\textheight}{8.5 in}
\setlength{\textwidth}{6.5 in}
\setlength{\parindent}{0 in}

\titleformat*{\section}{\LARGE\bfseries}
\renewcommand{\o}[1]{$\overline{#1}$}
\renewcommand{\O}[1]{$\mathcal{O}(#1)$} %% For the clear looking O notation.

\definecolor{darkgreen}{rgb}{0,0.42,0}
\newcommand{\answer}[1]{} 
\newcommand{\rubric}[1]{{\leavevmode\color{brown}#1}}
\newcommand{\ZZ}[1]{\mathbb Z / #1 \mathbb Z}
\newcommand{\notes}[1]{\textcolor{red}{#1}}

\DeclareUnicodeCharacter{2212}{-}

\title{\textbf{Math 410 Homework 2}}
\author{By: Hanting Zhang \hspace{5mm} Due Date: \textbf{Jan. 29, 2022}}
\date{} 
\begin{document}
\maketitle

Exercises 8, 9, 12, 26, 36, pp. 21-23.

\begin{enumerate}
    \item [8.] 
    \begin{enumerate}
        \item [(a)] Since $G$ is a subset of $\mathbb C$, it suffices to prove that $G$ is a subgroup of $\mathbb C$. For any $g, h \in G$, we need to show that $gh^{-1} \in G$. By definiton there exist some $n, m \in \mathbb Z^+$ such that $g^n = h^m = 1$. 
        
        We want to find $N$ such that $(gh^{-1})^N = 1$. Let $N = nm$ and notice that $(gh^{-1})^{nm} = g^{nm} * h^{-nm} = (g^n)^m * (h^m)^{-n} = 1^m * 1 ^{-n} = 1$. Hence $gh^{-1} \in G$, and $G$ is a subgroup of $\mathbb C$, which makes it a group in general.
        \item [(b)] Clearly $1 \in G$, but $1 + 1 = 2 \notin G$. So $G$ is not closed under addition.
    \end{enumerate}

    \item [9.]
    \begin{enumerate}
        \item [(a)] Again we prove that $G$ is a group by proving it is a subgroup of $\mathbb R$. 
        
        For any $g, h \in G$, there are some $a, b, c, d \in \mathbb Q$ with $g = a + b \sqrt 2$ and $h = c \sqrt d$. then $g - h = (a - c) + (b - d)\sqrt 2$. Clearly $a - c$ and $b - d$ are rational, so $g - h \in G$, as desired. 

        \item [(b)] Let $g$ be a non-zero element of $G$ such that $a + b \sqrt 2 = g$ for some $a, b \in \mathbb Q$ (where $a$ and $B$ are not both $0$). Then note that $1 / g = 1 / (a + b \sqrt 2)$ is in $G$, since \[\frac 1 {a + b \sqrt 2} = \frac {a - b\sqrt 2} {(a + b \sqrt 2) (a - b \sqrt 2)} = \frac {a - b \sqrt 2} {a^2 - 2b^2}.\] Letting $x = \frac a {a^2 - 2b^2}$ and $y = \frac {-b} {a^2 - 2b^2}$, we have $1 / g = x + y \sqrt 2$. 
        Both $x$ and $y$ are rational, since they are made up of rational expressions. Hence $1 / g$ (in $\mathbb R$) is the inverse of $g$ in $G$. 

        \item[Note.] This makes $G$ a \textit{field}. In fact it is the field $\mathbb Q[\sqrt 2]$, the result of adjoining $\sqrt 2$ to $\mathbb Q$. 
    \end{enumerate}
    
    \item [12.] We can just calculate the orders:
    \begin{align*}    
        |\overline{1}| &= 1 \\
        \overline{-1}^2 = 1 \implies |\overline {-1}| &= 2  \\
        \overline{5}^2 = \overline{25} = \overline 1 \implies |\overline{5}| &= 2 \\
        \overline{7} = -\overline{5} \implies |\overline{7}| &= 2 \\
        \overline{-7} = \overline{5} \implies |\overline{-7}| &= 2 \\
        \overline{13} = \overline{1} \implies |\overline{13}| &= 1
    \end{align*}

    \item [26.] We proceed by proving the goal axioms for $H$:
    
    \textit{Identity:} Let $1_H = 1_G$. Then for all $h \in H$, $1_H *_H h = 1_G *_G h = h$ by definition.

    \textit{Associativity:} For any $h_1, h_2, h_3 \in H$, $(h_1 *_H h_2) *_H h_3 = (h_1 *_G h_2) *_G h_3 = h_1 *_G (h_2 *_G h_3) = h_1 *_H (h_2 *_G h_3)$. Here we use the associativity of $G$.

    \textit{Closure:} Given by assumption.

    \textit{Inverses:} Given by assumption.

    \item [36.] Write it out in a table!
    
    \begin{center}
        \begin{tabular}{ c | c c c c }
         & 1 & $a$ & $b$ & $c$ \\ 
        \hline
        1 & 1 & $a$ & $b$ & $c$ \\  
        $a$ & $a$ &  &  &  \\  
        $b$ & $b$ &  &  &  \\ 
        $c$ & $c$ &  &  &  \\ 
        \end{tabular}
    \end{center}

    With loss of generality we can swap between $b$ and $c$ by relabeling them $c = b'$ and $b = c'$. Hence for $aa = ?$, we only need to consider cases $aa = 1$ and $aa = b$. 
    
    \textbf{Case $aa = b$}: Then $ab = 1$ or $ab = c$. In the first case, we have $aab = a1 \Rightarrow bb = a \Rightarrow aaaa = a \Rightarrow aaa = 1$. But $a$ cannot be of order $3$, thus we must have $ab = c$. So the final entry $ac$ is $1$ because that's the only choice left:    
    
    \begin{center}
        \begin{tabular}{ c | c c c c }
         & 1 & $a$ & $b$ & $c$ \\ 
        \hline
         1  &  1  & $a$ & $b$ & $c$ \\  
        $a$ & $a$ & $b$ & $c$ & 1 \\  
        $b$ & $b$ &  &  &  \\ 
        $c$ & $c$ &  &  &  \\ 
        \end{tabular}
    \end{center}

    Now $ba$ is either $c$ or 1. By the same logic it must be $c$ and $ac = 1$. This makes it easy to fill out the rest of the table $bb = 1$, $bc = a$, $cb = a$, $cc = b$.

    \begin{center}
        \begin{tabular}{ c | c c c c }
         & 1 & $a$ & $b$ & $c$ \\ 
        \hline
         1  &  1  & $a$ & $b$ & $c$ \\  
        $a$ & $a$ & $b$ & $c$ &  1  \\  
        $b$ & $b$ & $c$ &  1  & $a$ \\ 
        $c$ & $c$ &  1  & $a$ & $b$ \\ 
        \end{tabular}
    \end{center}

    But here, we see that $aa = b$ and so $aaaa = bb = 1$. Thus $a$ has order 4, so the table is \textbf{not} the one we're looking for. 

    \textbf{Case $aa = 1$}: Then we must have $ab = ba = c$ and $ac = ca = b$. More deductions show that, $bb = 1$, $cc = 1$, $bc = cb = a$. The tables is:

    \begin{center}
        \begin{tabular}{ c | c c c c }
         & 1 & $a$ & $b$ & $c$ \\ 
        \hline
         1  &  1  & $a$ & $b$ & $c$ \\  
        $a$ & $a$ & $1$ & $c$ & $b$ \\  
        $b$ & $b$ & $c$ & 1 & $a$ \\ 
        $c$ & $c$ & $b$ & $a$ & 1 \\ 
        \end{tabular}
    \end{center}

    So this is the unique table of $G$. Clearly $G$ is abelian.
    
\end{enumerate}

Exercises 3, 9, pp. 27-28.

We will have to use the fact that $D_{2n} = \{1, r, \dots, r^{n-1}, s, sr, \dots, sr^{n-1}\}$. This is not easy to prove formally because we don't have a formal definition of a presentation yet, so for now I will take it to be true.

\begin{enumerate}
    \item [3.] Since $x$ is not a rotation, it must be of the form $sr^i$ for $0 \le i < n$. Then $(sr^i)^2 = sr^isr^i = ssr^{-i}r^i = 1 * 1 = 1$. Thus $sr^i$ has order 2. 
    
    We can combine $s$ and $sr$ to get $ssr = r$. Hence $s$ and $sr$ generate $r$, which together will generate $D_{2n}$.

    \item[9.] Note that the orientation of the tetrahedron is determined the orientation of a single edge. Call the edge $ab$. Then there are 4 choices for $a$ and $3$ choices for $b$, giving a total of $12$ orientations, or $12$ possible rigid motions we can do.

\end{enumerate}
Exercises 2, 4, 13, 16, 20, pp. 32-34.

\begin{enumerate}
    \item[2.] Write everything in cycle notation. We have $\sigma = (1,13,5,10) (3,15,8) (4,14,11,7,12,9)$ and $\tau = (1,14) (2,9,15,13,4) (3,10) (5,12,7) (8,11)$.
    
    Then quick calculations show that: 
    \begin{align*}
        \sigma^2 &= (1,5) (5,10) (3,15,8) (4,11,12) (14,7,9) \\
        \sigma \tau &= (1,11,3) (2,4) (5,9,8,7,10,15) (13,14) \\
        \tau \sigma &= (1,4) (2,9) (3,13,12,15,11,5) (8,10,14) \\
        \tau^2 \sigma = \tau (\tau\sigma) &= (1,2,15,8,3,4,14,11,12,13,7,5,10)
    \end{align*}

    \item[4.] 
    \begin{enumerate}
        \item [(a)] $S_3$ has 6 elements: $\{(), (12), (23), (13), (123), (132)\}$. Direct calculation shows that these elements have orders $0, 2, 2, 2, 3, 3$.
        \item [(b)] $S_4$ has 24 elements: 
        \begin{enumerate}
            \item 1 identity: $()$; order = 0;
            \item 6 transpositions: $(12), (23), (34), (14), (13), (24)$; order = 2;
            \item 3 disjoint products of transpositions: $(12)(34), (23)(14), (13)(24)$; order = 2;
            \item 8 3-cycles: $(123), (132), (124), (142), (134), (143), (234), (243)$; order = 3;
            \item 6 4-cycles: $(1234), (1243), (1324), (1342), (1423), (1432)$; order = 4.
        \end{enumerate}
    \end{enumerate} 

    \item[13.] $(\Rightarrow)$: If $\sigma$ is a product of commuting 2-cycles. Let $\sigma = (a_1 b_1) (a_2 b_2) \dots (a_k b_k)$. Then \[\sigma^2 = (a_1 b_1) (a_2 b_2) \dots (a_k b_k)(a_1 b_1) (a_2 b_2) \dots (a_k b_k) = (a_1 b_1)^2 \dots (a_k b_k)^2,\] since all the factors commute. But squaring a 2-cycle makes the term vanish, so $\sigma^2 = 1$ and $\sigma$ has order 2.
    
    $(\Leftarrow)$: Suppose $\sigma$ has order 2. Decompose $\sigma$ as the product of disjoint cycles $c_1, \dots c_k$, with lengths $\ell_1, \dots, \ell_k$. Since $\sigma^2 = 1$ and disjoint cycles do not affect each other, we must have $c_1^2 = c_2^2 = \dots = c_k^2 = 1$. Then $\ell_1 = \dots = \ell_k = 2$ and everything is a 2-cycle, as desired.

    \item[16.] We make a combinatorial argument. To form an $m$-cycle we must choose $m$ objects out of $n$, where order matters. This gives $n(n-1) \dots (n - m + 1)$ possible choices. However, for each cycle, we have $m$ different representations created by shifting the cycle over $m$ times. Hence we have overcounted by a factor of $m$. Thus the final answer is 
    \[\frac{n(n-1)\dots(n-m+1)}{m}.\] 

    \item[20.] We know that $S_3 = \{(), (12), (23), (13), (123), (132)\}$ and that $(12)(23) = (123)$. So $(12)$ and $(23)$ can generate $()$, $(123)$, and $(132)$. Then $(132)(12) = (23)$. Hence $a = (12)$ and $b = (23)$ generate $S_3$. The relations are at least $a^2 = b^2 = 1$. 
    This isn't enough though, because it tells us nothing about how $ab = (123)$ behaves. Thus we need another relation $(ab)^3 = 1$ to constrain $(123)$. This gives \[S_3 = \langle a, b : a^2 = b^2 = 1, (ab)^3 = 1\rangle.\]
\end{enumerate}

Exercises 17, 18, pp. 40.

\begin{enumerate}
    \item[17.] Let $i : G \to G$ be the map $g \mapsto g^{-1}$. Then $i(gh) = (gh)^{-1} = h^{-1}g^{-1}$. Clearly $i(gh) = i(g)i(h)$ if and only if $h^{-1}$ commutes with $g^{-1}$ for all $g, h \in G$, i.e. $G$ is abelian.
    \item[18.] Let $s : G \to G$ be the map $g \mapsto g^2$. Then $s(gh) = (gh)^2 = ghgh$. If $G$ is abelian, then $ghgh = gghh = g^2 h^2 = s(g)s(h)$. Thus $s$ is a homomorphism. Conversely, if $s(gh) = s(g)s(h)$, then we can take $ghgh = gghh$ and multiply by $g^{-1}$ on the left and $h^{-1}$ on the right to cancel. The result is $gh = hg$. This holds for any $g, h \in G$, therefore $G$ is abelian.
\end{enumerate}

Exercises 18, 19, pp. 45.

\begin{enumerate}
    \item [18.] Let $H$ be a left action on $A$. The relation $a \sim b$ defined by $a = hb$ for some $h \in H$ defines a equivalence relation:
    
    \textit{Reflexive:} Let $h = 1$. Then clearly $a = 1a$ for all $a \in A$. Hence $a \sim a$ for all $a \in A$. 

    \textit{Symmetric:} Let $a \sim b$ with $a = hb$. Then $h^{-1}a = b$. Thus $b \sim a$.

    \textit{Transitive:} Let $a \sim b$ and $b \sim c$ with $a = h_1 b$ and $b = h_2 c$. Then substitute $b$ to see that $a = (h_1 h_2) c$. Hence $a \sim c$.

    \item [19.] Define $f : H \to \mathcal O$ to be the map $h \mapsto hx$. If $f(x) = f(y)$, then $hx = hy$. Cancelling implies $x = y$; thus $f$ is injective. Furthermore, for any $y \in \mathcal O$, we know by definition that $y \sim x$. Hence there is some $h \in H$ such that $y = hx$. Thus $f(h) = y$, and $f$ is surjective. The two combine show that $f$ is bijective. We conclude that $|H| = |\mathcal O|$. 
    
    But this is true for every orbit $\mathcal O$, so all the orbits $\mathcal O_1, \mathcal O_2, \dots, \mathcal O_k$ have size $|H|$. Since the orbits partition $G$, we have 
    \[|G = \sum_{i = 1}^k |\mathcal O_i| = \sum_{i = 1}^k |H| = k|H|.\]
    (We may do the sum since $G$ is finite.) Therefore we have \textit{Lagrange's Theorem}: $|H|$ divides $|G|$ for any subgroup $H$ of a finite group $G$.
\end{enumerate}

\end{document}