%%%%%%%%%%%%%%%%%%%%%%%%%%%%%%%%%%%%%%%%%%%%%%
%%                                          %%
%% USE THIS FILE TO SUBMIT YOUR SOLUTIONS   %%
%%                                          %%
%% You must have the usamts.tex file in     %%
%% the same directory as this file.         %%
%% You do NOT need to submit this file or   %%
%% usamts.tex with your solutions.  You     %%
%% only need to submit the output PDF file. %%
%%                                          %%
%% DO NOT ALTER THE FILE usamts.tex         %%
%%                                          %%
%% If you have any questions or problems    %%
%% using this file, or with LaTeX in        %%
%% general, please go to the LaTeX          %%
%% forum on the Art Of Problem Solving      %%
%% web site, and post your problem.         %%
%%                                          %%
%%%%%%%%%%%%%%%%%%%%%%%%%%%%%%%%%%%%%%%%%%%%%%

%%%%%%%%%%%%%%%%%%%%%%%%%%%%%%%%%%%%%%%%%%%%
%% DO NOT ALTER THE FOLLOWING LINES
\documentclass[12pt]{article}
\usepackage{amsmath,amssymb,amsthm,amsfonts,tabularx}
\usepackage[pdftex]{graphicx}
% \usepackage{mathptmx}
% \usepackage{geometry}
%  \geometry{
%  b5paper,
%  left=15mm,
%  right=15mm,
%  top=15mm,
%  bottom.
% }
\graphicspath{ {./images/} }
\usepackage{fancyhdr}
\pagestyle{fancy}
\usepackage{setspace}
\usepackage{csquotes}
%%%%%%%%%%%%%%%%%%%%%%%%%%%%%%%%%%%%%%%%%%%
%%                                       %%
%% Students: DO NOT MODIFY THIS FILE!!!  %%
%%                                       %%
%%%%%%%%%%%%%%%%%%%%%%%%%%%%%%%%%%%%%%%%%%%

%% USAMTS style sheet
%% last modified: 23-Jul-2004

%% Student and Year/Round data
\newcommand{\realname}[1]{\newcommand{\printrealname}{#1}}
\newcommand{\pset}[1]{\newcommand{\printpset}{#1}}

%% Pagestyle setup
\setlength{\headheight}{0.75in}
\setlength{\oddsidemargin}{0in}
\setlength{\evensidemargin}{0in}
\setlength{\voffset}{-.5in}
\setlength{\headsep}{10pt}
\setlength{\textwidth}{6.5in}
\setlength{\headwidth}{6.5in}
\setlength{\textheight}{8in}
\lhead{Math 410}
\chead{\Large \textbf{Homework \printpset}}
\rhead{\printrealname}
\rfoot{Page \thepage}
\renewcommand{\headrulewidth}{0.5pt}
\renewcommand{\footrulewidth}{0.3pt}
\setlength{\textwidth}{6.5in}


\renewcommand{\baselinestretch}{1}
%% DO NOT ALTER THE ABOVE LINES
%%%%%%%%%%%%%%%%%%%%%%%%%%%%%%%%%%%%%%%%%%%%


%% If you would like to use Asymptote within this document (which is optional), 
%% you can find out how at the following URL:
%%
%%   http://www.artofproblemsolving.com/Wiki/index.php/Asymptote:_Advanced_Configuration
%%
%% As explained there, you will want to uncomment the line below.  But be
%% sure to check the website because there are several other steps that must 
%% be followed.
%% \usepackage{asymptote}

\newtheorem*{prop}{Proposition}
\newtheorem*{corollary}{Corollary}
\newtheorem*{lemma}{Lemma}
\theoremstyle{remark}
\newtheorem*{defn}{Definition}

\newtheoremstyle{named}{}{}{}{}{\bfseries}{.}{.5em}{\thmnote{Problem #3}}
\theoremstyle{named}
\newtheorem*{theorem}{Theorem}

%% Enter your real name here
%% Example: \realname{David Patrick}
\realname{Hanting Zhang}
\pset{6}
\mathclass{425A}

\renewcommand{\bf}{\mathbf}
\renewcommand{\implies}{\Rightarrow}
\newcommand{\coimplies}{\Leftarrow}

\begin{document}

\begin{theorem}[5]
    For any two real sequences \(\{a_n\}, \{b_n\}\), prove that 
    \[\limsup_{n \to \infty} (a_n + b_n) \le \limsup_{n \to \infty} a_n + \limsup_{n \to \infty} b_n,\]
    provided the sum on the right is not of the form \(\infty - \infty\).
\end{theorem}

\begin{proof}
    Set \(a^* = \limsup_{n \to \infty} a_n\), \(b^* = \limsup_{n \to \infty} b_n\), and \(c^* = \limsup_{n \to \infty} (a_n + b_n)\).

    We do not consider the case \(a^* = \infty, b^* = -\infty\). If at least one of \(a^* = \infty\) or \(b^* = \infty\) (or both), then we necessarily have \(c^* \le \infty\). 
    
    Otherwise, if \(a^* = -\infty\), then \(a_n \to -\infty\). Then any subsequence of \(\{a_n + b_n\}\) also tends to \(-\infty\). Thus \(c^* = -\infty \le -\infty + b^*\). The same argument can be done in the case where \(b^* = -\infty\).

    Hence we may assume that \(a^*, b^* \in \mathbb R\). By Theorem 3.17 in the textbook, we have for any \(\epsilon > 0\), there are \(N_s, N_t \in \mathbb N\) such that \(n \ge \max(N_s, N_t)\) implies \(a_n < a^* + \epsilon/2\) and \(b_n < b^* + \epsilon/2\). 
    Thus \(a_n + b_n \le a^* + b^* + \epsilon\), and all subsequential limits of \(\{a_n + b_n\}\) are bounded by \(a^* + b^* + \epsilon\). 
    By definition \(\sup E = c^*\), where \(E\) is the set of subsequential limits of \(\{a_n + b_n\}\), so we must have \(c^* \le a^* + b^* + \epsilon\). Since \(\epsilon > 0\) is arbitrary, we conclude that 
    \[c^* \le a^* + b^* \implies \limsup_{n \to \infty} (a_n + b_n) \le \limsup_{n \to \infty} a_n + \limsup_{n \to \infty} b_n.\]
\end{proof}

\begin{theorem}[7]
    Prove that the convergence of \(\sum a_n\) implies the convergence of 
    \[\sum \frac{\sqrt{a_n}}{n},\]
    if \(a_n \ge 0\).
\end{theorem}

\begin{proof}
    The idea to this proof is that we must somehow linearize \(\frac{\sqrt{a_n}}{n}\) by bounding it with \(p_n a_n + q_n\). Then all that is required for \[\sum \frac{\sqrt{a_n}}{n}\] to converge is for 
    \[\sum p_n a_n + q_n = \sum p_n a_n + \sum q_n\] to converge. 

    Indeed, pick \(p_n = 1\) and \(q_n = \frac{1}{n^2}\). Since \(a_n, n \ge 0\), we have 
    \begin{align*}
        0 &\le a_n^2 + \frac{a_n}{n^2} + \frac{1}{n^4} \\
        \implies \frac{a_n}{n^2} &\le a_n^2 + 2\frac{a_n}{n^2} + \frac{1}{n^4} = \left(a_n + \frac{1}{n^2}\right) \\
        \implies \frac{\sqrt{a_n}}{n} &\le a_n + \frac{1}{n^2}.
    \end{align*}
    Then both \(\sum p_n a_n = \sum a_n\) and \(\sum q_n = \sum n^{-2}\) clearly converge. Hence \(\sum \sqrt{a_n}/n\) converges.
\end{proof}

\begin{theorem}[8]
    If \(\sum a_n\) converges, and if \(\{b_n\}\) is monotonic and bounded, prove that \(\sum a_n b_n\) converges. 
\end{theorem}

\begin{proof}
    Since \(\{b_n\}\) is bounded and montonic, it converges to some \(b \in \mathbb R\). If \(b_n\) is increasing, set \(c_n = b - b_n\), otherwise \(c_n = b_n - b\). This new sequence \(c_n\) is decreasing and converges to \(0\) by construction, therefore we may apply Theorem 3.42 from the textbook to obtain the convergence of \(\sum a_n c_n\). 
    Whether we have \(c_n = b - b_n\) or \(c_n = b_n - b\), the sum \(\sum a_n c_n\) differs from \(\sum a_n b_n\) by some constant of \(\pm \sum a_n b\). Hence \(\sum a_n b_n\) converges.
\end{proof}

\end{document}