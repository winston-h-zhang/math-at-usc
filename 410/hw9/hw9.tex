%%%%%%%%%%%%%%%%%%%%%%%%%%%%%%%%%%%%%%%%%%%%%%
%%                                          %%
%% USE THIS FILE TO SUBMIT YOUR SOLUTIONS   %%
%%                                          %%
%% You must have the usamts.tex file in     %%
%% the same directory as this file.         %%
%% You do NOT need to submit this file or   %%
%% usamts.tex with your solutions.  You     %%
%% only need to submit the output PDF file. %%
%%                                          %%
%% DO NOT ALTER THE FILE usamts.tex         %%
%%                                          %%
%% If you have any questions or problems    %%
%% using this file, or with LaTeX in        %%
%% general, please go to the LaTeX          %%
%% forum on the Art Of Problem Solving      %%
%% web site, and post your problem.         %%
%%                                          %%
%%%%%%%%%%%%%%%%%%%%%%%%%%%%%%%%%%%%%%%%%%%%%%

%%%%%%%%%%%%%%%%%%%%%%%%%%%%%%%%%%%%%%%%%%%%
%% DO NOT ALTER THE FOLLOWING LINES
\documentclass[12pt]{article}
\usepackage{amsmath,amssymb,amsthm,amsfonts,tabularx}
\usepackage[pdftex]{graphicx}
\usepackage[dvipsnames]{xcolor}
\usepackage{fancyhdr}
\usepackage{parskip}
\usepackage[shortlabels]{enumitem}
\pagestyle{fancy}
\usepackage{setspace}
\newcommand{\realname}[1]{\newcommand{\printrealname}{#1}}
\newcommand{\pset}[1]{\newcommand{\printpset}{#1}}
\newcommand{\mathclass}[1]{\newcommand{\printmathclass}{#1}}

%% Pagestyle setup
\setlength{\headheight}{0.75in}
\setlength{\oddsidemargin}{0in}
\setlength{\evensidemargin}{0in}
\setlength{\voffset}{-.5in}
\setlength{\headsep}{10pt}
\setlength{\textwidth}{6.5in}
\setlength{\headwidth}{6.5in}
\setlength{\textheight}{8in}
\lhead{Math \printmathclass}
\chead{\Large \textbf{Homework \printpset}}
\rhead{\printrealname}
\rfoot{Page \thepage}
\renewcommand{\headrulewidth}{0.5pt}
\renewcommand{\footrulewidth}{0.3pt}
\setlength{\textwidth}{6.5in}
\renewcommand{\baselinestretch}{1}
\setenumerate[0]{label=(\alph*)}

\newtheorem*{prop}{Proposition}
\newtheorem*{corollary}{Corollary}
\newtheorem*{lemma}{Lemma}
\theoremstyle{remark}
\newtheorem*{defn}{Definition}

\newtheoremstyle{named}{}{}{}{}{\bfseries}{.}{.5em}{\thmnote{Problem #3}}
\theoremstyle{named}
\newtheorem*{theorem}{Theorem}

%% DO NOT ALTER THE ABOVE LINES
%%%%%%%%%%%%%%%%%%%%%%%%%%%%%%%%%%%%%%%%%%%%


%% If you would like to use Asymptote within this document (which is optional), 
%% you can find out how at the following URL:
%%
%%   http://www.artofproblemsolving.com/Wiki/index.php/Asymptote:_Advanced_Configuration
%%
%% As explained there, you will want to uncomment the line below.  But be
%% sure to check the website because there are several other steps that must 
%% be followed.
%% \usepackage{asymptote}

%% Enter your real name here
%% Example: \realname{David Patrick}
\realname{Hanting Zhang}
\pset{9}
\mathclass{410}

\newcommand{\todo}{\textcolor{red}{\textbf{TODO} }}
\renewcommand{\a}{\alpha}
\renewcommand{\b}{\beta}
\newcommand{\Z}{\mathbb Z}
\newcommand{\F}{\mathbb F}
\newcommand{\Q}{\mathbb Q}
\renewcommand{\bf}{\mathbf}
\renewcommand{\implies}{\Rightarrow}
\newcommand{\coimplies}{\Leftarrow}
\newcommand{\rad}{\text{rad}\hspace*{0.7mm}}
\renewcommand{\em}{\varnothing}
\renewcommand{\mod}{\text{ mod }}

\begin{document}

Exercises 3, 7, 8, 10 pp. 277-279.

\begin{theorem}[3]
    Let \(R\) be a Euclidean Domain. Let \(m\) be the minimum integer in the set of norms of nonzero elements of \(R\). Prove that every nonzero element of \(R\) of norm \(m\) is a unit. Deduce that a nonzero element of norm zero (if such a element exists) is a unit.
\end{theorem}

\begin{theorem}[7]
    Find a generator for the ideal \((85, 1 + 13i)\) in \(\Z[i]\), i.e., a greatest common divisor of 85 and \(1 = 13i\), by the Euclidean Algorithm. Do the same for the ideal \((47 - 13i, 53 + 56i)\).
\end{theorem}

\begin{theorem}[8]
    Let \(F = \Q(\sqrt{D})\) be a quadratic field with associated quadratic integer ring \(\mathcal O\) and field norm \(N\) as in Section 7.1.
    \begin{enumerate}
        \item Suppose \(D\) is \(-1, -2, -3, -7\) or \(-11\). Prove that \(\mathcal O\) is a Euclidean Domain with respect to \(N\). [Modify the proof for \(\Z[i]\) (\(D = -1\)) in the text.]
        \item Suppose that \(D = -43, -67\) or \(-163\). Prove that \(\mathcal O\) is not a Euclidean Domain with respect to any norm. [Apply the same proof as for \(D = -19\) in the text.]
    \end{enumerate}
\end{theorem}

\begin{theorem}[10]
    Prove that the quotient ring \(\Z[i]/I\) is finite for any nonzero ideal \(I\) of \(\Z[i]\).
\end{theorem}

Exercises 1, 3, 4, 5, 6 pp. 282-283.

\begin{theorem}[1]
    Prove that in a Principal Ideal Domain two ideals \((a)\) and \((b)\) are comaximal if and only if a greatest common divisor of \(a\) and \(b\) is 1 (in which case \(a\) and \(b\) are said to be \textit{coprime} or \textit{relatively prime}.)
\end{theorem}

\begin{theorem}[3]
    Prove that a quotient of a P.I.D. by a prime ideal is once again a P.I.D..
\end{theorem}

\begin{theorem}[4]
    Let \(R\) be an integral domain. Prove that if the following two conditions hold then \(R\) is a P.I.D.:
    \begin{enumerate}
        \item [(i)] any two nonzero elements \(a\) and \(b\) in \(R\) have a greatest common divisor which can be written in the form \(ra + sb\) for some \(r, s \in R\), and
        \item [(ii)] if \(a_1, a_2, a_3, \dots\) are nonzero elements of \(R\) such that \(a_{i + 1} \mid a_i\) for all \(i\), then there is a positive integer \(N\) such that \(a_n\) is a unit times \(a_N\) for all \(n \ge N\). 
    \end{enumerate}
\end{theorem}

\begin{theorem}[5]
    Let \(R\) be the quadratic integer ring \(\Z[\sqrt{-5}]\). Define the ideals \(I_2 = (2, 1 + \sqrt[]{-5})\), \(I_3 = (3, 2 + \sqrt[]{-5})\), and \(I_3' = (3, 2 - \sqrt[]{-5})\).
    \begin{enumerate}
        \item Prove that \(I_2\), \(I_3\), and \(I_3'\) are nonprincipal ideals in \(R\).
        \item Prove that the product of two nonprincipal ideals can be principal by showing that \(I_2^2\) is the principal ideal generated by 2, i.e., \(I_2^2 = (2)\).
        \item Prove similarily that \(I_2 I_3 = (1 - \sqrt[]{-5})\) and \(I_2 I_3' = (1 + \sqrt[]{-5})\) are principal. Conclude that the principal ideal \((6)\) is the product of 4 ideals: \((6) = I_2^2 I_3 I_3'\).
    \end{enumerate}
\end{theorem}

\begin{theorem}[6]
    Let \(R\) be an integral domain and suppose that every \textit{prime} ideal in \(R\) is principal. This exercise proves that every ideal of \(R\) is principal, i.e., \(R\) is a P.I.D.
    \begin{enumerate}
        \item Assume that the set of ideals of \(R\) that are not principal is nonempty and prove that this set has a maximal element under inclusion (which, by hypothesis, is not prime). [Use Zorn's Lemma.]
        \item Let \(I\) be an ideal which is maximal with respect to being nonprincipal, and let \(a, b\in R\) with \(ab \in I\) but \(a \notin I\) and \(b \notin I\). Let \(I_a = (I, a)\) be the ideal generated by \(I\) and \(a\), let \(I_b = (I, b)\) be the ideal generated by \(I\) and \(b\), and define \(J = \{r \in R \mid rI_a \subseteq J\}\). Prove that \(I_a = (\a)\) and \(J = (\b)\) are principal ideals in \(R\) with \(I \subset I_b\) and \(I_a J = (\a \b) \subseteq I\).
        \item If \(x \in I\) show that \(x = s \a\) for some \(s \in J\). Deduce that \(I = I_a J\) is principal, a contradiction, and conclude that \(R\) is a P.I.D.
    \end{enumerate}
\end{theorem}

Exercises 6, 8 pp. 282-283.

\todo

\end{document}