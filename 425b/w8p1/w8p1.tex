\documentclass[12pt]{article}
\usepackage{amsmath,amssymb,amsthm,amsfonts,tabularx}
\usepackage[pdftex]{graphicx}
\usepackage[dvipsnames]{xcolor}
\usepackage{fancyhdr}
\usepackage{parskip}
\usepackage[shortlabels]{enumitem}
\pagestyle{fancy}
\usepackage{setspace}
\usepackage{pgfplots}
\pgfplotsset{compat=1.18}
\newcommand{\realname}[1]{\newcommand{\printrealname}{#1}}
\newcommand{\pset}[1]{\newcommand{\printpset}{#1}}
\newcommand{\mathclass}[1]{\newcommand{\printmathclass}{#1}}

%% Pagestyle setup
\setlength{\headheight}{0.75in}
\setlength{\oddsidemargin}{0in}
\setlength{\evensidemargin}{0in}
\setlength{\voffset}{-.5in}
\setlength{\headsep}{10pt}
\setlength{\textwidth}{6.5in}
\setlength{\headwidth}{6.5in}
\setlength{\textheight}{8in}
\lhead{Math \printmathclass}
\chead{\Large \textbf{\printpset}}
\rhead{\printrealname}
\rfoot{Page \thepage}
\renewcommand{\headrulewidth}{0.5pt}
\renewcommand{\footrulewidth}{0.3pt}
\setlength{\textwidth}{6.5in}
\renewcommand{\baselinestretch}{1}
\setenumerate[0]{label=(\alph*)}
\newcommand{\todo}{\textcolor{red}{\textbf{TODO }}}

\newtheorem*{prop}{Proposition}
\newtheorem*{corollary}{Corollary}
\newtheorem{lemma}{Lemma}
\theoremstyle{remark}
\newtheorem*{defn}{Definition}
\newtheorem*{remark}{Remark}

\newtheoremstyle{named}{}{}{}{}{\bfseries}{.}{.5em}{\thmnote{Problem #3}}
\theoremstyle{named}
\newtheorem*{theorem}{Theorem}
\allowdisplaybreaks

%% DO NOT ALTER THE ABOVE LINES
%%%%%%%%%%%%%%%%%%%%%%%%%%%%%%%%%%%%%%%%%%%%


%% If you would like to use Asymptote within this document (which is optional), 
%% you can find out how at the following URL:
%%
%%   http://www.artofproblemsolving.com/Wiki/index.php/Asymptote:_Advanced_Configuration
%%
%% As explained there, you will want to uncomment the line below.  But be
%% sure to check the website because there are several other steps that must 
%% be followed.
%% \usepackage{asymptote}

%% Enter your real name here
%% Example: \realname{David Patrick}
\realname{Hanting Zhang}
\pset{W8P1}
\mathclass{425B}

\renewcommand{\a}{\alpha}
\renewcommand{\b}{\beta}
\renewcommand{\d}{\delta}
\newcommand{\e}{\varepsilon}
\newcommand{\Z}{\mathbb Z}
\newcommand{\N}{\mathbb N}
\newcommand{\Q}{\mathbb Q}
\newcommand{\R}{\mathbb R}
\newcommand{\C}{\mathbb C}
\renewcommand{\bf}{\mathbf}
\newcommand{\id}[1]{\text{id}_{#1}}
\renewcommand{\implies}{\Rightarrow}
\newcommand{\coimplies}{\Leftarrow}
\renewcommand{\em}{\varnothing}
\renewcommand{\Im}{\text{Im}}
\newcommand{\abs}[1]{|#1|}
\newcommand{\bigabs}[1]{\left|#1\right|}
\newcommand{\Rloc}{\mathcal R_{\text{loc}}}

%% should be reset

\begin{document}

\begin{theorem}[2.1]
    This exercise gives you some practice working with some technical aspects of convolutions and approximate identities. To minimize the monotony of the write-up, challenge yourself to find shortcuts to make your solution as efficient as possible.
    
    For each \(n \in \N\), define \(\phi_n : \R \to \R\) by \(\phi_n = n 1_{I_n}\), where \(I_n = \left(-\frac{1}{2n}, \frac{1}{2n}\right)\). Define \(f : \R \to \R\) by \(f(x) = |x| 1_{(-2, 2)}(x)\). Write down an (\(n\)-dependent) formula for the function \(g_n = \phi_n * f\). You should write your formula explicitly enough so that is does not contain any integrals. Use basic calculus considerations to identify intervals where \(\phi_n * f\) is concave up, concave down, increasing, decreasing, differentiable, etc.
\end{theorem}

\begin{proof}
    Think about the convolution visually as we slide \(\phi_n\) across \(f\). With this in mind, the behavior of the convolution reduces to casework.
    \begin{enumerate}
        \item When \(x \leq -2 - \frac{1}{2n}\) or \(2 + \frac{1}{2n} \leq x\), the \(\phi_n\) block is disjoint from \((-2, 2)\), so the integral is zero.
        \item When \(x \in (-2 - \frac{1}{2n}, -2 + \frac{1}{2n})\), the interval where the integral is non-zero is \((-2, 2) \cap (x - \frac{1}{2n}, x + \frac{1}{2n}) = (-2, x+\frac{1}{2n})\), thus the integral is:
        \begin{align*}
            \int_{-2}^{x+\frac{1}{2n}} -ny dy &= \left . -\frac{ny^2}{2}\right |_{-2}^{x+\frac{1}{2n}} \\
            &= \frac{4n}{2} - \frac{n(x + \frac{1}{2n})^2}{2}.
        \end{align*}
        Note that we may replace \(|y|\) with \(-y\) because \(y < 0\) throughout.
        \item When \(x \in (-2 + \frac{1}{2n}, -\frac{1}{2n})\), the interval where the integral is non-zero is \((x-\frac{1}{2n}, x+\frac{1}{2n})\), thus the integral is:
        \begin{align*}
            \int_{x-\frac{1}{2n}}^{x+\frac{1}{2n}} -ny dy &= \left . -\frac{ny^2}{2}\right |_{x-\frac{1}{2n}}^{x+\frac{1}{2n}} \\
            &= \frac{n(x - \frac{1}{2n})^2}{2} - \frac{n(x + \frac{1}{2n})^2}{2} \\
            &= \frac{n}{2}(x - \frac{1}{2n} + x + \frac{1}{2n})(x - \frac{1}{2n} - (x + \frac{1}{2n})) \\
            &= \frac{n}{2}(2x)(-2/2n) = -x
        \end{align*}
        Note that we may replace \(|y|\) with \(-y\) because \(y < 0\) throughout.
        \item When \(x \in (-\frac{1}{2n}, \frac{1}{2n})\), things get a bit more complicated because we can no longer get rid of the absolute values. The relevant interval we want is still \(((x - \frac{1}{2n}, x + \frac{1}{2n}))\), but we need to split between \(x < 0\) and \(x > 0\):
        \begin{align*}
            \int_{x - \frac{1}{2n}}^{x + \frac{1}{2n}} n|y|dy &= \int_{x - \frac{1}{2n}}^{0} -nydy + \int_{0}^{x + \frac{1}{2n}} ny dy \\
            &= \left . \frac{ny^2}{2}\right |_{0}^{x-\frac{1}{2n}} + \left . \frac{ny^2}{2}\right |_{0}^{x+\frac{1}{2n}} \\
            &= \frac{n(x - \frac{1}{2n})^2}{2} + \frac{n(x + \frac{1}{2n})^2}{2} \\
            &= nx^2 + \frac{1}{4n} \\
        \end{align*}
        \item When \(x \in (\frac{1}{2n}, 2 - \frac{1}{2n})\), we can abuse symmetry and reuse the result from case (c) to deduce that the integral must be:
        \begin{align*}
            \int_{x-\frac{1}{2n}}^{x+\frac{1}{2n}} ny dy &= x
        \end{align*}
        \item When \(x \in (2 - \frac{1}{2n}, 2 + \frac{1}{2n})\), we can abuse symmetry and reuse the result from case (b) to deduce that the integral must be:
        \begin{align*}
            \int_{x-\frac{1}{2n}}^{2} ny dy &= \frac{4n}{2} - \frac{n(x + \frac{1}{2n})^2}{2}.
        \end{align*}
    \end{enumerate}
    Thus, in total, our function is 
    \begin{align*}
        g(x) = \begin{cases}
            0 & x \leq -2 - \frac{1}{2n} \lor 2 + \frac{1}{2n} \leq x \\
            -\frac{n}{2}\left(x + \frac{1}{2n}\right)^2 + 2n & x \in [-2 - \frac{1}{2n}, -2 + \frac{1}{2n}] \\
            -x & x \in [-2 + \frac{1}{2n}, -\frac{1}{2n}] \\
            nx^2 + \frac{1}{4n} & x \in [-\frac{1}{2n}, \frac{1}{2n}] \\
            x & x \in [\frac{1}{2n}, 2 - \frac{1}{2n}, ] \\
            -\frac{n}{2}\left(x - \frac{1}{2n}\right)^2 + 2n & x \in [2 - \frac{1}{2n}, 2 + \frac{1}{2n}].
        \end{cases}
    \end{align*}
\end{proof}

\end{document}