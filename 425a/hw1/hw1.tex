%%%%%%%%%%%%%%%%%%%%%%%%%%%%%%%%%%%%%%%%%%%%%%
%%                                          %%
%% USE THIS FILE TO SUBMIT YOUR SOLUTIONS   %%
%%                                          %%
%% You must have the usamts.tex file in     %%
%% the same directory as this file.         %%
%% You do NOT need to submit this file or   %%
%% usamts.tex with your solutions.  You     %%
%% only need to submit the output PDF file. %%
%%                                          %%
%% DO NOT ALTER THE FILE usamts.tex         %%
%%                                          %%
%% If you have any questions or problems    %%
%% using this file, or with LaTeX in        %%
%% general, please go to the LaTeX          %%
%% forum on the Art Of Problem Solving      %%
%% web site, and post your problem.         %%
%%                                          %%
%%%%%%%%%%%%%%%%%%%%%%%%%%%%%%%%%%%%%%%%%%%%%%

%%%%%%%%%%%%%%%%%%%%%%%%%%%%%%%%%%%%%%%%%%%%
%% DO NOT ALTER THE FOLLOWING LINES
\documentclass[12pt]{article}
\usepackage{amsmath,amssymb,amsthm,amsfonts,tabularx}
\usepackage[pdftex]{graphicx}
\graphicspath{ {./images/} }
\usepackage{fancyhdr}
\pagestyle{fancy}
\usepackage{setspace}
\usepackage{csquotes}
%%%%%%%%%%%%%%%%%%%%%%%%%%%%%%%%%%%%%%%%%%%
%%                                       %%
%% Students: DO NOT MODIFY THIS FILE!!!  %%
%%                                       %%
%%%%%%%%%%%%%%%%%%%%%%%%%%%%%%%%%%%%%%%%%%%

%% USAMTS style sheet
%% last modified: 23-Jul-2004

%% Student and Year/Round data
\newcommand{\realname}[1]{\newcommand{\printrealname}{#1}}
\newcommand{\pset}[1]{\newcommand{\printpset}{#1}}
\newcommand{\mathclass}[1]{\newcommand{\printmathclass}{#1}}

%% Pagestyle setup
\setlength{\headheight}{0.75in}
\setlength{\oddsidemargin}{0in}
\setlength{\evensidemargin}{0in}
\setlength{\voffset}{-.5in}
\setlength{\headsep}{10pt}
\setlength{\textwidth}{6.5in}
\setlength{\headwidth}{6.5in}
\setlength{\textheight}{8in}
\lhead{Math \printmathclass}
\chead{\Large \textbf{Homework \printpset}}
\rhead{\printrealname}
\rfoot{Page \thepage}
\renewcommand{\headrulewidth}{0.5pt}
\renewcommand{\footrulewidth}{0.3pt}
\setlength{\textwidth}{6.5in}


\renewcommand{\baselinestretch}{1}
%% DO NOT ALTER THE ABOVE LINES
%%%%%%%%%%%%%%%%%%%%%%%%%%%%%%%%%%%%%%%%%%%%


%% If you would like to use Asymptote within this document (which is optional), 
%% you can find out how at the following URL:
%%
%%   http://www.artofproblemsolving.com/Wiki/index.php/Asymptote:_Advanced_Configuration
%%
%% As explained there, you will want to uncomment the line below.  But be
%% sure to check the website because there are several other steps that must 
%% be followed.
%% \usepackage{asymptote}

\newtheorem*{prop}{Proposition}
\newtheorem*{corollary}{Corollary}
\newtheorem*{lemma}{Lemma}
\theoremstyle{remark}
\newtheorem*{defn}{Definition}

\newtheoremstyle{named}{}{}{}{}{\bfseries}{.}{.5em}{\thmnote{Problem #3}}
\theoremstyle{named}
\newtheorem*{theorem}{Theorem}

%% Enter your real name here
%% Example: \realname{David Patrick}
\realname{Hanting Zhang}
\pset{1}
\mathclass{425A}
\renewcommand{\bf}{\mathbf}

\begin{document}

\begin{theorem}[1]
    If $r$ is rational $(r \neq 0)$ and $x$ is irrational, prove that $r + x$ and $rx$ are irrational.
\end{theorem}

\begin{proof}
    We prove the contrapositive of both statements. 

    If $r$ is rational and $r + x$ is rational, then we may express both as fractions $r = a / b$ and $r + x = c / d$. Hence $x = (r + x) - x = c / d - a / b = (cb - ad) / bd$, which is clearly rational. Thus completes the first proof.
    
    If $r$ is rational and $rx$ is rational, then again write $r = a / b$ and $rx = c/d$. So then $x = rx / r = (ca)/(bd)$, which is again rational, as desired.
\end{proof}

\begin{theorem}[2]
    Prove that there is no rational number whose square is 12.
\end{theorem}

\begin{proof}
    Assume for the sake of contradiction that there exists some $q \in \mathbb Q$ such that $q^2 = 12$. Then $4 \mid q^2 \implies 2 \mid q$. Hence write $q = 2p$, and substitute to simplify $4p^2 = 12 \implies p^2 = 3$. Now $p$ is rational so we can write it as a reduced fraction $a / b$. 

    Then $a^2/b^2= 3$, implying $a^2= 3b^2$. This is an equation over the integers, so 3 must divide $a$. Write $a = a'/3$. Substitiuting, we have another integer equation $3a'^2= b^2$. By the same logic, 3 divides $b$. But now we conclude that 3 is a common factor of $a$ and $b$, contradicting our assumption that $a/b$ is reduced!
    
    Hence such a $q$ cannot exist.
\end{proof}

\end{document}