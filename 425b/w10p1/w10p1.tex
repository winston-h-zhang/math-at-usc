\documentclass[12pt]{article}
\usepackage{amsmath,amssymb,amsthm,amsfonts,tabularx}
\usepackage[pdftex]{graphicx}
\usepackage[dvipsnames]{xcolor}
\usepackage{fancyhdr}
\usepackage{parskip}
\usepackage[shortlabels]{enumitem}
\pagestyle{fancy}
\usepackage{setspace}
\usepackage{pgfplots}
\pgfplotsset{compat=1.18}
\newcommand{\realname}[1]{\newcommand{\printrealname}{#1}}
\newcommand{\pset}[1]{\newcommand{\printpset}{#1}}
\newcommand{\mathclass}[1]{\newcommand{\printmathclass}{#1}}

%% Pagestyle setup
\setlength{\headheight}{0.75in}
\setlength{\oddsidemargin}{0in}
\setlength{\evensidemargin}{0in}
\setlength{\voffset}{-.5in}
\setlength{\headsep}{10pt}
\setlength{\textwidth}{6.5in}
\setlength{\headwidth}{6.5in}
\setlength{\textheight}{8in}
\lhead{Math \printmathclass}
\chead{\Large \textbf{\printpset}}
\rhead{\printrealname}
\rfoot{Page \thepage}
\renewcommand{\headrulewidth}{0.5pt}
\renewcommand{\footrulewidth}{0.3pt}
\setlength{\textwidth}{6.5in}
\renewcommand{\baselinestretch}{1}
\setenumerate[0]{label=(\alph*)}
\newcommand{\todo}{\textcolor{red}{\textbf{TODO }}}

\newtheorem*{prop}{Proposition}
\newtheorem*{corollary}{Corollary}
\newtheorem{lemma}{Lemma}
\theoremstyle{remark}
\newtheorem*{defn}{Definition}
\newtheorem*{remark}{Remark}

\newtheoremstyle{named}{}{}{}{}{\bfseries}{.}{.5em}{\thmnote{Problem #3}}
\theoremstyle{named}
\newtheorem*{theorem}{Theorem}
\allowdisplaybreaks

%% DO NOT ALTER THE ABOVE LINES
%%%%%%%%%%%%%%%%%%%%%%%%%%%%%%%%%%%%%%%%%%%%


%% If you would like to use Asymptote within this document (which is optional), 
%% you can find out how at the following URL:
%%
%%   http://www.artofproblemsolving.com/Wiki/index.php/Asymptote:_Advanced_Configuration
%%
%% As explained there, you will want to uncomment the line below.  But be
%% sure to check the website because there are several other steps that must 
%% be followed.
%% \usepackage{asymptote}

%% Enter your real name here
%% Example: \realname{David Patrick}
\realname{Hanting Zhang}
\pset{W10P1}
\mathclass{425B}

\renewcommand{\a}{\alpha}
\renewcommand{\b}{\beta}
\renewcommand{\d}{\delta}
\newcommand{\e}{\varepsilon}
\newcommand{\Z}{\mathbb Z}
\newcommand{\N}{\mathbb N}
\newcommand{\Q}{\mathbb Q}
\newcommand{\R}{\mathbb R}
\newcommand{\C}{\mathbb C}
\renewcommand{\bf}{\mathbf}
\newcommand{\id}[1]{\text{id}_{#1}}
\renewcommand{\implies}{\Rightarrow}
\newcommand{\coimplies}{\Leftarrow}
\renewcommand{\em}{\varnothing}
\renewcommand{\Im}{\text{Im}}
\newcommand{\abs}[1]{|#1|}
\newcommand{\bigabs}[1]{\left|#1\right|}
\newcommand{\Rloc}{\mathcal R_{\text{loc}}}

%% should be reset

\begin{document}

\begin{theorem}[3.1]
    Prove Proposition 3.3 for complex inner product spaces, using the following strategy. Choose \(w_v \in W\) such that (36) holds. Fix \(z \in W\) and consider the two functions
    \begin{align*}
        f_z : \R \to [0, \infty), \hspace*{4mm} f_z(\a) = \|v - w_v + \a z\|^2, \\
        g_z : \R \to [0, \infty), \hspace*{4mm} g_z(\b) = \|v - w_v + i\b z\|^2.
    \end{align*}
    Argue that \(f_z'(0) = g_z'(0) = 0\). Then, use this to show that \(\langle v - w_v, z\rangle = 0\). Conclude that \(v - w_v \in W^\perp\).
\end{theorem}

\begin{theorem}[3.2]
    Prove Proposition 3,7, using the following outline.
    \begin{enumerate}
        \item Given \(v \in V\), let \((w_n)_{n = 1}^\infty\) be a sequence in \(W\) such that \(\|w_n - v\| \to \d(v)\) as \(n \to \infty\), where \(\d(v) = \inf_{w \in W}\|v - w\|\). (In fewer than 10 words, cite a reason why such a sequence exists.) Prove that
        \begin{align*}
            \|w_n - w_m\|^2 = 2\|w_n - v\|^2 + 2\|w_m - v\|^2 - 4 \left \| \frac{w_n + w_m}{2} - v \right \|^2,
        \end{align*}
        by applying the parallelogram law to \(w_n - w_m = (w_n - v) - (w_m - v)\).
        \item Use the identity from part (a), together with the definition of \(\d(v)\), to prove that \((w_n)_{n = 1}^\infty\) is a Cauchy sequence. 
        \item Let \(w_v\) denote the element of \(W\) to which the sequence \((w_n)_{n = 1}^\infty\) converges. (The fact that \(w_v\) exists is guaranteed by the completeness of \(W\), together with part (b).) Give a short argument for why \(\|v - w_v\| = \d(v)\).
    \end{enumerate}
\end{theorem}

\begin{theorem}[3.3]
    Prove the second half of Corollary 3.8. That is, prove that if \((V, \langle \cdot, \cdot, \rangle)\) is a real or complex Hilbert space and \(W\) is a subspace of \(V\), which is not complete, then \(V \neq W \oplus W^\perp\).
\end{theorem}

\end{document}