\documentclass[12pt]{article}
\usepackage{amsmath,amssymb,amsthm,amsfonts,tabularx}
\usepackage[pdftex]{graphicx}
\usepackage[dvipsnames]{xcolor}
\usepackage{fancyhdr}
\usepackage{parskip}
\usepackage[shortlabels]{enumitem}
\pagestyle{fancy}
\usepackage{setspace}
\newcommand{\realname}[1]{\newcommand{\printrealname}{#1}}
\newcommand{\pset}[1]{\newcommand{\printpset}{#1}}
\newcommand{\mathclass}[1]{\newcommand{\printmathclass}{#1}}

%% Pagestyle setup
\setlength{\headheight}{0.75in}
\setlength{\oddsidemargin}{0in}
\setlength{\evensidemargin}{0in}
\setlength{\voffset}{-.5in}
\setlength{\headsep}{10pt}
\setlength{\textwidth}{6.5in}
\setlength{\headwidth}{6.5in}
\setlength{\textheight}{8in}
\lhead{Math \printmathclass}
\chead{\Large \textbf{\printpset}}
\rhead{\printrealname}
\rfoot{Page \thepage}
\renewcommand{\headrulewidth}{0.5pt}
\renewcommand{\footrulewidth}{0.3pt}
\setlength{\textwidth}{6.5in}
\renewcommand{\baselinestretch}{1}
\setenumerate[0]{label=(\alph*)}
\newcommand{\todo}{\textcolor{red}{\textbf{TODO }}}

\newtheorem*{prop}{Proposition}
\newtheorem*{corollary}{Corollary}
\newtheorem*{lemma}{Lemma}
\theoremstyle{remark}
\newtheorem*{defn}{Definition}

\newtheoremstyle{named}{}{}{}{}{\bfseries}{.}{.5em}{\thmnote{Problem #3}}
\theoremstyle{named}
\newtheorem*{theorem}{Theorem}
\allowdisplaybreaks

%% DO NOT ALTER THE ABOVE LINES
%%%%%%%%%%%%%%%%%%%%%%%%%%%%%%%%%%%%%%%%%%%%


%% If you would like to use Asymptote within this document (which is optional), 
%% you can find out how at the following URL:
%%
%%   http://www.artofproblemsolving.com/Wiki/index.php/Asymptote:_Advanced_Configuration
%%
%% As explained there, you will want to uncomment the line below.  But be
%% sure to check the website because there are several other steps that must 
%% be followed.
%% \usepackage{asymptote}

%% Enter your real name here
%% Example: \realname{David Patrick}
\realname{Hanting Zhang}
\pset{W2P1}
\mathclass{425B}

\renewcommand{\a}{\alpha}
\renewcommand{\b}{\beta}
\renewcommand{\d}{\delta}
\newcommand{\e}{\varepsilon}
\newcommand{\Z}{\mathbb Z}
\newcommand{\N}{\mathbb N}
\newcommand{\Q}{\mathbb Q}
\newcommand{\R}{\mathbb R}
\newcommand{\C}{\mathbb C}
\renewcommand{\bf}{\mathbf}
\newcommand{\id}[1]{\text{id}_{#1}}
\renewcommand{\implies}{\Rightarrow}
\newcommand{\coimplies}{\Leftarrow}
\renewcommand{\em}{\varnothing}
\renewcommand{\Im}{\text{Im}}
\newcommand{\abs}[1]{|#1|}
\newcommand{\bigabs}[1]{\left|#1\right|}

\begin{document}

\begin{theorem}[2.1]
    Suppose \((a_n)^\infty_{n = 0}\) and \((b_n)^\infty_{n = 0}\) are sequences of complex numbers , and the series \(\sum^\infty_{n = 0} a_n z^n\) and \(\sum^\infty_{n = 0} b_n z^n\) have radii of convergence \(R_1\) and \(R_2\), respectively. Show that the radius of convergence \(R\) of the Cauchy product of these two series satisfies \(R \geq \min\{R_1, R_2\}\). Give an example of two series where strict inequalities folds, \(R > \min \{R_1, R_2\}\). 
\end{theorem}

\begin{proof}
    By proposition 1.2, the series \(\sum_{n = 0}^\infty a_n z^n\) and \(\sum_{n = 0}^\infty b_n z^n\) converges absolutely for values \(z\) such that \(|z| < R_1\) and \(|z| < R_2\), respectively. By Merten's Theorem, the Cauchy product at any point \(z\) such that \(|z| < \min \{R_1, R_2\}\) converges to the product of each series at \(z\). Thus, since we converge on the interval \((0, \min\{R_1, R_2\})\), the radius of convergence for the Cauchy product must at least be \(R \geq \min \{R_1, R_2\}\), as desired.
\end{proof}

An example of the strict inequality is as follows: Let \(f(z) = (1 + z)^{1/2}\) and \(g(z) = (1 + z)^{-1/2}\). We can construct the Maclaurin series of \(f(z)\) and \(g(z)\), which each have radii of convergence 1. This follows because \(f(z)\) blows up at \(z = -1\) and \(g(z)\) blows up at \(x = 1\). However, the Cauchy product of the series' of \(f(z)\) and \(g(z)\) is simply \((1 + z)^{1/2}(1 + z)^{-1/2} = 1 = 1 + 0z^1 + 0z^2 + \cdots\); this has radius of convergence \(\infty > 1\), as desired.

\begin{theorem}[2.2]
    Prove that the Cauchy product of two absolutely convergent series is itself absolutely convergent.
\end{theorem}

\begin{proof}
    Let \(\sum a_n\) and \(\sum b_n\) be two absolutely convergent series, i.e. \(\sum |a_n| < A\) and \(\sum |b_n| < B\) for some constants \(A, B\). 

    Then we have:
    \begin{align*}
        \sum_{n = 0}^N |c_n| &= \sum_{n = 0}^N \left |\sum_{k = 0}^n a_k b_{n - k}\right | \\
        &\leq \sum_{n = 0}^N \sum_{k = 0}^n |a_k| |b_{n - k}| \\
        &= |a_0||b_0| + |a_0||b_1| + |a_1||b_0| + |a_0||b_2| + |a_1||b_1| + |a_2||b_0| + \cdots \\
        &< \sum_{n = 0}^N |a_n| \sum_{k = 0}^{N - n} |b_k| \\
        &< \sum_{n = 0}^N |a_n| B \\
        &< AB
    \end{align*}
    This is independent of \(N\), thus as \(N \to \infty\), \(\sum |c_n|\) is bounded by \(AB\), which proves the absolute convergence of \(\sum c_n\).
\end{proof}

\end{document}