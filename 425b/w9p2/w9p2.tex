\documentclass[12pt]{article}
\usepackage{amsmath,amssymb,amsthm,amsfonts,tabularx}
\usepackage[pdftex]{graphicx}
\usepackage[dvipsnames]{xcolor}
\usepackage{fancyhdr}
\usepackage{parskip}
\usepackage[shortlabels]{enumitem}
\pagestyle{fancy}
\usepackage{setspace}
\usepackage{pgfplots}
\pgfplotsset{compat=1.18}
\newcommand{\realname}[1]{\newcommand{\printrealname}{#1}}
\newcommand{\pset}[1]{\newcommand{\printpset}{#1}}
\newcommand{\mathclass}[1]{\newcommand{\printmathclass}{#1}}

%% Pagestyle setup
\setlength{\headheight}{0.75in}
\setlength{\oddsidemargin}{0in}
\setlength{\evensidemargin}{0in}
\setlength{\voffset}{-.5in}
\setlength{\headsep}{10pt}
\setlength{\textwidth}{6.5in}
\setlength{\headwidth}{6.5in}
\setlength{\textheight}{8in}
\lhead{Math \printmathclass}
\chead{\Large \textbf{\printpset}}
\rhead{\printrealname}
\rfoot{Page \thepage}
\renewcommand{\headrulewidth}{0.5pt}
\renewcommand{\footrulewidth}{0.3pt}
\setlength{\textwidth}{6.5in}
\renewcommand{\baselinestretch}{1}
\setenumerate[0]{label=(\alph*)}
\newcommand{\todo}{\textcolor{red}{\textbf{TODO }}}

\newtheorem*{prop}{Proposition}
\newtheorem*{corollary}{Corollary}
\newtheorem{lemma}{Lemma}
\theoremstyle{remark}
\newtheorem*{defn}{Definition}
\newtheorem*{remark}{Remark}

\newtheoremstyle{named}{}{}{}{}{\bfseries}{.}{.5em}{\thmnote{Problem #3}}
\theoremstyle{named}
\newtheorem*{theorem}{Theorem}
\allowdisplaybreaks

%% DO NOT ALTER THE ABOVE LINES
%%%%%%%%%%%%%%%%%%%%%%%%%%%%%%%%%%%%%%%%%%%%


%% If you would like to use Asymptote within this document (which is optional), 
%% you can find out how at the following URL:
%%
%%   http://www.artofproblemsolving.com/Wiki/index.php/Asymptote:_Advanced_Configuration
%%
%% As explained there, you will want to uncomment the line below.  But be
%% sure to check the website because there are several other steps that must 
%% be followed.
%% \usepackage{asymptote}

%% Enter your real name here
%% Example: \realname{David Patrick}
\realname{Hanting Zhang}
\pset{W9P2}
\mathclass{425B}

\renewcommand{\a}{\alpha}
\renewcommand{\b}{\beta}
\renewcommand{\d}{\delta}
\renewcommand{\t}{\theta}
\newcommand{\e}{\varepsilon}
\newcommand{\Z}{\mathbb Z}
\newcommand{\N}{\mathbb N}
\newcommand{\Q}{\mathbb Q}
\newcommand{\R}{\mathbb R}
\newcommand{\C}{\mathbb C}
\renewcommand{\bf}{\mathbf}
\newcommand{\id}[1]{\text{id}_{#1}}
\renewcommand{\implies}{\Rightarrow}
\newcommand{\coimplies}{\Leftarrow}
\renewcommand{\em}{\varnothing}
\renewcommand{\Im}{\text{Im}}
\newcommand{\abs}[1]{|#1|}
\newcommand{\bigabs}[1]{\left|#1\right|}
\newcommand{\Rloc}{\mathcal R_{\text{loc}}}

%% should be reset

\begin{document}

\begin{theorem}[2.1]
    This exercise is about the formulas (28) and (29) in the definition of trigonometric polynomials.
    \begin{enumerate}
        \item Show that two formulas (28) and (29) are equivalent ways to define the space of complex valued trigonometric functions. That is, given a trigonometric polynomial of the form (28), find the numbers \(a_n, b_n\) so that the same polynomial can be written in the form (29). Then go the other direction -- give a formula for the \(c_n\)'s in terms of the \(a_n\)'s and \(b_n\)'s.
        \item Using part (a), formulate a condition on the coefficients \(c_n\) so that (28) defined a real-valued function. Your condition should be as general as possible.
    \end{enumerate}
\end{theorem}

\begin{proof}
    We proceed with each part separately.
    \begin{enumerate}
        \item We have \(e^{in \t} = \cos (n\t) + i \sin (n\t)\), so 
        \begin{align*}
            p(\t) &= \sum_{n = -N}^{N} c_n e^{i n \t} = \sum_{n = -N}^{N} c_n \cos(n\t) + c_n i \sin(n\t) \\
            &= c_0 + \sum_{n = 1}^{N} c_n \cos(n\t) + c_{-n} \cos(-n\t) + c_n i \sin(n\t) + c_{-n} i \sin(-n\t) \\
            &= c_0 + \sum_{n = 1}^{N} (c_n + c_{-n}) \cos(n\t) + (c_n - c_{-n}) i \sin(n\t).
        \end{align*}
        Thus \(a_0 = c_0\), \(a_n = c_n + c_{-n}\), and \(b_n = i(c_n - c_{-n})\).

        On the other hand, we have \(\cos x = \frac{1}{2}(e^{ix} + e^{-ix})\) and \(\sin x = \frac{1}{2i}(e^{ix} - e^{-ix})\)
        \begin{align*}
            p(\t) &= a_0 + \sum_{n = 1}^{N} a_n \cos(n\t) + b_n \sin(n\t) \\
            &= a_0 + \sum_{n = 1}^{N} \frac{a_n}{2} (e^{in\t} + e^{-in\t}) + \frac{b_n}{2i} (e^{in\t} - e^{-in\t}) \\
            &= a_0 + \sum_{n = 1}^{N} \frac{a_n - i b_n}{2} e^{in\t} + \frac{a_n + i b_n}{2} e^{-in\t} \\
            &= a_0 + \sum_{n = 1}^{N} \frac{a_n - i b_n}{2} e^{in\t} + \sum_{n = -1}^{-N} \frac{a_{-n} + i b_{-n}}{2} e^{in\t}.
        \end{align*}
        Thus we have \(c_0 = a_0\), \(c_n = \frac{1}{2} (a_n - i b_n)\) for \(1 \leq n \leq N\), and \(c_n = \frac{1}{2} (a_{-n} + i b_{-n})\) for \(-N \leq n \leq -1\).

        \item From part (a), for \(p(\t)\) to be real, we need \(a_n\) and \(b_n\) to be real, thus \(c_n + c_{-n}\) and \(i(c_n - c_{-n})\) must both be real. The first condition implies that \(\Im(c_n) = -\Im(c_{-n})\); the second implies that \(\text{Re}(c_n) = \text{Re}(c_{-n})\). Thus we need exactly \(c_n = \overline{c_{-n}}\) for \(0 \leq n \leq N\).
    \end{enumerate}
\end{proof}

\begin{theorem}[2.2]
    For \(F = \R\) or \(\C\), prove that the space \(P(S^1; F)\) defined in Section 2.5 is an \(F\)-algebra that separates points and vanishes at no point of \(S^1\). If \(F = \C\), also show that \(P(S^1; F)\) is self-adjoint.
\end{theorem}

\begin{proof}
    The sums of polynomials are clearly polynomials. The products of polynomials are clearly polynomials. The scalar product of polynomials are clearly polynomials. Thus \(P(S^1;F)\) is an \(F\)-algebra. It also separates points with the element \(z\) and vanishes nowhere with the element \(1\).

    Suppose \(F = \C\) and \(p(z) \in P(S^1;\C)\). Note that for \(z \in S^1\), we have \(1/z = \overline{z} / \|z\| = \overline{z}\). Thus
    \begin{align*}
        \overline{p(z)} &= \overline{\sum_{n = -N}^{N} c_n z^n} = \sum_{n = -N}^{N} \overline{c_n} \overline{z}^n = \sum_{n = -N}^{N} \overline{c_n} \frac{1}{z^n} = \sum_{n = -N}^{N} \overline{c_{-n}} z^n \in P(S^1;\C).
    \end{align*}
\end{proof}

\begin{theorem}[2.3]
    Let \(P^+(S^1;\C)\) be the space of trigonometric polynomials of the form
    \begin{align*}
        p(z) = \sum_{n = 0}^{N} c_n z^n, \hspace*{4mm} z \in S^1, N \in \N, c_n \in \C.
    \end{align*}
    Prove that \(P^+(S^1;\C)\) is a complex algebra that separates points and vanishes at no point of \(S^1\), yet \(p^+(S^1;\C)\) is not dense in \((C(S^1;\C), \|\cdot\|_u)\). Why does this not contradict the (complex) Stone-Weierstrass Theorem?
\end{theorem}

\begin{proof}
    The same argument from Exercise 2.2 applies; \(P^+(S^1;\C)\) is an complex algebra. However, it is not self-adjoint: For all \(f(z) = z^n\), \(n \geq 0\), we have \(\int_{-\pi}^{\pi} f(e^{i\t}) e^{i\t} d\t = \int_{-\pi}^{\pi} e^{i(n + 1)\t}d\t = 0\). Thus for any \(p \in P^+(S^1;\C)\), 
    \begin{align*}
        \int_{-\pi}^{\pi} p(e^{i\t}) e^{i\t} d\t = \sum_{n = 0}^{N} \int_{-\pi}^{\pi} c_n (e^{i\t})^n e^{i\t} d\t = 0.
    \end{align*}
    But if \(n = -1\), then \(\int_{-\pi}^{\pi} e^{-i\t}e^{i\t}d\t = 2\pi \neq 0\), so \(\overline{z} \notin P^+(S^1;\C)\), i.e. it is not self-adjoint.
\end{proof}

\end{document}