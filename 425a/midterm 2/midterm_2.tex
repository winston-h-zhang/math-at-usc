%%%%%%%%%%%%%%%%%%%%%%%%%%%%%%%%%%%%%%%%%%%%%%
%%                                          %%
%% USE THIS FILE TO SUBMIT YOUR SOLUTIONS   %%
%%                                          %%
%% You must have the usamts.tex file in     %%
%% the same directory as this file.         %%
%% You do NOT need to submit this file or   %%
%% usamts.tex with your solutions.  You     %%
%% only need to submit the output PDF file. %%
%%                                          %%
%% DO NOT ALTER THE FILE usamts.tex         %%
%%                                          %%
%% If you have any questions or problems    %%
%% using this file, or with LaTeX in        %%
%% general, please go to the LaTeX          %%
%% forum on the Art Of Problem Solving      %%
%% web site, and post your problem.         %%
%%                                          %%
%%%%%%%%%%%%%%%%%%%%%%%%%%%%%%%%%%%%%%%%%%%%%%

%%%%%%%%%%%%%%%%%%%%%%%%%%%%%%%%%%%%%%%%%%%%
%% DO NOT ALTER THE FOLLOWING LINES
\documentclass[12pt]{article}
\usepackage{amsmath,amssymb,amsthm,amsfonts,tabularx}
\usepackage[pdftex]{graphicx}
\usepackage{fancyhdr}
\usepackage{parskip}
\pagestyle{fancy}
\usepackage{setspace}
\newcommand{\realname}[1]{\newcommand{\printrealname}{#1}}
\newcommand{\pset}[1]{\newcommand{\printpset}{#1}}
\newcommand{\mathclass}[1]{\newcommand{\printmathclass}{#1}}

%% Pagestyle setup
\setlength{\headheight}{0.75in}
\setlength{\oddsidemargin}{0in}
\setlength{\evensidemargin}{0in}
\setlength{\voffset}{-.5in}
\setlength{\headsep}{10pt}
\setlength{\textwidth}{6.5in}
\setlength{\headwidth}{6.5in}
\setlength{\textheight}{8in}
\lhead{Math \printmathclass}
\chead{\Large \textbf{Homework \printpset}}
\rhead{\printrealname}
\rfoot{Page \thepage}
\renewcommand{\headrulewidth}{0.5pt}
\renewcommand{\footrulewidth}{0.3pt}
\setlength{\textwidth}{6.5in}
\renewcommand{\baselinestretch}{1}

\newtheorem*{prop}{Proposition}
\newtheorem*{corollary}{Corollary}
\newtheorem*{lemma}{Lemma}
\theoremstyle{remark}
\newtheorem*{defn}{Definition}

\newtheoremstyle{named}{}{}{}{}{\bfseries}{.}{.5em}{\thmnote{Problem #3}}
\theoremstyle{named}
\newtheorem*{theorem}{Theorem}

%% DO NOT ALTER THE ABOVE LINES
%%%%%%%%%%%%%%%%%%%%%%%%%%%%%%%%%%%%%%%%%%%%


%% If you would like to use Asymptote within this document (which is optional), 
%% you can find out how at the following URL:
%%
%%   http://www.artofproblemsolving.com/Wiki/index.php/Asymptote:_Advanced_Configuration
%%
%% As explained there, you will want to uncomment the line below.  But be
%% sure to check the website because there are several other steps that must 
%% be followed.
%% \usepackage{asymptote}

%% Enter your real name here
%% Example: \realname{David Patrick}
\realname{Hanting Zhang}
\pset{Midterm 2}
\mathclass{425A}

\renewcommand{\bf}{\mathbf}
\renewcommand{\implies}{\Rightarrow}
\newcommand{\coimplies}{\Leftarrow}

\begin{document}

\begin{theorem}[1]
    Determine the convergence or divergence in (a)-(c).
    \begin{enumerate}
        \item[(a)] 
        \[\sum_{k = 1}^\infty (-1)^k \frac{\sqrt{k + 1} - \sqrt{k}}{\sqrt{k}}\]
        \item[(b)] \[\sum_{k = 1}^\infty \frac{k!}{(k + 2)!}\]
        \item[(c)] \[\sum_{k = 1}^\infty \frac{2^k}{k^k}\]
    \end{enumerate}
\end{theorem}

\begin{proof}
    We do each separately:
    \begin{enumerate}
        \item[(a)] \textbf{TODO}
        \item[(b)] \textbf{TODO}
        \item[(c)] \textbf{TODO}
    \end{enumerate}
\end{proof}

\begin{theorem}[2]
    Consider the power series \(\sum a_n z^n\) and assume that the coefficients \(a_n\) are integers, infinitely many of which are not zero. Prove that the radius of convergence \(R \le 1\).
\end{theorem}

\begin{proof}
    \textbf{TODO:} radius of convergence
\end{proof}

\begin{theorem}[3]
    Consider a function \(f : M \to \mathbb R\). It's graph is the set,
    \[G(f) = \{(x, y) \in M \times \mathbb R \mid y = f(x)\}.\]
    \begin{enumerate}
        \item[(a)] Prove that if \(f\) is continuous then \(G(f)\) is closed as a subset of \(M \times \mathbb R\). 
        \item[(b)] Prove that if \(f\) is continuous and \(M\) is compact then \(G(f)\) is compact.
        \item[(c)] Prove that if \(G(f)\) is compact then \(f\) is continuous. 
    \end{enumerate}
\end{theorem}

\begin{proof}
    We do each part separately:
    \begin{enumerate}
        \item[(a)] \textbf{TODO:} \(\implies\)
        \item[(b)] \textbf{TODO:} \(\implies\)
        \item[(c)] \textbf{TODO:} \(\implies\)
    \end{enumerate}    
\end{proof}

\begin{theorem}[4]
    Let \(I = [0, 1]\) and let \(F : I \to I\) be continuous. Prove that \(F\) has at least one fixed point. Quoting a fixed points theorem is not acceptable.
\end{theorem}

\begin{proof}
    \textbf{TODO:} fixed point, no thm
\end{proof}

\begin{theorem}[5]
    Let \(X\) and \(Y\) be metric spaces and \(F : X \to Y\) be a continuous mapping onto \(Y\). If \(D\) is a dense subset of \(X\), prove that \(F(D)\) is a dense subset of \(Y\). 
\end{theorem}

\begin{proof}
    \textbf{TODO:} dense \(\implies\) dense
\end{proof}

\textbf{EXTRA CREDIT (10 POINTS):}

\begin{theorem}[6]
    Prove the convergence or divergence of 
    \[\sum_{n = 1}^\infty\frac{\sin n}{n}.\]
\end{theorem}

\begin{proof}
    \textbf{TODO} 
\end{proof}

\end{document}