\documentclass[12pt]{article}
\usepackage{amsmath,amssymb,amsthm,amsfonts,tabularx}
\usepackage[pdftex]{graphicx}
\usepackage[dvipsnames]{xcolor}
\usepackage{fancyhdr}
\usepackage{parskip}
\usepackage[shortlabels]{enumitem}
\pagestyle{fancy}
\usepackage{setspace}
\newcommand{\realname}[1]{\newcommand{\printrealname}{#1}}
\newcommand{\pset}[1]{\newcommand{\printpset}{#1}}
\newcommand{\mathclass}[1]{\newcommand{\printmathclass}{#1}}

%% Pagestyle setup
\setlength{\headheight}{0.75in}
\setlength{\oddsidemargin}{0in}
\setlength{\evensidemargin}{0in}
\setlength{\voffset}{-.5in}
\setlength{\headsep}{10pt}
\setlength{\textwidth}{6.5in}
\setlength{\headwidth}{6.5in}
\setlength{\textheight}{8in}
\lhead{Math \printmathclass}
\chead{\Large \textbf{\printpset}}
\rhead{\printrealname}
\rfoot{Page \thepage}
\renewcommand{\headrulewidth}{0.5pt}
\renewcommand{\footrulewidth}{0.3pt}
\setlength{\textwidth}{6.5in}
\renewcommand{\baselinestretch}{1}
\setenumerate[0]{label=(\alph*)}
\newcommand{\todo}{\textcolor{red}{\textbf{TODO }}}

\newtheorem*{prop}{Proposition}
\newtheorem*{corollary}{Corollary}
\newtheorem{lemma}{Lemma}
\theoremstyle{remark}
\newtheorem*{defn}{Definition}
\newtheorem*{remark}{Remark}

\newtheoremstyle{named}{}{}{}{}{\bfseries}{.}{.5em}{\thmnote{Problem #3}}
\theoremstyle{named}
\newtheorem*{theorem}{Theorem}
\allowdisplaybreaks

%% DO NOT ALTER THE ABOVE LINES
%%%%%%%%%%%%%%%%%%%%%%%%%%%%%%%%%%%%%%%%%%%%


%% If you would like to use Asymptote within this document (which is optional), 
%% you can find out how at the following URL:
%%
%%   http://www.artofproblemsolving.com/Wiki/index.php/Asymptote:_Advanced_Configuration
%%
%% As explained there, you will want to uncomment the line below.  But be
%% sure to check the website because there are several other steps that must 
%% be followed.
%% \usepackage{asymptote}

%% Enter your real name here
%% Example: \realname{David Patrick}
\realname{Hanting Zhang}
\pset{W3P1}
\mathclass{425B}

\renewcommand{\a}{\alpha}
\renewcommand{\b}{\beta}
\renewcommand{\d}{\delta}
\newcommand{\e}{\varepsilon}
\newcommand{\Z}{\mathbb Z}
\newcommand{\N}{\mathbb N}
\newcommand{\Q}{\mathbb Q}
\newcommand{\R}{\mathbb R}
\newcommand{\C}{\mathbb C}
\renewcommand{\bf}{\mathbf}
\newcommand{\id}[1]{\text{id}_{#1}}
\renewcommand{\implies}{\Rightarrow}
\newcommand{\coimplies}{\Leftarrow}
\renewcommand{\em}{\varnothing}
\renewcommand{\Im}{\text{Im}}
\newcommand{\abs}[1]{|#1|}
\newcommand{\bigabs}[1]{\left|#1\right|}

\begin{document}

\begin{theorem}[1.1]
    Prove that the integral \(\int_{1}^{\infty} \cos(x)dx\) does not converge.
\end{theorem}

\begin{proof}
    Simply integrate:
    \begin{align*}
        \int_{1}^{\infty} \cos(x) dx = \lim_{b \to \infty} \int_{1}^{b} \cos(x) dx = \lim_{b \to \infty}(-\sin(b) + \sin(1))
    \end{align*}
    which has no limit \(b \to \infty\) since \(\sin(b)\) oscillates forever.
\end{proof}

\begin{theorem}[1.2]
    Consider the two integrals
    \[\int_{\pi}^{\infty}\frac{\cos x}{x}dx \hspace*{4mm} \text{and} \hspace*{4mm} \int_{\pi}^{\infty} \frac{\sin x}{x^2}dx\]
    \begin{enumerate}
        \item Prove that one of these two integrals converges absolutely, but the other does not.
        \item Prove that both integrals converge, to the same value.
    \end{enumerate}
\end{theorem}

\begin{proof}
    We proceed part by part.
    \begin{enumerate}
        \item We claim that the first does not converge absolutely while the second does. 
        
        Indeed, consider the union of closed intervals of radius \(\pi/2\) centered along \(\pi\mathbb{N}\). On each interval, the value of \(|\cos(x)| \geq 1/2\), thus 
        \begin{align*}
            \int_{\pi}^{\infty} \frac{|\cos x|}{x} dx &\geq \sum_{n = 1}^\infty \int_{\pi n - \pi/2}^{\pi n + \pi/2} \frac{|\cos x|}{x} dx \\
            &\geq \sum_{n = 1}^\infty \int_{\pi n - \pi/2}^{\pi n + \pi/2} \frac{1}{2x} dx \\
            &\geq \sum_{n = 1}^\infty \frac{\pi}{2\pi n + \pi} = \infty,
        \end{align*}
        where the final sum diverges due to the diverge of the harmonic series.

        Meanwhile, we have
        \begin{align*}
            \int_{\pi}^{\infty} \frac{|\sin x|}{x^2} dx \leq \int_{\pi}^{\infty} \frac{1}{x^2}dx,
        \end{align*}
        and this obviously converges since the exponent is \(> 1\).

        \item However, it is the case that both integrals converge to the same value. Indeed, we can show this indirectly via integration by parts:
        
        \begin{align*}
            \int_{\pi}^{\infty} \frac{\cos x}{x} dx = \left. \frac{\sin x}{x}\right|_\pi^\infty+\int_\pi^\infty \frac{\sin x}{x^2}dx = \int_\pi^\infty \frac{\sin x}{x^2}dx
        \end{align*}

        Thus since the \(\sin\) integral converges, so does the \(\cos\) integral, and they must have the same value.
    \end{enumerate}
\end{proof}

\begin{theorem}[1.3]
    Prove that the integral \(\int_{1}^{\infty}\cos(x^2)dx\) does not converge absolutely.
\end{theorem}

\begin{proof}
    Substitute \(x^2 \mapsto u\) to get \(\int_{1}^{\infty}\frac{\cos u}{2 u} du\). By Exercise 1.1, we know that this does not converge absolutely.
\end{proof}

\begin{theorem}[1.4]
    Find a function \(f \in \mathcal{R}_{\text{loc}}((0, 1])\) such that \(\int_{0}^{1} f(x) dx\) converges, but not absolutely.
\end{theorem}

\begin{proof}
    Let \(I_k = (2^k, 2^{k + 1}]\). Note that \(\bigcup_{i = 0}^\infty I_k = (0, 1]\). Now define 
    \[f(x) = \sum_{k = 0}^\infty (-1)^k\frac{2^{k + 1}}{k} 1_{I_k}.\]
    Thus
    \begin{align*}
        \int_{0}^{1} f(x) dx &= \int_{0}^{1} \sum_{k = 0}^\infty (-1)^k\frac{2^{k + 1}}{k} 1_{I_k} dx = \sum_{k = 0}^\infty (-1)^k \int_{0}^{1}\frac{2^{k + 1}}{k} 1_{I_k} = \sum_{k = 0}^\infty (-1)^k\frac{1}{k}
    \end{align*}
    which converges. However, similarly, we have
    \begin{align*}
        \int_{0}^{1} |f(x)| dx &= \int_{0}^{1} \sum_{k = 0}^\infty \frac{2^{k + 1}}{k} 1_{I_k} dx = \sum_{k = 0}^\infty \int_{0}^{1}\frac{2^{k + 1}}{k} 1_{I_k} = \sum_{k = 0}^\infty \frac{1}{k},
    \end{align*}
    which diverges.
\end{proof}

\end{document}