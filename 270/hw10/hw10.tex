\documentclass[10pt]{article}
\usepackage{fullpage}
\usepackage[noend]{algorithmic}
\usepackage{algorithm}

\usepackage{amsfonts}
\usepackage{amsmath,amsthm,amssymb}
\usepackage{nicefrac}
\usepackage{latexsym}
\usepackage{epic}
\usepackage{epsfig}
\usepackage{amscd}
\usepackage{url}
% \usepackage{pstricks,pst-node,pst-plot,pst-tree}
\usepackage{verbatim}
\usepackage{setspace}
\usepackage{hyperref}
\usepackage{enumerate}
\usepackage{times}

\newtheorem{lemma}{Lemma}
\newtheorem{fact}[lemma]{Fact}
\newtheorem{theorem}[lemma]{Theorem}
\newtheorem{corollary}[lemma]{Corollary}
\newtheorem{proposition}[lemma]{Proposition}
\newtheorem{claim}[lemma]{Claim}

\newtheorem{definition}{Definition}
\newtheorem{example}[definition]{Example}
\newtheorem{remark}[definition]{Remark}

\providecommand{\half}{\ensuremath{\nicefrac{1}{2}}\xspace}
\providecommand{\third}{\ensuremath{\nicefrac{1}{3}}\xspace}
\providecommand{\quarter}{\ensuremath{\nicefrac{1}{4}}\xspace}

\providecommand{\SetCard}[1]{\ensuremath{| #1 |}\xspace}
\providecommand{\SET}[1]{\ensuremath{\{ #1 \}}\xspace}
\providecommand{\Abs}[1]{\ensuremath{| #1 |}\xspace}
\providecommand{\Set}[2]{\ensuremath{\SET{#1 \mid #2}}\xspace}

\providecommand{\Kth}[1]{\ensuremath{{#1}^{\rm th}}}
\providecommand{\Ceiling}[1]{\ensuremath{\lceil {#1} \rceil}\xspace}
\providecommand{\Floor}[1]{\ensuremath{\lfloor {#1} \rfloor}\xspace}

\providecommand{\PROB}{\ensuremath{{\rm Prob}}\xspace}
\providecommand{\Prob}[2][]{\ensuremath{%
\ifthenelse{\equal{#1}{}}{\PROB[#2]}{\PROB_{#1}[#2]}}\xspace}
\providecommand{\ProbC}[3][]{\Prob[#1]{#2\;|\;#3}}
\providecommand{\Expect}[2][]{\ensuremath{%
\ifthenelse{\equal{#1}{}}{\mathbb{E}}{\mathbb{E}_{#1}}%
\left[#2\right]}\xspace}
\providecommand{\ExpectC}[3][]{\Expect[#1]{#2\;|\;#3}}

\providecommand{\Compl}[1]{\overline{#1}}

\newcommand{\OPT}[2][]{\ensuremath{\text{OPT}_{#1}(#2)}\xspace}

\title{\bf Homework 10}
\author{Winston (Hanting) Zhang}
\date{Friday, December 6, by 23:00}

\begin{document}
\maketitle 

\paragraph{Late Submission.} \textbf{Regardless of how many late days you have remaining, this assignment cannot be submitted more than three days late, i.e., the latest possible submission date is December 9 at 23:00. This is so we can release sample solutions before the weekend, and you can use them to study.}
  
\paragraph{General Instructions.} The following assignment is meant to be challenging, and we anticipate that it will take most of you at least 10--15 hours to complete, so please allow yourself plenty of time to work on it.
We highly recommend reading the entire assignment right away --- you never know when inspiration will strike.
Please provide a formal mathematical proof for all your claims, and  present runtime guarantees for your algorithms using asymptotic (big-$O/\Omega/\Theta$) notation, unless stated otherwise. You may assume that all basic arithmetic operations (multiplication, subtraction, division, comparison, etc.) take constant time. In problems involving graphs, we typically use $n$ and $m$ to denote the number of nodes and edges, respectively, unless otherwise stated.

\paragraph{Collaboration.}  Please carefully check the collaboration policy on the course website. When in doubt, ask an instructor.

\paragraph{Consulting outside sources.} Please carefully check the policy regarding outside sources on the course website. Again, when in doubt, ask an instructor.

\paragraph{Submission.} Homework submission will be through the Gradescope system. Instructions and links have been provided through the course website and Piazza. The only accepted format is PDF. Only typed solutions will be accepted, and we highly recommend producing your solutions using \LaTeX~(the text markup language we are also using for this assignment).

\paragraph{Recommended practice problems (do not hand in):} KT Problems 8.3, 8.4, 8.5, 8.6, 8.12, 8.13, 8.15, 8.16, 8.19, 8.26, 8.27, 8.28, 8.31, 8.37


\newpage

When you do reductions, you are welcome to use any of the problems that were listed at the end of the 11:00am lecture on November 22 (see notes) --- they will also be listed at the beginning of the 2:00pm lecture on November 29. If you want to use any other problem, you have to prove that problem NP-complete yourself first.
Unless we give a hint otherwise, hardness proofs can be accomplished reasonably easily using the problems explicitly discussed in class: \textsc{3SAT}, \textsc{Independent Set}, \textsc{Vertex Cover}, \textsc{Set Cover}.

\subsection*{Problem 1. [10]}
  Imagine that you are planning the allocation of animals to enclosures in a mountain lion rehabilitation program. The layout of the rehabilitation center is given to you as an undirected graph $G=(V,E)$ with $n$ nodes and $m$ edges, in which the nodes are the enclosures, and the edges capture proximity, i.e., an edge between enclosures $u$ and $v$ means that if you put mountain lions in $u$ and $v$, they might interact (see below).
  There are $k_M \geq 0$ male mountain lions, and $k_F \geq 0$ female mountain lions. For each mountain lion $i$, you are given their sex and their \emph{territoriality} $t_i \geq 0$ (integer). Each enclosure can only contain one animal, because animals of the same sex might fight, and animals of opposite sex might mate, leaving you with even more mountain lions to take care of.
  If a male mountain lion $i$ is put in an enclosure adjacent to (one or more) other male mountain lions, due to his territoriality, he will create territoriality trouble $t_i$; similarly, a female mountain lion $i$ in an enclosure adjacent to (one or more) other female mountain lions will create territoriality trouble $t_i$.
  You can put male and female mountain lions in enclosures adjacent to each other without any resulting territoriality problems.

  Your goal is to assign the mountain lions to enclosures so as to minimize the total territoriality trouble.\footnote{Bonus problem: Practice saying ``territoriality trouble'' ten times in a row fast!} Phrase this problem as a decision problem and prove that the decision problem is NP-complete.

\begin{definition}
  Define \textsc{Territoriality} as the following decision problem:

  Given a undirected graph \(G = (V, E)\), \(k_M\) lions with territorialities \(t_i\) and \(k_F\) lionesses with territorialities \(t_i\), decide whether there exists an assignment of the nodes such that the total territoriality is less than some \(K\).
\end{definition}

\begin{proposition}
  \textsc{Territoriality} is in NP.
\end{proposition}

\begin{proof}
  Given an instance of \textsc{Territoriality} and an assignment of lions and lionesses, we can certify the assignment with the following: For each lion \(i\), iterate through the edges incident the its node. If any adjacent nodes are lions, then add \(t_i\) to the total territoriality. Then for each lioness \(i\), iterate through the edges incident the its node. If any adjacent nodes are lionesses, then add \(t_i\) to the total territoriality. If the total territoriality in the end is at most \(K\), return ``yes,'' otherwise return ``no.'' This iterates through each node once and each edge twice, so the running time is \(O(V + 2E)\), which is polynomial. Thus we have an efficient certifier, and \textsc{Territoriality} is in NP.
\end{proof}

\begin{proposition}
  \textsc{Territoriality} is NP-Hard.
\end{proposition}

\begin{proof}
  We perform a reduction from \textsc{Independent Set} to \textsc{Territoriality}. Given an instance of \textsc{Independent Set} with graph \(G = (V, E)\) and an integer \(k\), create the following \textsc{Territoriality} instance on the same graph \(G\): Have \(k\) lions and \(n - k\) lionesses. The territoriality of each lion \(i\) is \(t_i = 1\) and the territoriality of each lioness \(i\) is \(t_i = 0\). Set \(K = 0\), i.e. the total territoriality must be exactly \(0\). 
  
  We claim that this is a Karp reduction from \textsc{Independent Set} to \textsc{Territoriality}. Clearly, this construction function run in polynomial time. We next show that our reduction is correct.
  
  Suppose we had a ``yes'' instance of \textsc{Independent Set}. Then there is some \(S \subseteq V\) of size \(k\) such that no two nodes in \(S\) share an edge. Then for the constructed \textsc{Territoriality}, assign the \(k\) lions to \(S\). Because no two nodes in \(S\) share an edge, the total territoriality of the lions is \(0\). Because the territorialities of the lionesses are \(0\), it doesn't matter if they share an edge. Thus the total territoriality of this assignment is 0; thus this is a ``yes'' instance of \textsc{Territoriality}.

  We procedd by the contrapositive. Suppose we had a ``yes'' instance of \textsc{Territoriality}. Then there is some assignment of the \(k\) lions that don't share edges between each other, because otherwise our territoriality would at least be \(1\). This chosen set \(S\) of \(k\) nodes is then a independent set of size \(k\) in \(G\). This we have a ``yes'' instance of \textsc{Independent Set}.

  We have proven that \textsc{Territoriality} is NP-hard and in NP, thus it is NP-complete.
\end{proof}

\subsection*{Problem 2. [10]}
  Remember the \textsc{Thanksgiving Hugs} problem? Well, now you will show it NP-complete. You have $n$ guests. For each guest $i$, you are given three integers $h_i, s_i, f_i$. This means that guest $i$ will be available to hug during the time interval $[s_i, f_i]$. You want to hug them for $h_i \geq 0$ time. You can only hug one person at a time, and hugs cannot be interrupted/resumed.

  Show that deciding whether you can hug everyone for the desired amount of time is NP-complete.

  Hint 1: \textsc{Subset Sum}

  Hint 2: Have one guest whom you want to hug for one unit of time who is only available for exactly one unit of time at a specific time.

\begin{proposition}
  Thanksgiving Hugs is in NP.
\end{proposition}

\begin{proof}
  Suppose we had an input and certificate for \textsc{Thanksgiving Hugs} with \(n\) guests and 3 integers \(h_i, s_i, f_i\) for each guest \(i\). The certificate is a set of \(n\) times \(t_i\) indicating the time to start each hug \(h_i\). Our certifier first sorts the times in ascending order. Then we iterate and check the following conditions for each time \(t_i\):
  \begin{enumerate}
    \item \([t_i, t_i + h_i] \subseteq [s_i, f_i]\). This checks that each hugs conforms to the constraints of our problem.
    \item \(t_{i - 1} + h_{i - 1} \leq t_i\) and \(t_i + h_i \leq t_{i + 1}\). This checks that all of the hugs are disjoint and do not overlap with each other.
  \end{enumerate}
  If any of these checks fail, we immediately terminate and answer ``no.'' Otherwise, if we go through all the times, we anwer ``yes.'' This is clearly polynomial as our certifier sorts and makes a linear scan, which both take polynomial time in the size of the input.
\end{proof}

\begin{proposition}
  Thanksgiving Hugs is NP-hard.
\end{proposition}

\begin{proof}
  We will perform a reduction from \textsc{Subset Sum} to \textsc{Thanksgiving Hugs}. Suppose we have an instance of \textsc{Subset Sum}, that is \(n\) integers \(x_1, \ldots, x_n\) and a target integer \(T\). If \(\sum_{i = 1}^n x_i < T\), then output is the impossible \((s_1 = 0, f_1 = 0, h_1 = 1)\). Otherwise, output the following \textsc{Thanksgiving Hugs} instance: Create \(n\) triples where triple \(i\) has \(h_i = x_i\) and \(s_i = 0, f_i = 1 + \sum_{n=1}^n x_i\). We also create one extra triple such that \(h_0 = 1\), \(s_0 = T\), and \(f_0 = T+1\). 
  
  We claim that this construction is a Karp reduction from \textsc{Subset Sum} to \textsc{Thanksgiving Hugs}. Clearly the function creating the instances run in polynomial time. Now we will prove the correctness of our reduction.

  Suppose we have a "yes" instance of \textsc{Subset Sum}, then there exists a set of integers that add up to \(T\). That implies that we have a subset of our \(n + 1\) hugs that add up to \(T\). We could start hugging people in this subset one by one with no gaps in between hug times until we reach time \(T\). Then we hug the additional person we've added from \(T\) to \(T + 1\). At this point, we've hugged for a total of \(T + 1\) time, and we have \(1 + \sum_{i = 1}^n x_i - 1 - W\) time left. Our final deadline is \(1 + \sum_{i = 1}^n x_i\), so we have exactly enough time left to hug everyone else. Thus we have a ``yes'' instance of \textsc{Thanksgiving Hugs}.

  Now the other side. Suppose we have a ``no'' instance of \textsc{Subset Sum}, i.e. we have \(n\) integers \(x_i\) with no subset that sums to \(T\). If it is the case that \(\sum_{i = 1}^n x_i < T\), then automatically we have a ``no'' instance \((s_1 = 0, f_1 = 0, h_1 = 1)\) for \textsc{Thanksgiving Hugs}.
  
  Otherwise, suppose for the sake of contradiction that the \textsc{Thanksgiving Hugs} instance we construct is a ``yes.'' Then we have \(1 + \sum_{i = 1}^n x_i\) total hugging time and deadlines between \(0\) and \(1 + \sum_{i + 1}^n x_i\), so we cannot have any gaps between our hugs. We also know that we must have a hug between \(T\) and \(T + 1\). Thus we be continuously hugging some \(k\) people between \(0\) to \(T\). Thus the hugging times of this subset of \(k\) people sum to \(T\)! But this is a contradiction since \(h_i = x_i\), and we assumed that no subset of the \(x_i\) sum to \(T\). Thus we must also have a ``no'' instance of \textsc{Thanksgiving Hugs}.
  
  We have proven that \textsc{Thanksgiving Hugs} is NP-hard and in NP, thus it is NP-complete.
\end{proof}

\subsection*{Problem 3. [10]}
  A city government is trying to improve their public transportation system. Their idea is to make their Central Station the main hub, allowing you to get to/from Central Station in at most two hops from anywhere in the city.\footnote{It's not clear if that's a very good idea --- such centralization would mean that if they ever need to temporarily close the Central Station, it will be a disaster. But hey --- they didn't ask your opinion about what to do; they only wanted to rely on your algorithmic expertise.}
  This is modeled as follows: the current transportation system is given as an undirected graph $G=(V,E)$. The Central Station is a given node $s \in V$. They want to add a smallest possible set of additional edges $E'$ such that in the new graph $G'=(V,E \cup E')$, every node has a path of length at most 2 hops to $s$.

  Phrase this problem as a decision problem and prove that it is NP-complete.

\begin{definition}
  We define the \textsc{Station} decision problem as follows:

  Given an undirected graph \(G = (V, E)\), and a start node \(s\), decide whether there exists a set \(E'\) of \(k\) edges you can add to \(E\) such that every node in the graph \(G' = (V, E \cup E')\) can be reached within 2 edges of s. Denote a node ``reachable'' if it can be reached within 2 edges of s.
\end{definition}

\begin{proposition}
  \textsc{Station} is in NP.
\end{proposition}

\begin{proof}
  
\end{proof}

\begin{proposition}
  \textsc{Station} is NP-complete.
\end{proposition}

\begin{proof}
  We perform a reduction from \textsc{Set Cover} to \textsc{Station}. Given an instance of \textsc{Set Cover}, that is, a universe \(U = \{x_1, \ldots, x_n\}\) and \(m\) subsets \(S_j \subseteq U\), and a target number of subsets \(k\), we construct the following instance of \textsc{Station}: 
  \begin{enumerate}
    \item Create a start node \(s\).
    \item Create \(m\) subset nodes \(s_1, \ldots, s_m\), and make the subset nodes fully connected to each other, i.e. for every \(1 \leq j_1, j_2 \leq m\), add the edge \((s_{j_1}, s_{j_2})\).
    \item Create \(n\) element nodes \(x_1, \ldots, x_n\). 
    \item For each subset \(S_j\), and for every \(x_i \in S_j\), add an edge \((x_i, s_j)\).
  \end{enumerate}
  Call this graph \(G\).

  We claim that this construction is a Karp reduction from \textsc{Set Cover} to \textsc{Station}. Indeed, the construction itself runs in polynomial time, since we we just linearly iterating through the subsets. Now we will provide the correctness of our reduction.

  Suppose we have a ``yes'' instance of \textsc{Set Cover}. Then there is a family of subsets \(I\) with size \(k\) such that \(\bigcup_{S_j \in I} S_j = U\). Then in \(G\), simply add the \(k\) edges \((s, s_j)\) for each \(S_j \in I\). Since \(\bigcup_{S_j \in I} S_j = U\), every \(x_i\) is in some \(S_j \in I\). Thus every node \(x_i\) is connected to that same \(S_j\). We just added an edge \(s \to s_j\), so now we have a path \(s \to s_j \to x_i\). This holds for all \(x_i\), so all of the element nodes are reachable. Furthermore, since all the \(s_j\) nodes are connected to each other, we can literally take the path \(s \to s_j \to s_k\) for any \(S_j \in I\) and \(1 \leq k \leq m\). So all the subset nodes are reachable as well. Thus all the nodes of \(G\) are reachable, and \(G\) is a ``yes'' instance of \textsc{Station}. 

  Now for the other side, we process by contrapositive. Suppose the constructed graph \(G\) is a ``yes'' instance of \textsc{Station}. Then there exists a set of \(k\) edges such that adding those \(k\) edges makes everything in \(G\) reachable. Ideally, we could use the same idea from above and say that all the edges connect to a family of \(s_{j_1}, \ldots, s_{j_k}\) nodes, which creates \(I = \{S_{j_1}, \ldots, S_{j_k}\}\) that solves the \textsc{Set Cover} instance. However, there are potentially ``bad'' edges that don't connect from \(s\) to \(s_j\) directly. So first, we show that these ``bad'' edges can be removed. There are 3 types of ``bad'' edges: 1. edges between the element nodes \(x_i\); 2. edges going from \(s\) to \(x_i\); and 3. edges going between \(s_j\) and \(x_i\). (Note we can't have edges between the subset nodes \(s_j\) since that layer is already fully connected.) We proceed by casework:
  \begin{enumerate}
    \item Suppose one of our \(k\) edges connects \((x_i, x_j)\). Since all nodes in \(G\) are reachable, let's analyze paths to \(x_i\) and \(x_j\). Now, if both \(x_i\) and \(x_j\) are reachable via other nodes, i.e. either \(x_i\) and \(x_j\) directly connect to \(s\), or there is some other node \(v \neq x_i, x_j\) such that \(s \to v \to x_i, x_j\). Then we can simply remove the edge \((x_i, x_j)\). This doesn't affect the fact that \(x_i\) and \(x_j\) are reachable, and removing \((x_i, x_j)\) doesn't affect any other nodes, so all nodes in \(G\) are still reachable.
    
    Otherwise, without loss of generality suppose \(x_j\) is only reachable via \(s \to x_i \to x_j\). Note that we can assume \(x_i\) connects to \(s\), or else both \(x_i\) would be at least 2 away from \(s\), making \(x_j\) at least 3 away from \(s\), thus contradicting the reachablility of \(x_j\). Then again we can simply remove \((x_i, x_j)\) and add \((s, x_j)\). Again this doesn't affect any other nodes, and we maintain that \(x_i\) and \(x_j\) are both still reachable. 

    Thus in this case, we can always remove the \((x_i, x_j)\) edge while maintaining a valid ``yes'' instance.

    \item Suppose one of our \(k\) edges connects \((s, x_i)\). Since all of the \(S_j\) in total form a cover of \(U\), there is at least one \(j\) such that \(x_i \in S_j\). Then we might as well remove \((s, x_i)\) and add \((s, s_j)\). This doesn't destroy any paths to other element nodes \(x_j\), like \(s \to x_i \to x_j\), since in the above case we proved that we can remove all \((x_i, x_j)\) edges. Thus in this case, we can always remove the \((s, x_i)\) edge while maintaining a valid ``yes'' instance.
    
    \item Finally, suppose one of our \(k\) edges connects \((s_j, x_i)\). Again, there is at least one other \(j'\) such that \(x_i \in S_{j'}\). Then we might as well remove \((s_j, x_i)\) and add \((s, s_{j'})\). This doesn't destroy any paths from like \(s \to x_i \to s_j\), since from the above we showed that we can remove all edges between \(s \to x_i\). Furthermore, we maintain that \(x_i\) is reachable via \(s \to s_{j'} \to x_i\). Thus in this case too, we maintain a valid ``yes'' instance.
  \end{enumerate}

  So, we have shown that we can remove all ``bad'' edges and replace them with connections bewteen \((s, s_j)\). Then take these \(k\) connections and collect the subsets \(S_j\) that are connected. Since all the element nodes of \(G\) are reachable via some \(s \to s_j \to x_i\), we have that \(S_j\) forms a cover of \(U\). Thus we also have a ``yes'' instance of \textsc{Set Cover}. 
  
  Thus we have a proper reduction from \textsc{Set Cover} to \textsc{Station}, so \textsc{Station} is NP-hard, and thus also NP-complete.
\end{proof}

\begin{definition}
  We define the following function from instances of \textsc{3-SAT} to \textsc{Station}: Given a \textsc{3-SAT} decision problem on the proposition \(p\) with literals \(x_1, \ldots, x_k\) and \(m\) clauses \(C_1, \ldots, C_m\). Constuct the following graph: 
  \begin{enumerate}
    \item Create the start vertex \(s\).
    \item Create \(2k\) literal vertices \(x_i, \overline{x_i}\) for each \(1 \leq i \leq k\). 
    \item Add an edge \((x_i, \overline{x_i})\) for each \(1 \leq i \leq k\).
    \item Create \(m\) clause vertices \(c_j\) for each \(1 \leq j \leq m\).
    \item For each clause \(C_j\) and for each of the 3 literals in the clause, if a literal \(x_i\) is present, then add an edge \((c_j, x_i)\), otherwise if \(\neg x_i\) is present, then add an edge \((c_j, \overline{x_i})\). 
    \item For each literal \(x_i\), add \(2k + 1\) nodes \(u_{i, \ell}\), for \(1 \leq \ell \leq k\), and add edges \((x_i, u_{i, \ell})\) and \((\overline{x_i}, u_{i, \ell})\).
  \end{enumerate}
  Call this constructed graph \(G_p = (V, E)\)
\end{definition}

\begin{proposition}
  The function from \textsc{3-SAT} to \textsc{Station} given by \(p \mapsto G_p\) is a Karp reduction.
\end{proposition}

\begin{proof}
  We must show that \(p\) is satisfiable if and only if \(G_p\) is solvable. We show both directions. 

  Suppose \(p\) is satisfiable. Then there exist some assignment of the \(x_i\) to ``true'' and ``false'' such that \(p\) is true. Then, for each \(1 \leq i \leq k\), if \(x_i\) is ``true'', then add an edge \((s, x_i)\). Otherwise if \(x_i\) is ``false'', then add an edge \((s, \overline{x_i})\). We claim that adding these \(k\) edges make all nodes in \(G_p\) reachable. Let's do casework on the literal nodes, the clause nodes, and the \(u\) nodes separately:
  \begin{enumerate}
    \item If \(s\) connects to \(x_i\), then \(\overline{x_i}\) is reachable from \(s \to x_i \to \overline{x_i}\). If \(s\) connects to \(\overline{x_i}\), then we have to opposite \(s \to \overline{x_i} \to x_i\). This holds for all \(i\), so all the literal nodes are reachable.
    \item We know that \(p\) is true, so each clause \(C_j\) must also be true. Thus there is some literal \(x_i\) or \(\neg x_i\) in the clause that is true. We only need one of these since doing \(\text{true} \lor \text{anything}\) is ``true''. 
    If \(x_i\) is true, recall that \(c_j\) connects to \(x_i\) from step 5 of our construction. Then \(c_j\) is reachable via \(s \to x_i \to c_j\). If \(\neg x_i\) is true, then \(x_i\) is false and recall that \(c_j\) connects to \(\overline{x_i}\) from step 5 of our construction. Thus \(c_j\) is reachable via \(s \to \overline{x_i} \to c_j\). 

    This holds for all \(c_j\), so all the clause nodes are reachable.
    \item All the \(u\) nodes connect to either \(x_i\) or \(\overline{x_i}\), so we can reach any \(u\) node via \(s \to x_i \to u_{i, \ell}\) or \(s \to \overline{x_i} \to u_{i, \ell}\). 
  \end{enumerate}
  Thus \(G_p\) is solvable, given that \(p\) is satisfiable.

  Now for the other direction. Suppose that \(G_p\) solvable. We want to show that \(p\) is satisfiable. 

  Unfortunately, we cannot directly use the same idea of connecting to \(x_i\) or \(\overline{x_i}\) because we might be able to ``cheat'' a solution by connecting to the clause nodes directly. Let's call any solution to \(G_p\) that connects to non-literal nodes or connects to both a literal and its complement (i.e. connecting to both \(x_i\) and \(\overline{x_i}\)) as ``degenerate'' solutions. We claim in fact that no degenerate solutions exist!

  Think about how to make the \(u\) nodes reachable. We either have to connect to them directly via \(s \to u\), or go through \(x_i\)/\(\overline{x_i}\). 
\end{proof}

\subsection*{Problem 4. [10]}
  Prove that the following problem is undecidable: Given a C++ program $P$ which takes as input a graph $G$ with edge costs $c_e$, does $P$ correctly compute a Minimum Spanning Tree of its input (for each input graph)?

  Note: This problem will not make sense until Tuesday at the earliest, possibly Thursday.

\begin{proof}
  Denote the problem by \textsc{ComputeMST}. We will give  reduction from \textsc{Halt} to \textsc{ComputeMST}. For any program \(P\) instance of \textsc{Halt}, define a program \(Q\) which takes as input a simple graph \(G\) as follows: \(Q\) will first run \(P(P)\). Then it will return the MST of \(G\) by running Kruskal's Algorithm. Our reduction is computable in polynomial time as the code for \(P\) is given and Kruskal's Algorithm is computable.

  Now let's analyze the behavior of \(Q\). Suppose \(P\) is a ``yes'' instance of \textsc{Halt}. Then \(P(P)\) terminates, and thus \(Q\) terminates -- therefore \(Q\) is a ``yes'' instance of \textsc{ComputeMST}. For the other side, suppose \(P\) is a ``no'' instance of \textsc{Halt}. Then \(P(P)\) does not terminate, and thus \(Q\) does not terminate -- therefore \(Q\) is a ``no'' instance of \textsc{ComputeMST}.

  So we have constructed a reduction from \textsc{Halt} to \textsc{ComputeMST}. Thus \textsc{ComputeMST} is undecidable.
\end{proof}

\subsection*{Problem 0. [0]}
\textbf{Chocolate Problem (1 chocolate bar)}: 
Imagine that you own a toll road, and want to set prices.
The toll road has $m$ ramps to get on/off, numbered $1, 2, \ldots, m$ from start to end of the road (with 1 being the start and $m$ the endpoint).
This divides your toll road into segments $T_j = [j,j+1)$.
You are given a list of $n$ drivers.
For each driver $i$, you are told the starting and ending ramp of their trip $s_i < t_i$, as well as a budget $b_i$. 
To drive from $s_i$ to $t_i$, the driver has to drive along all the intermediate segments $T_{s_i}, T_{s_i+1}, \ldots, T_{t_i-1}$.

You get to choose a price $p_j \geq 0$ for each segment $T_j$.
If the sum of prices on the trip is affordable to the driver (at most $b_i$), then they will drive on your toll road and pay you the total for the segments.
If the sum exceeds $b_i$, driver $i$ will take surface streets for the entire trip, and not pay you anything.
Your goal is to choose prices for the segments that maximize your revenue from the drivers under this model.

Phrase this problem as a decision problem, and prove that it is NP-complete. 

\end{document}