\documentclass[12pt]{article}
\usepackage{amsmath,amssymb,amsthm,amsfonts,tabularx}
\usepackage[pdftex]{graphicx}
\usepackage[dvipsnames]{xcolor}
\usepackage{fancyhdr}
\usepackage{parskip}
\usepackage[shortlabels]{enumitem}
\pagestyle{fancy}
\usepackage{setspace}
\newcommand{\realname}[1]{\newcommand{\printrealname}{#1}}
\newcommand{\pset}[1]{\newcommand{\printpset}{#1}}
\newcommand{\mathclass}[1]{\newcommand{\printmathclass}{#1}}

%% Pagestyle setup
\setlength{\headheight}{0.75in}
\setlength{\oddsidemargin}{0in}
\setlength{\evensidemargin}{0in}
\setlength{\voffset}{-.5in}
\setlength{\headsep}{10pt}
\setlength{\textwidth}{6.5in}
\setlength{\headwidth}{6.5in}
\setlength{\textheight}{8in}
\lhead{Math \printmathclass}
\chead{\Large \textbf{\printpset}}
\rhead{\printrealname}
\rfoot{Page \thepage}
\renewcommand{\headrulewidth}{0.5pt}
\renewcommand{\footrulewidth}{0.3pt}
\setlength{\textwidth}{6.5in}
\renewcommand{\baselinestretch}{1}
\setenumerate[0]{label=(\alph*)}
\newcommand{\todo}{\textcolor{red}{\textbf{TODO }}}

\newtheorem*{prop}{Proposition}
\newtheorem*{corollary}{Corollary}
\newtheorem*{lemma}{Lemma}
\theoremstyle{remark}
\newtheorem*{defn}{Definition}

\newtheoremstyle{named}{}{}{}{}{\bfseries}{.}{.5em}{\thmnote{Problem #3}}
\theoremstyle{named}
\newtheorem*{theorem}{Theorem}
\allowdisplaybreaks

%% DO NOT ALTER THE ABOVE LINES
%%%%%%%%%%%%%%%%%%%%%%%%%%%%%%%%%%%%%%%%%%%%


%% If you would like to use Asymptote within this document (which is optional), 
%% you can find out how at the following URL:
%%
%%   http://www.artofproblemsolving.com/Wiki/index.php/Asymptote:_Advanced_Configuration
%%
%% As explained there, you will want to uncomment the line below.  But be
%% sure to check the website because there are several other steps that must 
%% be followed.
%% \usepackage{asymptote}

%% Enter your real name here
%% Example: \realname{David Patrick}
\realname{Hanting Zhang}
\pset{W2P1}
\mathclass{425B}

\renewcommand{\a}{\alpha}
\renewcommand{\b}{\beta}
\renewcommand{\d}{\delta}
\newcommand{\e}{\varepsilon}
\newcommand{\Z}{\mathbb Z}
\newcommand{\N}{\mathbb N}
\newcommand{\Q}{\mathbb Q}
\newcommand{\R}{\mathbb R}
\newcommand{\C}{\mathbb C}
\renewcommand{\bf}{\mathbf}
\newcommand{\id}[1]{\text{id}_{#1}}
\renewcommand{\implies}{\Rightarrow}
\newcommand{\coimplies}{\Leftarrow}
\renewcommand{\em}{\varnothing}
\renewcommand{\Im}{\text{Im}}
\newcommand{\abs}[1]{|#1|}
\newcommand{\bigabs}[1]{\left|#1\right|}

\begin{document}

\begin{theorem}[3.1]
    Prove that
    \[\lim_{L \to \infty} \sum_{\ell = 0}^{2L + 1} \sum_{j + k = \ell} x_{j, k} = -\frac{2}{3}.\]
\end{theorem}

\begin{proof}
    We may rearrange the finite sum \(\sum_{\ell = 0}^{2L + 1} \sum_{j + k = \ell} x_{j, k}\) into \(\sum_{n = 0}^L \sum_{m = n}^{2L + 1 - n} x_{n, m}\). Formally checking that this is a bijection is quite painful, but the geometric argument is that the second sum simply counts the entries column by column instead of along the diagonals.

    Now, the definition of \(x_{n, m}\) gives:
    \begin{align*}
        \sum_{n = 0}^L \sum_{m = n}^{2L + 1 - n} x_{n, m} &= \sum_{n = 0}^L \left(-1 + \sum_{m = n + 1}^{2L + 1 - n} x_{n, m}\right) \\ 
        \text{(since \(m > n\) we have \(x_{n, m} = 2^{n - m}\))  } &= \sum_{n = 0}^L \left(-1 + \sum_{m = n + 1}^{2L + 1 - n} 2^{n - m}\right) \\
        \text{(reindex with \(m' = m - n\))   } &= \sum_{n = 0}^L \left(-1 + \sum_{m' = 1}^{2L + 1 - 2n} \frac{1}{2^{m'}}\right) \\
        &= \sum_{n = 0}^L \left(-1 + 1 - \frac{1}{2^{2L + 1 - 2n}}\right) \\
        &= -\frac{1}{2} \sum_{n = 0}^L \frac{1}{4^{L - n}} = -\frac{1}{2} \sum_{n = 0}^L \frac{1}{4^{n}}
    \end{align*}

    Now as \(L \to \infty\), we have \(-\frac{1}{2} \sum_{n = 0}^L \frac{1}{4^{n}} \to -\frac{1}{2} \left(\frac{1}{1 - 1/4}\right) = -2/3\), as desired.
\end{proof}

\begin{theorem}[3.2]
    Let \((x_{n, m})_{n, m = 0}^\infty\) be a double sequence of complex numbers. Let \(\phi : \mathbb{N}_0 \to \mathbb{N}_0 \times \mathbb{N}_0\) be a bijection, \(\phi(n) = (j(n), k(n))\). Prove that if \(\sum_{n = 0}^\infty \sum_{m = 0}^\infty |x_{n, m}|\) converges, then so does the series \(\sum_{n = 0}^\infty x_{j(n), k(n)}\), and
    \[\sum_{n = 0}^\infty \sum_{m = 0}^\infty |x_{n, m}| = \sum_{n = 0}^\infty x_{j(n), k(n)}.\]
\end{theorem}

\begin{lemma}
    Let \(\phi : \mathbb{N}_0 \to \mathbb{N}_0 \times \mathbb{N}_0\) be a bijection, \(\phi(n) = (j(n), k(n))\). For any \(L\), there exists some \(N\) such that \([0, L] \times [0, L] \subseteq \phi([0, N])\). i.e. \(\phi\) will always ``fill up" the \(L \times L\) square.
\end{lemma}

\begin{proof}
    Proof by contradiction. Suppose this wasn't the case for all \(N\). Then \(\phi\) would not be bijective, since there exists some \((x, y) \in [0, L] \times [0, L]\) with no preimage.
\end{proof}

\begin{proof}
    Let \(\e > 0\). Define 
    \[y_m = \sum_{n = 0}^\infty|x_{m, n}|, \hspace*{4mm} z_n = \sum_{m = 0}^\infty |x_{m, n}|, \hspace{4mm} A = \sum_{n = 0}^\infty \sum_{m = 0}^\infty x_{n, m}.\]
    By the same argument as Lemma 3.3, there exists some \(M_0\) and \(N_0\) such that, for any \(M > M_0\) and \(N > N_0\),
    \[\sum_{m = M + 1}^\infty y_m < \frac{\e}{2}, \hspace*{4mm} \sum_{n = N + 1}^\infty x_n < \frac{\e}{2}.\]

    To show that \(\sum_{n = 0}^\infty x_{j(n), k(n)}\) converges, it suffices to show that there exists some \(P\) such that
    \[\bigabs{A - \sum_{n = 0}^P x_{j(n), k(n)}} < 2 \e.\]

    Indeed, choose \(L = \max(M, N)\). Then by our above lemma, there is some \(P\) such \([0, L] \times [0, L] \subseteq \phi([0, P])\). Let \(S = [0, P] \setminus \phi^{-1}([0, L] \times [0, L])\) be the points in the interval that don't get mapped into the square. Also note that \(S \subseteq [L + 1, \infty) \times [0, \infty) \cup [0, \infty) \times [L + 1, \infty)\). Thus we have, 
    \begin{align*}
        \bigabs{\sum_{n = 0}^P x_{j(n), k(n)} - \sum_{j = 0}^L \sum_{k = 0}^L x_{j, k}} &\leq \sum_{n \in S} |x_{j(n), k(n)}| \\
        &\leq \sum_{(i, j) \in [L + 1, \infty) \times [0, \infty)} |x_{j, k}| + \sum_{(i, j) \in [0, \infty) \times [L + 1, \infty)} |x_{j, k}| \\
        &\leq \sum_{j = L + 1}^\infty y_j + \sum_{k = L + 1}^\infty x_k < \e.
    \end{align*}

    Thus we have, 
    \begin{align*}
        \bigabs{A - \sum_{n = 0}^P x_{j(n), k(n)}} &\leq \bigabs{A - \sum_{j = 0}^L \sum_{k = 0}^L x_{j, k}} + \bigabs{\sum_{n = 0}^P x_{j(n), k(n)} - \sum_{j = 0}^L \sum_{k = 0}^L x_{j, k}} \\
        &< \e + \e = 2\e
    \end{align*}
    as desired. Thus our proof is complete.
\end{proof}

\end{document}